\documentclass{llncs}

% --- -----------------------------------------------------------------
% --- Document-specific definitions.
% --- -----------------------------------------------------------------

\usepackage{amsmath,amsfonts,amssymb,amstext}
\usepackage{latexsym,ifthen,url}
\usepackage[pdftex,usenames,dvipsnames]{color}


% \usepackage{amsthm}
% \theoremstyle{plain}
% \newtheorem{theorem}{Theorem}
% \newtheorem{lemma}{Lemma}
% \newtheorem{claim}{Claim}
% \newtheorem{corollary}{Corollary}

% \theoremstyle{definition}
% \newtheorem{definition}{Definition}

% \theoremstyle{remark}
% \newtheorem{remark}{Remark}
% \newtheorem{example}{Example}

% \renewenvironment{proof}{\vspace{1mm}\noindent{\sc Proof.}}{\hfill$\square$\newline\vspace{1mm}}
\DeclareMathAlphabet{\mathscript}{OT1}{pzc}{m}{it} 

\usepackage[pdftex,colorlinks,backref]{hyperref}
\hypersetup{colorlinks=false, linkcolor=blue, menucolor=blue, urlcolor=blue, citecolor=blue}
\renewcommand*{\backref}[1]{}
\def\sectionautorefname{Section}
\def\subsectionautorefname{Section}
\def\subsubsectionautorefname{Section}
\def\appendixautorefname{Appendix}
\def\definitionautorefname{Definition}
\def\claimautorefname{Claim}
\def\lemmaautorefname{Lemma}
\def\theoremautorefname{Theorem}

%%%%%%%%%%%%%%%%%%%%%%%%%%%%%%%%%%%%%%%%%%%%%%%%%%%%%%%%%%%%%%%%%%%%%%%%%%%%%%%%
%\newcommand{\newsequenceofgames}[1]{
%  \newcounter{#1}
%  \setcounter{#1}{-1}
%
%  \ifx \GameID \undefined
%    \newcommand{\GameID}{#1}
%  \else
%    \renewcommand{\GameID}{#1}
%  \fi
%
%  \ifx \PrevLabel \undefined
%    \newcommand{\PrevLabel}{\GameID.NULL}
%  \else
%    \renewcommand{\PrevLabel}{\GameID.NULL}
%  \fi
%
%  \ifx \ThisLabel \undefined
%    \newcommand{\ThisLabel}{\GameID.NULL}
%  \else
%    \renewcommand{\ThisLabel}{\GameID.NULL}
%  \fi
%}
%
%
%\newcommand{\nextgame}[1]{
%  \let\PrevLabel\ThisLabel
%  \renewcommand{\ThisLabel}{\GameID.#1}
%  \refstepcounter{\GameID}\label{\GameID.#1}
%  \paragraph{Game~\arabic{\GameID}.}
%}
%
%\newcommand{\thisgame}{{\ref{\ThisLabel}}\xspace}
%\newcommand{\prevgame}{{\ref{\PrevLabel}}\xspace}
%%%%%%%%%%%%%%%%%%%%%%%%%%%%%%%%%%%%%%%%%%%%%%%%%%%%%%%%%%%%%%%%%%%%%%%%%%%%%%%%

\usepackage{color}
\usepackage{marginnote}
%\newcommand{\todo}[1]{{\marginnote{\textcolor{red}{TODO}}\textcolor{red}{(todo: #1)}}}
%
%\newcommand{\heading}[1]{\paragraph{\sc #1}}

\usepackage{xspace}
%\newcommand{\x}{\ensuremath{\cdot}}
%\newcommand{\Z}{\ensuremath{\mathbb{Z}}\xspace}
%\newcommand{\N}{\ensuremath{\mathbb{N}}\xspace}
%\newcommand{\Q}{\ensuremath{\mathbb{Q}}\xspace}
%\newcommand{\R}{\ensuremath{\mathbb{R}}\xspace}
%\newcommand{\F}{\ensuremath{\mathbb{F}}\xspace}
%\newcommand{\G}{\ensuremath{\mathbb{G}}\xspace}
%
%\newcommand{\ceil}[1]{\left\lceil #1 \right\rceil}
%\newcommand{\floor}[1]{\left\lfloor #1 \right\rfloor}
%\newcommand{\abs}[1]{\left| #1 \right|}
%
%\newcommand{\rand}{\stackrel{{\scriptscriptstyle\$}}{\leftarrow}}
%\newcommand{\testequal}{\stackrel{{\scriptscriptstyle ?}}{=}}
%
%\newcommand{\Adversary}{\ensuremath{\mathcal{A}}\xspace}
%\newcommand{\AdversaryB}{\ensuremath{\mathcal{B}}\xspace}
%\newcommand{\Algo}{\ensuremath{\mathcal{A}}\xspace}
%\newcommand{\AlgoB}{\ensuremath{\mathcal{B}}\xspace}
%\newcommand{\Challenger}{\ensuremath{\mathcal{C}}\xspace}
%\newcommand{\Oracle}{\ensuremath{\mathcal{O}}\xspace}

\usepackage{cleveref}
\usepackage{thmtools,thm-restate}

%%% Local Variables:
%%% mode: latex
%%% TeX-master: "main"
%%% End:

%%%%%                   %%%%%%%%%%%%%%%%%%%%%                   %%%%%
%%%%% Take out unneeded commands from this file to avoid bloat. %%%%%
%%%%%                   %%%%%%%%%%%%%%%%%%%%%                   %%%%%

%%%%%%%%%%%%%%%%%%%%%%%%%%%%%%%%%%%%%%%%%%%%%%%%%%%%%%%%%%%%%%%%%%%%%%%%%%%%%%%%%%%%%%%%%%%%%%%%%%%%%%
%%%%                              GROUPS/DISTRIBUTIONS/SETS/LISTS                                 %%%%
%%%%%%%%%%%%%%%%%%%%%%%%%%%%%%%%%%%%%%%%%%%%%%%%%%%%%%%%%%%%%%%%%%%%%%%%%%%%%%%%%%%%%%%%%%%%%%%%%%%%%%
%%%  %%%

% number sets
\newcommand{\x}{\ensuremath{\cdot}}
\newcommand{\Z}{\ensuremath{\mathbb{Z}}\xspace}
\newcommand{\N}{\ensuremath{\mathbb{N}}\xspace}
\newcommand{\Q}{\ensuremath{\mathbb{Q}}\xspace}
\newcommand{\R}{\ensuremath{\mathbb{R}}\xspace}
\newcommand{\F}{\ensuremath{\mathbb{F}}\xspace}

%% Groups, fields and rings
\newcommand{\nats}{\mathbb{N}} % Set of natural numbers
\newcommand{\ints}{\mathbb{Z}} % Set of integers
\newcommand{\prms}{\mathbb{P}} % Set of prime numbers
\newcommand{\Zp}{\ints_p} % Integers modulo p
\newcommand{\Zq}{\ints_q} % Integers modulo q
\newcommand{\Zn}{\ints_N} % Integers modulo N
\newcommand{\Zns}{\ints_{N^2}^*}
\newcommand{\Jn}{\mathbb{J}_N}
\newcommand{\qrn}{\mathbb{QR}_N}
\newcommand{\qrni}{\mathbb{QR}_{N_i}}
\newcommand{\G}{\mathbb{G}} % Group!

% operators
\newcommand{\defeq}{\coloneqq}
%\newcommand{\rand}{\xleftarrow{\smash{\raisebox{-1.75pt}{$\scriptscriptstyle\$$}}}}
\newcommand{\rand}{\stackrel{{\scriptscriptstyle\$}}{\leftarrow}}
\newcommand{\xor}{\oplus}
\newcommand{\concat}{\mathbin\Vert}

%% Distributions
\newcommand{\distr}{\mathfrak{D}}
\newcommand{\ind}{\in_{_{\distr}}}

%% Set-shaped things
\newcommand{\bits}{\{0,1\}} % Bit set
\newcommand{\msgspace}{\mathcal{M}} % Message Space
\newcommand{\tagspace}{\mathcal{T}} % Tag Space. SPPPAAAACEEE!!! SPPPPAAAAAAAAAACCCEEEEE!!!!
\newcommand{\msgspacea}{\msgspace_1} % Message Space Uno!
\newcommand{\msgspaceb}{\msgspace_2} % Message Space Dos!
\newcommand{\repspace}{\mathbf{Y}_{\secpar,\h}} % Message Reprentative Space
\newcommand{\hrepspace}{\mathbf{Y}_{\secpar+\beta,\h}} % Message Reprentative Space
\newcommand{\keyspace}{\mathcal{K}} % Key Space
\newcommand{\sigspace}{\mathfrak{S}} % Signature Space
\newcommand{\sigspacea}{\sigspace_1} % Siganture Space Uno!
\newcommand{\sigspaceb}{\sigspace_2} % Siganture Space Dos!
\newcommand{\inr}{\in_{_R}} % Uniformily random in
\newcommand{\intvalab}[2]{\llbracket #1 , #2 \rrbracket} % Int val a to b
\newcommand{\intval}[1]{\llbracket 1, #1 \rrbracket}  %Int val from 1 to x
\newcommand{\intvalz}[1]{\llbracket 0, #1 \rrbracket} % Int val from 0 to x
\newcommand{\dom}{\mathsf{Domain}} % A domain of some sort.
\newcommand{\ccls}{\mathcal{C}_{\lambda}} % Circuit class
\newcommand{\cclsf}{\mathfrak{F}_{\lambda}} % Circuit class F
\newcommand{\lang}{L}
%\newcommand{\smplset}{[N^2]}
\newcommand{\smplsetqrn}{[\floor{N/4}]}
\newcommand{\smplset}{[\floor{N/2}]}
\newcommand{\hdom}{\mathcal{U}} % domain of a hash function
\newcommand{\himg}{\mathcal{V}} % image of a hash function

%% Lists
\newcommand{\slista}{\mathcal{S}_1} % The list currently known as S
\newcommand{\slistb}{\mathcal{S}_2} % The list currently known as S
\newcommand{\mlista}{\mathcal{M}_1} % The list currently known as M
\newcommand{\mlistb}{\mathcal{M}_2} % The list currently known as M
\newcommand{\qlista}{\mathcal{Q}_1} % The list currently known as Q
\newcommand{\qlistb}{\mathcal{Q}_2} % The list currently known as Q
\newcommand{\alist}{\mathcal{A}} %  The list currently known as A
\newcommand{\clist}{\mathcal{C}} %  The list currently known as F
\newcommand{\hlist}{\mathcal{H}} %  The list currently known as H
\newcommand{\glist}{\mathcal{G}} %  The list currently known as G
\newcommand{\flist}{\mathcal{F}} %  The list currently known as F


%%%%%%%%%%%%%%%%%%%%%%%%%%%%%%%%%%%%%%%%%%%%%%%%%%%%%%%%%%%%%%%%%%%%%%%%%%%%%%%%%%%%%%%%%%%%%%%%%%%%%%
%%%%                                  ADVERSARIES AND SUCH                                        %%%%
%%%%%%%%%%%%%%%%%%%%%%%%%%%%%%%%%%%%%%%%%%%%%%%%%%%%%%%%%%%%%%%%%%%%%%%%%%%%%%%%%%%%%%%%%%%%%%%%%%%%%%
%% Adversaries
\newcommand{\advA}{\mathcal{A}} % Adversary 
\newcommand{\adv}{\mathcal{A}} % Adversary
\newcommand{\advB}{\mathcal{B}} % Simulator
\newcommand{\simr}{\mathcal{B}} % Simulator
\newcommand{\advC}{\mathcal{C}} % Challenger
\newcommand{\chal}{\mathcal{C}} % Challener
\newcommand{\advD}{\mathcal{D}} % Distinguisher
\newcommand{\advE}{\mathcal{E}} % Extractor
\newcommand{\extr}{\mathcal{E}} % Extractor
\newcommand{\dist}{\mathcal{D}} % Distinguiser
\newcommand{\advF}{\mathcal{F}} % Forger
\newcommand{\forger}{\mathcal{F}} % Forger

%% Such
\newcommand{\advtg}{\mathbf{Adv}} % The adversary's advantage
\newcommand{\zk}{\mathbf{ZK}}
\newcommand{\simsnd}{\mathbf{SimSnd}}
\newcommand{\snd}{\mathbf{Snd}}

%%%%%%%%%%%%%%%%%%%%%%%%%%%%%%%%%%%%%%%%%%%%%%%%%%%%%%%%%%%%%%%%%%%%%%%%%%%%%%%%%%%%%%%%%%%%%%%%%%%%%%
%%%%                                   ALGORITHMS/PROCEDURES                                      %%%%
%%%%%%%%%%%%%%%%%%%%%%%%%%%%%%%%%%%%%%%%%%%%%%%%%%%%%%%%%%%%%%%%%%%%%%%%%%%%%%%%%%%%%%%%%%%%%%%%%%%%%%
%% Syntax never killed anyody right?
\newcommand{\rnd}{\leftarrow_{\mbox{\tiny{\$}}}} % Randomised algorithm output
\newcommand{\get}{\leftarrow} % Normal Algorithm output
\newcommand{\true}{\mathtt{TRUE}} % TRUE boolean flag
\newcommand{\false}{\mathtt{FALSE}} % FALSE boolean flag

%% Setups and the like.
\newcommand{\setup}{\mathsf{Setup}} % Setup Algorithm, that give us the groups and such.
\newcommand{\keygen}{\mathsf{KeyGen}} % Key Generation Algorithm.
\newcommand{\gen}{\mathsf{Gen}} % Generation Algorithm.
\newcommand{\ggen}{\mathsf{GrpGen}} % Group Generation Algorithm.
\newcommand{\pgen}{\mathsf{PGen}} % Parameters Generation Algorithm.
\newcommand{\kgen}{\mathsf{KGen}} % Key Generation Algorithm.
\newcommand{\genmod}{\mathsf{GenMod}} 

\newcommand{\solve}{\mathsf{Solve}}
\newcommand{\f}{\mathsf{F}}

\newcommand{\preproc}{\mathsf{Preproc}}
\newcommand{\encode}{\mathsf{Encode}}
\newcommand{\decode}{\mathsf{Decode}}


%% Obfuscation
\newcommand{\io}{\mathscript{i}\mathcal{O}}
\newcommand{\dio}{\mathscript{di}\mathcal{O}}
\newcommand{\eo}{\mathscript{e}\mathcal{O}}

%% Encryption Schemes.
\newcommand{\otle}{\mathtt{OTLE}}
\newcommand{\tpke}{\mathtt{TPKE}}
\newcommand{\tlp}{\mathtt{TLP}}
\newcommand{\tlppp}{\mathtt{TLP \mhyphen PP}}
\newcommand{\stlp}{\mathtt{sTLP}}
\newcommand{\wtlp}{\mathtt{wTLP}}
\newcommand{\pptlp}{\mathtt{ppTLP}}
\newcommand{\re}{\mathtt{RE}}
\newcommand{\tre}{\mathtt{TRE}}
\newcommand{\trepp}{\mathtt{TRE \mhyphen PP}}
\newcommand{\hm}{\mathtt{H}}
\newcommand{\htre}{\mathtt{HTRE}}
\newcommand{\ftre}{\mathtt{FTRE}}
\newcommand{\pke}{\mathtt{PKE}}
\newcommand{\he}{\mathtt{HE}}
\newcommand{\e}{\mathtt{E}}
\newcommand{\fhe}{\mathtt{FHE}}
\newcommand{\we}{\mathtt{WE}}
\newcommand{\eowe}{\mathtt{EOWE}}
\newcommand{\owe}{\mathtt{OWE}}
\newcommand{\fwe}{\mathtt{FWE}}
\newcommand{\ofwe}{\mathtt{OFWE}}
\newcommand{\eofwe}{\mathtt{EOFWE}}
\newcommand{\tbe}{\mathtt{TBE}}
%\newcommand{\tbenc}{\mathtt{TBE}}
\newcommand{\penc}{\mathtt{PE}}
% Encryption Algorithms
\newcommand{\enc}{\mathsf{Enc}}
\newcommand{\dec}{\mathsf{Dec}}
\newcommand{\decf}{\mathsf{Dec}_f}
\newcommand{\decs}{\mathsf{Dec}_s}
\newcommand{\eval}{\mathsf{Eval}}
% Tag Encryption Algorithms
\newcommand{\tenc}{\mathsf{TEnc}}
\newcommand{\tdec}{\mathsf{TDec}}
% Puncturable Encryption Algorithms
\newcommand{\punct}{\mathsf{Punct}}
\newcommand{\pdec}{\mathsf{PDec}}
% Functional Encryption Algorithms
\newcommand{\fe}{\mathtt{FE}}
\newcommand{\msk}{\mathsf{msk}}
\newcommand{\FF}{\mathcal{F}}
\newcommand{\ID}{\mathcal{ID}}
\newcommand{\MM}{\mathcal{M}}
\newcommand{\OO}{\mathcal{O}}
\newcommand{\XX}{\mathcal{X}}
\newcommand{\YY}{\mathcal{Y}}
\newcommand{\ff}{f}
\newcommand{\xx}{x}
\newcommand{\yy}{y}
\newcommand{\Sim}{\mathcal{S}}
\newcommand{\simgen}{\mathsf{\widetilde{Gen}}}
\newcommand{\simkeygen}{\mathsf{\widetilde{KeyGen}}}
\newcommand{\simenc}{\mathsf{\widetilde{Enc}}}
\newcommand{\trfe}{\mathsf{TRFE}}


\newcommand{\ro}{\mathsf{RO}}
\newcommand{\gss}{\mathtt{GSS}}
\newcommand{\fac}{\mathtt{Factor}}
\newcommand{\factor}{\mathsf{Factor}}
\newcommand{\dla}{\mathsf{DL}_\adv}
\newcommand{\gnr}{\mathsf{GNR}}
\newcommand{\fail}{\mathsf{FAIL}}
\newcommand{\face}{\mathsf{FACTOR}}
\newcommand{\gsse}{\mathsf{GSS}}
\newcommand{\success}{\mathsf{SUCCESS}}
\newcommand{\forge}{\mathsf{FRG}} 
\newcommand{\event}{\mathsf{E}} 


%% Signature Schemes.
\newcommand{\ots}{\mathtt{OTS}} % One-time Signature Scheme
\newcommand{\sig}{\mathtt{Sig}} % Signature Scheme.
% Signature Algorithms
\newcommand{\sign}{\mathsf{Sign}} % Signing Algorithm.
\newcommand{\verify}{\mathsf{Verify}} % Verification Algorithm.
\newcommand{\vrfy}{\mathsf{Vrfy}} % Verification Algorithm.


%% Commitments
\newcommand{\tc}{\mathtt{TC}}
\newcommand{\nitc}{\mathtt{NITC}}
\newcommand{\com}{\mathsf{Com}}
\newcommand{\cvrfy}{\mathsf{ComVrfy}}
\newcommand{\decom}{\mathsf{Dec}}
\newcommand{\dvrfy}{\mathsf{DecVrfy}}
\newcommand{\fdecom}{\mathsf{FDec}}
\newcommand{\fdvrfy}{\mathsf{FDecVrfy}}


%% NIZK/SNARK
\newcommand{\nizk}{\mathtt{NIZK}}
\newcommand{\niwi}{\mathtt{NIWI}}

%% Proof systems
\newcommand{\poe}{\mathtt{PoE}}
\newcommand{\prv}{\mathsf{Prover}} % Prover
\newcommand{\vrf}{\mathsf{Verifier}} %Verifier
\newcommand{\snark}{\mathtt{SNARK}}
\newcommand{\prove}{\mathsf{Prove}}
%\newcommand{\simul}{\mathsf{Sim}}
\newcommand{\crs}{\mathsf{crs}} % Common Reference String


%% Hashes, Chameleon or otherwise.
\newcommand{\CRH}{\mathsf{CRHF}} % Collision Resistant Hash Function.
% Collision Resitant Hash Algorithms.
\newcommand{\hash}{\mathsf{Hash}} % Hash Algorithm.
\newcommand{\h}{\mathtt{H}}

%% Oracles
\newcommand{\ddhvf}{\mathsf{DDHvf}} % The DDH Oracle
\newcommand{\dss}{\mathsf{DSSvf}} % The Decisional Sequential Squaring oracle
\newcommand{\deco}{\mathsf{DEC}} %decryption oracle
\newcommand{\signo}{\mathsf{SIGN}} %signing oracle
\newcommand{\kgeno}{\mathsf{KEYGEN}} %signing oracle

%%%%%%%%%%%%%%%%%%%%%%%%%%%%%%%%%%%%%%%%%%%%%%%%%%%%%%%%%%%%%%%%%%%%%%%%%%%%%%%%%%%%%%%%%%%%%%%%%%%%%%
%%%%                                         ASSUMPTIONS                                          %%%%
%%%%%%%%%%%%%%%%%%%%%%%%%%%%%%%%%%%%%%%%%%%%%%%%%%%%%%%%%%%%%%%%%%%%%%%%%%%%%%%%%%%%%%%%%%%%%%%%%%%%%%
\newcommand{\dl}{\ensuremath{\mathsf{DLog}}\xspace} % The Discrete Logarithm Assumption
\newcommand{\ddh}{\ensuremath{\mathsf{DDH}}\xspace} % The Decisional Diffie-Hellman Assumption
\newcommand{\dlin}{\ensuremath{\mathsf{DLin}}\xspace} % The Descision Linear Assumption
\newcommand{\gdlin}{\ensuremath{\mathsf{GapDLin}}\xspace} % The Descision Linear Assumption
\newcommand{\ssa}{\ensuremath{\mathsf{SS}}\xspace}
\newcommand{\sss}{\ensuremath{\mathsf{SSS}}\xspace}
\newcommand{\dcr}{\ensuremath{\mathsf{DCR}}\xspace} % The Decisional Diffie-Hellman Assumption

%%%%%%%%%%%%%%%%%%%%%%%%%%%%%%%%%%%%%%%%%%%%%%%%%%%%%%%%%%%%%%%%%%%%%%%%%%%%%%%%%%%%%%%%%%%%%%%%%%%%%%
%%%%                                    KEYS & CONSTANTS                                          %%%%
%%%%%%%%%%%%%%%%%%%%%%%%%%%%%%%%%%%%%%%%%%%%%%%%%%%%%%%%%%%%%%%%%%%%%%%%%%%%%%%%%%%%%%%%%%%%%%%%%%%%%%
%% Keys
\newcommand{\gk}{\mathsf{gk}} % Description of the pairing group
\newcommand{\pk}{\mathsf{pk}} % Public Key.
\newcommand{\sk}{\mathsf{sk}} % Secret Key (Overloads to Signing Key).
\newcommand{\ek}{\mathsf{ek}} % Encryption Key.
\newcommand{\dk}{\mathsf{dk}} % Decryption Key.
\newcommand{\td}{\mathsf{td}} % Trapdoor.
\newcommand{\vk}{\mathsf{vk}} % Verfication Key.
\newcommand{\otsk}{\mathsf{sk_{OT}}} % One Time Signing Key.
\newcommand{\csk}{\mathsf{sk^*_{OT}}} % One Time Signing
\newcommand{\otvk}{\mathsf{vk_{OT}}} % One Time Verification Key.
\newcommand{\cvk}{\mathsf{vk^*_{OT}}} % One Time Verification Key.
\newcommand{\spr}{\mathsf{sp}} % System Parameters.
\newcommand{\pp}{\mathsf{pp}}
\newcommand{\ppe}{\mathsf{pp}_e} % Encryption parameters
\newcommand{\ppd}{\mathsf{pp}_d} % Decryption parameters
\newcommand{\ppei}{\mathsf{pp}_{e,i}} % Encryption parameters
\newcommand{\ppdi}{\mathsf{pp}_{d,i}} % Decryption parameters
\newcommand{\ppej}{\mathsf{pp}_{e,j}} % Encryption parameters
\newcommand{\ppdj}{\mathsf{pp}_{d,j}} % Decryption parameters
\newcommand{\ppeft}{\mathsf{pp}_{e,T_i,\FF'}} % Encryption parameters
\newcommand{\ppdft}{\mathsf{pp}_{d,T_i,\FF'}} % Encryption parameters
\newcommand{\param}{\mathsf{par}} % parameters 
\newcommand{\oracle}{\mathcal{O}}
\newcommand{\funs}{\mathsf{Funs}[\hdom, \himg]}
\newcommand{\rerand}{\mathsf{Rand}}

%% Constants

%%%%%%%%%%%%%%%%%%%%%%%%%%%%%%%%%%%%%%%%%%%%%%%%%%%%%%%%%%%%%%%%%%%%%%%%%%%%%%%%%%%%%%%%%%%%%%%%%%%%%%
%%%%                                          NAMES                                               %%%%
%%%%%%%%%%%%%%%%%%%%%%%%%%%%%%%%%%%%%%%%%%%%%%%%%%%%%%%%%%%%%%%%%%%%%%%%%%%%%%%%%%%%%%%%%%%%%%%%%%%%%%
\newcommand{\hexa}[1]{\mathtt{0x#1}} % Hexadecimal values

%% Security Defs
\newcommand{\ufcma}{\mathsf{UF \mhyphen CMA}}
\newcommand{\ufnma}{\mathsf{UF \mhyphen NMA}}
\newcommand{\rom}{\mathsf{ROM}}
%\newcommand{\sm}{\mathsf{StdM}}

\newcommand{\cpa}{\mathsf{CPA}}
\newcommand{\cca}{\mathsf{CCA}}
\newcommand{\ror}{\mathsf{RoR}}
%%Known Schemes
\newcommand{\ktbe}{\mathtt{KiltzTBE}}





%%%%%%%%%%%%%%%%%%%%%%%%%%%%%%%%%%%%%%%%%%%%%%%%%%%%%%%%%%%%%%%%%%%%%%%%%%%%%%%%%%%%%%%%%%%%%%%%%%%%%%
%%%%                                           MISC                                               %%%%
%%%%%%%%%%%%%%%%%%%%%%%%%%%%%%%%%%%%%%%%%%%%%%%%%%%%%%%%%%%%%%%%%%%%%%%%%%%%%%%%%%%%%%%%%%%%%%%%%%%%%%
\mathchardef\hyphen="2D % Hyphen. Because it has to be done this way for reasons.
\mathchardef\mhyphen="2D % Hyphen. Because it has to be done this way for reasons.
\newcommand{\tab}{\hspace*{2em}} % Tabs for code
\newcommand{\half}{\frac{1}{2}} % Half
\newcommand{\qtr}{\frac{1}{4}} % Quarter
\newcommand{\et}{$e^{\text{th}}$} % e-th cause why not make it easy?
\newcommand{\secpar}{\ensuremath{\lambda\xspace}} % Security Parameter
\newcommand{\seck}{1{^\secpar}} % Unary represntation of our security parameter
\newcommand{\seckt}{1^{2\secpar}} % Unary represntation of our security parameter DOUBLED!
\newcommand{\negl}{\mathsf{negl}}
\newcommand{\poly}{\mathsf{poly}}
\newcommand{\polylog}{\mathsf{polylog}}
\newcommand{\inp}{\mathsf{input}}
\newcommand{\tilT}{\tilde{T}}
\newcommand{\undT}{\underline{T}}
\newcommand{\undE}{\underline{\epsilon}}
\newcommand{\dep}{\mathsf{depth}}
\newcommand{\aux}{\mathsf{aux}}
%\newcommand{\max}{\mathsf{max}}

\newcommand{\win}{\mathsf{Win}}
\newcommand{\query}{\mathsf{Query}}
\newcommand{\games}{\mathsf{G}} % game
\newcommand{\expe}{\mathsf{Exp}} % game
\newcommand{\te}{t_{e}} % run time of encryption
\newcommand{\tfd}{t_{fd}} % run time of fast decryption
\newcommand{\tsd}{t_{sd}} % run time of slow decryption
\newcommand{\st}{\mathsf{st}} % state
\newcommand{\gap}{\epsilon} % gap
\newcommand{\ugap}{\underline{\gap}}

\newcommand{\epke}{\epsilon_{\pke}}
\newcommand{\etbe}{\epsilon_{\tbe}}
\newcommand{\etlp}{\epsilon_{\tlp}}
\newcommand{\eots}{\epsilon_{\ots}}

\newcommand{\simul}{\mathsf{Sim}}

\newcommand{\clock}{\mathcal{C}}
\newcommand{\rel}{\mathcal{R}}

\newcommand{\circC}{C_{\sk}} % circuit with hardcoded secret key
\newcommand{\obfC}{\widetilde{C}_{\sk}} % obfuscated circuit with hardcoded secret key

\newcommand{\plm}{\mathsf{p}(\secpar)}
\newcommand{\qlm}{\mathsf{q}(\secpar)}
%\newcommand{\plm}{\alpha(\lambda)}
%\newcommand{\qlm}{\poly(\lambda,1/(\varepsilon - \alpha(\secpar)))}
\usepackage{stmaryrd}
\newcommand{\heading}[1]{\paragraph{\sc #1}}

\newcommand{\T}{\mathsf{T}}
\newcommand{\np}{\mathtt{NP}}

\newcommand{\ceil}[1]{\left\lceil #1 \right\rceil}
\newcommand{\floor}[1]{\left\lfloor #1 \right\rfloor}
\newcommand{\abs}[1]{\left| #1 \right|}
\newcommand{\sd}{\mathbb{SD}}
\mathchardef\mhyphen="2D
\newcommand{\ord}{\varphi(N)/2}
\newcommand{\ordqrn}{\varphi(N)/4}
\newcommand{\estord}{\floor{N/2}}
\newcommand{\estordqrn}{\floor{N/4}}
\newcommand{\prot}[1]{\langle #1 \rangle}

\newcommand{\fm}{a} % first message
\newcommand{\sm}{c} % second message
\newcommand{\tm}{z} % third message

\newcommand{\frset}{A} %first set
\newcommand{\scset}{\mathcal{C}} %second set
\newcommand{\thset}{\mathcal{Z}} %third set
\newcommand{\sset}{\mathcal{S}}
\newcommand{\wset}{\mathcal{W}}

\newcommand{\pr}{\mathsf{P}} % Prover algorithm
\newcommand{\vr}{\mathsf{V}} %Verifier algorithm



%%%%%%%%%%%%%%%%%%%%%%%%%%%%%%%%%%%%%%%%%%%%%%%%%%%%%%%%%%%%%%%%%%%%%%%%%%%%%%%%%%%%%%%%%%%%%%%%%%%%%%
%%%%                                    COMMENTS AND NOTES                                        %%%%
%%%%%%%%%%%%%%%%%%%%%%%%%%%%%%%%%%%%%%%%%%%%%%%%%%%%%%%%%%%%%%%%%%%%%%%%%%%%%%%%%%%%%%%%%%%%%%%%%%%%%%
\newlength{\strutdepth}%
\settodepth{\strutdepth}{\strutbox}%
\newcommand{\saqib}[1]{%
	\noindent{\bfseries
	\color{orange}{#1}\color{black}}%
    \strut\vadjust{\kern-\strutdepth%
        \vtop to \strutdepth{%
            \baselineskip\strutdepth%
            \vss\llap{{\large\color{blue}Saqib\quad\color{black}}}\null%
        }%
    }%
}
\newcommand{\peter}[1]{%
	\noindent{\bfseries
	\color{orange}{#1}\color{black}}%
    \strut\vadjust{\kern-\strutdepth%
        \vtop to \strutdepth{%
            \baselineskip\strutdepth%
            \vss\llap{{\large\color{green}Peter\quad\color{black}}}\null%
        }%
    }%
}


\newcommand{\daniel}[1]{{\color{blue} \textbf{Daniel:} #1}}
\newcommand{\christoph}[1]{{\color{blue} \textbf{Christoph:} #1}}
\newcommand{\tibor}[1]{{\color{blue} \textbf{Tibor:} #1}}


\newcommand{\etal}{\emph{et~al.}\xspace}
\newcommand{\eg}{\emph{e.g.}\xspace}
\newcommand{\ie}{\emph{i.e.}\xspace}


\newcommand*{\numero}{n\kern-.1em \raise.7ex\vbox{\hbox{\tiny \ensuremath{\circ}}\kern.5pt}}

\newcommand{\todo}[1]{{\marginnote{\textcolor{red}{TODO}}\textcolor{red}{(todo: #1)}}}

%for submission
\crefname{appendix}{supplementary material}{supplementary materials}
\Crefname{appendix}{Supplementary Material}{Supplementary Materials}

%%%%%%%%%%%%%%%%%%%%%%%%%%%%%%%%%%%%%%%%%%%%%%%%%%%%%%%%%%%%%%%%%%%%%%%%%%%%%%%%
\newcommand{\newsequenceofgames}[1]{
  \newcounter{#1}
  \setcounter{#1}{-1}

  \ifx \GameID \undefined
    \newcommand{\GameID}{#1}
  \else
    \renewcommand{\GameID}{#1}
  \fi

  \ifx \PrevLabel \undefined
    \newcommand{\PrevLabel}{\GameID.NULL}
  \else
    \renewcommand{\PrevLabel}{\GameID.NULL}
  \fi

  \ifx \ThisLabel \undefined
    \newcommand{\ThisLabel}{\GameID.NULL}
  \else
    \renewcommand{\ThisLabel}{\GameID.NULL}
  \fi
}


\newcommand{\nextgame}[1]{
  \let\PrevLabel\ThisLabel
  \renewcommand{\ThisLabel}{\GameID.#1}
  \refstepcounter{\GameID}\label{\GameID.#1}
  \paragraph{Game~\arabic{\GameID}.}
}

\newcommand{\thisgame}{{\ref{\ThisLabel}}\xspace}
\newcommand{\prevgame}{{\ref{\PrevLabel}}\xspace}


% workaround for making sure that list/tuples break at the comma
% modified from https://tex.stackexchange.com/questions/19094/allowing-line-break-at-in-inline-math-mode-breaks-citations/19100#19100
\mathchardef\breakingcomma\mathcode`\,
{\catcode`,=\active
    \gdef,{\breakingcomma\discretionary{}{}{}}
}
\newcommand{\mathlist}[1]{\ensuremath{\mathcode`\,=\string"8000 #1}}
%%%%%%%%%%%%%%%%%%%%%%%%%%%%%%%%%%%%%%%%%%%%%%%%%%%%%%%%%%%%%%%%%%%%%%%%%%%%%%%%
%%% Local Variables:
%%% mode: latex
%%% TeX-master: "main"
%%% End:




%\usepackage{hyperref}

%\usepackage{cleveref}

%\usepackage{graphicx}

%\usepackage{subcaption}

\makeatletter
\newcommand\rraggedright{%
  \let\\\@centercr\@rightskip\@flushglue \rightskip\@rightskip
  \leftskip\z@skip}
\makeatother

\author{\vspace{-5mm}}
\institute{\vspace{-5mm}}

\pagestyle{plain}

% --- -----------------------------------------------------------------
% --- The document starts here.
% --- -----------------------------------------------------------------
\begin{document}

\title{Simple, Fast, and Efficient Non-Malleable Non-Interactive Timed Commitments}


\maketitle
\begin{abstract}
Timed commitment schemes, introduced by Boneh and Naor (CRYPTO 2000) can be used to achieve fairness in secure computation protocols in a simple and elegant way.
The only known non-malleable construction in the standard model is due to Katz, Loss, and Xu (TCC 2020). This construction requires general-purpose zero knowledge proofs with specific properties, and it suffers from an inefficient commitment protocol, which requires the committing party to solve a computationally expensive puzzle.

We propose new constructions of non-malleable non-interactive timed commitments, which combine (an extension of) the Naor-Yung paradigm used to construct IND-CCA secure encryption with a non-interactive ZK proofs for a simple algebraic language. This yields much simpler and more efficient non-malleable timed commitments in the standard model.

Furthermore, our constructions also compare favourably to known constructions of timed commitments in the random oracle model, as they achieve several further interesting properties that make the schemes very practical. This includes the possibility of using a homomorphism for the forced opening of multiple commitments in the sense of Malavolta and Thyagarajan (CRYPTO 2019), and they are the first constructions to achieve \emph{public verifiability}, which seems particularly useful to apply the homomorphism in practical applications.
\end{abstract}


%!TEX root=main.tex
\section{Introduction}\label{sec:intro}

Timed commitments make it possible to commit to a message with respect to some time parameter $T \in \N$, such that (1) the commitment is \emph{binding} for the committing party, (2) it is \emph{hiding} the committed message for $T$ units of time (\eg, seconds, minutes, days), but (3) it can also forcibly be opened after time $T$ in case the committing party refuses to open the commitment or becomes unavailable. This idea goes back to a seminal work by Rivest, Shamir, and Wagner \cite{RSW96} introducing the strongly related notion of \emph{time-lock puzzles}, and Boneh and Naor~\cite{C:BonNao00} extended this idea to \emph{timed commitments}, which have the additional feature that an opening to the commitment can be efficiently verified (and thus the commitment can be opened efficiently).

\paragraph{Achieving fairness via timed commitments.}
One prime application of timed commitmens is to achieve \emph{fairness} in secure two- or multi-party protocols. For instance, consider a simple sealed-bid auction protocol with $n$ bidders $B_1, \ldots, B_n$, where every bidder $B_i$ commits to its bid $x_i$ and publishes the commitment $c_i = \com(x_i,r_i)$ using randomness $r_i$. When all bidders have published their commitments, everyone reveals their bid $x_i$ along with $r_i$, such that everyone can publicly verify that the claimed bid $x_i$ is indeed consistent with the initial commitment $c_i$. The bidder with the maximal bid wins the auction.
For this to be most practical, we want commitments to be \emph{non-interactive}.

Now suppose that after the first $(n-1)$ bidders $B_1, \ldots, B_{n-1}$ have opened their commitments $(x_i, r_i)$, the last bidder $B_n$ claims that it has ``lost'' its randomness $r_{i^*}$, \eg, by accidentally deleting it. However, $B_n$ also argues strongly and quite plausibly that it has made the highest bid $x_{i^*}$. This is a difficult situation to resolve in practice:
\begin{itemize}
	\item \emph{$B_n$ might indeed be honest.} In this case, it would be fair to accept its highest bid $x_{i^*}$. One could argue that it is $B_n$'s own fault and thus it should not win the auction, but at the same time a seller might strongly argue to accept the bid, as it is interested in maximising the price, and if $B_n$'s claim is indeed true, then discarding the real highest bit could be considerd unfair by the seller.
	\item \emph{However, $B_n$ might also be cheating.} Maybe it didn't commit to the highest bid, and now $B_n$ tries to ``win'' the auction in an unfair way.
\end{itemize}

Timed commitments can resolve this situation very elegantly and without the need to resort to a third party that might collude with bidders, and thus needs to be trusted, or which might not even be available in certain settings, \eg, in fully decentralized protocols, such as blockchain-based applications. 
In a timed commitment scheme, the parties create their commitments $c_i = \com(x_i,r_i, T)$ with respect to a suitable time parameter $T$ for the given application. In case one party is not able to or refuses to open its commitment, the other parties can force the commitment open in time $T$ and thus resolve a potential dispute. 


\subsection{Requirements on Practical Timed Commitments}
Several challenges arise when constructing timed commitments that can be used in practical applications.

\begin{description}
	\item[Consistency of standard and forced opening.] A first challenge to resolve when constructing a timed commitment scheme is to guarantee that the availability of an alternative way to open a commitment, by using the forced decommitment procedure, does not break the \emph{binding} property. Standard and forced opening must be guaranteed to reveal the same message. Otherwise, a malicious party could create a commitment where standard and forced openings yields different values. Then it could decide in the opening phase whether it provide the ``real'' opening, or whether it refuses to open, such that the other parties will perform the forced opening.
	
	\item[Non-interactivity.] Having non-interactive commitments is generally desirable to obtain protocols that do not require all parties to be online at the same time. Furthermore, certain applications inherently require the commitment scheme to be non-interactive. This includes, for example, protocols where the commitments are published in a public ledger (\eg, a decentralized blockchain). Several examples of such applications are described in \cite{C:MalThy19}. Non-interactivity also avoids concurrent executions of the commitment protocol, which simplifies the security model significantly.


	\item[Non-malleability.] 
	Non-malleability of a commitment guarantees that no party can turn a given commitment $c$ that decommits to some value $x$ into another commitment $c'$ which decommits to a different value $x'$, such that $x$ and $x'$ are related in some meaningful way.
	%
	For instance, in the above example of an auction, a malicious party $B_n$ could first wait for all other parties to publish their commitments. Then it would select the commitment $c_i$ which most likely contains the highest bid $x_i$, and exploit the malleability of to create a new commitment $c_n$, which is derived from $c_i$ and opens to $x_i + 1$. Hence, $B_n$ would be able win the auction with a bid that is only slightly larger than the 2\textsuperscript{nd} highest bit, which does not meet the intuitive security expectations on a secure auctioning protocol.

	In order to achieve non-malleability for timed commitments, a recent line of works has explored the idea of \emph{non-malleable time-locked commitments} and \emph{puzzles} \cite{TCC:KatLosXu20,EPRINT:EFKP20a,EC:BDDNO21}.
%
	% Ephraim is FO with TLPs, not homomorphic
	% EC21 TLPs in UC, automatically achieve NM, RO, no public verifiability, not homomorphic
%
	Existing constructions of timed commitments are either malleable, rely on the random oracle model, or require the sender of the commitment to invest as much effort to commit to a value as for the receiver to forcibly open the commitment. 
%
	The only known standard model construction by Katz \etal \cite{TCC:KatLosXu20} relies on non-interactive zero-knowledge proofs (NIZKs) for \emph{general} NP relations with very specific properties. 



	

	\item[Force opening many commitments at once via homomorphism.] 
	Yet another interesting property that can make timed commitments more practical is a possibility to aggregate multiple commitments into a \emph{single} one, such that it is sufficient to force open only this commitment. The idea of homomorphic time-lock \emph{puzzles} was introduced by Malavolta and Thyagarajan~\cite{C:MalThy19}. We consider the adoption of this idea to timed commitments. 

	A homomorphic timed commitment scheme allows to efficiently evaluate a circuit $C$ over a set of commitments $c_1, \ldots, c_n$, where $c_i$ is a commitment to some value $x_i$ for all $i$, to obtain a commitment $c$ to $C(x_1, \ldots, x_n)$. 
	%
	If there are multiple parties $B_{i_1}, \ldots, B_{i_z}$ that refuse to open their commitments and it is not necessary to recover the full committed messages $x_{i_1}, \ldots, x_{i_z}$, but recovering $C(x_{i_1}, \ldots, x_{i_z})$ is sufficient, then one can use the homomorphism to compute a signle commitment $c$ that needs to be opened. Malavolta and Thyagarajan~\cite{C:MalThy19} describe several interesting applications, includings e-voting and sealed-bid auctions over blockchains, multi-party coin flipping, and multi-party contract signing.

	\item[Public verifiability of commitments.] 
	Another property is \emph{public verifiability} of a timed commitment, which requires that one can efficiently check whether a commitment is well-formed, such that a forced decommitment will yield a correct result. 
%This property was first suggested for time-lock puzzles by \cite{EPRINT:EFKP20a}.

	Without public verifiability, timed commitments might not provide practical solutions for certain applications. For instance, a malicious party could output a malformed commitment that cannot be opened in time $T$, such that a protocol would fail again in case the malicious party refuses to open the commitment. This could pose a problem in time-sensitive applications, in particular if a large time parameter $T$ is used, and also give rise do Denial-of-Service attacks.
	Note that public verifiability is particularly relevant for homomorphic commitments. When many commitments are aggregated into a single one, then it is essentiall that all these commitments are well-formed, as otherwise the forced opening may fail. Public verifiability allows to efficiently decide which subset of commitments is well-formed, and thus to include only these in the homomorphic aggregate.

	Note that the requirement of public verifiability rules out several natural ways to achieve non-malleability, such as the Fujisaki-Okamoto transform \cite{C:FujOka99,JC:FujOka13} used by Ephraim \etal \cite{EPRINT:EFKP20a}. It seems that even in the random oracle model ZK proofs are required.
	% However, even the construction by Katz \etal \cite{TCC:KatLosXu20}, which intensively uses ZK proofs, does not yet achieve public verifiability. \todo{is this true? or only not yet defined?}
	\item[Public verifiability of forced opening.] In scenarios when the forced opening is executed by untrusted party, it is desirable to be able efficiently check that forced opening has been executed properly without redoing an expensive sequential computation. This particularly useful when the forced opening computation is outsourced to untrusted server. This property was first suggested for time-lock puzzles by \cite{EPRINT:EFKP20a}. 
\end{description}









\subsection{Our Contributions}

We provide a simpler and more efficient approach to construct practical non-malleable timed commitments. We give the first constructions that simultaneously achieve non-interactivity, non-malleability, linear (\ie, additive) or multiplicative homomorphism, public verifiability of commitments and public verifiability of forced opening. 
Instead of relying on expensive ZK proofs for general NP languages as prior work, we show how to use Fiat-Shamir \cite{C:FiaSha86} NIZKs derived from Sigma protocols for simple algebraic languages. Our constructions can be instantiated in the standard model by leveraging techniques from Libert \etal \cite{Libert2021OneShotFN} and more efficiently in the random oracle model.

In more detail, we make the following contributions.
\begin{enumerate}
\item We begin by extending the formal definitions of prior work to cover public verifiability of forced opening and homomorphic properties in the setting of non-malleable non-interactive timed commitments. 
%Public verifiability allows to anyone verify that the claimed opening computed by force is indeed value which is contained in the given commitment. This is useful in settings where forced opening is executed by some third party which is not necessarily trusted. Moreover, if we are only interested in the result of some computation on a set of NITCs, one can instead of force opening all unopened commitments, simply homomorphically combine commitments and then force open the resulting commitment. We remark, that for these to work, one must be sure that all commitments are properly generated, which is easy to check using $\cvrfy$ algorithm of NITC. Hence, one must at first check that all commitments are properly generated, then homomorphically combine commitments to obtain a final commitments, which is then opened by force. If some untrusted party is executing these steps, then everybody is able to verify that computation was executed properly, by verifying that all commitments are generated properly, homomorphically combining commitments and then checking a proof of forced opening that was provided by untrusted party with respect to the final commitment which was computed by us. 
\item We then give four constructions of non-interactive non-malleable timed commitments. All our constructions rely on a variation of the double encryption paradigm by Naor and Yung \cite{STOC:NaoYun90}, which was also used by Katz \etal \cite{TCC:KatLosXu20}. 

However, in contrast to \cite{TCC:KatLosXu20}, we do not start from a timed public key encryption scheme, but build our timed commitment from scratch. This enables us avoid two out of the three NIZK proofs in their construction, and lets us replace the third by a proof for a variation of the DDH relation over groups of unknown order. We are able to instantiate the given NIZK both in the standard model and in the random oracle model \cite{CCS:BelRog93}. Like the construction from \cite{TCC:KatLosXu20} we support public verifiability of commitments. Another important advantage of our constructions over that of Katz \etal \cite{TCC:KatLosXu20} is that it allows for fast commitment, whereas \cite{TCC:KatLosXu20} requires solving a puzzle for commitments.Additionally we achieve, public verifiability of forced opening and homomorphic properties. 
\end{enumerate}

%We give a comparison of our schemes with these works in ....


%\item\todo{some of the informations in this paragraph are not totally correct} We can also make this construction non-interactive by using a single simulation sound NIZK. This is a far simpler requirement than what is needed for the construction of Katz et al., who need three NIZKs and also require that the simulated prover runs in time independently of the size of a witness. 

In comparison, the non-interactive construction of David \etal~\cite{EC:BDDNO21} is in the programmable random oracle model, while ours can also be instantiated in the standard model. David \etal achieve fast commitments, however the construction does not provide public verifiability of commitments, public verifiability of forced opening nor homomorphic properties. The work of Ephraim et al. \cite{EPRINT:EFKP20a} does support fast commitments and public verifiability of forced opening, but is also in the (auxiliary non-programmable) random oracle model and does not support public verifiability of commitments and homomorphic properties.


\todo{Connect table to rest of intro.}

\begin{table}
\begin{center}
\begin{tabular}{llclcclllc}
\textbf{Construction}         & \textbf{Hom.} & \textbf{Std.} & \textbf{Setup} & \textbf{Com?} & \textbf{FDec?} & $|\textbf{Com}|$ & \textbf{$|\pi|$} & $t_{Com}$   & \textbf{Tight} \\
\hline
\cite{EPRINT:EFKP20a}         & ---           & \xmark        & ---            & \xmark        & \cmark         & $O(1)$           &                  & $O(\log T)$ & \cmark         \\
\cite{TCC:KatLosXu20}         & ---           & \cmark        & priv.          & \cmark        & \xmark         & $O(1)$           &                  & $O(T)$      & \cmark         \\
\cite{CCS:TCLM21}             & linear        & \xmark        & pub            & \cmark        & \xmark         & $O(\secpar)$     &                  & $O(1)$      & \xmark         \\
\hline
\Cref{sec:linearNITCstdmodel} & linear        & \cmark        & priv.          & \cmark        & \cmark         & $O(\log\secpar)$ &                  & $O(1)$      & \cmark         \\
\Cref{sec:multNITCstdmodel}   & mult.         & \cmark        & priv.          & \cmark        & \cmark         & $O(\log\secpar)$ &                  & $O(1)$      & \cmark         \\
\Cref{sec:linear-ROM}         & linear        & \xmark        & priv.          & \cmark        & \cmark         & $O(1)$           &                  & $O(1)$      & \cmark         \\
\Cref{sec:mult-ROM}           & mult.         & \xmark        & priv.          & \cmark        & \cmark         & $O(1)$           &                  & $O(1)$      & \cmark         \\
\hline
\end{tabular}
\caption{\label{tab:comparison-related-work}Comparison of our constructions with related work. Column \textbf{Hom.} indicates whether the construction provides a linear/multiplicative homomorphism, \textbf{Std.} whether the construction has a standard-model proof, \textbf{Com?} whether it is publicly verifiable that commitments are well-formed, \textbf{FDec?} efficient public verifiability of forced decommitments, $|\textbf{Com}|$ is the size of commitments, \textbf{$|\pi|$} the size of proofs,  $t_{Com}$ the running time of the commitment algorithm, and \textbf{Tight} whether the proof avoids running the forced decommitment algorithm to respond to CCA queries.}
\end{center}
\end{table}






%%% Local Variables:
%%% mode: latex
%%% TeX-master: "main"
%%% End:

%!TEX root=main.tex

\subsection{Technical Overview}\label{sec:techoverview}
The \emph{binding} property of our commitment scheme will be relatively easy to argue, therefore let us focus on the \emph{hiding} property and non-malleability. Like in \cite{TCC:KatLosXu20}, we prove this by considering an IND-CCA security experiment, where the adversary has access to a forced decommitment oracle. 
Even though the forced decommitment can be performed in polynomial time, this polynomial may be very large, if the time parameter $T$ is large. Since the experiment needs to perform a forced decommitment for \emph{every} decommitment query of the adversary, this would incur a very significant overhead and a highly lossy reduction.
Hence, following Katz \etal \cite{TCC:KatLosXu20}, we aim to build commitment schemes where a reduction can perform a fast decommitment. 

Recall that a classical approach to achieve IND-CCA security is to apply the Naor-Yung paradigm \cite{STOC:NaoYun90}. A natural approach to construct non-malleable timed commitments is therefore to apply this paradigm as follows. A commitment $c = (c_1, c_2, \pi)$ to a message $m$ consists of a time-lock puzzle $c_1$ opening to $m$, a public key encryption of $m$, and a simulation-sound zero knowledge proof $\pi$ that both contain the same message $m$, everything with respect to public parameters contained in a public common reference string.
This scheme may potentially achieve all desired properties:
\begin{itemize}
	\item Consistency of regular and forced opening can be achieved by using a suitable time-lock puzzle and public-key encryption scheme. 
	\item The commitment is non-interactive.
	\item IND-CCA security follows from the standard Naor-Yung argument.
	\item The time-lock puzzle in the above construction can be instantiated based on repeated squaring \cite{RSW96}, possibly using the variant of \cite{C:MalThy19} that combines repeated squaring with Paillier encryption \cite{EC:Paillier99} to achieve a linear homomorphism. 
	\item Public verifiability can be achieved by using a suitable proof system for $\pi$.
\end{itemize}

Furthermore, in the IND-CCA security proof, we can perform fast opening by decrypting $c_2$ with the secret key of the public key encryption scheme, which is indistinguishable from a forced opening using $c_1$ by the soundness of the proof. 
%
However, it turns out that concretely instantiating this scheme in a way that yields a practical construction is non-trivial and requires a very careful combination of different techniques.

\paragraph{Triple Naor-Yung.}
First of all, note that repeated squaring modulo a composite number $N = PQ$, where $P$ and $Q$ are different primes, is currently the only available choice to achieve a practical time-lock puzzle, hence we are bound to using this puzzle to instantiate $c_1$. Conveniently, this puzzle allows for a linear (\ie, additive) homomorphism by following \cite{C:MalThy19}.
%
Then, in order to be able to instanatiate $\pi$ efficiently, it would be convenient to use a standard Sigma protocol, which can then be made non-interactive via the Fiat-Shamir transform \cite{C:FiaSha86} in the random oracle model, or by leveraging techniques from Libert \etal \cite{Libert2021OneShotFN} in the standard model. Since practically efficient Sigma protocols are only known for algebraic languages, such as that defined by the DDH relation, for example, we have to choose $c_2$ in a way which is ``algebraically compatible'' with $c_1$ and the available proofs $\pi$. If we instantiate $c_1$ with the homomorphic TLP from \cite{C:MalThy19}, then a natural candidate would be to instantiate $c_2$ also with Paillier encryption.  Here we face the first technical difficulty: 
\begin{itemize}
	\item Efficient proof systems for $\pi$ are only available, if both $c_1$ and $c_2$ use the same modulus $N$. Hence, we have to instantiate both with the same modulus.
	\item When arguing that $c_1$ hides the committed message $m$ in the Naor-Yung argument of the security proof, we will have to replace $c_1$ with a random puzzle, using the \emph{strong sequential squaring} (SSS) assumption. At the same time, we have to be able to respond to decommitment queries using the decryption key of $c_2$. But this decryption key is the factorization $P, Q$ of the common modulus $N$, and we cannot reduce to the hardness of SSS while knowing the factorization of $N$.
\end{itemize}
The first candidate approach to overcome this difficulty is to replace the Paillier encryption used in $c_2$ with an encryption scheme that does not require knowledge of the factorization of $N$, such as the ``Paillier ElGamal'' scheme from \cite{C:MalThy19}, which is defined over the subgroup $\Jn$ of elements of $\Zn$ having Jacobi symbol 1, and which uses a discrete logarithm to decrypt but still requires the factorization of $N$ to be hidden in order to be secure.

However, now we run into another difficulty. In the Naor-Yung argument, we will also have to replace $c_2$ with an encryption of a random message, in order to argue that our commitment scheme is hiding. In this part of the proof, we cannot know the secret key of $c_2$, that is, neither the aforementioned discrete logarithm, nor the factorization of $N$. However, we also cannot use $c_1$ to respond to decommitment queries, because then we would have to solve the time-lock puzzle, which cannot be done fast without knowledge of the factorization of $N$.

We resolve this problem by using ``triple Naor-Yung''. In our linearly homomorphic constructions, a commitment to $m$ will have the form $(c_1, c_2, c_3, \pi)$, where $c_1$ and $c_2$ are Paillier-ElGamal encryptions of $m$, and $c_3$ is the Paillier-style time-lock puzzle based on repeated squaring from \cite{C:MalThy19}. All are with respect to the same modulus $N$, and thus allow for an efficient Sigma-protocol-based proof $\pi$ that $c_1$, $c_2$, and $c_3$ all contain the same message. In the Naor-Yung-style IND-CCA security proof, we will first replace $c_3$ with a random puzzle, while using the discrete logarithm of the public key that corresponds to $c_1$ to perform fast decommitments. When we then replace $c_2$ with an encryption of a random message, we use the discrete logarithm of the public key that corresponds to $c_1$ to answer decommitment queries. Finally, we switch to using the discrete logarithm of the secret key corresponding to $c_{2}$ for decommitment queries, and replace $c_{1}$ with an encryption of a random message. Hence, throughout the argument we never require the factorization of $N$ for fast decommitments.


\paragraph{Standard Naor-Yung works for multiplicative homomorphism.}
Next, we observe that the standard (\ie, ``two-ciphertext'') Naor-Yung approach works, if a \emph{multiplicative} homomorphism (or no homomorphism at all) is required. Concretely, a commitment will have the form $(c_1, c_2, \pi)$, where $c_1$ is an ElGamal encryption and $c_2$ uses the ``sequential-squaring-with-ElGamal-encryption'' idea of \cite{C:MalThy19}. By replacing the underlying group to the subgroup $\qrn$ of quadratic residues modulo $N$, we can rely on the DDH assumption in $\qrn$ and thus do not require the factorization of $N$ to be hidden when replacing the ElGamal encryption $c_1$ with an encryption of a random message.
While the construction idea and high-level arguments are very similar, the underlying groups and detailed arguments are somewhat different, and thus we have to give a separate proof.




\paragraph{On separate proofs in the standard model and the ROM}
The constructions sketched above can be instantiated relatively efficiently in the standard model, using the one-shot Fiat-Shamir arguments in the standard model by Libert \etal \cite{Libert2021OneShotFN}. However, these proofs repeat the underlying Sigma protocol a logarithmic number of times, and thus it would be interesting to also consider constructions in the random oracle model.
Since the syntactical definitions and properties of proof systems in the random oracle model are slightly different from that in \cite{Libert2021OneShotFN}, we give separate proofs for both random oracle constructions as well. 

\paragraph{Shared randomness} To obtain commitments of smaller size we additionally apply the shared randomness technique from \cite{SCN:BiaMasVen16}, where instead of producing two or three independent encryptions of the same message, we reuse the same randomness for encryption. This allows to save one group element in case of the standard Naor-Yung constructions and two group elements in the case of triple Naor-Yung.

\subsection{Further related work}
Time-lock puzzles based on randomized encodings were introduced in \cite{TCC:BDGM19}, but all known constructions of timed commitments rely on the repeated squaring puzzles of \cite{RSW96}.
Timed commitments are also related to time-lock encryption scheme \cite{liu2018build} and time-released encryption \cite{cryptoeprint:2020/739}, albeit with different properties. The construction in \cite{liu2018build} is based on an external ``computational reference clock'' (instantiated with a public block chain), whose output can be used to decrypt, such that decrypting parties do not have to perform expensive computations by solving a puzzle. The constructions of Chvojka \etal \cite{cryptoeprint:2020/739} are based on repeated squaring, however, the main difference is that the time needed for decryption starts to run from the point when $setup$ is executed and not from the point when ciphertext is created. 


% \subsection{Technical Overview}\label{sec:techoverview}

% % Katz et al:
% % - Slow encryption, have to perform seq squaring in encryption, therefore. Still only for one fixed value of T, because the NIZK language CRS depends on T.

% % - NY proof, well-formed commitment (can be force-openend with suitable parameter),
% % third proof not sufficient to reveal randomness. Three proofs in total....

% % - We need only one, for plaintext equality (sim sound), due to same modulus can be instantiated very efficiently.

% % In our case, T is fixed...

% Similar to the construction by Katz \etal \cite{TCC:KatLosXu20}, our is based on the Naor-Yung double-encryption approach \cite{STOC:NaoYun90}. However, we apply it in a very different way. 


% \todo{Explain why triple encryption}

% \paragraph{Main idea of the construction.}
% The CRS of our commitment scheme essentially consists of a time parameter $T$, an RSA modulus $N$, a generator $g$ of the group $\qrn$ of quadratic residues modulo $N$ and two numbers $h_{1}, h_{2} \in \Z_{N}$, where
% \[
% h_{1} = g^{k} \qquad\text{and}\qquad h_{2} = g^{2^{T}} 
% \]
% for $k \rand \smplset$.
% Note that this $(h_{1}, h_{2})$ can be seen as two ElGamal public keys in the group $\qrn$, where the corresponding secret keys $k$ and $2^{T} \bmod \ord$ are discarded.

% A commitment to a message $m \in \Z_{N}$ contains $c = (c_{0}, c_{1}, c_{2})$ where
% \[
% c_{0} = g^{r}, \qquad c_{1} = h_{1}^{r} \cdot m, \qquad c_{2} = h_{2}^{r} \cdot m,
% \]
% for $r \rand \smplset$.
% Note that $c$ can be viewed as two ElGamal ciphertexts $(c_{0}, c_{1})$ and $(c_{0}, c_{2})$ that share the same randomness $r$ and encrypt the same message $m$ with respect to the two public keys $(h_{1}, h_{2})$.
% ElGamal ciphertext $(c_{0}, c_{2})$ can be force-opened by repeated squaring, by computing $m = c_{2} / c_{0}^{2^{T}}$.
% Furthermore, the fact that there exists a secret key $k$ for ciphertext $(c_{0}, c_{1})$ will allow for answering decommitment queries in the security experiment quickly, that is, without the need to solve a sequential squaring puzzle, by computing $m = c_{1} / c_{0}^{k}$.
% Additionally, a commitment contains a NIZK proof that both  $(c_{0}, c_{1})$ and $(c_{0}, c_{2})$ commit to the same message. 

% Note that the commitment scheme is hiding based on the DDH assumption (for $(c_{0}, c_{1})$) and the strong sequential squaring assumption (for $(c_{0}, c_{2})$) in $\qrn$. It is perfectly binding by the perfect correctness of ElGamal with honestly generated public key in $\qrn$.

% \paragraph{Simplicity and efficiency.}
% We are able to obtain a very significant improvement in computational efficiency and conceptual simplicity when compared to \cite{TCC:KatLosXu20}. We achieve this for the following two reasons.
% \begin{enumerate}
% \item One key advantage of our construction is that both ciphertexts  $(c_{0}, c_{1})$ and $(c_{0}, c_{2})$ are defined over the \emph{same group} $\qrn$. This enables a very simple and efficient Fiat-Shamir-style proof, simply setting $h := h_{1}/h_{2}$ and then proving that
% \[
% \left(g, h, c_{0}, \frac{c_{1}}{c_{2}} \right)
% =
% \left(g, h, g^{r}, \frac{h_{1}^{r} \cdot m}{h_{2}^{r} \cdot m} \right)
% =
% \left(g, h, g^{r}, h^{r} \right)
% \]
% forms a DDH-tuple. Note that this can be achieved using a simple Sigma-protocol-based simulation-sound ZK proof of membership in the language of DDH tuples in $\qrn$ as proposed by Chaum and Pedersen \cite{C:ChaPed92}, which can be made non-interactive using the Fiat-Shamir transform \cite{C:FiaSha86}. Hence, in contrast to  \cite{TCC:KatLosXu20}, who use double encryption with two ciphertexts over different groups modulo $N_{1} \neq N_{2}$, no NIZKs for general NP languages are necessary.
% \item Furthermore, committing to a message essentially consists of a few standard exponentiations in $\qrn$ and computing the Fiat-Shamir NIZK. In particular, in contrast to \cite{TCC:KatLosXu20} it is not necessary to perform $T$ sequential squarings also when \emph{committing} to the message, but only for forced opening.
% \end{enumerate}
 




% %On the technical level our construction has some similarities with the construction of Katz \etal , however we make several observations which are crucial to obtain fast encryption and allows to simplify the construction, which leads to significant efficiency improvements. We recall that the construction by Katz \etal is based on timed-public key encryption (TPKE) which is then generically turned into non-malleable non-interactive timed commitment using two NIZKs. One of the NIZKs is used to guarantee that commitment has been generated properly and the second NIZK is used to verify the validity of the opening. The proposed construction of TPKE is based on the Naor-Yung paradigm combining simulation sound NIZK and two time-locked encryptions produced using two independent RSA modulus $N_1, N_2$. The disadvantage of this construction is an expensive encryption that depends on the hardness parameter $T$ and hence yields an inefficient commitment algorithm. Concretely, one has to compute values $x_i^{2^T} \bmod N_i$ for $i \in [2]$ using $T$ repeated squarings. 

% %Our first observation is that if we can achieve fast encryption and hence fast commitment which is perfectly binding, then we are able to build NITC directly using the Naor-Yung paradigm without using additional NIZKs. Since encryption is fast and perfectly binding, then it is sufficient reveal as opening the message $m$ with the randomness $r$, which were used to produce commitment, and everyone is able to efficiently check that these values are correct. Therefore we do not need any NIZK to prove validity of the opening. Moreover, when relying on the Naor-Yung paradigm the language for the SS-NIZK can be designed in such a way, that it guarantees proper generation of a commitment. Hence, we do not need additional NIZK to enforce this property. 

% %To achieve fast commitment we precompute the value $h_1:=g^{2^T} \bmod N$ where $g$ is an generator of $\qrn$ in trusted setup. This can be done efficiently, since while executing the setup the factorization of $N$ is known. When encrypting a message, we produce an ElGamal-style ciphertext with respect to $g, h_1$. Since the order of the group $\qrn$ is not known, we sample randomness from $\smplset$ which we show is statistically indistinguishable from sampling from the set $[\ord]$. The second ciphertext is ElGamal encryption with respect to secret key $g, h_2: = g^k \bmod N$ where $k$ is chosen uniformly at random from $\smplset$. Lastly, we use SS-NIZK to prove equality of ciphertexts. All of these computations are independent of $T$ and hence the resulting commitment is efficient. In order to be able to prove security of our construction, we need DDH assumption holds in $\qrn$ even if the factorization of $N$ is known. This is however implied by hardness of DDH in the large subgroups of $\Zn^*$ as was shown in \cite{C:CouPetPoi16}.

% %We are able to shorten the ciphertext size by encrypting a message using shared randomness as proposed by Biagioni \etal \cite{SCN:BiaMasVen16}. This adjustment allows to design an efficient Sigma protocol that the commitment is properly generated. Applying the Fiat-Shamir transform \cite{C:FiaSha86} to the Sigma protocol we obtain SS-NIZK. The Sigma protocol essentially proves that the tuple specified by the commitment and the public key is a DDH tuple. Since the order of the group $\qrn$ is not know, we have to design the Sigma protocol in hidden order group. Such protocols are less efficient than Sigma protocols where order of the group is known. 
% %\todo{However, we realize that to obtain SS-NIZK, it is sufficient to prove that the constructed Sigma protocol has negligible soundness and we do not have to care about special soundness. In this way we are able to avoid the strong RSA assumption and therefore we are able to achieve proofs of smaller size. Usually in Sigma protocols in hidden order groups, the first value is sampled from very large interval which increases the overall size of the proof. } Since sampling from $\smplset$ is statistically indistinguishable from sampling from $[\ord]$ this additionally reduces the size of the resulting proof.


% We stress that our construction does not require \emph{special soundness}, which is usually more expensive to achieve in hidden-order groups since to obtain reasonable security guarantees one has to either run the underlying protocols several times in parallel or use large modulus $N$ \cite{SPEED:BKSST}. Instead, for our construction a negligible soundness error is sufficient and therefore our constructions do not suffer with the mentioned drawbacks.
% %so that we can use smaller integers, which reduces computational complexity and proof size further.


%%% Local Variables:
%%% mode: latex
%%% TeX-master: "main"
%%% End:

%!TEX root=./main.tex
\section{Preliminaries}
We denote our security parameter by $\lambda$. For $n \in \mathbb{N}$ we write $1^n$ to denote the $n$-bit string of all ones. For any element $x$ in a set $X$, we use $x \rand X$ to indicate that we choose $x$ uniformly at random from $X$. For simplicity we model all algorithms as Turing machines, however, all adversaries are modeled as non-uniform polynomial-size circuits to simplify concrete time bounds in the security definitions of non-interactive timed commitments and the strong sequential squaring assumption. All algorithms are randomized, unless explicitly defined as deterministic. For any PPT algorithm $A$, we define $x \leftarrow A(1^\secpar,a_1,\ldots,a_n)$ as the execution of $A$ with inputs security parameter $\secpar$, $a_1,\ldots,a_n$ and fresh randomness and then assigning the output to $x$. We write $[n]$ to denote the set of integers $\{1, \dots, n\}$ and $\floor{x}$ to denote the greatest integer that is less than or equal to $x$.



\paragraph{Non-interactive timed commitments.}
The following definition of a non-interactive timed commitment scheme is from~\cite{TCC:KatLosXu20}.
\begin{definition}
\label{def:nitc_syntax}
A non-interactive timed commitments scheme $\nitc$ with message space $\msgspace$ is a tuple of algorithms $\nitc = (\pgen, \com, \cvrfy, \dvrfy, \allowbreak \fdecom)$ with the following syntax.
\begin{itemize}
\item $\crs \leftarrow \pgen(\seck, T)$ is a probabilistic algorithm that takes as input the security parameter $\seck$ and a hardness parameter $T$ and outputs a common reference string $\crs$ and a secret key.
\item $(c, \pi_\com, \pi_\dec) \leftarrow \com(\crs, m)$ is a probabilistic algorithm that takes as input a common reference string $\crs$ and a message $m$ and outputs a commitment $c$ and proofs $\pi_\com, \pi_\dec$.
\item $0/1 \leftarrow \cvrfy(\crs, c, \pi_\com)$ is a deterministic algorithm that takes as input a common reference string $\crs$, a commitment $c$ and proof $\pi_\com$ and outputs $0$ (reject) or $1$ (accept).
\item $0/1 \leftarrow \dvrfy(\crs, c, m, \pi_\dec)$ is a deterministic algorithm that takes as input a common reference string $\crs$, a commitment $c$, a message $m$ and proof $\pi_\dec$ and outputs $0$ (reject) or $1$ (accept).
%\item $m \leftarrow \decom(\crs,\sk, c)$ is a deterministic algorithm that takes as input a common reference string $\crs$, a secret key $\sk$ and a commitment $c$ and outputs $m \in \msgspace \cup \{\bot\}$. The running time of $\decom$ should be independent of $T$.
\item $m \leftarrow \fdecom(\crs, c)$ is a deterministic forced decommit algorithm that takes as input a common reference string $\crs$ and a ciphertext $c$ and outputs $m \in \msgspace \cup \{\bot\}$ in time at most $T \cdot \poly(\secpar)$.
\end{itemize}
We say $\nitc$ is correct if for all $\secpar, T \in \nats$ and all $m \in \msgspace$ holds:
\[\Pr\left[
\begin{aligned}
%\decom(\crs, \sk, c) = 
\fdecom(\crs, c) = m \\
\land \; \cvrfy(\crs, c, \pi_\com) = 1 \\
\land \; \dvrfy(\crs, c, m, \pi_\dec)=1
\end{aligned}
: 
\begin{aligned}
      \crs \leftarrow \pgen(\seck, T) \\
      (c, \pi_\com, \pi_\dec) \leftarrow \com(\crs, m) \\
    \end{aligned}
\right] = 1.
\]
\end{definition}

%As the names of the decryption algorithms in \Cref{def:tpke_syntax} suggest, it should hold that $\tfd$ is much smaller than $\tsd$. Now we state the original security definition, however, adjusted to the computational model which is used in this paper. 

%\paragraph{Timed commitments.}
%The following definition of an interactive timed commitment scheme is adjusted definition from~\cite{TCC:KatLosXu20} to an interactive setting.
%\begin{definition}
%\label{def:tpke_syntax}
%An interactive timed commitments scheme $\tc$ with message space $\msgspace$ is a tuple $\tc = (\pgen, \com = \prot{C,R}, \cvrfy, \dvrfy, \decom, \allowbreak \fdecom)$ with the following syntax.
%\begin{itemize}
%\item $\crs \leftarrow \pgen(\seck, T)$ is a probabilistic algorithm that takes as input the security parameter $\seck$ and a hardness parameter $T$ and outputs a common reference string $\crs$ and a secret key.
%\item $(c, \pi_\com, \pi_\dec) \leftarrow \com(\crs, m) = \prot{C(m),R}(\crs)$ is an interactive protocol between two PPT algorithms $C$ and $R$, where $C$ takes as input a common reference string $\crs$ and a message $m$ and $R$ takes as input $\crs$. At the end of the protocol $C$ and $R$ has a joint  output a commitment $c$ and proof $\pi_\com$ and additionally $C$ has private output $\pi_\dec$.
%\item $0/1 \leftarrow \cvrfy(\crs, c, \pi_\com)$ is a deterministic algorithm that takes as input a common reference string $\crs$, a commitment $c$ and proof $\pi_\com$ and outputs $0$ (reject) or $1$ (accept).
%\item $0/1 \leftarrow \dvrfy(\crs, c, m, \pi_\dec)$ is a deterministic algorithm that takes as input a common reference string $\crs$, a commitment $c$, a message $m$ and proof $\pi_\dec$ and outputs $0$ (reject) or $1$ (accept).
%%\item $m \leftarrow \decom(\crs,\sk, c)$ is a deterministic algorithm that takes as input a common reference string $\crs$, a secret key $\sk$ and a commitment $c$ and outputs $m \in \msgspace \cup \{\bot\}$. The running time of $\decom$ should be independent of $T$.
%\item $m \leftarrow \fdecom(\crs, c)$ is a deterministic forced decommit algorithm that takes as input a common reference string $\crs$ and a ciphertext $c$ and outputs $m \in \msgspace \cup \{\bot\}$ in time at most $T \cdot \poly(\secpar)$.
%\end{itemize}
%We say $\tc$ is correct if for all $\secpar, T \in \nats$ and all $m \in \msgspace$ holds:
%\[\Pr\left[
%\begin{aligned}
%%\decom(\crs, \sk, c) = 
%\fdecom(\crs, c) = m \\
%\land \; \cvrfy(\crs, c, \pi_\com) = 1 \\
%\land \; \dvrfy(\crs, c, m, \pi_\dec)=1
%\end{aligned}
%: 
%\begin{aligned}
%      \crs \leftarrow \pgen(\seck, T) \\
%      (c, \pi_\com, \pi_\dec) \leftarrow \com(\crs, m) \\
%    \end{aligned}
%\right] = 1.
%\]
%\end{definition}
%
%\begin{definition}
%We say that timed commitment scheme $(\pgen, \com, \cvrfy, \allowbreak \dvrfy, \decom, \fdecom)$ is non-interactive timed commitment $\nitc$ if $(c, \pi_\com, \pi_\dec) \leftarrow \com(\crs, m)$ is a probabilistic algorithm that takes as input a common reference string $\crs$ and a message $m$ and outputs a commitment $c$ and proofs $\pi_\com, \pi_\dec$. All other algorithms are defined as for an interactive timed commitment scheme.
%
%\end{definition}

%\begin{definition}
%\label{def:nitc_cca}
%A timed commitment scheme $\tc$ is \emph{IND-CCA secure} with gap $0 < \gap < 1$ if there exists a polynomial $\tilde{T}(\cdot)$ such that for all polynomials $T(\cdot) \geq \tilde{T}(\cdot)$ and every non-uniform polynomial-size adversary $\adv = \{(\adv_{1,\secpar}, \adv_{2, \secpar})\}_{\secpar \in \nats}$, where the depth of $\adv_{2, \secpar}$ is at most $T^{\gap}(\secpar)$, there exists a negligible function $\negl(\cdot)$ such that for all $\secpar \in \nats$ it holds 
%\[ \advtg^{\tc}_{\adv} = 
%\left| \Pr\left[ 
%    b = b'
%    \;:\;
%    \begin{aligned}
%    \crs \leftarrow \pgen(\seck, T(\secpar)) \\
%      (m_0, m_1, \st) \leftarrow \adv_{1,\secpar}^{\deco(\cdot)}(\crs) \\
%      b \rand \bits\\
%      (c^*, \pi_\com, \pi_\dec) \leftarrow \com(\crs, m_b) \\
%      b' \leftarrow \adv_{2,\secpar}^{\deco(\cdot)}(c^*, \pi_\com, \st)
%    \end{aligned}
%    \right] -  \half \right|
%\leq \negl(\secpar),  
%\]
%where $|m_0|=|m_1|$ and the oracle ${\deco(c)}$ returns value which is equal to $\fdecom(\crs, c)$ with the restriction that $\adv_{2, \secpar}$ is not allowed to query the oracle $\deco(\cdot)$ for decommitment of the challenge commitment $c^*$. We require that $\deco(\cdot)$ must be able to answer decommitment queries in time which is independent of $T$.
%\end{definition}

\begin{definition}
\label{def:nitc_cca}
A non-interactive timed commitment scheme $\nitc$ is \emph{IND-CCA secure} with gap $0 < \gap < 1$ if there exists a polynomial $\tilde{T}(\cdot)$ such that for all polynomials $T(\cdot) \geq \tilde{T}(\cdot)$ and every non-uniform polynomial-size adversary $\adv = \{(\adv_{1,\secpar}, \adv_{2, \secpar})\}_{\secpar \in \nats}$, where the depth of $\adv_{2, \secpar}$ is at most $T^{\gap}(\secpar)$, there exists a negligible function $\negl(\cdot)$ such that for all $\secpar \in \nats$ it holds 
\[ \advtg^{\nitc}_{\adv} = 
\left| \Pr\left[ 
    b = b'
    \;:\;
    \begin{aligned}
    \crs \leftarrow \pgen(\seck, T(\secpar)) \\
      (m_0, m_1, \st) \leftarrow \adv_{1,\secpar}^{\deco(\cdot)}(\crs) \\
      b \rand \bits\\
      (c^*, \pi_\com, \pi_\dec) \leftarrow \com(\crs, m_b) \\
      b' \leftarrow \adv_{2,\secpar}^{\deco(\cdot)}(c^*, \pi_\com, \st)
    \end{aligned}
    \right] -  \half \right|
\leq \negl(\secpar),  
\]
where $|m_0|=|m_1|$ and the oracle ${\deco(c)}$ returns value which is equal to $\fdecom(\crs, c)$ with the restriction that $\adv_{2, \secpar}$ is not allowed to query the oracle $\deco(\cdot)$ for decommitment of the challenge commitment $c^*$. We require that $\deco(\cdot)$ must be able to answer decommitment queries in time which is independent of $T$. In case of an interactive timed commitment $\com(\crs, m_b)$ corresponds to interactive protocol between $C(\crs, m_b)$ and $\adv_{2,\secpar}^{\deco(\cdot)}(\st)$.
\end{definition}

We remark that requirement that $\deco(\cdot)$ should answer oracle queries in time independent of $T$ is necessary to obtain sound proof when reducing to a time sensitive assumption such as the strong sequential squaring assumption. This in particular means that in the security proof decommitment queries can not be simply answered by $\fdecom$ algorithm whose runtime depends on $T$, but there must exist another way to answer them. 

%Now we state a binding property for a non-interactive timed commitment scheme. 

\begin{definition}
\label{def:nitc-bnd}
We define $\mathsf{BND \mhyphen CCA}_\adv(\secpar)$ experiment as follows:
\begin{enumerate}
\item $\crs \leftarrow \pgen(\seck, T(\secpar))$;
\item $(m, c, \pi_\com, \pi_\dec, m', \pi_\dec') \leftarrow \adv_{\secpar}^{\deco(\cdot)}(\crs)$, where the oracle $\deco(c)$ returns the value which is equal to $\fdecom(\crs, c)$;
\item Outputs 1 iff $\cvrfy(\crs, c, \pi_\com)= 1$ and either:
\begin{itemize}
\item $m \neq m' \; \land \; \dvrfy(\crs, c, m, \pi_\dec) = \dvrfy(\crs, c, m', \pi_\dec') = 1$;
\item $\dvrfy(\crs, c, m, \pi_\dec) = 1 \; \land \; \fdecom(\crs, c) \neq m$.
\end{itemize}
\end{enumerate}
A non-interactive timed commitment scheme $\nitc$ is \emph{BND-CCA secure} if for all non-uniform polynomial-size adversaries $\adv = \{\adv_{\secpar}\}_{\secpar \in \nats}$ there is a negligible function $\negl(\cdot)$ such that for all $\secpar \in \N$ 
\[ \advtg^{\nitc}_{\adv} = 
\Pr\left[ \mathsf{BND \mhyphen CCA}_\adv(\secpar) = 1 \right] \leq \negl(\secpar). 
\]
\end{definition}

%Now we state a binding property for a timed commitment scheme. 
%
%\begin{definition}
%\label{def:nitc-bnd}
%A  timed commitment scheme $tc$ is \emph{perfectly binding} if for all $\secpar, T \in \N, m, m' \in \msgspace$ such that $m \neq m'$, for all $\crs$ in the support of $\pgen(\secpar, T)$, for all $c, \pi_\com, \pi_\dec, \pi_\dec'$ it holds 
%\[ \Pr\left[ 
%    \begin{aligned}
%    (\dvrfy(\crs, c, m, \pi_\dec) = \dvrfy(\crs, c, m', \pi_\dec') = 1 \\
%    \land \cvrfy(\crs, c, \pi_\com)= 1)
%    \lor(\cvrfy(\crs, c, \pi_\com)= 1  \\
%    \land \dvrfy(\crs, c, m, \pi_\dec) = 1 \; \land \; \fdecom(\crs, c) \neq m) \\
%    \end{aligned}
%    \right]
%= 0,
%\]
%\end{definition}

Next we define a novel property of NITC which allows for efficient verification that forced decommitment was executed correctly without the need to execute expensive sequential computation. This was first suggested in terms of time-lock puzzles by cite{} and denoted as public verifiability.

\begin{definition}
\label{def:nitc_pubver}
A non-interactive timed commitments scheme $\nitc$ is publicly verifiable if $\fdecom$ additionally outputs a proof $\pi_\fdecom$ and it has an additional algorithm $\fdvrfy$ with the following syntax:
\begin{itemize}
\item $0/1 \leftarrow \fdvrfy(\crs, c, m, \pi_\fdecom)$ is a deterministic algorithm that takes as input a common reference string $\crs$, a commitment $c$, a message $m$ and a proof $\pi_\fdecom$ and outputs 0 (reject) or 1 (accept) in time $\poly(\log T, \secpar)$.
\end{itemize}

Moreover, publicly verifiable NITC must fulfil further properties:
\begin{itemize}
\item \emph{Completeness} for all $\secpar, T \in \nats$ and all $m \in \msgspace$ holds:
\[\Pr\left[
\begin{aligned}
\fdvrfy(\crs, c, m, \pi_\fdecom)=1
\end{aligned}
: 
\begin{aligned}
      \crs \leftarrow \pgen(\seck, T) \\
      (c, \pi_\com, \pi_\dec) \leftarrow \com(\crs, m) \\
      (m, \pi_\fdecom) \leftarrow \fdecom(\crs, c) \\
    \end{aligned}
\right] = 1.
\]
\item \emph{Soundness} for all non-uniform polynomial-size adversaries $\adv = \{\adv_{\secpar}\}_{\secpar \in \nats}$ there is a negligible function $\negl(\cdot)$ such that for all $\secpar \in \N$ \todo{should I require to output the proof that c was correctly committed?}
\[\Pr\left[
\begin{aligned}
\fdvrfy(\crs, c, m', \pi_\fdecom') = 1 \\
 \land \; \cvrfy(\crs, c, \pi_\com) = 1 \\
\land \; m \neq m' \\
\end{aligned}
: 
\begin{aligned}
      \crs \leftarrow \pgen(\seck, T) \\
      (c, \pi_\com, m', \pi_\fdecom') \leftarrow \adv_\secpar(\crs) \\
      (m,\pi_\fdecom) \leftarrow \fdecom(\crs,c)
    \end{aligned}
\right] \leq \negl(\secpar).
\]
\end{itemize}

\end{definition}


\paragraph{Non-interactive zero-knowledge proofs.}
Next we recall definition of a simulation-sound non-interactive proof system (SS-NIZK). 

\begin{definition} 
A \emph{non-interactive proof system} for an NP language $L$ with relation $\rel$ is a pair of algorithms $(\prove, \vrfy)$, which work as follows:
\begin{itemize}
\item $\pi \leftarrow \prove(s,w)$ is a PPT algorithm which takes as input a statement $s$ and a witness $w$ such that $(s,w) \in \rel$ and outputs a proof $\pi$.
\item $\vrfy(s, \pi) \in \{0,1\}$ is a deterministic algorithm which takes as input a statement $s$ and a proof $\pi$ and outputs either 1 or 0, where 1 means that the proof is ``accepted'' and 0 means it is ``rejected''.
\end{itemize}
We say that a non-interactive proof system is \emph{complete}, if for all $(s, w) \in \rel$ holds:
\[\Pr[\vrfy(s,\pi)=1:\pi \leftarrow \prove(s,w)] =1.\] 
\end{definition}

Next we define \emph{zero-knowledge} property for non-interactive proof system in the random oracle model. The simulator $\simul$ of a non-interactive zero-knowledge proof system is modelled as a stateful algorithm which has two modes: $(\pi, \st) \leftarrow \simul(1, \st, s)$  for answering proof queries and $(v, \st) \leftarrow \simul(2, \st, u)$ for answering random oracle queries. The common state $\st$ is updated after each operation.



\begin{definition}[Zero-Knowledge in the Random Oracle Model]
Let $(\prove, \allowbreak \vrfy)$ be a non-interactive proof system for a relation $\rel$ which may make use of a hash function $H : \hdom \rightarrow \himg$. Let $\funs$ be the set of all functions from the set $\hdom$ to the set $\himg$. We say that $(\prove, \vrfy)$ is \emph{non-interactive zero-knowledge proof in the random oracle model (NIZK)}, if there exists an efficient simulator $\simul$ such that for all non-uniform polynomial-size adversaries $\adv = \{\adv_\secpar\}_{\secpar \in \nats}$ there exists a negligible function $\negl(\cdot)$ such that for all $\secpar \in \nats$ 
\[\zk_\adv^\nizk = 
\left| \Pr\left[ \adv_\secpar^{\prove^H(\cdot, \cdot), H(\cdot)} = 1 \right] -  \Pr\left[\adv_\secpar^{\simul_1(\cdot, \cdot), \simul_2(\cdot)} \right] = 1 \right|
\leq \negl(\secpar),
\]
where 
\begin{itemize}
\item $H$ is a function sampled uniformly at random from $\funs$,
\item $\prove^H$ corresponds to the $\prove$ algorithm, having oracle access to $H$,
\item $\pi \leftarrow \simul_1(s, w)$ takes as input $(s, w) \in \rel$, and outputs the first output of $(\pi, \st) \leftarrow \simul(1, \st, s)$,
\item $v \leftarrow \simul_2(u)$ takes as input $u \in \hdom$ and outputs the first output of $(v, \st) \leftarrow \simul(2, \st, u)$.
\end{itemize}
\end{definition}

\begin{definition}[One-Time Simulation Soundness]
Let $(\prove, \vrfy)$ be a non-interactive proof system for an NP language $L$ with zero-knowledge simulator $\simul$. We say that $(\prove, \vrfy)$ is \emph{one-time simulation sound} in the random oracle model, if for all non-uniform polynomial-size adversaries $\adv = \{\adv_\secpar\}_{\secpar \in \nats}$ there exists a negligible function $\negl(\cdot)$ such that for all $\secpar \in \nats$ 
\[
\simsnd_\adv^\nizk = \Pr\left[
\begin{aligned}
x \notin L \land (s, \pi) \neq (s', \pi') \\
\land \vrfy^{\simul_2(\cdot)}(s, \pi) = 1
\end{aligned}
:(x, \pi) \leftarrow \adv_\secpar^{\simul_1(\cdot), \simul_2(\cdot)} \right] \leq \negl(\secpar),
\]
where $\simul_1(\cdot)$ is a single query oracle which on input $s'$ returns the first output of $(\pi', \st) \leftarrow \simul(1, \st, s)$ and $\simul_2(u)$ returns the first output of $(v, \st) \leftarrow \simul(2, \st, u)$.
\end{definition}



\paragraph{Complexity assumptions.}
We base our construction on the strong sequential squaring assumption in the group of quadratic residues. Precisely, let $p, q$ be strong primes (i.e., such that $p = 2p'+1,  q= 2q'+1$ for primes $p', q'$). We denote by $\varphi(\cdot)$ Euler's totient function and by $\qrn$ the cyclic group of quadratic residues modulo $N$ which has order $|\qrn| = \frac{\varphi(N)}{4} = \frac{(p-1)(q-1)}{4}$. To efficiently sample a random element $x$ from $\qrn$, we can sample $r \rand \Zn^*$ and let $x:= r^2 \bmod N$. When the factors $p,q$ are known, then it easy to check if the given element is a generator of $\qrn$ by checking if $x^{p'} \neq 1 \bmod N \land x^{q'} \neq 1 \bmod N$. Therefore we are able to efficiently sample a random generator of $\qrn$. Let $\genmod$ be a probabilistic polynomial-time algorithm which on input $\seck$ outputs two $\secpar$-bit strong primes $p$ and $q$, modulus $N= pq$ and a random generator $g$ of $\qrn$. Now we state the strong sequential squaring assumption. 


\begin{figure}[h]
\begin{center}
\begin{tabular}{l}
$\mathsf{ExpSSS}_{\adv}^{b}(\secpar)$: \\
\hline
$(p, q, N,g) \leftarrow \genmod(\seck)$ \\
$\st \leftarrow \adv_{1, \secpar}(N, T(\lambda),g)$\\
$x \rand \qrn$ \\
$\text{if } b = 0: y:=x^{2^{T(\secpar)}} \bmod N$\\
$\text{if } b = 1: y \rand \Jn$\\
return	$b' \leftarrow \adv_{2,\secpar}(x, y, \st)$
\end{tabular}
\end{center}
\caption{\label{fig:sss}Security experiment for the strong sequential squaring assumption.}
\end{figure}


\begin{definition}[Strong Sequential Squaring Assumption (SSS)]\label{def:sssa}
Consider the security experiment $\mathsf{ExpSSS}_{\adv}^{b}(\secpar)$ in \Cref{fig:sss}. The \emph{strong sequential squaring assumption} with gap $0< \gap <1$ holds relative to $\genmod$ if there exists a polynomial $\tilde{T}(\cdot)$ such that for all polynomials $T(\cdot) \geq \tilde{T}(\cdot)$ and for every non-uniform polynomial-size adversary $\adv = \{(\adv_{1,\secpar}, \adv_{2,\secpar})\}_{\secpar \in \nats}$, where the depth of $\adv_{2,\secpar}$ is at most $T^{\gap}(\secpar)$, there exists a negligible function $\negl(\cdot)$ such that for all $\secpar \in \nats$ 
\[\advtg^\sss_\adv = \abs{\Pr[\mathsf{ExpSSS}_{\adv}^{0}(\secpar) = 1] - \Pr[\mathsf{ExpSSS}_{\adv}^{1}(\secpar) = 1]} \leq \negl(\secpar).\]
\end{definition}

Next we define DDH experiment in the group of quadratic residues modulo $N$ where the factors of $N$ are given to an adversary. 

\begin{figure}[h]
\begin{center}
\begin{tabular}{l}
$\mathsf{ExpDDH}_{\adv}^{b}(\secpar)$: \\
\hline
$(p, q, N,g) \leftarrow \genmod(\seck)$ \\
$\alpha, \beta \rand \Z_{\varphi(N)}$ \\
$\text{if } b = 0: \gamma = a\cdot b \bmod \varphi(N)$\\
$\text{if } b = 1: \gamma \rand \Z_{\varphi(N)}$ \\
return	$b' \leftarrow \adv_{\secpar}(N,p,q,g,g^\alpha, g^\beta, g^\gamma)$
\end{tabular}
\end{center}
\caption{\label{fig:ddh}Security experiment for DDH in $\qrn$.}
\end{figure}

Castagnos \etal \cite{C:CouPetPoi16} have shown that DDH problem is hard in $\qrn$ assuming that DDH is hard in the subgroups of $\Zn^*$ of order $p'$ and $q'$. This is shown in the proof of their Theorem 9. We remark that even though in \cite[Proof of Theorem 9]{C:CouPetPoi16} the prime factors $p,q$ are not given to the DDH adversary in the group $\qrn$, but the proof relies on the fact that the constructed reduction knows factors $p, q$. Therefore the proof holds even if $p,q$ are given to DDH adversary in $\qrn$ as input. 

\begin{theorem}[Decisional Diffie-Hellman in $\qrn$ \cite{C:CouPetPoi16}]\label{thm:ddh}
Consider the security experiment $\mathsf{ExpDDH}_{\adv}^{b}(\secpar)$ in \Cref{fig:ddh}.
If the DDH assumption holds relative to $\genmod$ in the both large prime-order subgroups of $\Zn^*$, then 
\[\advtg^\ddh_\adv = \abs{\Pr[\mathsf{ExpDDH}_{\adv}^{0}(\secpar) = 1] - \Pr[\mathsf{ExpDDH}_{\adv}^{1}(\secpar) = 1]} \leq \negl(\secpar).\]
\end{theorem} 

When designing an efficient simulation sound NIZK for our scheme, we rely on factoring assumption. 

\begin{definition}[Factoring Assumption]
The \emph{factoring assumption} holds relative to $\genmod$ if  for every non-uniform polynomial-size adversary $\adv = \{\adv_{\secpar}\}_{\secpar \in \nats}$ there exists a negligible function $\negl(\cdot)$ such that for all $\secpar \in \nats$ 
\[ \advtg_\adv^\fac = 
\Pr\left[ 
		N = p'q'
		\;:\;
    \begin{aligned}
    	(p, q, N) \leftarrow \genmod(\seck) \\
			p',q' \leftarrow \adv_{\secpar}(N), \\
			\text{ such that }  p', q' \in \N; p', q'>1
    \end{aligned}
    \right] 
\leq \negl(\secpar). 
\]
\end{definition}

To argue that our proof system fulfils required properties, we make of use the following lemma which states that it is possible factorize $N$ if a positive multiple of $\varphi(N)$ is known. The proof of this lemma is part of an analysis of \cite[Theorem 8.50]{books/crc/KatzLindell2014}.
\begin{lemma}\label{factor-lemma}
Let $(p,q,N)  \leftarrow \genmod(\seck)$ and let $M = \alpha\varphi(N)$ for some positive integer $\alpha\in \Z^+$. There exists a PPT algorithm $\factor(N, M)$ which, on input $(N,M)$, outputs $p',q' \in \N$, $p', q'>1$ such that $N = p'q'$ with probability at least $1-2^{-\secpar}$. 
\end{lemma}


\paragraph{On sampling random exponents for $\qrn$.}
Since in our construction the order $\ord$ of the group $\qrn$ is unknown, we use the set $\smplset$ whenever we should sample from the set $[\ord]$ without knowing the factorization of $N$. Sampling from $\smplset$ is statistically indistinguishable from sampling from $[\ord]$. 

\begin{definition}[Statistical Distance]
Let $X$ and $Y$ be two random variables over a finite set $S$. The statistical distance between $X$ and $Y$ is defined as 
\[\sd(X, Y) = \half \sum_{s \in S} \abs{\Pr[X = s] - \Pr[Y = s]}.\]
\end{definition}


\begin{lemma}\label{sampling-lemma}
Let $p,q$ be strong primes, $N=pq$ and  $X$ and $Y$ be random variables defined on domain $[\floor{N/2}]$ as follows:
\[
\Pr[X = r] = 1/\floor{N/2} \; \forall r \in [\floor{N/2}] \text{ and } \Pr[Y = r] = 
\begin{cases} 
     2/\varphi(N) & \forall r \in [\varphi(N)/2] \\
     0 & \text{otherwise}. 
   \end{cases}
\]
Then 
\[
\sd(X, Y) \leq \frac{1}{p}+\frac{1}{q}-\frac{1}{N}.
\]
\end{lemma}


\todo{Move to appendix.}
\begin{proof}
The following computation proves the lemma:
\begin{align*}
\sd(X, Y) = \half \sum_{r \in [\floor{N/2}]} \abs{\Pr[X = r] - \Pr[Y = r]} = \\ 
\half \left( \sum_{r = 1}^{\varphi(N)/2} \abs{\Pr[X = r] - \Pr[Y = r]} + \sum_{r = \varphi(N)/2+1}^{\floor{N/2}} \abs{\Pr[X = r] - \Pr[Y = r]} \right) = 
\\
\half \left( \sum_{r = 1}^{\varphi(N)/2} \abs{\frac{1}{\floor{N/2}} - \frac{2}{\varphi(N)}} + \sum_{r = \varphi(N)/2+1}^{\floor{N/2}} \abs{\frac{1}{\floor{N/2}} - 0} \right) \leq \\
\half \left( \sum_{r = 1}^{\varphi(N)/2} \abs{\frac{2}{N} - \frac{2}{\varphi(N)}} + \sum_{r = \varphi(N)/2+1}^{\floor{N/2}} \abs{\frac{1}{\floor{N/2}} - 0} \right) = \\
\half \left( \varphi(N)/2 \abs{\frac{2(\varphi(N) - N)}{\varphi(N)N}} + (\floor{N/2}-\varphi(N)/2) \frac{1}{\floor{N/2}} \right) = \\
 \half \left(\frac{(N-\varphi(N))}{N} + 1 - \frac{\varphi(N)/2}{{\floor{N/2}}} \right) \leq 
\half \left(\frac{(N-\varphi(N))}{N} + 1 - \frac{\varphi(N)/2}{{N/2}} \right) = \\
 \half\frac{2(N-\varphi(N))}{N} = 
\frac{(N-(N-p-q+1))}{N} = \frac{p+q-1}{N} = \frac{1}{p}+\frac{1}{q}-\frac{1}{N}.
\end{align*}
\end{proof}



%\begin{lemma}\label{sampling-lemma}
%Let $p,q$ be strong primes, $N=pq$ and  $X$ and $Y$ be random variables defined on domain $[\floor{N/4}]$ as follows:
%\[
%\Pr[X = r] = 1/\floor{N/4} \; \forall r \in [\floor{N/4}] \text{ and } \Pr[Y = r] = 
%\begin{cases} 
%     4/\varphi(N) & \forall r \in [\varphi(N)/4] \\
%     0 & \text{otherwise}. 
%   \end{cases}
%\]
%Then 
%\[
%\sd(X, Y) \leq \frac{1}{p}+\frac{1}{q}-\frac{1}{N}.
%\]
%\end{lemma}
%
%
%\todo{Move to appendix.}
%\begin{proof}
%The following computation proves the lemma:
%\begin{align*}
%\sd(X, Y) = \half \sum_{r \in [\floor{N/4}]} \abs{\Pr[X = r] - \Pr[Y = r]} = \\ 
%\half \left( \sum_{r = 1}^{\varphi(N)/4} \abs{\Pr[X = r] - \Pr[Y = r]} + \sum_{r = \varphi(N)/4+1}^{\floor{N/4}} \abs{\Pr[X = r] - \Pr[Y = r]} \right) = 
%\\
%\half \left( \sum_{r = 1}^{\varphi(N)/4} \abs{\frac{1}{\floor{N/4}} - \frac{4}{\varphi(N)}} + \sum_{r = \varphi(N)/4+1}^{\floor{N/4}} \abs{\frac{1}{\floor{N/4}} - 0} \right) \leq \\
%\half \left( \sum_{r = 1}^{\varphi(N)/4} \abs{\frac{4}{N} - \frac{4}{\varphi(N)}} + \sum_{r = \varphi(N)/4+1}^{\floor{N/4}} \abs{\frac{1}{\floor{N/4}} - 0} \right) = \\
%\half \left( \varphi(N)/4 \abs{\frac{4(\varphi(N) - N)}{\varphi(N)N}} + (\floor{N/4}-\varphi(N)/4) \frac{1}{\floor{N/4}} \right) = \\
% \half \left(\frac{(N-\varphi(N))}{N} + 1 - \frac{\varphi(N)/4}{{\floor{N/4}}} \right) \leq 
%\half \left(\frac{(N-\varphi(N))}{N} + 1 - \frac{\varphi(N)/4}{{N/4}} \right) = \\
% \half\frac{2(N-\varphi(N))}{N} = 
%\frac{(N-(N-p-q+1))}{N} = \frac{p+q-1}{N} = \frac{1}{p}+\frac{1}{q}-\frac{1}{N}.
%\end{align*}
%\end{proof}






%%% Local Variables:
%%% mode: latex
%%% TeX-master: "main"
%%% End:

%!TEX root=main.tex
\section{Standard Model Constructions}
In this section we construct two non-malleable non-interactive timed commitment schemes whose security can be proven in standard model and which are either linearly (i.e., additively) or multiplicatively homomorphic.  he constructions rely on non-interactive zero-knowledge proofs in the common reference string model. 
%!TEX root=main.tex

\subsection{Non-Interactive Zero-Knowledge Proofs}
We recall the definition of a simulation-sound non-interactive proof system (SS-NIZK) that we take from Libert~\etal \cite{Libert2021OneShotFN}. 

\begin{definition} 
A \emph{non-interactive zero-knowledge proof system} $\Pi$ for an NP language $L$ associated with a relation $\rel$ is a tuple of four PPT algorithms $(\gen_\param, \gen_L, \prove, \vrfy)$, which work as follows:
\begin{itemize}
%\item $\param \leftarrow \gen_\param(\seck)$ takes as input binary representation of a security parameter $\seck$ and outputs public parameters $\param$.
\item $\crs \leftarrow \setup(\seck, L)$ takes a security parameter $\seck$ and the description of a language $L$.  It outputs a a common reference string $\crs$. 
%\item $\crs \leftarrow \gen_L(\seck, L)$ takes as input binary representation of a security parameter $\seck$ and the description of language $L$ which specifies length of statements $n$.  It outputs a language dependent part $\crs_L$ of the common reference string $\crs:=(\param, \crs_L)$. 
%\item $\crs \leftarrow \gen_L(\seck, L, \tau_L)$ takes as input binary representation of a security parameter $\seck$, the description of language $L$ which specifies length of statements $n$ and a membership testing trapdoor $\tau_L$ for $L$.  It outputs a language dependent part $\crs_L$ of the common reference string $\crs:=(\param, \crs_L)$. 
\item $\pi \leftarrow \prove(\crs,s,w)$ is a PPT algorithm which takes as input the common reference string $\crs$, a statement $s$, and a witness $w$ such that $(s,w) \in \rel$ and outputs a proof $\pi$.
\item $0/1 \leftarrow \vrfy(\crs, s, \pi)$ is a deterministic algorithm which takes as input the common reference string $\crs$, a statement $s$ and a proof $\pi$ and outputs either 1 or 0, where 1 means that the proof is ``accepted'' and 0 means it is ``rejected''.
\end{itemize}
Moreover, $\Pi$ should satisfy the following properties. 
%For simplification we denote below by $\setup$ an algorithm that runs successively $\gen_\param$ and $\gen_L$ to generate a common reference string. 
\begin{itemize}
\item \emph{Completeness:}  for all $(s, w) \in \rel$ holds:
\[\Pr[\vrfy(\crs, s,\pi)=1:\crs \leftarrow \setup(\seck, L), \pi \leftarrow \prove(\crs, s,w)] =1.\] 
\item \emph{Soundness:} for all non-uniform polynomial-size adversaries $\adv = \{\adv_\secpar\}_{\secpar \in \nats}$ there exists a negligible function $\negl(\cdot)$ such that for all $\secpar \in N$
\[
\snd_\adv^\nizk = \Pr\left[
\begin{aligned}
s \notin L \land \\
\vrfy(\crs, s, \pi) = 1
\end{aligned}
:
\begin{aligned}
(\crs \leftarrow \setup(\seck, L) \\
(\pi, s) \leftarrow \adv_\secpar(\crs, \tau_L)
\end{aligned} \right] \leq \negl(\secpar),
\]
where $\tau_L$ is membership testing trapdoor. 
\item \emph{Zero-Knowledge:} there is a PPT simulator $(\simul_1, \simul_2)$, such that for all non-uniform polynomial-size adversaries $\adv = \{\adv_\secpar\}_{\secpar \in \nats}$ there exists a negligible function $\negl(\cdot)$ such that for all $\secpar \in \nats$:
\begin{align*}
\zk_\adv^\nizk =& 
\left| \Pr\left[ \adv_\secpar^{\prove(\crs, \cdot, \cdot),}(\crs, \tau_L) = 1:\crs \leftarrow \setup(\seck, L) \right] \right. \\
&\left. - \Pr\left[ \adv_\secpar^{\oracle(\crs, \tau, \cdot, \cdot),}(\crs, \tau_L) = 1: (\crs, \tau) \leftarrow \simul_1(\seck, L) \right] \right|
\leq \negl(\secpar).
\end{align*}
Here $\tau_L$ is a membership testing trapdoor for language $L$; $\prove(\crs, \cdot, \cdot)$ is an oracle that outputs $\bot$ on input $(s,w) \notin \rel$ and outputs a valid proof $\pi \leftarrow \prove(\crs, s, w)$ otherwise; $\oracle(\crs, \tau, \cdot, \cdot)$ is an oracle that outputs $\bot$ on input $(s,w) \notin \rel$ and outputs a simulated proof $\pi \leftarrow \simul_2(\crs, \tau, s)$ on input $(s,w) \in \rel$. Note that the simulated proof is generated independently of the witness $w$.
\end{itemize}

\end{definition}

\begin{remark} 
We have slightly modified the soundness and zero-knowledge definitions compared to \cite{Libert2021OneShotFN}. Our soundness definition is adaptive and an adversary is given as input also a membership testing trapdoor $\tau_L$. This notion is implied by the simulation-soundness as defined in \Cref{def:simsnd}. Our zero-knowledge definition provides a membership testing trapdoor $\tau_L$ as an input for an adversary, whereas the definition of \cite{Libert2021OneShotFN} lets an adversary generate the language $L$ itself. The definition of \cite{Libert2021OneShotFN} works in our constructions, too, but we prefer to base our constructions on a slightly weaker definition. 
\end{remark}

\begin{definition}[One-Time Simulation Soundness]\label{def:simsnd}
A NIZK for an NP language $L$ with zero-knowledge simulator $\simul = (\simul_0, \simul_1)$ is \emph{one-time simulation sound}, if for all non-uniform polynomial-size adversaries $\adv = \{\adv_\secpar\}_{\secpar \in \nats}$ there exists a negligible function $\negl(\cdot)$ such that for all $\secpar \in \nats$ 
\[
\simsnd_\adv^\nizk = \Pr\left[
\begin{aligned}
s \notin L \land \\
(s, \pi) \neq (s', \pi') \land \\
\vrfy(\crs, s, \pi) = 1
\end{aligned}
:
\begin{aligned}
(\crs, \tau) \leftarrow \simul_1(\seck, L) \\
(s, \pi) \leftarrow \adv_\secpar^{\simul_2(\crs, \tau, \cdot)}(\crs, \tau_L)
\end{aligned} \right] \leq \negl(\secpar),
\]
where $\tau_L$ is a membership testing trapdoor for language $L$ and $\simul_2(\crs, \tau, \cdot)$ is a single query oracle which on input $s'$ returns $\pi' \leftarrow \simul(\crs, \tau, s')$.
\end{definition}

Libert~\etal \cite{Libert2021OneShotFN} show that given an additively homomorphic encryption scheme, one can build a trapdoor Sigma protocol for the language defined below. Moreover, any trapdoor Sigma protocol can be turned into an unbounded simulation sound NIZK which directly implies existence of a one-time simulation sound NIZK. Since we use the term \emph{trapdoor Sigma protocol} only as intermediate notion and never instantiate it, we do not state formal definition and only reference it for brevity. For more details about trapdoor Sigma protocols see e.g. \cite{Libert2021OneShotFN}. 

\begin{lemma}[Lemma D.1~\cite{Libert2021OneShotFN}]\label{lem:tsp}
Let $(\gen, \enc, \dec)$ be an additively homomorphic encryption scheme where the message space $M$, randomness space $R$ and the ciphertext space $C$ form groups $(M, +), (R,+)$ and $(C, \cdot)$. Let the encryption scheme be such that for any public key $\pk$ generated using $(\pk, \sk) \leftarrow \gen(\seck)$, any messages $m_1, m_2 \in M$ and randomness $r_1, r_2 \in R$ holds
\[\enc(\pk, m_1;r_1) \cdot \enc(\pk, m_2;r_2) = \enc(\pk, m_1+m_2; r_1+r_2).\]
Let $S$ be a finite set of public cardinality such that uniform sampling from $S$ is computationally indistinguishable from uniform sampling from $R$. 
Then there is an trapdoor Sigma protocol for the language $L:= \{c \in C| \exists r \in R: c = \enc(\pk, 0; r)\}$ of encryptions of zero, where $\pk$ is fixed by the language.  
\end{lemma}

\begin{remark}
We note that Libert~\etal required that the order of the group $(R,+)$ is public and that this group is efficiently samplable, which is used in their proof of the zero-knowledge property. This is however, not necessary, since it is sufficient to be able to sample from a distribution which is computationally indistinguishable from the uniform distribution. This results in computational indistinguishability of real and simulated transcripts. In case of our constructions, we will sample randomness from a distribution which is statistically close and hence indistinguishable from the uniform distribution over $R$, which yields that the real and the simulated transcripts are statistically indistinguishable. 
\end{remark}

Additionally, Libert~\etal construct a simulation sound non-interactive argument system from any trapdoor Sigma protocol relying on a strongly unforgeable one-time signature, a lossy public-key encryption scheme, an admissible hash function and a correlation intractable hash function. 

\begin{theorem}[Thm B.1, Thm. B.2~\cite{Libert2021OneShotFN}]\label{thm:nizk}
Let $(\gen_\param, \gen_L, \prove, \vrfy)$ be a trapdoor Sigma protocol for an NP language $L$. Then given a strongly unforgeable one-time signature scheme, $\rel$-lossy public-key encryption scheme, a correlation intractable hash function and an admissible hash function, there is an unbounded simulation sound non-interactive zero-knowledge proof system for the language $L$. 
\end{theorem}

We note that in order to achieve negligible soundness error, it is needed to run the underlying trapdoor Sigma protocol $\mathcal{O}(\log \secpar)$ times in parallel. One run of the trapdoor Sigma protocol of Libert~\etal for $L$, as defined above, corresponds to sending one ciphertext of the homomorphic encryption scheme and one random element $r \in R$.
\todo{say a few words about efficiency}

\subsection{Standard-Model Instantiation of SS-NIZKs}\label{sec:nizk-crs}
In this section we provide simulation sound NIZK proof systems for languages $L_1$ and $L_2$ that are used in our constructions. The languages are defined in the following way:  
\[
L_1 = \left\{(c_0, c_1, c_2, c_3)| \exists (m,r):
\begin{aligned}
       (\land_{i=1}^3 c_i = h_i^{rN}(1+N)^m \bmod N^2) \land \\
       c_0 = g^r \bmod N\\
    \end{aligned}
    \right\} \text{ and } 
\]
\[
L_2 = \left\{(c_0, c_1, c_2)| \exists (m,r):
\begin{aligned}
       (\land_{i=1}^2 c_i = h_i^{r}m \bmod N) \land
       c_0 = g^r \bmod N\\
    \end{aligned}
    \right\},   
\]
where $g, h_1, h_2, h_3, N$ are parameters defining the languages. 

Note that $L_1$ consists of all ciphertexts $(c_0, c_1\cdot (c_2)^{-1}, c_3\cdot (c_2)^{-1})$ that are encryptions of zero, where the corresponding public key is defined as $\pk:=(g, (h_1\cdot (h_2)^{-1}), (h_3\cdot (h_2)^{-1}), N)$ and encryption is defined as $\enc(\pk:=(g,h,h'),m): c:=g^r \bmod N, c':=h^{rN}(1+N)^m \bmod N^2, c':=h'^{rN}(1+N)^m \bmod N^2$. $L_2$ consists of all ciphertexts $(c_0, c_1\cdot (c_2)^{-1})$ that are encryptions of zero, where the corresponding public key is defined as $\pk:=(g, (h_1\cdot (h_2)^{-1})), N)$ and encryption is defined as $\enc(\pk:=(g,h),m): c:=g^r, c':=hg^m \bmod N$. Hence, both encryption schemes are additively homomorphic and by \Cref{lem:tsp} we obtain a trapdoor Sigma protocol for the languages $L_1, L_2$. By \Cref{thm:nizk} this yields unbounded simulation-sound NIZKs for these languages.

%Note that these languages consists of all ciphertexts $(c_0, c_1\cdot (c_2)^{-1}, c_3\cdot (c_2)^{-1})$ that are encryptions of zero, where the corresponding public key is defined as $\pk:=(g, (h_1\cdot (h_2)^{-1}), (h_3\cdot (h_2)^{-1}), N)$. For $L_1$ encryption is defined as $\enc(\pk:=(g,h,h'),m): c:=g^r \bmod N, c':=h^{rN}(1+N)^m \bmod N^2, c':=h'^{rN}(1+N)^m \bmod N^2$. For $L_2$ encryption is defined as $\enc(\pk:=(g,h,h'),m): c:=g^r, c':=hg^m, c':=h'g^m \bmod N$. Hence, both encryption schemes are additively homomorphic and by \Cref{lem:tsp} we obtain a trapdoor Sigma protocol for the languages $L_1, L_2$. By \Cref{thm:nizk} this yields unbounded simulation-sound NIZKs for these languages.








%%% Local Variables:
%%% mode: latex
%%% TeX-master: "main"
%%% End:

%!TEX root=./main.tex
\subsection{Construction of Linearly Homomorphic Non-Malleable NITC}
We start with a construction of linearly homomorphic non-malleable NITC. In our construction we rely on SS-NIZK for the following language: 

\[
L = \left\{(c_0, c_1, c_2)| \exists (m,r):
\begin{aligned}
       (\land_{i=1}^3 c_i = h_i^{rN}(1+N)^m \bmod N^2) \land \\
       c_0 = g^r \bmod N\\
    \end{aligned}
    \right\}, 
\]
where $g, h_1, h_2, h_3, N$ are parameters specifying the language.

%In the following we use $\crs_\nizk \leftarrow \nizk.\setup(\seck, L)$ to denote the following sequence of instructions: $\param \leftarrow \nizk.\gen_\param(\seck)$, $\crs_L \leftarrow \nizk.\gen_L(\seck, L)$, $\crs_\nizk:=(\param, \crs_L)$ .


\begin{figure}[h!]
\begin{center}
\begin{tabular}{|ll|}
\hline
$\underline{\pgen(\seck, T)}$ 							   & $\underline{\com(\crs, m)}$ \\
$(p, q_, N, g) \leftarrow \genmod(\seck)$ & $r \rand \smplset$  \\
$\varphi(N):= (p-1)(q-1)$   & $c_0:= g^r \bmod N$ \\
$k_1, k_2 \rand \smplset$ & For $i \in [3]: c_i:= h_i^{rN}(1+N)^m \bmod N^2$\\
$t:= 2^T \bmod \varphi(N)/2$ & $\Phi := (c_0, c_1, c_2, c_3), w := (m, r)$ \\
For $i \in [2]: h_i:= g^{k_i} \bmod N$ &  $\pi \leftarrow \nizk.\prove(\crs_\nizk, \Phi, w)$\\
$h_3:=g^{t} \bmod N$ &  $c := (c_0, c_1, c_2, c_3, \pi)$\\
$\crs_\nizk \leftarrow \nizk.\setup(\seck, L)$ &  $\pi_\com:= \bot, \pi_\dec: = r$ \\
return $\crs:= (N,T,g,h_1,h_2, h_3, \crs_\nizk)$ & return $(c, \pi_\com, \pi_\dec)$\\
%$\crs_\nizk \leftarrow \nizk.\setup(\seck, L, (k_1, k_2, p,q))$ & return $(c, \pi_\com, \pi_\dec)$\\
%$\crs_L \leftarrow \nizk.\gen_L(\seck, L, k_1)$ & \\ 
%$\crs_\nizk:=(\param, \crs_\nizk)$ & \\
%return $\crs$     & \\
                                             &\\
$\underline{\cvrfy(\crs, c, \pi_\com)}$     & $\underline{\dvrfy(\crs,c, m, \pi_\dec)}$ \\
Parse $c$ as $(c_0, c_1, c_2, c_3, \pi)$  & Parse $c$ as $(c_0, c_1, c_2, c_3, \pi)$ \\
return $\nizk.\vrfy(\crs_\nizk, (c_0, c_1, c_2, c_3), \pi)$  & if $ \land_{i=1}^3 c_i = h_i^{\pi_\dec N}(1+N)^m  \bmod N^2$ \\
								& $\land c_0 = g^{\pi_\dec} \bmod N$\\
								 & \tab return 1 	\\
& return 0 \\

                                             &\\
$\underline{\fdecom(\crs,c)}$ & $\underline{\fdvrfy(\crs,c, m, \pi_\fdecom)}$ \\
Parse $c$ as $(c_0, c_1, c_2, c_3, \pi)$ & Parse $c$ as $(c_0, c_1, c_2, c_3, \pi)$\\
if $\nizk.\vrfy(\crs_\nizk, (c_0, c_1, c_2, c_3), \pi)$ &if $c_3 = \pi_\fdecom^N (1+N)^m \bmod N^2$ \\
\tab Compute $ \pi_\fdecom:=c_0^{2^T} \bmod N$ & \tab return 1 \\
\tab $m:=\frac{c_3 \cdot \pi_\fdecom^{-N} (\bmod N^2) -1}{N}$ & return 0\\
\tab return $(m,\pi_\fdecom)$ & \\
return $\bot$ &\\
                                             &\\
$\underline{\eval(\crs,\oplus_N, c_1, \dots, c_n)}$ &  \\
Parse $c_i$ as $(c_{i,0}, c_{i,1}, c_{i,2}, c_{i,3}, \pi_i)$ & \\
\multicolumn{2}{|l|}{Compute $c_0 := \prod_{i=1}^n c_{i,0} \bmod N, c_1:= \bot, c_2:=\bot, c_3 := \prod_{i=1}^n c_{i,3} \bmod N^2, \pi:= \bot$}  \\
return $c := (c_0, c_1, c_2, c_3, \pi)$ & \\
%$\underline{\decom(\crs, \sk, c)}$     & $\underline{\fdecom(\crs,c)}$ \\
%Parse $c$ as $(c_0, c_1, c_2, \pi)$  & Parse $c$ as $(c_0, c_1, c_2, \pi)$ \\
%if $\nizk.\vrfy((c_0, c_1, c_2), \pi)= 1$  & if $\nizk.\vrfy((c_0, c_1, c_2), \pi)= 1$\\
%\tab Compute $y_1:= c_0^{k} \bmod N$  &   \tab Compute $ y_2:=c_0^{2^T} \bmod N$ \\
%\tab return $c_1 \cdot y_1^{-1} \bmod N$ & \tab return $c_2 \cdot y_2^{-1} \bmod N$ \\
%return $\bot$ & return $\bot$\\
\hline          
\end{tabular}
\caption{Construction of Linearly Homomorphic NITC in Standard Model. \\ $\oplus_N$ refers to addition $\bmod N$}
\label{table:nitc-lh}
\end{center}
\end{figure}

%\begin{figure}[h!]
%\begin{center}
%\begin{tabular}{|ll|}
%\hline
%$\underline{\pgen(\seck, T)}$ 							   & $\underline{\com(\pk, m)}$ \\
%$(p, q_, N) \leftarrow \genmod(\seck)$ & $r \rand \smplset$  \\
%$\varphi(N):= (p-1)(q-1)$   & Compute $c_0:= g^r \bmod N$ \\
%Sample random generator $g$ of $\Jn$ & For $i \in [3]: c_i:= h_i^{rN}(1+N)^m \bmod N^2$\\
%$k_1, k_2 \rand \smplset$ & $\Phi := (h_1c_0, c_1, c_2, c_3), w := (m, r)$ \\
%$t:= 2^T \bmod \varphi(N)/2$ &  $\pi \leftarrow \nizk.\prove(\Phi, w)$\\
%For $i \in [2]: h_i:= g^{k_i} \bmod N$ &  $c := (c_0, c_1, c_2, c_3, \pi)$\\
%$h_3:=g^{t} \bmod N$ &  $\pi_\com:= \bot, \pi_\dec: = r$ \\
%%$\crs \leftarrow \nizk.\setup(\seck)$ & \\
%return $\crs:= (N,T,g,h_1,h_2, h_3)$ & return $(c, \pi_\com, \pi_\dec)$\\
%%return $\crs$     & \\
%                                             &\\
%$\underline{\cvrfy(\crs, c, \pi_\com)}$     & $\underline{\dvrfy(\crs,c, m, \pi_\dec)}$ \\
%Parse $c$ as $(c_0, c_1, c_2, c_3 \pi)$  & Parse $c$ as $(c_0, c_1, c_2, c_3 \pi)$ \\
%return $\nizk.\vrfy((c_0, c_1, c_2, c_3,), \pi)$  & if $ \land_{i=1}^3 c_i = h_i^{rN}(1+N)^m  \bmod N^2 \land c_0 = g^r \bmod N$\\
% & \tab return 1 \\
%& return 0 \\
%                                             &\\
%$\underline{\fdecom(\crs,c)}$ & $\underline{\fdvrfy(\crs,c, m, \pi_\fdecom)}$ \\
%Parse $c$ as $(c_0, c_1, c_2, c_3, \pi)$ & Parse $c$ as $(c_0, c_1, c_2, c_3, \pi)$\\
%if $\nizk.\vrfy((c_0, c_1, c_2,c_3), \pi)= 1$& if $\nizk.\vrfy((c_0, c_1, c_2,c_3), \pi)= 1 \land $\\
%\tab Compute $ \pi_\fdecom:=c_0^{2^T} \bmod N$ &  $c_3 = \pi_\fdecom^N (1+N)^m \bmod N^2$\\
%\tab Compute $m:=\frac{c_3 \cdot \pi_\fdecom^{-N} (\bmod N^2) -1}{N}$ &\tab return 1\\
%\tab return $(m,\pi_\fdecom)$ & return 0\\
%return $\bot$ & \\
%
%%$\underline{\decom(\crs, \sk, c)}$     & $\underline{\fdecom(\crs,c)}$ \\
%%Parse $c$ as $(c_0, c_1, c_2, \pi)$  & Parse $c$ as $(c_0, c_1, c_2, \pi)$ \\
%%if $\nizk.\vrfy((c_0, c_1, c_2), \pi)= 1$  & if $\nizk.\vrfy((c_0, c_1, c_2), \pi)= 1$\\
%%\tab Compute $y_1:= c_0^{k} \bmod N$  &   \tab Compute $ y_2:=c_0^{2^T} \bmod N$ \\
%%\tab return $c_1 \cdot y_1^{-1} \bmod N$ & \tab return $c_2 \cdot y_2^{-1} \bmod N$ \\
%%return $\bot$ & return $\bot$\\
%
%\hline          
%\end{tabular}
%\caption{NY Construction of NITC}
%\label{table:nitc}
%\end{center}
%\end{figure}



\begin{theorem}
If  $(\nizk.\setup, \nizk.\prove, \nizk.\vrfy)$ is a one-time simulation-sound non-interactive zero-knowledge proof system for $L$, the strong sequential squaring assumption with gap $\gap$ holds relative to $\genmod$, the Decisional Composite Residuosity assumption holds relative to $\genmod$, and the Decisional Diffie-Hellman assumption holds relative to $\genmod$ in $\Jn$, then \mathlist{(\pgen, \com, \cvrfy, \dvrfy, \fdecom)} defined in \Cref{table:nitc-lh} is an IND-CCA-secure non-interactive timed commitment scheme with $\ugap$, for any $\ugap < \gap$. 
\end{theorem}

\begin{proof}
Completeness is implied by the completeness of the NIZK and can be verified by inspection. 

%Our construction is based on the Naor-Yung paradigm where we combine three Paillier-type ciphertexts with shared randomness.  


%Similarly to \cite{SCN:BiaMasVen16} we define a PPT algorithm $\rerand$ in Figure~\ref{fig:rerand}, which takes two ciphertexts generated with independent randomness, both public keys, only one secret key (in our case $k$) and randomness which was used to encrypt a message using the public key $g, h_1$. 
%%We assume that modulus $N$ is implicitly known.\todo{Why not explicit?} 
%% which corresponds to the secret key which is given as the input. 
%
%\begin{figure}[tb]
%\centering
%\begin{minipage}{0.75\textwidth}
%$\underline{\rerand(c:= (g^{r}, h_1^{r}\cdot m), c':= (g^{r'}, h_2^{r'}\cdot m), N, h_1, h_2, k, r)}:$
%\vspace{-2mm}
%\begin{itemize}
%\item $c_0:= g^{r}\cdot{g^{r'}} = g^{r+r'} \bmod N$;
%\item $c_1:= (g^{r'})^k \cdot h_1^{r}\cdot m  =  h_1^{r'}\cdot h_1^{r}\cdot m = h_1^{r+r'}\cdot m \bmod N$;
%\item $c_2:=h_2^{r} \cdot h_2^{r'}\cdot m = h_2^{r+r'}\cdot m \bmod N$.
%\end{itemize}
%\end{minipage}
%\caption{\label{fig:rerand}Algorithm $\rerand$.}
%\end{figure}
%
%
%
%
%It is straightforward to see that the ciphertext returned by $\rerand$ is perfectly distributed to the ciphertext produced using a shared randomness where the pair $(c_0, c_1)$ encrypts a message $m$ and the pair $(c_0, c_2)$ encrypts message $m'$.


% We note that if we use value $2^T$ as a secret key, then in order to compute $c_2$ we have to execute $T$ repeated squarings, but since $T$ is polynomial in $\secpar$ this computations is considered to be efficient.

\newsequenceofgames{NITC-LH}
To prove security we define a sequence of games $\games_0 - \games_{12}$.  For $i \in \{0,1,\dots,12\}$ we denote by $\games_i = 1$ the event that the adversary $\adv = \{(\adv_{1,\secpar}, \adv_{2, \secpar})\}_{\secpar \in \nats}$ outputs $b'$ in the game $\games_i$ such that $b = b'$.
In individual games we use the algorithm $\decom$ define in \Cref{fig:deco} to answer decommitment queries efficiently. 
\begin{figure}[h!]
\begin{center}
\begin{tabular}{|l|}
\hline
$\underline{\decom(\crs, c, i)}$\\
Parse $c$ as $(c_0, c_1, c_2,, c_3, \pi)$\\
if $\nizk.\vrfy(\crs_\nizk, (c_0, c_1, c_2, c_3), \pi)= 1$\\
\tab Compute $y:= c_0^{k_i} \bmod N$\\
\tab return $\frac{c_i \cdot y^{-N} (\bmod N^2) -1}{N}$\\
return $\bot$\\
\hline          
\end{tabular}
\caption{Decommitment oracle}
\label{fig:deco}
\end{center}
\end{figure}

\nextgame{G0}
Game $\games_\thisgame$ corresponds to the original security experiment where decommitment queries are answered using $\fdecom$.

\nextgame{DecOracle}
In game $\games_\thisgame$ decommitment queries are answered using $\decom$ with $i:=1$ meaning that secret key $k_1$ and ciphertext $c_1$ are used.

\begin{lemma}\label{nitc-lh:flem}
\[
\left|\Pr[\games_\prevgame = 1] - \Pr[\games_\thisgame = 1]\right| \leq \snd^\nizk_\advB.
\]
\end{lemma}

Notice that if $c_1$ and $c_3$ contain the same message, both oracles answer decommitment queries consistently. Let $\event$ denote the event that the adversary $\adv$ asks a decommitment query $c$ such that its decommitment using the key $k_1$ is different from its decommitment using $\fdecom$. Since $\games_\prevgame$ and $\games_\thisgame$ are identical until $\event$ does not happen, we bound the probability of $\event$. Concretely, we have

\[
\left|\Pr[\games_\prevgame = 1] - \Pr[\games_\thisgame = 1]\right| \leq \Pr[\event]. 
\]

We construct an adversary $\advB$ that breaks soundness of the NIZK. It is given as input $\crs_\nizk$ together with a membership testing trapdoor $\tau_L:=(k_1, k_2, t)$ where $t:=2^T \bmod \varphi(N)/2$. 

The adversary $\advB_{\secpar}(\crs_\nizk, \tau_L)$ proceeds as follows:
\vspace{-2mm}
\begin{enumerate}
\item It computes $h_1:= g^{k_1} \bmod N, h_2:= g^{k_2} \bmod N, h_3:= g^{t} \bmod N$ using the membership testing trapdoor $\tau_L$ and sets $\crs:=(N, T, g, h_1, h_2, h_3, \crs_\nizk)$.\todo{How is $\tau_L$ defined exactly?}
\item The it runs $(m_0, m_1, \st) \leftarrow \adv_{1, \secpar}(\crs)$ and answers decommitment queries using $k_1$.
\item It samples $b \rand \bits, r \rand \smplset$ and computes $c_0^*:=g^r, c_1^*:=h_1^{rN}(1+N)^{m_b}, c_2^*:=h_2^{rN}(1+N)^{m_b}, c_3^*:=h_3^{rN}(1+N)^{m_b}$. It sets $(s:=\mathlist(c_0^*, c_1^*, c_2^*, c_3^*), w:=(m,r))$ and runs $\pi^* \leftarrow \nizk.\prove(s,w)$.
\item Next, it runs $b' \leftarrow \adv_{2, \secpar}((c_0^*, c_1^*, c_2^*, c_3^*), \pi^*, \st)$ and answers decommitment queries using $k_1$.
\item Finally, it checks whether there exists a decommitment query $c: = (c_0, c_1, c_2, c_3, \pi)$ such that $\dec(\crs, c, 1) \neq \dec(\crs,c,2)$. If this is the case, it returns $((c_0, c_1, c_2, c_3), \pi)$. Notice that this can be done efficiently with the knowledge of $t$.
\end{enumerate}

$\advB$ simulates $\games_\thisgame$ perfectly and if the event $\event$ happens, it outputs a valid proof for a statement which is not in the specified language $L$. Therefore
\[\Pr[\event] \leq \snd^\nizk_\advB,\]
which concludes the proof of the lemma.  

%Let $\gnr$ denote the event that the sampled $g$ in $\kgen$ is a generator of $\qrn$. Recall that $N = pq$ where $p = 2p'+1$ and $q = 2q'+1$. Because $g$ is sampled uniformly at random and $\qrn$ has $\varphi(|\qrn|) = (p'-1)(q'-1)$ generators, this event happens with overwhelming probability. Concretely, $\Pr[\gnr] = 1-\frac{1}{p'}-\frac{1}{q'}+\frac{1}{p'q'}$.
%Therefore the following holds.
%\begin{lemma}\label{tpke3:flem} 
%\begin{align*}
%\Pr[\games_\thisgame = 1] &= \Pr[\games_\thisgame = 1| \gnr]\Pr[\gnr] + \Pr[\games_\thisgame = 1| \overline{\gnr}]\Pr[\overline{\gnr}] \\
%&\leq \Pr[\games_\thisgame = 1| \gnr]\Pr[\gnr] + \Pr[\overline{\gnr}] \\
%&= \Pr[\games_\thisgame = 1| \gnr]\left( 1-\frac{1}{p'}-\frac{1}{q'}+\frac{1}{p'q'} \right) + \frac{1}{p'}+\frac{1}{q'}-\frac{1}{p'q'}.
%\end{align*}
%\end{lemma}




\nextgame{SimulProof}
Game $\games_\thisgame$ proceeds exactly as the previous game but we run the zero-knowledge simulator $(\crs, \tau) \leftarrow \simul_1(\seck, L)$ in $\pgen$ and produce a simulated proof for the challenge commitment as $\pi \leftarrow \simul_2(\crs, \tau, (c_0^*,c_1^*,c_2^*,c_3^*))$. By zero-knowledge security of underlying NIZK we directly obtain
\begin{lemma}
\[
\left|\Pr[\games_\prevgame = 1] - \Pr[\games_\thisgame = 1]\right| \leq \zk^\nizk_\advB.
\]
\end{lemma}

We construct an adversary $\advB = \{\advB_\secpar\}_{\secpar \in \N}$ against zero-knowledge security of NIZK as follows:
$\advB_\secpar(\crs_\nizk, \tau_L):$
\vspace{-2mm}
\begin{enumerate}
\item Sets $\crs:=\mathlist(N, T(\secpar), g, h_1, h_2, h_3, \crs_\nizk)$. and runs $(m_0, m_1, \st) \leftarrow \adv_{1, \secpar}(\crs)$ and answers decommitment queries using $k_1$ which is included in $\tau_L$.
\item Samples $b \rand \bits, r \rand \smplset$ and computes $c_0^*:=g^r, c_1^*:=h_1^{rN}(1+N)^{m_b}, c_2^*:=h_2^{rN}(1+N)^{m_b}, c_3^*:=h_3^{rN}(1+N)^{m_b}$. It submits $(s:=\mathlist(c_0^*, c_1^*, c_2^*, c_3^*), w:=(m,r))$ to its oracle and obtains proof $\pi^*$ as answer.
\item Runs $b' \leftarrow \adv_{2, \secpar}((c_0^*, c_1^*, c_2^*, c_3^*), \pi^*, \st)$ and answers decommitment queries using $k_1$.
\item Returns the truth value of $b=b'$.
\end{enumerate}
If the proof $\pi^*$ is generated using $\nizk.\prove$, then $\advB$ simulates $\games_\prevgame$ perfectly. Otherwise $\pi^*$ is generated using $\simul_1$ and $\advB$ simulates $\games_\thisgame$ perfectly. This proofs the lemma.



\nextgame{RndExp}
In $\games_\thisgame$ we sample $r$ uniformly at random from $[\varphi(N)/2]$. 

\begin{lemma}
\[
\left|\Pr[\games_\prevgame = 1] - \Pr[\games_\thisgame = 1]\right| \leq \frac{1}{p}+\frac{1}{q}-\frac{1}{N}.
\]
\end{lemma}
%At first we remark that for upper bounding the difference between the games we use a statistical argument. Because $r$ appears only in the exponent of the group generator, we later sample a random element from the group $\qrn$ which can be done efficiently. 
Since the only difference between the two games is in the set from which we sample $r$, to upper bound the advantage of adversary we can use \Cref{sampling-lemma}, which directly yields required upper bound.

%\nextgame{ReRand}
%In Game $\games_\thisgame$ we produce the challenge commitment by encrypting the challenge message using two independent random exponents $r \rand \smplset, r' \rand [\varphi(N)/4]$ to obtain $c:= (g^{r}, h_1^{r}\cdot m_b), c':= (g^{r'}, h_2^{r'}\cdot m_b)$ and then run $\rerand(c,c',N, \allowbreak h_1, h_2,k,r)$ to obtain resulting ciphertext $(c_0^*, c_1^*, c_2^*)$. Since $r'$ is sampled uniformly at random from $[\varphi(N)/4]$ the ciphertext distributions in both games  are the same. Therefore 
%
%\begin{lemma}
%\[
%\Pr[\games_\prevgame = 1] = \Pr[\games_\thisgame = 1].
%\]
%\end{lemma}


\nextgame{SSSA}
In $\games_\thisgame$ we sample $y_3 \rand \Jn$ and compute $c_3^*$ as $y_3^N (1+N)^{m_b}$.

Let $\tilT_\sss(\secpar)$ be the polynomial whose existence is guaranteed by the SSS assumption.
Let $\poly_\advB(\secpar)$ be the fixed polynomial which bounds the time required to execute Steps 1--2 and answer decommitment queries in Step 3 of the adversary $\advB_{2, \secpar}$ defined below. Set $\undT := (\poly_\advB(\secpar))^{1 / \ugap}$.  Set $\tilT_\nitc := \max(\tilT_\sss, \undT)$.
\begin{lemma}
From any polynomial-size adversary $\adv = \{(\adv_{1,\secpar}, \adv_{2, \secpar})\}_{\secpar \in \nats}$, where depth of $\adv_{2, \secpar}$ is at most $T^{\ugap}(\secpar)$ for some $T(\cdot) \geq \undT(\cdot)$ we can construct a polynomial-size adversary $\advB = \{(\advB_{1,\secpar}, \advB_{2, \secpar})\}_{\secpar \in \nats}$ where the depth of $\advB_{2, \secpar}$ is at most $T^{\gap}(\secpar)$ with
\[
\left|\Pr[\games_\prevgame = 1] - \Pr[\games_\thisgame = 1]\right| \leq \advtg_\advB^\sss.
\]
\end{lemma}

The adversary $\advB_{1,\secpar}(N, T(\secpar), g):$
\vspace{-2mm}
\begin{enumerate}
\item Samples $k_1, k_2 \rand \smplset$, computes $h_1 := g^{k_1} \bmod N, h_2 := g^{k_2} \bmod N,  h_3 := g^{2^{T(\secpar)}} \bmod N$, runs $(\crs_\nizk, \tau) \leftarrow \nizk.\simul_1(\seck, L)$ and sets $\crs:=\mathlist(N, T(\secpar), g, h_1, h_2, h_3, \crs_\nizk)$. Notice that value $h_3$ is computed by repeated squaring.
\item Runs $(m_0, m_1, \st) \leftarrow \adv_{1, \secpar}(\crs)$ and answers decommitment queries using $k_1$.
\item Outputs $(N,g,k_1, k_2, h_1,h_2,h_3, \crs_\nizk, \tau, m_0, m_1, \st)$
\end{enumerate}

The adversary $\advB_{2,\secpar}(x,y,(N,g,k_1, k_2, h_1,h_2,h_3, \crs_\nizk, \tau, m_0, m_1, \st)):$
\vspace{-2mm}
\begin{enumerate}
\item Samples $b \rand \bits$, computes $c_0^*:=x, c_1^*:=x^{k_1N}(1+N)^{m_b}, c_2^*:=x^{k_2N}(1+N)^{m_b}, c_3^*:=y^{N}(1+N)^{m_b}$.
\item Runs $\pi^* \leftarrow \simul_2(\crs_\nizk, \tau, (c_0^*, c_1^*, c_2^*, c_3^*))$.
\item Runs $b' \leftarrow \adv_{2, \secpar}((c_0^*, c_1^*, c_2^*, c_3^*), \pi^*), \st)$ and answers decommitment queries using $k_1$.
\item Returns the truth value of $b=b'$.
\end{enumerate}
Since $g$ is a generator of $\Jn$ and $x$ is sampled uniformly at random from $\Jn$ there exists some $r \in [\varphi(N)/2]$ such that $x = g^{r}$. Therefore when $y = x^{2^T} = (g^{2^T})^{r} \bmod N$, then $\advB$ simulates $\games_\prevgame$ perfectly. Otherwise $y$ is random value and $\advB$ simulates $\games_\thisgame$ perfectly. 

Now we analyse the running time of the constructed adversary. Adversary $\advB_1$ computes $h_3$ by $T(\secpar)$ consecutive squarings and because $T(\secpar)$ is polynomial in $\secpar$, $\advB_1$ is efficient. Moreover, $\advB_2$ fulfils the depth constraint:
\[ \dep(\advB_{2,\secpar}) = \poly_\advB(\secpar) + \dep(\adv_{2,\secpar}) \leq \undT^{\ugap}(\secpar) + T^{\ugap}(\secpar) \leq 2 T^{\ugap}(\secpar) = o(T^{\gap}(\secpar)). \] 

Also $T(\cdot) \geq \tilT_\nitc(\cdot) \geq \tilT_\sss(\cdot)$ as required.

%\nextgame{RndExp2}
%In $\games_\thisgame$ we stop to use $\rerand$ algorithm. Concretely, we sample $r \rand [\varphi(N)/4], y_2 \rand \qrn$ and compute challenge ciphertext as $c^*:=(g^r, h_1^r \cdot m_b, y_2 \cdot m_b)$. The ciphertext has the same distribution as in the previous game. Therefore 
%
%\begin{lemma}
%\[
%\Pr[\games_\prevgame = 1] = \Pr[\games_\thisgame = 1].
%\]
%\end{lemma}
%\nextgame{RndExp3}




\nextgame{DCR1}
In $\games_\thisgame$ we sample $y_3 \rand \Zns$ such that it has Jacobi symbol 1 and compute $c_3^*$ as $y_3(1+N)^{m_b}$. 

\begin{lemma}\label{lem:dcr}
\[
\left|\Pr[\games_\prevgame = 1] - \Pr[\games_\thisgame = 1]\right| \leq \advtg_\advB^\dcr.
\]
\end{lemma}
We construct an adversary $\advB = \{\advB_\secpar\}_{\secpar \in \N}$ against DCR.

$\advB_{\secpar}(N,y):$
\vspace{-2mm}
\begin{enumerate}
\item Samples $g, y_3, x \rand \Jn, k_1, k_2 \rand \smplset$, computes $h_1 := g^{k_1} \bmod N, h_2 := g^{k_2} \bmod N,  h_3 := g^{2^{T}} \bmod N$, runs $(\crs_\nizk, \tau) \leftarrow \nizk.\simul_1(\seck, L)$ and sets $\crs:=\mathlist(N, T(\secpar), g, h_1, h_2, h_3, \crs_\nizk)$. Notice that value $h_3$ is computed by repeated squaring.
\item Runs $(m_0, m_1, \st) \leftarrow \adv_{1, \secpar}(\crs)$ and answers decommitment queries using $k_1$.
\item Samples $b \rand \bits, w \rand \Zns$ such that $\left( \frac{y}{N} \right)= \left( \frac{w}{N} \right)$. We remark that computing Jacobi symbol can be done efficiently without knowing factorization of N.
\item Computes $c_0^*:=x, , c_1^*:=x^{k_1N}(1+N)^{m_b}, c_2^*:=x^{k_2N}(1+N)^{m_b}, c_3^*:=yw^{N}(1+N)^{m_b}$. Runs $\pi^* \leftarrow \simul_2(\crs_\nizk, \tau, (c_0^*, c_1^*, c_2^*, c_3^*))$.
\item Runs $b' \leftarrow \adv_{2, \secpar}((c_0^*, c_1^*, c_2^*, c_3^*), \pi^*, \st)$ and answers decommitment queries using $k_1$.
\item Returns the truth value of $b=b'$.
\end{enumerate}
If $y = v^N \bmod N^2$ then $yw^N = v^N w^N = (vw)^N$ and hence $yw^N$ is N-th residue. Moreover, the Jacobi symbol of $yw$ is 1, since the Jacobi symbol is multiplicatively homomorphic. Therefore $\advB$ simulates $\games_\prevgame$ perfectly. Otherwise $y$ is uniform random element in $\Zns$ then $yw^N$ is also uniform among all elements of $\Zns$ that have Jacobi symbol 1 and $\advB$ simulates $\games_\thisgame$ perfectly. This proofs the lemma. We remark that at this point $c_3^*$ does not reveal any information about $m_b$. To argue this we rely on the facts that  if $x = y \bmod N$ then $\left( \frac{x}{N} \right)= \left( \frac{y}{N} \right)$ and that there is an isomorphism $f:\Zn^* \times \Zn \rightarrow\Zns$ given by $f(u,v)=u^N(1+N)^v = u^N(1+vN) \bmod N^2$ (see e.g. \cite[Proposition 13.6]{books/crc/KatzLindell2014}).  Since $f(u,v) \bmod N = u^N + u^NvN \bmod N = u^N \bmod N$, then if $\left( \frac{f(u,v)}{N} = 1 \right)$ then it must hold that $\left( \frac{f(u,r)}{N} = 1 \right)$ for all $r \in \Zn$. This implies that random element $f(u,v)$ in $\Zns$ with $\left( \frac{f(u,v)}{N} \right) = 1$ has uniformly random distribution of $v$ in $\Zn$. Therefore if $yw^N = u^N(1+N)^v \bmod N^2$ then  $yw^N(1+N)^{m_b}  = u^N(1+N)^{m_b+v} \bmod N^2$. Since $v$ is uniform in $\Zn$, $(m_b + v)$ is also uniform in $\Zn$, which means that ciphertext $c_3^*$ does not reveal any information about $m_b$.    

\nextgame{RndExp2}
In $\games_\thisgame$ we sample $k_2$ uniformly at random from $[\varphi(N)/2]$. 

\begin{lemma}
\[
\left|\Pr[\games_\prevgame = 1] - \Pr[\games_\thisgame = 1]\right| \leq \frac{1}{p}+\frac{1}{q}-\frac{1}{N}.
\]
\end{lemma}

Again using a statistical argument this lemma directly follows from \Cref{sampling-lemma}.

\nextgame{DDH}
In $\games_\thisgame$ we sample $y_2 \rand \Jn$ and compute $c_2^*$ as  $y_2^N(1+N)^{m_b}$. 

\begin{lemma}\label{lem:ddh}
\[
\left|\Pr[\games_\prevgame = 1] - \Pr[\games_\thisgame = 1]\right| \leq \advtg_\advB^\ddh.
\]
\end{lemma}
We construct an adversary $\advB = \{\advB_\secpar\}_{\secpar \in \N}$ against DDH in the group $\Jn$. %Given \Cref{thm:ddh} this implies an adversary against DDH in large prime-order subgroups of $\Zn^*$.

$\advB_{\secpar}(N,g,g^\alpha, g^\beta, g^\gamma):$
\vspace{-2mm}
\begin{enumerate}
\item Samples $k_1 \rand \smplset$, computes $h_1 := g^{k_1} \bmod N,  h_3 := g^{2^{T}} \bmod N$, runs $(\crs_\nizk, \tau) \leftarrow \nizk.\simul_1(\seck, L)$ and sets $\crs:=(N, T, g, h_1, h_2:=g^\alpha, h_3, \crs_\nizk)$.
\item Runs $(m_0, m_1, \st) \leftarrow \adv_{1, \secpar}(\crs)$ and answers decommitment queries using $k_1$.
\item Samples $b \rand \bits, y_3 \rand \Zns$ such that it has Jacobi symbol 1 and computes $(c_0^*, c_1^*, c_2^*, c_3^*):=(g^\beta, (g^\beta)^{k_1 N}(1+N)^{m_b}, (g^{\gamma})^N(1+N)^{m_b}, y_3(1+N)^{m_b}).$ Runs $\pi^* \leftarrow \simul_2(\crs_\nizk, \tau, (c_0^*, c_1^*, c_2^*, c_3^*))$.
\item Runs $b' \leftarrow \adv_{2, \secpar}((c_0^*, c_1^*, c_2^*,c_3^*), \pi^*, \st)$ and answers decommitment queries using $k_1$.
\item Returns the truth value of $b=b'$.
\end{enumerate}
If $\gamma = \alpha\beta$, then $\advB$ simulates $\games_\prevgame$ perfectly. Otherwise $g^\gamma$ is uniform random element in $\Jn$ and $\advB$ simulates $\games_\thisgame$ perfectly. This proofs the lemma.

\nextgame{RndExp3}
In $\games_\thisgame$ we sample $k_2$ uniformly at random from $\smplset$. 

\begin{lemma}
\[
\left|\Pr[\games_\prevgame = 1] - \Pr[\games_\thisgame = 1]\right| \leq \frac{1}{p}+\frac{1}{q}-\frac{1}{N}.
\]
\end{lemma}

This lemma directly follows from \Cref{sampling-lemma}.

\nextgame{DCR2}
In $\games_\thisgame$ we sample $y_2 \rand \Zns$ such that it has Jacobi symbol 1 and compute $c_2^*$ as $y_2(1+N)^{m_b}$. 

\begin{lemma}
\[
\left|\Pr[\games_\prevgame = 1] - \Pr[\games_\thisgame = 1]\right| \leq \advtg_\advB^\dcr.
\]
\end{lemma}
This can be proven in similar way as \Cref{lem:dcr}. We remark that at this point $c_2^*$ does not reveal any information about $m_b$.



\nextgame{SimSnd}

In $\games_\thisgame$ we answer decommitment queries using $\dec$ with $i:=2$ which means that secret key $k_2$ and ciphertext $c_2$ are used. 

\begin{lemma}
\[
\left|\Pr[\games_\prevgame = 1] - \Pr[\games_\thisgame = 1]\right| \leq \simsnd^\nizk_\advB. 
\]
\end{lemma}

Let $\event$ denote the event that adversary $\adv$ asks a decommitment query $c$ such that its decommitment using the key $k_1$ is different from its decommitment using the key $k_2$. Since $\games_\prevgame$ and $\games_\thisgame$ are identical until $\event$ does not happen, by the standard argument it is sufficient to upper bound the probability of happening $\event$. Concretely,  

\[
\left|\Pr[\games_\prevgame = 1] - \Pr[\games_\thisgame = 1]\right| \leq \Pr[\event]. 
\]

We construct an adversary $\advB$ that breaks one-time simulation soundness of the NIZK and it is given as input $\crs_\nizk$ together with a membership testing trapdoor $\tau_L:=(k_1, k_2, t)$ where $t:=2^T \bmod \varphi(N)/2$. 

The adversary $\advB_{\secpar}^{\simul_2}(\crs_\nizk, \tau_L):$
\vspace{-2mm}
\begin{enumerate}
\item Computes $h_1:= g^{k_1} \bmod N, h_2:= g^{k_2} \bmod N, h_3:= g^{t} \bmod N$ using the membership testing trapdoor $\tau_L$ and sets $\crs:=(N, T, g, h_1, h_2, h_3, \crs_\nizk)$.
\item Runs $(m_0, m_1, \st) \leftarrow \adv_{1, \secpar}(\crs)$ and answers decommitment queries using $k_2$.
\item Samples $b \rand \bits, x \rand \Jn, y_2, y_3 \rand \Zns$ and computes $(c_0^*, c_1^*, c_2^*, c_3^*):=(x, x^{k_1 N} (1+N)^{m_b}, y_2 (1+N)^{m_b}, y_3 (1+N)^{m_b})$. Forwards $(c_0^*, c_1^*, c_2^*, c_3^*)$ to simulation oracle $\simul_2$ and obtains a proof $\pi^*$.
\item Runs $b' \leftarrow \adv_{2, \secpar}((c_0^*, c_1^*, c_2^*, c_3^*), \pi^*, \st)$ and answers decommitment queries using $k_2$.
\item Find a decommitment query $c: = (c_0, c_1, c_2, c_3, \pi)$ such that $\dec(\crs, c, 1) \neq \dec(\crs,c,2)$ and returns $((c_0, c_1, c_2, c_3), \pi)$
\end{enumerate}

$\advB$ simulates $\games_\thisgame$ perfectly and if the event $\event$ happens, it outputs a valid proof for a statement which is not in the specified language $L$. Therefore
\[\Pr[\event] \leq \simsnd^\nizk_\advB,\]
which concludes the proof of the lemma.  

%\nextgame{ReRand2}
%In $\games_\thisgame$ we use the key $t$ and randomness $r'$ as input for rerandomization. Concretely we compute $\rerand(c,c',h_1,h_2,t,r')$. This is just conceptual change since the ciphertext distributions are the same in both games and therefore 
%
%\begin{lemma}
%\[
%\left|\Pr[\games_\prevgame = 1] = \Pr[\games_\thisgame = 1]\right|.
%\]
%\end{lemma}
%
%\nextgame{RndExp4}
%In $\games_\thisgame$ we sample $r$ uniformly at random from $\varphi(N)$. 
%
%\begin{lemma}
%\[
%\left|\Pr[\games_\prevgame = 1] - \Pr[\games_\thisgame = 1]\right| \leq \frac{1}{N}.
%\]
%\end{lemma}

\nextgame{RndExp4}
In $\games_\thisgame$ we sample $k_1$ uniformly at random from $[\varphi(N)/2]$. 

\begin{lemma}
\[
\left|\Pr[\games_\prevgame = 1] - \Pr[\games_\thisgame = 1]\right| \leq \frac{1}{p}+\frac{1}{q}-\frac{1}{N}.
\]
\end{lemma}

This lemma directly follows from \Cref{sampling-lemma}.

\nextgame{DDH2}
In $\games_\thisgame$ we sample $y_1 \rand \Jn$ and compute $c_1^*$ as  $y_1^{N} (1+N)^{m_b}$. 

\begin{lemma}
\[
\left|\Pr[\games_\prevgame = 1] - \Pr[\games_\thisgame = 1]\right| \leq \advtg_\advB^\ddh.
\]
\end{lemma}
This can be proven in similar way as \Cref{lem:ddh}.

\nextgame{DCR3}
In $\games_\thisgame$ we sample $y_1 \rand \Zns$ such that it has Jacobi symbol 1 and compute $c_1^*$ as $y_1(1+N)^{m_b}$. 

\begin{lemma}
\[
\left|\Pr[\games_\prevgame = 1] - \Pr[\games_\thisgame = 1]\right| \leq \advtg_\advB^\dcr.
\]
\end{lemma}
This can be proven in similar way as \Cref{lem:dcr}. We remark that at this point $c_1^*$ does not reveal any information about $m_b$.

\begin{lemma}\label{nitc-lh:llem}
\[
\Pr[\games_\thisgame = 1] = \half.
\]
\end{lemma}

Clearly, $c_0^*$ is uniform random element in $\Jn$ and hence it does not contain any information about the challenge message. Since $y_1, y_2, y_3$ are sampled uniformly at random from $\Zns$ the ciphertexts $c_1^*, c_2^*, c_3^*$ are also uniform random elements in $\Zns$ and hence do not contain any information about the challenge message $m_b$. Therefore, an adversary can not do better than guessing.

By combining Lemmas \ref{nitc-lh:flem} - \ref{nitc-lh:llem} we obtain the following:
\begin{align*}
\advtg^{\nitc}_{\adv} &= \left| \Pr[\games_0 = 1] - \half \right| \leq \sum_{i=0}^{12} \left|\Pr[\games_i = 1] - \Pr[\games_{i+1} = 1] \right| + \left|\Pr[\games_{13}- \half\right| \\
 &\leq  \snd^\nizk_\advB + \zk^\nizk_\advB + \advtg^{\sss}_{\advB} + \simsnd^{\nizk}_{\advB} + 2 \advtg^{\ddh}_{\advB} + 3 \advtg^{\dcr}_{\advB} \\ &+ 4 \left( \frac{1}{p}+\frac{1}{q}-\frac{1}{N} \right),
\end{align*}
which concludes the proof.
\end{proof}

\begin{theorem}\label{bnd-cca-lh}
$(\pgen, \com, \cvrfy, \dvrfy, \fdecom)$ defined in \Cref{table:nitc-lh} is a BND-CCA-secure non-interactive timed commitment scheme. 
\end{theorem}

\begin{proof}
We show that the construction is actually perfectly binding. This is straightforward to show since Paillier encryption is perfectly binding. Therefore there is exactly one message/randomness pair $(m,r)$ which can pass the check in $\dvrfy$. Therefore the first winning condition of BND-CCA experiment happens with probability 0. Moreover, since $\pgen$ is executed by the challenger, the value $h_3$ is computed correctly and therefore $\fdecom$ reconstructs always the correct message $m$. Therefore the second winning condition of BND-CCA experiment happens with probability 0 as well.
\end{proof}

\begin{theorem}\label{pv-lh}
If $\nizk = (\nizk.\setup, \nizk.\prove, \nizk.\vrfy)$ is a non-interactive zero-knowledge proof system for $L$, then \mathlist{(\pgen, \com, \cvrfy, \dvrfy, \fdecom, \fdvrfy)} defined in \Cref{table:nitc-lh} is a publicly verifiable non-interactive timed commitment scheme.
\end{theorem}

\begin{proof}
Completeness is straightforward to verify. 

To prove the soundness notice that if commitment verifies, then we know that $c_0 = g^r$ and $c_3 = h_3^r(1+N)^m$ for honestly generated $g$ and $h_3$ and some $r$ and $m$. Otherwise, an adversary would be able to break soundness of the proof system. Since there is an isomorphism $f:\Zn^* \times \Zn \rightarrow\Zns$ given by $f(a,b)=a^N(1+N)^b \bmod N^2$ (see e.g. \cite[Proposition 13.6]{books/crc/KatzLindell2014}) there exist unique values $\pi_\fdecom$ and $m$ such that $c_3=\pi_\fdecom^N(1+N)^m \bmod N^2$ and therefore adversary is not able to provide different message $m'$ fulfilling required equation.

Finally, the running time of $\fdvrfy$ is efficient, since it is independent of $T$.
\end{proof}

It is straightforward to verify that considering $\eval$ algorithm, our construction yields linearly homomorphic NITC. 
\begin{theorem}\label{hom-lh}
The NITC \mathlist{(\pgen, \com, \cvrfy, \dvrfy, \fdecom, \fdvrfy, \eval)} defined in \Cref{table:nitc-lh} is a linearly homomorphic non-interactive timed commitment scheme.
\end{theorem}




 

%\begin{figure}[h!]
%\begin{center}
%\begin{tabular}{|ll|}
%\hline
%$\underline{\kgen(\seck, T)}$ 							   & $\underline{\decf(\sk, c)}$\\
%$(p, q_, N) \leftarrow \genmod(\seck)$ &  Parse $c$ as $(c_0, c_1, c_2, \pi)$\\
%$\varphi(N):= (p-1)(q-1)$   & if $\nizk.\vrfy((c_0, c_1, c_2), \pi)= 1$ \\
%$g\rand \qrn$ & \tab Compute $y_1:= c_0^{k} \bmod N$\\
%$k \rand \varphi(N)/4$ & \tab return $c_1 \cdot H(y_1)^{-1} \bmod N$ \\
%$t:= 2^T \bmod \varphi(N)/4$ & return $\bot$\\
%$h_1:= g^k \bmod N$ & \\
%$h_2:=g^{t} \bmod N$ & \\
%%$\crs \leftarrow \nizk.\setup(\seck)$ & \\
%$\pk:= (N,T,g,h_1,h_2), \sk:= (N, k)$ & \\
%return $(\pk, \sk)$       & \\
%                                             &\\
%$\underline{\enc(\pk, m)}$           & $\underline{\decs(\pk,c)}$ \\
%$r \rand [\floor{N/4}]$     & Parse $c$ as $(c_0, c_1, c_2, \pi)$ \\
%Compute $c_0:= g^r \bmod N$ & if $\nizk.\vrfy((c_0, c_1, c_2), \pi)= 1$\\
%For $i \in [2]: y_i:= h_i^r \bmod N$   &   \tab Compute $ y_2:=c_0^{2^T} \bmod N$ \\
%For $i \in [2]: c_i:= H(y_i) \cdot m \bmod N$ & \tab return $c_2 \cdot H(y_2)^{-1} \bmod N$ \\
%$\Phi := (c_0, c_1, c_2), w := (m, r)$& return $\bot$\\
%$\pi \leftarrow \nizk.\prove(\Phi, w)$  &  \\
%return $c \leftarrow (c_0, c_1, c_2, \pi)$ &  \\
%
%\hline          
%\end{tabular}
%\caption{NY Construction of TPKE from SSSA}
%\label{table:tpke-elgamal}
%\end{center}
%\end{figure}




%%% Local Variables:
%%% mode: latex
%%% TeX-master: "main"
%%% End:

%!TEX root=./main.tex
\section{Construction of Non-Malleable NITC}
In our construction we rely on SS-NIZK for the following language: 

\[
L = \left\{(c_0, c_1, c_2)| \exists (m,r):
\begin{aligned}
       (\land_{i=1}^3 c_i = h_i^{r}(m \bmod N) \land
       c_0 = g^r \bmod N\\
    \end{aligned}
    \right\}, 
\]
where $g, h_1, h_2, h_3, N$ are parameters specifying the language.


In the following we use $\crs_\nizk \leftarrow \nizk.\setup(\seck, L, \tau_L)$ to denote the following sequence of instructions: $\param \leftarrow \nizk.\gen_\param(\seck)$, $\crs_L \leftarrow \nizk.\gen_L(\seck, L, k_1)$, $\crs_\nizk:=(\param, \crs_L)$ .

\begin{figure}[h!]
\begin{center}
\begin{tabular}{|ll|}
\hline
$\underline{\pgen(\seck, T)}$ 							   & $\underline{\com(\crs, m)}$ \\
$(p, q_, N) \leftarrow \genmod(\seck)$ & $r \rand \smplset$  \\
$\varphi(N):= (p-1)(q-1)$   & $c_0:= g^r \bmod N$ \\
Sample random generator $g$ of $\Jn$ & For $i \in [3]: c_i:= h_i^{r}m \bmod N$\\
$k_1, k_2 \rand \smplset$ & $\Phi := (c_0, c_1, c_2, c_3), w := (m, r)$ \\
$t:= 2^T \bmod \varphi(N)/2$ &  $\pi \leftarrow \nizk.\prove(\crs_\nizk, \Phi, w)$\\
For $i \in [2]: h_i:= g^{k_i} \bmod N$ &  $c := (c_0, c_1, c_2, c_3, \pi)$\\
$h_3:=g^{t} \bmod N$ &  $\pi_\com:= \bot, \pi_\dec: = r$ \\
$\crs_\nizk \leftarrow \nizk.\setup(\seck, L, (k_1, k_2, p,q))$ & return $(c, \pi_\com, \pi_\dec)$\\
%$\crs_L \leftarrow \nizk.\gen_L(\seck, L, k_1)$ & \\ 
%$\crs_\nizk:=(\param, \crs_L)$ & \\
return $\crs:= (N,T,g,h_1,h_2, h_3, \crs_\nizk)$ & \\
                                             &\\
$\underline{\cvrfy(\crs, c, \pi_\com)}$     & $\underline{\dvrfy(\crs,c, m, \pi_\dec)}$ \\
Parse $c$ as $(c_0, c_1, c_2, c_3, \pi)$  & Parse $c$ as $(c_0, c_1, c_2, c_3, \pi)$ \\
return $\nizk.\vrfy(\crs_\nizk,(c_0, c_1, c_2, c_3), \pi)$  & if $ \land_{i=1}^3 c_i = h_i^{\pi_\dec}m  \bmod N \land c_0 = g^{\pi_\dec} \bmod N$\\
 & \tab return 1 \\
& return 0 \\
                                             &\\
$\underline{\fdecom(\crs,c)}$ & $\underline{\fdvrfy(\crs,c, m, \pi_\fdecom)}$ \\
Parse $c$ as $(c_0, c_1, c_2, c_3, \pi)$ & Parse $c$ as $(c_0, c_1, c_2, c_3, \pi)$\\

Compute $ y:=c_0^{2^T} \bmod N$ &  if $c_3 = \pi_\fdecom^N (1+N)^m \bmod N^2$\\
$m:=c_3 \cdot y^{-1} \bmod N$ &\tab return 1\\
return $m$ & return 0\\

%$\underline{\decom(\crs, \sk, c)}$     & $\underline{\fdecom(\crs,c)}$ \\
%Parse $c$ as $(c_0, c_1, c_2, \pi)$  & Parse $c$ as $(c_0, c_1, c_2, \pi)$ \\
%if $\nizk.\vrfy((c_0, c_1, c_2), \pi)= 1$  & if $\nizk.\vrfy((c_0, c_1, c_2), \pi)= 1$\\
%\tab Compute $y_1:= c_0^{k} \bmod N$  &   \tab Compute $ y_2:=c_0^{2^T} \bmod N$ \\
%\tab return $c_1 \cdot y_1^{-1} \bmod N$ & \tab return $c_2 \cdot y_2^{-1} \bmod N$ \\
%return $\bot$ & return $\bot$\\

\hline          
\end{tabular}
\caption{NY Construction of NITC}
\label{table:nitc}
\end{center}
\end{figure}

%\begin{figure}[h!]
%\begin{center}
%\begin{tabular}{|ll|}
%\hline
%$\underline{\pgen(\seck, T)}$ 							   & $\underline{\com(\pk, m)}$ \\
%$(p, q_, N) \leftarrow \genmod(\seck)$ & $r \rand \smplset$  \\
%$\varphi(N):= (p-1)(q-1)$   & Compute $c_0:= g^r \bmod N$ \\
%Sample random generator $g$ of $\Jn$ & For $i \in [3]: c_i:= h_i^{rN}(1+N)^m \bmod N^2$\\
%$k_1, k_2 \rand \smplset$ & $\Phi := (h_1c_0, c_1, c_2, c_3), w := (m, r)$ \\
%$t:= 2^T \bmod \varphi(N)/2$ &  $\pi \leftarrow \nizk.\prove(\Phi, w)$\\
%For $i \in [2]: h_i:= g^{k_i} \bmod N$ &  $c := (c_0, c_1, c_2, c_3, \pi)$\\
%$h_3:=g^{t} \bmod N$ &  $\pi_\com:= \bot, \pi_\dec: = r$ \\
%%$\crs \leftarrow \nizk.\setup(\seck)$ & \\
%return $\crs:= (N,T,g,h_1,h_2, h_3)$ & return $(c, \pi_\com, \pi_\dec)$\\
%%return $\crs$     & \\
%                                             &\\
%$\underline{\cvrfy(\crs, c, \pi_\com)}$     & $\underline{\dvrfy(\crs,c, m, \pi_\dec)}$ \\
%Parse $c$ as $(c_0, c_1, c_2, c_3 \pi)$  & Parse $c$ as $(c_0, c_1, c_2, c_3 \pi)$ \\
%return $\nizk.\vrfy((c_0, c_1, c_2, c_3,), \pi)$  & if $ \land_{i=1}^3 c_i = h_i^{rN}(1+N)^m  \bmod N^2 \land c_0 = g^r \bmod N$\\
% & \tab return 1 \\
%& return 0 \\
%                                             &\\
%$\underline{\fdecom(\crs,c)}$ & $\underline{\fdvrfy(\crs,c, m, \pi_\fdecom)}$ \\
%Parse $c$ as $(c_0, c_1, c_2, c_3, \pi)$ & Parse $c$ as $(c_0, c_1, c_2, c_3, \pi)$\\
%if $\nizk.\vrfy((c_0, c_1, c_2,c_3), \pi)= 1$& if $\nizk.\vrfy((c_0, c_1, c_2,c_3), \pi)= 1 \land $\\
%\tab Compute $ \pi_\fdecom:=c_0^{2^T} \bmod N$ &  $c_3 = \pi_\fdecom^N (1+N)^m \bmod N^2$\\
%\tab Compute $m:=\frac{c_3 \cdot \pi_\fdecom^{-N} (\bmod N^2) -1}{N}$ &\tab return 1\\
%\tab return $(m,\pi_\fdecom)$ & return 0\\
%return $\bot$ & \\
%
%%$\underline{\decom(\crs, \sk, c)}$     & $\underline{\fdecom(\crs,c)}$ \\
%%Parse $c$ as $(c_0, c_1, c_2, \pi)$  & Parse $c$ as $(c_0, c_1, c_2, \pi)$ \\
%%if $\nizk.\vrfy((c_0, c_1, c_2), \pi)= 1$  & if $\nizk.\vrfy((c_0, c_1, c_2), \pi)= 1$\\
%%\tab Compute $y_1:= c_0^{k} \bmod N$  &   \tab Compute $ y_2:=c_0^{2^T} \bmod N$ \\
%%\tab return $c_1 \cdot y_1^{-1} \bmod N$ & \tab return $c_2 \cdot y_2^{-1} \bmod N$ \\
%%return $\bot$ & return $\bot$\\
%
%\hline          
%\end{tabular}
%\caption{NY Construction of NITC}
%\label{table:nitc}
%\end{center}
%\end{figure}



\begin{theorem}
If $(\nizk.\gen_\param, \nizk.\gen_L, \nizk.\prove, \nizk.\vrfy)$ is a one-time simulation-sound non-interactive zero-knowledge proof system for $L$, the strong sequential squaring assumption with gap $\gap$ holds relative to $\genmod$, the Decisional Composite Residuosity assumption holds relative to $\genmod$, and the Decisional Diffie-Hellman assumption holds relative to $\genmod$ in $\Jn$, then \mathlist{(\pgen, \com, \cvrfy, \dvrfy, \fdecom)} defined in \Cref{table:nitc} is an IND-CCA-secure non-interactive timed commitment scheme with $\ugap$, for any $\ugap < \gap$. 
\end{theorem}

\begin{proof}
Completeness is implied by the completeness of the NIZK and can be verified by inspection. 

Our construction is based on the Naor-Yung paradigm where we combine three Paillier-type ciphertext with shared randomness.  
%Similarly to \cite{SCN:BiaMasVen16} we define a PPT algorithm $\rerand$ in Figure~\ref{fig:rerand}, which takes two ciphertexts generated with independent randomness, both public keys, only one secret key (in our case $k$) and randomness which was used to encrypt a message using the public key $g, h_1$. 
%%We assume that modulus $N$ is implicitly known.\todo{Why not explicit?} 
%% which corresponds to the secret key which is given as the input. 
%
%\begin{figure}[tb]
%\centering
%\begin{minipage}{0.75\textwidth}
%$\underline{\rerand(c:= (g^{r}, h_1^{r}\cdot m), c':= (g^{r'}, h_2^{r'}\cdot m), N, h_1, h_2, k, r)}:$
%\vspace{-2mm}
%\begin{itemize}
%\item $c_0:= g^{r}\cdot{g^{r'}} = g^{r+r'} \bmod N$;
%\item $c_1:= (g^{r'})^k \cdot h_1^{r}\cdot m  =  h_1^{r'}\cdot h_1^{r}\cdot m = h_1^{r+r'}\cdot m \bmod N$;
%\item $c_2:=h_2^{r} \cdot h_2^{r'}\cdot m = h_2^{r+r'}\cdot m \bmod N$.
%\end{itemize}
%\end{minipage}
%\caption{\label{fig:rerand}Algorithm $\rerand$.}
%\end{figure}
%
%
%
%
%It is straightforward to see that the ciphertext returned by $\rerand$ is perfectly distributed to the ciphertext produced using a shared randomness where the pair $(c_0, c_1)$ encrypts a message $m$ and the pair $(c_0, c_2)$ encrypts message $m'$.


% We note that if we use value $2^T$ as a secret key, then in order to compute $c_2$ we have to execute $T$ repeated squarings, but since $T$ is polynomial in $\secpar$ this computations is considered to be efficient.

\newsequenceofgames{NITC-MH}
To prove security we define a sequence of games $\games_0 - \games_12$.  For $i \in \{0,1,\dots,12\}$ we denote by $\games_i = 1$ the event that the adversary $\adv = \{(\adv_{1,\secpar}, \adv_{2, \secpar})\}_{\secpar \in \nats}$ outputs $b'$ in the game $\games_i$ such that $b = b'$.
In individual games we use the algorithm $\decom$ define in \Cref{fig:deco} to answer decommitment queries efficiently. 
\begin{figure}[h!]
\begin{center}
\begin{tabular}{|l|}
\hline
$\underline{\decom(\crs, c, i)}$\\
Parse $c$ as $(c_0, c_1, c_2,, c_3, \pi)$\\
if $\nizk.\vrfy(\crs_\nizk,(c_0, c_1, c_2, c_3), \pi)= 1$\\
\tab Compute $y:= c_0^{k_i} \bmod N$\\
\tab return $\frac{c_i \cdot y^{-N} (\bmod N^2) -1}{N}$\\
return $\bot$\\
\hline          
\end{tabular}
\caption{Decommitment oracle}
\label{fig:deco}
\end{center}
\end{figure}

\nextgame{G0}
Game $\games_\thisgame$ corresponds to the original security experiment where decommitment queries are answered using $\decom$ with $i:=1$ meaning that secret key $k_1$ and ciphertext $c_1$ are used.

%Let $\gnr$ denote the event that the sampled $g$ in $\kgen$ is a generator of $\qrn$. Recall that $N = pq$ where $p = 2p'+1$ and $q = 2q'+1$. Because $g$ is sampled uniformly at random and $\qrn$ has $\varphi(|\qrn|) = (p'-1)(q'-1)$ generators, this event happens with overwhelming probability. Concretely, $\Pr[\gnr] = 1-\frac{1}{p'}-\frac{1}{q'}+\frac{1}{p'q'}$.
%Therefore the following holds.
%\begin{lemma}\label{tpke3:flem} 
%\begin{align*}
%\Pr[\games_\thisgame = 1] &= \Pr[\games_\thisgame = 1| \gnr]\Pr[\gnr] + \Pr[\games_\thisgame = 1| \overline{\gnr}]\Pr[\overline{\gnr}] \\
%&\leq \Pr[\games_\thisgame = 1| \gnr]\Pr[\gnr] + \Pr[\overline{\gnr}] \\
%&= \Pr[\games_\thisgame = 1| \gnr]\left( 1-\frac{1}{p'}-\frac{1}{q'}+\frac{1}{p'q'} \right) + \frac{1}{p'}+\frac{1}{q'}-\frac{1}{p'q'}.
%\end{align*}
%\end{lemma}




\nextgame{SimulProof}
Game $\games_\thisgame$ proceeds exactly as the previous game but we run the zero-knowledge simulator $(\crs, \tau) \leftarrow \simul_1(\seck, L)$ in $\pgen$ and produce a simulated proof for the challenge commitment as $\pi \leftarrow \simul_2(\crs, \tau, (c_0^*,c_1^*,c_2^*,c_3^*))$. By zero-knowledge security of underlying NIZK we directly obtain
\begin{lemma}\label{nitc-mh:flem}
\[
\left|\Pr[\games_\prevgame = 1] - \Pr[\games_\thisgame = 1]\right| \leq \zk^\nizk_\advB.
\]
\end{lemma}

\nextgame{RndExp}
In $\games_\thisgame$ we sample $r$ uniformly at random from $[\varphi(N)/2]$. 

\begin{lemma}
\[
\left|\Pr[\games_\prevgame = 1] - \Pr[\games_\thisgame = 1]\right| \leq \frac{1}{p}+\frac{1}{q}-\frac{1}{N}.
\]
\end{lemma}
%At first we remark that for upper bounding the difference between the games we use a statistical argument. Because $r$ appears only in the exponent of the group generator, we later sample a random element from the group $\qrn$ which can be done efficiently. 
Since the only difference between the two games is in the set from which we sample $r$, to upper bound the advantage of adversary we can use \Cref{sampling-lemma}, which directly yields required upper bound.

%\nextgame{ReRand}
%In Game $\games_\thisgame$ we produce the challenge commitment by encrypting the challenge message using two independent random exponents $r \rand \smplset, r' \rand [\varphi(N)/4]$ to obtain $c:= (g^{r}, h_1^{r}\cdot m_b), c':= (g^{r'}, h_2^{r'}\cdot m_b)$ and then run $\rerand(c,c',N, \allowbreak h_1, h_2,k,r)$ to obtain resulting ciphertext $(c_0^*, c_1^*, c_2^*)$. Since $r'$ is sampled uniformly at random from $[\varphi(N)/4]$ the ciphertext distributions in both games  are the same. Therefore 
%
%\begin{lemma}
%\[
%\Pr[\games_\prevgame = 1] = \Pr[\games_\thisgame = 1].
%\]
%\end{lemma}


\nextgame{SSSA}
In $\games_\thisgame$ we sample $y_3 \rand \Jn$ and compute $c_3^*$ as $y_3 m_b$.

Let $\tilT_\sss(\secpar)$ be the polynomial whose existence is guaranteed by the SSS assumption.
Let $\poly_\advB(\secpar)$ be the fixed polynomial which bounds the time required to execute Steps 1--2 and answer decommitment queries in Step 3 of the adversary $\advB_{2, \secpar}$ defined below. Set $\undT := (\poly_\advB(\secpar))^{1 / \ugap}$.  Set $\tilT_\nitc := \max(\tilT_\sss, \undT)$.
\begin{lemma}
From any polynomial-size adversary $\adv = \{(\adv_{1,\secpar}, \adv_{2, \secpar})\}_{\secpar \in \nats}$, where depth of $\adv_{2, \secpar}$ is at most $T^{\ugap}(\secpar)$ for some $T(\cdot) \geq \undT(\cdot)$ we can construct a polynomial-size adversary $\advB = \{(\advB_{1,\secpar}, \advB_{2, \secpar})\}_{\secpar \in \nats}$ where the depth of $\advB_{2, \secpar}$ is at most $T^{\gap}(\secpar)$ with
\[
\left|\Pr[\games_\prevgame = 1] - \Pr[\games_\thisgame = 1]\right| \leq \advtg_\advB^\sss.
\]
\end{lemma}

The adversary $\advB_{1,\secpar}(N, T(\secpar), g):$
\vspace{-2mm}
\begin{enumerate}
\item Samples $k_1, k_2 \rand \smplset$, computes $h_1 := g^{k_1} \bmod N, h_2 := g^{k_2} \bmod N,  h_3 := g^{2^{T(\secpar)}} \bmod N$, runs $(\crs_\nizk, \tau) \leftarrow \nizk.\simul_1(\seck, L)$ and sets $\crs:=\mathlist(N, T(\secpar), g, h_1, h_2, h_3, \crs_\nizk)$. Notice that value $h_3$ is computed by repeated squaring.
\item Runs $(m_0, m_1, \st) \leftarrow \adv_{1, \secpar}(\crs)$ and answers decommitment queries using $k_1$.
\item Outputs $(N,g,k_1, k_2, h_1,h_2,h_3,\crs_\nizk, \tau, m_0, m_1, \st)$
\end{enumerate}

The adversary $\advB_{2,\secpar}(x,y,(N,g,k_1, k_2, h_1,h_2,h_3, \crs_\nizk, \tau, m_0, m_1, \st)):$
\vspace{-2mm}
\begin{enumerate}
\item Samples $b \rand \bits$, computes $c_0^*:=x, c_1^*:=x^{k_1} m_b, c_2^*:=x^{k_2}m_b, c_3^*:=y m_b$.
\item Runs $\pi^* \leftarrow \simul(\crs_\nizk, \tau, (c_0^*, c_1^*, c_2^*, c_3^*))$.
\item Runs $b' \leftarrow \adv_{2, \secpar}((c_0^*, c_1^*, c_2^*, c_3^*), \pi^*), \st)$ and answers decommitment queries using $k_1$.
\item Returns the truth value of $b=b'$.
\end{enumerate}
Since $g$ is a generator of $\Jn$ and $x$ is sampled uniformly at random from $\Jn$ there exists some $r \in [\varphi(N)/2]$ such that $x = g^{r}$. Therefore when $y = x^{2^T} = (g^{2^T})^{r} \bmod N$, then $\advB$ simulates $\games_\prevgame$ perfectly. Otherwise $y$ is random value and $\advB$ simulates $\games_\thisgame$ perfectly. We remark that at this point $c_3^*$ does not reveal any information about $m_b$.

Now we analyse the running time of the constructed adversary. Adversary $\advB_1$ computes $h_3$ by $T(\secpar)$ consecutive squarings and because $T(\secpar)$ is polynomial in $\secpar$, $\advB_1$ is efficient. Moreover, $\advB_2$ fulfils the depth constraint:
\[ \dep(\advB_{2,\secpar}) = \poly_\advB(\secpar) + \dep(\adv_{2,\secpar}) \leq \undT^{\ugap}(\secpar) + T^{\ugap}(\secpar) \leq 2 T^{\ugap}(\secpar) = o(T^{\gap}(\secpar)). \] 

Also $T(\cdot) \geq \tilT_\nitc(\cdot) \geq \tilT_\sss(\cdot)$ as required.  

%\nextgame{RndExp2}
%In $\games_\thisgame$ we stop to use $\rerand$ algorithm. Concretely, we sample $r \rand [\varphi(N)/4], y_2 \rand \qrn$ and compute challenge ciphertext as $c^*:=(g^r, h_1^r \cdot m_b, y_2 \cdot m_b)$. The ciphertext has the same distribution as in the previous game. Therefore 
%
%\begin{lemma}
%\[
%\Pr[\games_\prevgame = 1] = \Pr[\games_\thisgame = 1].
%\]
%\end{lemma}
%\nextgame{RndExp3}


\nextgame{RndExp2}
In $\games_\thisgame$ we sample $k_2$ uniformly at random from $[\varphi(N)/2]$. 

\begin{lemma}
\[
\left|\Pr[\games_\prevgame = 1] - \Pr[\games_\thisgame = 1]\right| \leq \frac{1}{p}+\frac{1}{q}-\frac{1}{N}.
\]
\end{lemma}

Again using a statistical argument this lemma directly follows from \Cref{sampling-lemma}.

\nextgame{DDH}
In $\games_\thisgame$ we sample $y_2 \rand \Jn$ and compute $c_2^*$ as  $y_2 m_b$. 

\begin{lemma}\label{lem-mh:ddh}
\[
\left|\Pr[\games_\prevgame = 1] - \Pr[\games_\thisgame = 1]\right| \leq \advtg_\advB^\ddh.
\]
\end{lemma}
We construct an adversary $\advB = \{\advB_\secpar\}_{\secpar \in \N}$ against DDH in the group $\Jn$. %Given \Cref{thm:ddh} this implies an adversary against DDH in large prime-order subgroups of $\Zn^*$.

$\advB_{\secpar}(N,g,g^\alpha, g^\beta, g^\gamma):$
\vspace{-2mm}
\begin{enumerate}
\item Samples $k_1 \rand \smplset$, computes $h_1 := g^{k_1} \bmod N,  h_3 := g^{2^{T}} \bmod N$, runs $(\crs_\nizk, \tau) \leftarrow \nizk.\simul_1(\seck, L)$ and sets $\crs:=\mathlist(N, T(\secpar), g, h_1, h_2: = g^\alpha, h_3, \crs_\nizk)$.
\item Runs $(m_0, m_1, \st) \leftarrow \adv_{1, \secpar}(\crs)$ and answers decommitment queries using $k_1$.
\item Samples $b \rand \bits, y_3 \rand \Jn$ and computes $(c_0^*, c_1^*, c_2^*, c_3^*):=(g^\beta, (g^\beta)^{k_1} m_b, (g^{\gamma}) m_b, y_3 m_b).$ Runs $\pi^* \leftarrow \simul(1, \st', (h_1, h_2, c_0^*, c_1^*, c_2^*, c_3^*))$.
\item Runs $b' \leftarrow \adv_{2, \secpar}((c_0^*, c_1^*, c_2^*,c_3^*), \pi^*, \st)$ and answers decommitment queries using $k_1$.
\item Returns the truth value of $b=b'$.
\end{enumerate}
If $\gamma = \alpha\beta$, then $\advB$ simulates $\games_\prevgame$ perfectly. Otherwise $g^\gamma$ is uniform random element in $\Jn$ and $\advB$ simulates $\games_\thisgame$ perfectly. This proofs the lemma. We remark that at this point $c_2^*$ does not reveal any information about $m_b$.

\nextgame{RndExp3}
In $\games_\thisgame$ we sample $k_2$ uniformly at random from $\smplset$. 

\begin{lemma}
\[
\left|\Pr[\games_\prevgame = 1] - \Pr[\games_\thisgame = 1]\right| \leq \frac{1}{p}+\frac{1}{q}-\frac{1}{N}.
\]
\end{lemma}

This lemma directly follows from \Cref{sampling-lemma}.

\nextgame{SimSnd}

In $\games_\thisgame$ we answer decommitment queries using $\dec$ with $i:=2$ which means that secret key $k_2$ and ciphertext $c_2$ are used. 

\begin{lemma}
\[
\left|\Pr[\games_\prevgame = 1] - \Pr[\games_\thisgame = 1]\right| \leq \simsnd^\nizk_\advB. 
\]
\end{lemma}

Let $\event$ denote the event that adversary $\adv$ asks a decommitment query $c$ such that its decommitment using the key $k_1$ is different from its decommitment using the key $k_2$. Since $\games_\prevgame$ and $\games_\thisgame$ are identical until $\event$ does not happen, by the standard argument it is sufficient to upper bound the probability of happening $\event$. Concretely,  

\[
\left|\Pr[\games_\prevgame = 1] - \Pr[\games_\thisgame = 1]\right| \leq \Pr[\event]. 
\]

We construct an adversary $\advB$ which breaks one-time simulation soundness of the NIZK. 

The adversary $\advB_{\secpar}^{\simul_1, \simul_2}:$
\vspace{-2mm}
\begin{enumerate}
\item Computes $\crs \leftarrow \pgen(\seck, T)$ as defined in the construction where the value $h_3$ is computed using repeated squaring instead.
\item Runs $(m_0, m_1, \st) \leftarrow \adv_{1, \secpar}(\crs)$ and answers decommitment queries using $k_2$.
\item Samples $b \rand \bits, x, y_2, y_3 \rand \Jn$ and computes $(c_0^*, c_1^*, c_2^*, c_3^*):=(x, x^{k_1} m_b,\allowbreak y_2 m_b, y_3 m_b)$. Forwards $(h_1, h_2, c_0^*, c_1^*, c_2^*, c_3^*)$ to simulation oracle $\simul_1$ and obtains a proof $\pi^*$.
\item Runs $b' \leftarrow \adv_{2, \secpar}((c_0^*, c_1^*, c_2^*, c_3^*), \pi^*, \st)$ and answers decryption queries using $k_2$.
\item Find a decommitment query $c: = (c_0, c_1, c_2, c_3, \pi)$ such that $\dec(\crs, c, 1) \neq \dec(\crs,c,2)$ and returns $((h_1, h_2, c_0, c_1, c_2, c_3), \pi)$
\end{enumerate}

$\advB$ simulates $\games_\thisgame$ perfectly and if the event $\event$ happens, it outputs a valid proof for a statement which is not in the specified language $L$. Therefore
\[\Pr[\event] \leq \simsnd^\nizk_\advB,\]
which concludes the proof of the lemma.  

%\nextgame{ReRand2}
%In $\games_\thisgame$ we use the key $t$ and randomness $r'$ as input for rerandomization. Concretely we compute $\rerand(c,c',h_1,h_2,t,r')$. This is just conceptual change since the ciphertext distributions are the same in both games and therefore 
%
%\begin{lemma}
%\[
%\left|\Pr[\games_\prevgame = 1] = \Pr[\games_\thisgame = 1]\right|.
%\]
%\end{lemma}
%
%\nextgame{RndExp4}
%In $\games_\thisgame$ we sample $r$ uniformly at random from $\varphi(N)$. 
%
%\begin{lemma}
%\[
%\left|\Pr[\games_\prevgame = 1] - \Pr[\games_\thisgame = 1]\right| \leq \frac{1}{N}.
%\]
%\end{lemma}

\nextgame{RndExp4}
In $\games_\thisgame$ we sample $k_1$ uniformly at random from $[\varphi(N)/4]$. 

\begin{lemma}
\[
\left|\Pr[\games_\prevgame = 1] - \Pr[\games_\thisgame = 1]\right| \leq \frac{1}{p}+\frac{1}{q}-\frac{1}{N}.
\]
\end{lemma}

This lemma directly follows from \Cref{sampling-lemma}.

\nextgame{DDH2}
In $\games_\thisgame$ we sample $y_1 \rand \Jn$ and compute $c_1^*$ as  $y_1 m_b$. 

\begin{lemma}
\[
\left|\Pr[\games_\prevgame = 1] - \Pr[\games_\thisgame = 1]\right| \leq \advtg_\advB^\ddh.
\]
\end{lemma}
This can be proven in similar way as \Cref{lem-mh:ddh}. We remark that at this point $c_1^*$ does not reveal any information about $m_b$.

\begin{lemma}\label{nitc-mh:llem}
\[
\Pr[\games_\thisgame = 1] = \half.
\]
\end{lemma}

Clearly, $c_0^*$ is uniform random element in $\Jn$ and hence it does not contain any information about the challenge message. Since $y_1, y_2, y_3$ are sampled uniformly at random from $\Jn$ the ciphertexts $c_1^*, c_2^*, c_3^*$ are also uniform random elements in $\Jn$ and hence do not contain any information about the challenge message $m_b$. Therefore, an adversary can not do better than guessing.

By combining Lemmas \ref{nitc-mh:flem} - \ref{nitc-mh:llem} we obtain the following:
\begin{align*}
&\advtg^{\nitc}_{\adv} = \left| \Pr[\games_0 = 1] - \half \right| \leq \sum_{i=0}^8 \left|\Pr[\games_i = 1] - \Pr[\games_{i+1} = 1] \right| + \left|\Pr[\games_{9}- \half\right| \\
 &\leq \zk^\nizk_\advB + \advtg^{\sss}_{\advB} + \simsnd^{\nizk}_{\advB} + 2 \advtg^{\ddh}_{\advB} + 4 \left( \frac{1}{p}+\frac{1}{q}-\frac{1}{N} \right),
\end{align*}
which concludes the proof.
\end{proof}

\begin{theorem}
$(\pgen, \com, \cvrfy, \dvrfy, \fdecom)$ defined in \Cref{table:nitc} is a BND-CCA-secure non-interactive timed commitment scheme. 
\end{theorem}

\begin{proof}
We show that the construction is actually perfectly binding. This is straightforward to show since Paillier encryption is perfectly binding. Therefore there is exactly one message/randomness pair $(m,r)$ which can pass the check in $\dvrfy$. Therefore the first winning condition of BND-CCA experiment happens with probability 0. Moreover, since $\pgen$ is executed by the challenger, the value $h_3$ is computed correctly and therefore $\fdecom$ reconstructs always the correct message $m$. Therefore the second winning condition of BND-CCA experiment happens with probability 0 as well.
\end{proof}

\begin{theorem}
If $\nizk = (\nizk.\prove, \nizk.\vrfy)$ is a non-interactive zero-knowledge proof system for $L$, then \mathlist{(\pgen, \com, \cvrfy, \dvrfy, \fdecom, \fdvrfy)} defined in \Cref{table:nitc} is a publicly verifiable non-interactive timed commitment scheme.
\end{theorem}

\begin{proof}
Completeness is straightforward to verify. 

To prove the soundness notice that if commitment verifies, then we know that $c_0 = g^r$ and $c_3 = h_3^r(1+N)^m$ for honestly generated $g$ and $h_3$ and some $r$ and $m$. Otherwise, an adversary would be able to break soundness of the proof system. Since there is an isomorphism $f:\Zn^* \times \Zn \rightarrow\Zns$ given by $f(a,b)=a^N(1+N)^b \bmod N^2$ (see e.g. \cite[Proposition 13.6]{books/crc/KatzLindell2014}) there exist unique values $\pi_\fdecom$ and $m$ such that $c_3=\pi_\fdecom^N(1+N)^m \bmod N^2$ and therefore adversary is not able to provide different message $m'$ fulfilling required equation.

Finally, the running time of $\fdvrfy$ is efficient, since it is independent of $T$.
\end{proof}

 

%\begin{figure}[h!]
%\begin{center}
%\begin{tabular}{|ll|}
%\hline
%$\underline{\kgen(\seck, T)}$ 							   & $\underline{\decf(\sk, c)}$\\
%$(p, q_, N) \leftarrow \genmod(\seck)$ &  Parse $c$ as $(c_0, c_1, c_2, \pi)$\\
%$\varphi(N):= (p-1)(q-1)$   & if $\nizk.\vrfy((c_0, c_1, c_2), \pi)= 1$ \\
%$g\rand \qrn$ & \tab Compute $y_1:= c_0^{k} \bmod N$\\
%$k \rand \varphi(N)/4$ & \tab return $c_1 \cdot H(y_1)^{-1} \bmod N$ \\
%$t:= 2^T \bmod \varphi(N)/4$ & return $\bot$\\
%$h_1:= g^k \bmod N$ & \\
%$h_2:=g^{t} \bmod N$ & \\
%%$\crs \leftarrow \nizk.\setup(\seck)$ & \\
%$\pk:= (N,T,g,h_1,h_2), \sk:= (N, k)$ & \\
%return $(\pk, \sk)$       & \\
%                                             &\\
%$\underline{\enc(\pk, m)}$           & $\underline{\decs(\pk,c)}$ \\
%$r \rand [\floor{N/4}]$     & Parse $c$ as $(c_0, c_1, c_2, \pi)$ \\
%Compute $c_0:= g^r \bmod N$ & if $\nizk.\vrfy((c_0, c_1, c_2), \pi)= 1$\\
%For $i \in [2]: y_i:= h_i^r \bmod N$   &   \tab Compute $ y_2:=c_0^{2^T} \bmod N$ \\
%For $i \in [2]: c_i:= H(y_i) \cdot m \bmod N$ & \tab return $c_2 \cdot H(y_2)^{-1} \bmod N$ \\
%$\Phi := (c_0, c_1, c_2), w := (m, r)$& return $\bot$\\
%$\pi \leftarrow \nizk.\prove(\Phi, w)$  &  \\
%return $c \leftarrow (c_0, c_1, c_2, \pi)$ &  \\
%
%\hline          
%\end{tabular}
%\caption{NY Construction of TPKE from SSSA}
%\label{table:tpke-elgamal}
%\end{center}
%\end{figure}




%%% Local Variables:
%%% mode: latex
%%% TeX-master: "main"
%%% End:


We defer the constructions of non-malleable non-interactive timed commitments in the random oracle model to \Cref{sec:rom-const}.


%%% Local Variables:
%%% mode: latex
%%% TeX-master: "main"
%%% End:

%!TEX root=main.tex
\section{Random Oracle Model Constructions}
We are able to obtain more efficient construction of non-malleable non-interactive timed commitments when we instantiate the non-interactive zero-knowledge proof systems in the random oracle model \cite{CCS:BelRog93}. Since the underlying languages and proofs differ in some subtle but crucial ways from the standard model constructions, we provide the constructions together with proofs in full detail.
%!TEX root=main.tex
\subsection{Non-Interactive Zero-Knowledge Proofs in the Random Oracle Model}
%Since we can build a very efficient NIZK proof system for our constructions in the random oracle model (ROM), we recall the definition of NIZKs in the ROM.  

\begin{definition} 
A \emph{non-interactive proof system} for an NP language $L$ with relation $\rel$ is a pair of algorithms $(\prove, \vrfy)$, which work as follows:
\begin{itemize}
\item $\pi \leftarrow \prove(s,w)$ is a PPT algorithm which takes as input a statement $s$ and a witness $w$ such that $(s,w) \in \rel$ and outputs a proof $\pi$.
\item $\vrfy(s, \pi) \in \{0,1\}$ is a deterministic algorithm which takes as input a statement $s$ and a proof $\pi$ and outputs either 1 or 0, where 1 means that the proof is ``accepted'' and 0 means it is ``rejected''.
\end{itemize}
We say that a non-interactive proof system is \emph{complete}, if for all $(s, w) \in \rel$ holds:
\[\Pr[\vrfy(s,\pi)=1:\pi \leftarrow \prove(s,w)] =1.\] 
\end{definition}

Next we define the \emph{zero-knowledge} property for non-interactive proof system in the random oracle model. The simulator $\simul$ of a non-interactive zero-knowledge proof system is modelled as a stateful algorithm which provides two modes, namely $(\pi, \st) \leftarrow \simul(1, \st, s)$  for answering proof queries and $(v, \st) \leftarrow \simul(2, \st, u)$ for answering random oracle queries. The common state $\st$ is updated after each operation.



\begin{definition}[Zero-Knowledge in the ROM]
Let $(\prove, \allowbreak \vrfy)$ be a non-interactive proof system for a relation $\rel$ which may make use of a hash function $H : \hdom \rightarrow \himg$. Let $\funs$ be the set of all functions from the set $\hdom$ to the set $\himg$. We say that $(\prove, \vrfy)$ is \emph{non-interactive zero-knowledge proof in the random oracle model (NIZK)}, if there exists an efficient simulator $\simul$ such that for all non-uniform polynomial-size adversaries $\adv = \{\adv_\secpar\}_{\secpar \in \nats}$ there exists a negligible function $\negl(\cdot)$ such that for all $\secpar \in \nats$ 
\[\zk_\adv^\nizk = 
\left| \Pr\left[ \adv_\secpar^{\prove^H(\cdot, \cdot), H(\cdot)} = 1 \right] -  \Pr\left[\adv_\secpar^{\simul_1(\cdot, \cdot), \simul_2(\cdot)} \right] = 1 \right|
\leq \negl(\secpar),
\]
where 
\begin{itemize}
\item $H$ is a function sampled uniformly at random from $\funs$,
\item $\prove^H$ corresponds to the $\prove$ algorithm, having oracle access to $H$,
\item $\pi \leftarrow \simul_1(s, w)$ takes as input $(s, w) \in \rel$, and outputs the first output of $(\pi, \st) \leftarrow \simul(1, \st, s)$,
\item $v \leftarrow \simul_2(u)$ takes as input $u \in \hdom$ and outputs the first output of $(v, \st) \leftarrow \simul(2, \st, u)$.
\end{itemize}
\end{definition}

\begin{definition}[One-Time Simulation Soundness]
Let $(\prove, \vrfy)$ be a non-interactive proof system for an NP language $L$ with zero-knowledge simulator $\simul$. We say that $(\prove, \vrfy)$ is \emph{one-time simulation sound} in the random oracle model, if for all non-uniform polynomial-size adversaries $\adv = \{\adv_\secpar\}_{\secpar \in \nats}$ there exists a negligible function $\negl(\cdot)$ such that for all $\secpar \in \nats$ 
\[
\simsnd_\adv^\nizk = \Pr\left[
\begin{aligned}
s \notin L \land (s, \pi) \neq (s', \pi') \\
\land \vrfy^{\simul_2(\cdot)}(s, \pi) = 1
\end{aligned}
:(s, \pi) \leftarrow \adv_\secpar^{\simul_1(\cdot), \simul_2(\cdot)} \right] \leq \negl(\secpar),
\]
where $\simul_1(\cdot)$ is a single query oracle which on input $s'$ returns the first output of $(\pi', \st) \leftarrow \simul(1, \st, s')$ and $\simul_2(u)$ returns the first output of $(v, \st) \leftarrow \simul(2, \st, u)$.
\end{definition}

\subsection{Efficient Instantiation of SS-NIZK in the ROM}\label{sec:nizk} 
In this section we provide efficient simulation sound NIZK proof systems in the ROM for languages $L_3$ and $L_4$ that are used in our constructions. The languages are defined in the following way:
\[
L_3 = \left\{(h_1, h_2, c_0, c_1, c_2)| \exists (m,r):
\begin{aligned}
       (\land_{i=1}^3 c_i = h_i^{rN}(1+N)^m \bmod N^2) \land \\
       c_0 = g^r \bmod N\\
    \end{aligned}
    \right\} \text{ and }
\]
\[
L_4 = \left\{(h_1, h_2, c_0, c_1, c_2)| \exists (m,r):
\begin{aligned}
       (\land_{i=1}^3 c_i = h_i^{r}m \bmod N) \land
       c_0 = g^r \bmod N\\
    \end{aligned}
    \right\}, 
\]
where $g, h_3, N$ are parameters defining the language. For the language $L_3$ this is equivalent to proving that $(c_0^N, (h_1\cdot (h_2)^{-1})^N , (c_1\cdot (c_2)^{-1}))$ and $(c_0^N, (h_3\cdot (h_2)^{-1})^N, (c_3\cdot (c_2)^{-1}))$ are two DDH tuples where all computations are done $\bmod N^2$. Similarly, for the language $L_4$ this is equivalent to proving that $(c_0, (h_1\cdot (h_2)^{-1}), (c_1\cdot (c_2)^{-1}))$ and $(c_0, (h_3\cdot (h_2)^{-1}), (c_3\cdot (c_2)^{-1}))$ are two DDH tuples where all computations are done $\bmod N$. Therefore, we can instantiate the required NIZKs using Sigma protocols that prove that given tuples are DDH tuples. These Sigma protocols can be turned into efficient simulation-sound NIZKs in the ROM using the Fiat-Shamir transformation \cite{C:FiaSha86}. We remark that we have to design a Sigma protocols in a hidden order group setting, since the order of the group $\Jn$ is known neither by the prover, nor by the verifier. Current constructions of Sigma protocols in hidden order groups are far less efficient than standard Sigma protocols. However, we observe that to obtain simulation-sound NIZKs, it is sufficient to design a Sigma protocol which has a negligible soundness error and we do not have to care about special soundness, and we are able to avoid the strong RSA assumption which is often needed in Sigma protocols in hidden order groups to prove special soundness. Therefore we are able to avoid both using an unnecessary large modulus $N$ and a large number of sequential repetitions as discussed in \cite{SPEED:BKSST}. 
%Hence it is possible to sample individual messages of our Sigma protocol  from the sets which are smaller than the sets typically used in Sigma protocols in hidden order groups where one is interested in special soundness. 
This results in particularly short proofs. 
%   Essentially, we use the Sigma protocol described in \cite{SCN:BiaMasVen16} which can be turn i Since the order of the group $\qrn$ is not known, we provide detailed proof that the constructed Sigma protocol is indeed complete, special sound, special honest verifier zero-knowledge and has quasi-unique responses.  

We recall at first definition of a Sigma protocol. 

\begin{definition} A Sigma protocol for an NP language $L$ and a corresponding binary relation $\rel$ is an interactive 3-move protocol $\Sigma$ between two interactive algorithms which we will call $\prv$ and $\vrf$. $\prv$ is given input a statement and a witness $(s, w)$ and the $\vrf$ is given as input a statement $s$. The protocol works as follows:
\begin{enumerate}
\item $\prv$ starts the protocol by computing a message $\fm$, called the \emph{commitment}, and sends $\fm$ to $\vrf$. \emph{commitment}, and a state $\st$. The message $\fm$ is sent to the $\vrf$.
\item $\vrf$ receives $\fm$ and samples a \emph{challenge} $\sm$ uniformly from a finite challenge space $\scset$ and sends it to the $\prv$.
\item $\prv$ receives the challenge $c$ end computes a \emph{response} $\tm \in \thset$. It sends the response $\tm$ to $\vrf$.
\item $\vrf$ runs a deterministic function $\vr(s,\fm,\sm,\tm)$ which outputs 0 or 1 meaning reject or accept, respectively.
\end{enumerate}
A triple $(\fm,\sm,\tm)$ is called a \emph{transcript} of the Sigma protocol. A transcript $(\fm,\sm,\tm)$ is called an \emph{accepting} transcript for $s$ if it cause $\vr(s,\fm,\sm,\tm) =1$. 
We say that a $\Sigma$-protocol is \emph{complete} if for all $(s, w) \in \rel$ it holds that whenever a $\prv(s,w)$ and a $\vrf(s)$ interact, then the $\vrf$ always accepts. 
\end{definition}


%\begin{definition}[Special Honest Verifier Zero-Knowledge]
%We say that a Sigma protocol for relation $\rel \subseteq \sset \times \wset$ with challenge space $\scset$ is \emph{special honest verifier zero-knowledge (SHVZK)}, if there exists a PPT simulator $\simul$ which takes as input $(s, \sm) \in \sset \times \scset$ such that:
%\begin{enumerate}
%\item  For all $s \in \sset, \sm \in \scset$ it holds:
%\[\Pr[\vr(s,\fm,\sm,\tm)=1: (\fm,\tm) \leftarrow \simul(s,\sm)] = 1.\]
%\item For all $(s, w) \in \rel$, if we compute 
%\[(\fm,\tm) \leftarrow \simul(s,\sm)\]
%with uniformly random $\sm \rand \scset$, then $(\fm,\sm,\tm)$ has the same distribution as a transcript of a conversation between the $\prv(s,w)$ and the $\vrf(s)$.
%\end{enumerate}
%\end{definition}

\begin{definition}[Honest Verifier Zero-Knowledge]
We say that a Sigma protocol for an NP language  $L$ is \emph{honest verifier zero-knowledge (HVZK)}, if there exists a PPT simulator $\simul$ which takes as input $s \in L$ and outputs transcript $(\fm,\sm,\tm)$ that is computationally indistinguishable from honest
transcripts resulting from interactions between $\prv$ and $\vrf$ on common input $s$.
\end{definition}

\begin{definition}[Soundness]
We say that a Sigma protocol for an NP language $L$ is \emph{sound}, if a proof for a statement $x \notin L$ output by any (even unbounded) is accepted only with negligible probability.
\end{definition}

\begin{definition}[Quasi Unique Responses]
Let $\Sigma$ be a Sigma protocol. We say that $\Sigma$ has \emph{quasi unique responses} if for every non-uniform polynomial-size adversary $\adv = \{\adv_\secpar\}_{\secpar \in \nats}$ there exists a negligible function $\negl(\cdot)$ such that for all $\secpar \in \nats$ 
\[\Pr\left[ \vr(s,\fm,\sm,\tm) = \vr(s,\fm,\sm,\tm' ) = 1 \land \tm \neq \tm' : (s, \fm, \sm , \tm, \tm')\leftarrow \adv_\secpar \right] \leq \negl(\secpar).\]
\end{definition}

Faust \etal \cite{INDOCRYPT:FKMV12} have shown that Sigma protocols fulfilling the above definitions can be turned into simulation-sound NIZK via the Fiat-Shamir transform. 

\begin{theorem}[\cite{INDOCRYPT:FKMV12}]\label{thm:fs}
Consider a non-trivial three-round public-coin honest verifier zero-knowledge interactive proof system $(\prv, \vrf)$ for an NP language $L$, with quasi unique responses. In the random oracle model, the proof system $(\prove, \vrfy)$ derived from $(\prv, \vrf)$ via the Fiat-Shamir transform is a simulation-sound NIZK with respect to its canonical simulator $\simul$.
\end{theorem}

Now we are ready to describe our Sigma protocol $\Sigma = (\prv, \vrf)$ for language $L_3$. At first, we recall that $f:\Zn^* \times \Zn \rightarrow \Zns$ defined as $f(x,y)=x^N(1+N)^y$ is an isomorphism. Since $g$ is generator of $\Jn$ and has order $\ord$, also $g^N \bmod N^2$ has order $\ord$. 
\begin{enumerate}
\item $\prv$ samples $\alpha \rand \smplset$, computes $a_0:=g^{N\alpha}, a_1:= (h_1\cdot h_2^{-1})^{N\alpha}, a_2:= (h_3\cdot h_2^{-1})^{N\alpha} \bmod N^2$ and sends $(a_0, a_1, a_2)$ to $\vrf$.
\item $\vrf$ samples $v \rand [2^d]$ and sends $v$ to $\prv$.
\item $\prv$ computes the response $z:= \alpha + v \cdot r$ and sends it to $\vrf$.
\item $\vrf$ accepts if and only if $g^{Nz} = a_0 \cdot c_0^{Nv} \land (h_1\cdot h_2^{-1})^{Nz} = a_1 \cdot (c_1\cdot c_2^{-1})^v  \land (h_3\cdot h_2^{-1})^{Nz} = a_2 \cdot (c_3\cdot c_2^{-1})^v$ where all computation are done $\bmod N^2$ and $z \in [\estord + v\estord]$.
\end{enumerate}

\begin{theorem}
\label{thm:SigmaROMFactoring}
If the factoring assumption holds relative $\genmod$, then the above defined protocol $\Sigma = (\prv, \vrf)$ is a Sigma protocol for language $L$ that is perfectly complete, honest verifier zero-knowledge, sound and has quasi-unique responses. 
\end{theorem}

\begin{proof} 
Completeness is straightforward to verify:
\begin{enumerate}
\item $g^{Nz} = g^{N(\alpha + v \cdot r)} = g^{N\alpha} + (g^r)^{Nv} = a_0 \cdot c_0^{Nv} \bmod N^2$;
\item $(h_1\cdot h_2^{-1})^{Nz} = (h_1\cdot h_2^{-1})^{N(\alpha + v \cdot r)} = (h_1\cdot h_2^{-1})^{N\alpha} \cdot (h_1^{Nr}\cdot (h_2^{Nr})^{-1})^v = a_1 \cdot (h_1^{Nr} \cdot m \cdot (h_2^{Nr} \cdot m)^{-1})^v = a_1 \cdot (c_1\cdot c_2^{-1})^v \bmod N^2$;
\item $(h_3\cdot h_2^{-1})^{Nz} = (h_3\cdot h_2^{-1})^{N(\alpha + v \cdot r)} = (h_3\cdot h_2^{-1})^{N\alpha} \cdot (h_3^{Nr}\cdot (h_2^{Nr})^{-1})^v = a_1 \cdot (h_3^{Nr} \cdot m \cdot (h_2^{Nr} \cdot m)^{-1})^v = a_2 \cdot (c_3\cdot c_2^{-1})^v \bmod N^2$;
\item Since $r, \alpha \in \smplset$ and $v \in [2^d]$, therefore $z:= \alpha + v \cdot r \in [\estord + v\estord]$. 
\end{enumerate}

\paragraph{HVZK.} The simulator $\simul((h_1, h_2, c_0, c_1, c_2))$ works as follows:
\begin{enumerate}
\item Samples $v \rand [2^d], z \rand [\estord + v\estord]$.
\item Computes $a_0:= g^{Nz} \cdot c_0^{-v}, a_1:= (h_1 \cdot h_2^{-1})^{Nz} \cdot (c_1 \cdot c_2^{-1})^{-v}, a_2:= (h_3 \cdot h_2^{-1})^{Nz} \cdot (c_3 \cdot c_2^{-1})^{-v} \bmod N^2$ and returns $((a_0, a_1, a_2), v, z)$.
\end{enumerate}
It is easy to verify that the produced transcript is valid and has the same distribution as an honest transcript. 

%\paragraph{SHVZK.} The simulator $\simul((c_0, c_1, c_2), v)$ works as follows:
%\begin{enumerate}
%\item Samples $z \rand [\estord + v\estord]$.
%\item Computes $a_0:= g^z \cdot c_0^{-v}, a_1:= (h_1 \cdot h_2^{-1})^z \cdot (c_1 \cdot c_2^{-1})^{-v}$ and returns $(a_0, a_1), z$.
%\end{enumerate}
%It is easy to verify that produced transcript is accepting and moreover if $v$ is sampled uniformly at random from $[2^d]$, then the given transcript has the same distribution as a honest transcript. 

\paragraph{Soundness.} Let $((a_0, a_1, a_2), v, z)$ is accepting transcript for $(h_1, h_2, c_0, c_1, c_2) \notin L_3$. That means $c_0 = g^{Nr_0}, (c_1\cdot (c_2)^{-1}) = (h_1\cdot (h_2)^{-1})^{Nr_1} \bmod N, (c_3\cdot (c_2)^{-1}) = (h_3\cdot (h_2)^{-1})^{Nr_2} \bmod N$ and $r_0 \neq r_1 \mod \varphi(N)/2$ or $r_0 \neq r_2 \mod \varphi(N)/2$. Let $a_0 = g^{N\alpha_0} \bmod N^2, a_1 = (h_1\cdot (h_2)^{-1})^{N\alpha_1} \bmod N^2, a_2 = (h_3\cdot (h_2)^{-1})^{N\alpha_2} \bmod N^2$. Considering the first verification equation, taking discrete logarithms to base $g^N$, the second verification equation with discrete logarithms to base $(h_1 \cdot h_2^{-1})^N$, and the third verification equation with discrete logarithms to base $(h_3 \cdot h_2^{-1})^N$, we obtain following equations:
\begin{align}
z = \alpha_0 + r_0 v \bmod \ord\\
z = \alpha_1 + r_1 v \bmod \ord \\
z = \alpha_2 + r_2 v \bmod \ord
\end{align}
Computing (1)-(2) we obtain
\begin{align*}
0 = (\alpha_0 - \alpha_1)+ (r_0-r_1)v \bmod \ord \\
v = \frac{\alpha_1 - \alpha_0}{r_0-r_1} \bmod \ord
\end{align*}
Computing (1)-(3) we obtain
\begin{align*}
0 = (\alpha_0 - \alpha_2)+ (r_0-r_2)v \bmod \ord \\
v = \frac{\alpha_2 - \alpha_0}{r_0-r_2} \bmod \ord
\end{align*}
Now notice that at least one of $(r_0 - r_1)$ and $(r_0 - r_2)$ is not zero and values $\alpha_0, \alpha_1, \alpha_1, r_0, r_1, r_2$ are fixed before challenge $v$ is provided by $\vrf$. Therefore, in order to provide an accepting proof, the adversary has to predict value $v$. If we choose $2^d < \ord$, then $v$ is unique and therefore the probability that the adversary produces an accepting transcript for a statement that is not in $L_3$ is at most $2^{-d}$, which can be set to be negligible in $\secpar$. This holds even for unbounded adversaries. 

%Let $((a_0, a_1), v, z), ((a_0, a_1), v', z')$ are two accepting transcripts such that $v \neq v'$. Then $g^z = a_0 \cdot g^{rv} \bmod N$ and $g^{z'} = a_0 \cdot g^{rv'} \bmod N$. Hence $r = (z-z') \cdot (v-v')^{-1}$ a since $v \neq v'$


\paragraph{Quasi Unique Responses.} We show that our Sigma protocol has quasi unique responses, otherwise we are able to factorize $N$. Assume that an adversary can output a statement $((h_1, h_2, c_0, c_1, c_2), (a_0, a_1), v, z, z')$ such that $(a_0, a_1, a_2), v, z)$ and $(a_0, a_1, a_2), v, z')$ are accepting transcripts for $(h_1, h_2, c_0, c_1, c_2)$ and $z \neq z'$. Therefore using the first verification equation it holds $g^{Nz} = a_0 \cdot c_0^v = g^{Nz'} \bmod N \implies z = z' \bmod \varphi(N)$. Hence, $z-z' = \alpha \cdot \ord \implies 2(z-z') = \alpha \cdot \varphi(N) $ for some $\alpha \neq 0$. Applying \Cref{factor-lemma} for $M=2(z-z')$ we are able to factorize $N$.

This concludes the proof.
\end{proof}

Next we describe our Sigma protocol $\Sigma = (\prv, \vrf)$ for language $L_4$:
\begin{enumerate}
\item $\prv$ samples $\alpha \rand \smplset$, computes $a_0:=g^\alpha, a_1:= (h_1\cdot h_2^{-1})^\alpha, a_2:= (h_3\cdot h_2^{-1})^\alpha \bmod N$ and sends $(a_0, a_1, a_2)$ to $\vrf$.
\item $\vrf$ samples $v \rand [2^d]$ and sends $v$ to $\prv$.
\item $\prv$ computes the response $z:= \alpha + v \cdot r$ and sends it to $\vrf$.
\item $\vrf$ accepts if and only if $g^z = a_0 \cdot c_0^v \land (h_1\cdot h_2^{-1})^z = a_1 \cdot (c_1\cdot c_2^{-1})^v  \land (h_3\cdot h_2^{-1})^z = a_2 \cdot (c_3\cdot c_2^{-1})^v$ where all computation are done $\bmod N$ and $z \in [\estord + v\estord]$.
\end{enumerate}

\begin{theorem}
If the factoring assumption holds relative $\genmod$, then the above defined protocol $\Sigma = (\prv, \vrf)$ is a Sigma protocol for language $L$ that is perfectly complete, honest verifier zero-knowledge, sound and has quasi-unique responses. 
\end{theorem}

\begin{proof} 
We prove required properties. 

\paragraph{Completeness.} We verify the given requirements:
\begin{enumerate}
\item $g^z = g^{\alpha + v \cdot r} = g^\alpha + (g^r)^v = a_0 \cdot c_0^v \bmod N$;
\item $(h_1\cdot h_2^{-1})^z = (h_1\cdot h_2^{-1})^{\alpha + v \cdot r} = (h_1\cdot h_2^{-1})^\alpha \cdot (h_1^r\cdot (h_2^r)^{-1})^v = a_1 \cdot (h_1^r \cdot m \cdot (h_2^r \cdot m)^{-1})^v = a_1 \cdot (c_1\cdot c_2^{-1})^v$;
\item $(h_3\cdot h_2^{-1})^z = (h_3\cdot h_2^{-1})^{\alpha + v \cdot r} = (h_3\cdot h_2^{-1})^\alpha \cdot (h_3^r\cdot (h_2^r)^{-1})^v = a_2 \cdot (h_3^r \cdot m \cdot (h_2^r \cdot m)^{-1})^v = a_2 \cdot (c_3\cdot c_2^{-1})^v$;
\item since $r, \alpha \in \smplset$ and $v \in [2^d]$, therefore $z:= \alpha + v \cdot r \in [ \estord + v\estord]$. 
\end{enumerate}

\paragraph{HVZK.} The simulator $\simul((h_1, h_2, c_0, c_1, c_2))$ works as follows:
\begin{enumerate}
\item Samples $v \rand [2^d], z \rand [\estord + v\estord]$.
\item Computes $a_0:= g^z \cdot c_0^{-v}, a_1:= (h_1 \cdot h_2^{-1})^z \cdot (c_1 \cdot c_2^{-1})^{-v}, a_2:= (h_3 \cdot h_2^{-1})^z \cdot (c_3 \cdot c_2^{-1})^{-v}$ and returns $((a_0, a_1, a_2), v, z)$.
\end{enumerate}
It is easy to verify that produced transcript is accepting and moreover it has the same distribution as a honest transcript. 

%\paragraph{SHVZK.} The simulator $\simul((c_0, c_1, c_2), v)$ works as follows:
%\begin{enumerate}
%\item Samples $z \rand [\estord + v\estord]$.
%\item Computes $a_0:= g^z \cdot c_0^{-v}, a_1:= (h_1 \cdot h_2^{-1})^z \cdot (c_1 \cdot c_2^{-1})^{-v}$ and returns $(a_0, a_1), z$.
%\end{enumerate}
%It is easy to verify that produced transcript is accepting and moreover if $v$ is sampled uniformly at random from $[2^d]$, then the given transcript has the same distribution as a honest transcript. 

\paragraph{Soundness.} Let $((a_0, a_1, a_2), v, z)$ be an accepting transcript for $(h_1, h_2, c_0, c_1, c_2) \notin L_4$. That means $c_0 = g^{r_0}, (c_1\cdot (c_2)^{-1}) = (h_1\cdot (h_2)^{-1})^{r_1} \bmod N, (c_3\cdot (c_2)^{-1}) = (h_3\cdot (h_2)^{-1})^{r_2} \bmod N$ and $r_0 \neq r_1 \mod \varphi(N)/2$ or $r_0 \neq r_2 \mod \varphi(N)/2$. Let $a_0 = g^{\alpha_0} \bmod N, a_1 = (h_1\cdot (h_2)^{-1})^{\alpha_1} \bmod N, a_2 = (h_3\cdot (h_2)^{-1})^{\alpha_2} \bmod N$. Considering the first verification equation with discrete logarithms to base $g$, the second verification equation with discrete logarithm to base $(h_1 \cdot h_2^{-1})$, and the third verification equation to base $(h_3 \cdot h_2^{-1})$, we obtain following equations:
\begin{align}
z = \alpha_0 + r_0 v \bmod \ord\\
z = \alpha_1 + r_1 v \bmod \ord \\
z = \alpha_2 + r_2 v \bmod \ord
\end{align}
Computing (1)-(2) we obtain
\begin{align*}
0 = (\alpha_0 - \alpha_1)+ (r_0-r_1)v \bmod \ord \\
v = \frac{\alpha_1 - \alpha_0}{r_0-r_1} \bmod \ord
\end{align*}
Computing (1)-(3) we obtain
\begin{align*}
0 = (\alpha_0 - \alpha_2)+ (r_0-r_2)v \bmod \ord \\
v = \frac{\alpha_2 - \alpha_0}{r_0-r_2} \bmod \ord
\end{align*}
Now notice that at least one of $(r_0 - r_1)$ and $(r_0 - r_2)$ is not zero and values $\alpha_0, \alpha_1, \alpha_1, r_0, r_1, r_2$ are fixed before challenge $v$ is provided by $\vrf$. Therefore in order to provide accepting proof, the adversary has to correctly guess value $v$. If we choose $2^d < \ord$ value $v$ is unique and therefore probability the adversary produce accepting transcript for a statement that is not in $L_4$ is at most $2^{-d}$ which can be set to be negligible in $\secpar$. This holds even for unbounded adversaries. 

%Let $((a_0, a_1), v, z), ((a_0, a_1), v', z')$ are two accepting transcripts such that $v \neq v'$. Then $g^z = a_0 \cdot g^{rv} \bmod N$ and $g^{z'} = a_0 \cdot g^{rv'} \bmod N$. Hence $r = (z-z') \cdot (v-v')^{-1}$ a since $v \neq v'$


\paragraph{Quasi Unique Responses.} We show that our Sigma protocol has quasi unique responses, otherwise we are able to factorize $N$. Assume that an adversary can output a statement $((h_1, h_2, c_0, c_1, c_2), (a_0, a_1), v, z, z')$ such that $(a_0, a_1, a_2), v, z)$ and $(a_0, a_1, a_2), v, z')$ are accepting transcripts for $(h_1, h_2, c_0, c_1, c_2)$ and $z \neq z'$. Therefore using the first verification equation it holds $g^z = a_0 \cdot c_0^v = g^{z'} \bmod N \implies z = z' \bmod \ord$. Hence, $z-z' = \alpha \cdot \ord \implies 2(z-z') = \alpha \cdot \varphi(N) $ for some $\alpha \neq 0$. Applying \Cref{factor-lemma} for $M=2(z-z')$ we are able to factorize $N$.
\end{proof}

By \Cref{thm:fs} our Sigma protocols can be turned into simulation-sound NIZKs with alogrithms $(\prove, \vrfy)$ in the random oracle model via the Fiat-Shamir transform. 




%%% Local Variables:
%%% mode: latex
%%% TeX-master: "main"
%%% End:

%!TEX root=./main.tex
%\section{Constructions of Non-Malleable NITC using NIZK in ROM}
\subsection{Construction of Linearly Homomorphic Non-Malleable NITC}
\label{sec:linear-ROM}
We define language for our construction of a linearly homomorphic NITC depicted in \Cref{table:nitc-lh-rom} which relies on a one-time simulation sound NIZKs in the ROM in the following way:

\[
L = \left\{(h_1, h_2, c_0, c_1, c_2, c_3)| \exists (m,r):
\begin{aligned}
       (\land_{i=1}^3 c_i = h_i^{rN}(1+N)^m \bmod N^2) \land \\
       c_0 = g^r \bmod N\\
    \end{aligned}
    \right\}, 
\]
where $g, h_3, N$ are parameters specifying the language.


\begin{figure}[h!]
\begin{center}
\begin{tabular}{|ll|}
\hline
$\underline{\pgen(\seck, T)}$ 							   & $\underline{\com(\crs, m)}$ \\
$(p, q_, N, g) \leftarrow \genmod(\seck)$ & $r \rand \smplset$  \\
$\varphi(N):= (p-1)(q-1)$   & $c_0:= g^r \bmod N$ \\
$k_1, k_2 \rand \smplset$ & For $i \in [3]: c_i:= h_i^{rN}(1+N)^m \bmod N^2$\\
$t:= 2^T \bmod \varphi(N)/2$ & $\Phi := (h_1, h_2, c_0, c_1, c_2, c_3), w := (m, r)$ \\
For $i \in [2]: h_i:= g^{k_i} \bmod N$ &  $\pi_\com \leftarrow \nizk.\prove(\Phi, w)$\\
$h_3:=g^{t} \bmod N$ &  $c := (c_0, c_1, c_2, c_3)$\\
return $\crs:= (N,T,g,h_1,h_2, h_3)$ &  $\pi_\dec: = r$ \\
%$\crs \leftarrow \nizk.\setup(\seck)$ & \\
 & return $(c, \pi_\com, \pi_\dec)$\\
%return $\crs$     & \\
                                             &\\
$\underline{\cvrfy(\crs, c, \pi_\com)}$     & $\underline{\dvrfy(\crs,c, m, \pi_\dec)}$ \\
Parse $c$ as $(c_0, c_1, c_2, c_3)$  & Parse $c$ as $(c_0, c_1, c_2, c_3)$ \\
return $\nizk.\vrfy((h_1, h_2, c_0, c_1, c_2, c_3), \pi)$  & if $ \land_{i=1}^3 c_i = h_i^{\pi_\dec N}(1+N)^m  \bmod N^2$ \\
 & $\land c_0 = g^{\pi_\dec} \bmod N$\\
 & \tab return 1 \\
& return 0 \\
                                             &\\
$\underline{\fdecom(\crs,c)}$ & $\underline{\fdvrfy(\crs,c, m, \pi_\fdecom)}$ \\
Parse $c$ as $(c_0, c_1, c_2, c_3)$ & Parse $c$ as $(c_0, c_1, c_2, c_3)$\\
Compute $ \pi_\fdecom:=c_0^{2^T} \bmod N$ & if $c_3 = \pi_\fdecom^N (1+N)^m \bmod N^2$\\
$m:=\frac{c_3 \cdot \pi_\fdecom^{-N} (\bmod N^2) -1}{N}$ &  \tab return 1\\
return $(m,\pi_\fdecom)$ & return 0\\


                                             &\\
$\underline{\eval(\crs,\oplus_N, c_1, \dots, c_n)}$ &  \\
Parse $c_i$ as $(c_{i,0}, c_{i,1}, c_{i,2}, c_{i,3})$ & \\
\multicolumn{2}{|l|}{Compute $c_0 := \prod_{i=1}^n c_{i,0} \bmod N, c_1:= \bot, c_2:=\bot, c_3 := \prod_{i=1}^n c_{i,3} \bmod N^2$} \\
return $c := (c_0, c_1, c_2, c_3, \pi)$ & \\
%$\underline{\decom(\crs, \sk, c)}$     & $\underline{\fdecom(\crs,c)}$ \\
%Parse $c$ as $(c_0, c_1, c_2, \pi)$  & Parse $c$ as $(c_0, c_1, c_2, \pi)$ \\
%if $\nizk.\vrfy((c_0, c_1, c_2), \pi)= 1$  & if $\nizk.\vrfy((c_0, c_1, c_2), \pi)= 1$\\
%\tab Compute $y_1:= c_0^{k} \bmod N$  &   \tab Compute $ y_2:=c_0^{2^T} \bmod N$ \\
%\tab return $c_1 \cdot y_1^{-1} \bmod N$ & \tab return $c_2 \cdot y_2^{-1} \bmod N$ \\
%return $\bot$ & return $\bot$\\
\hline          
\end{tabular}
\caption{Construction of Linearly Homomorphic NITC in ROM. \\ $\oplus_N$ refers to addition $\bmod N$}
\label{table:nitc-lh-rom}
\end{center}
\end{figure}

%\begin{figure}[h!]
%\begin{center}
%\begin{tabular}{|ll|}
%\hline
%$\underline{\pgen(\seck, T)}$ 							   & $\underline{\com(\pk, m)}$ \\
%$(p, q_, N) \leftarrow \genmod(\seck)$ & $r \rand \smplset$  \\
%$\varphi(N):= (p-1)(q-1)$   & Compute $c_0:= g^r \bmod N$ \\
%Sample random generator $g$ of $\Jn$ & For $i \in [3]: c_i:= h_i^{rN}(1+N)^m \bmod N^2$\\
%$k_1, k_2 \rand \smplset$ & $\Phi := (h_1c_0, c_1, c_2, c_3), w := (m, r)$ \\
%$t:= 2^T \bmod \varphi(N)/2$ &  $\pi \leftarrow \nizk.\prove(\Phi, w)$\\
%For $i \in [2]: h_i:= g^{k_i} \bmod N$ &  $c := (c_0, c_1, c_2, c_3, \pi)$\\
%$h_3:=g^{t} \bmod N$ &  $\pi_\com:= \bot, \pi_\dec: = r$ \\
%%$\crs \leftarrow \nizk.\setup(\seck)$ & \\
%return $\crs:= (N,T,g,h_1,h_2, h_3)$ & return $(c, \pi_\com, \pi_\dec)$\\
%%return $\crs$     & \\
%                                             &\\
%$\underline{\cvrfy(\crs, c, \pi_\com)}$     & $\underline{\dvrfy(\crs,c, m, \pi_\dec)}$ \\
%Parse $c$ as $(c_0, c_1, c_2, c_3 \pi)$  & Parse $c$ as $(c_0, c_1, c_2, c_3 \pi)$ \\
%return $\nizk.\vrfy((c_0, c_1, c_2, c_3,), \pi)$  & if $ \land_{i=1}^3 c_i = h_i^{rN}(1+N)^m  \bmod N^2 \land c_0 = g^r \bmod N$\\
% & \tab return 1 \\
%& return 0 \\
%                                             &\\
%$\underline{\fdecom(\crs,c)}$ & $\underline{\fdvrfy(\crs,c, m, \pi_\fdecom)}$ \\
%Parse $c$ as $(c_0, c_1, c_2, c_3, \pi)$ & Parse $c$ as $(c_0, c_1, c_2, c_3, \pi)$\\
%if $\nizk.\vrfy((c_0, c_1, c_2,c_3), \pi)= 1$& if $\nizk.\vrfy((c_0, c_1, c_2,c_3), \pi)= 1 \land $\\
%\tab Compute $ \pi_\fdecom:=c_0^{2^T} \bmod N$ &  $c_3 = \pi_\fdecom^N (1+N)^m \bmod N^2$\\
%\tab Compute $m:=\frac{c_3 \cdot \pi_\fdecom^{-N} (\bmod N^2) -1}{N}$ &\tab return 1\\
%\tab return $(m,\pi_\fdecom)$ & return 0\\
%return $\bot$ & \\
%
%%$\underline{\decom(\crs, \sk, c)}$     & $\underline{\fdecom(\crs,c)}$ \\
%%Parse $c$ as $(c_0, c_1, c_2, \pi)$  & Parse $c$ as $(c_0, c_1, c_2, \pi)$ \\
%%if $\nizk.\vrfy((c_0, c_1, c_2), \pi)= 1$  & if $\nizk.\vrfy((c_0, c_1, c_2), \pi)= 1$\\
%%\tab Compute $y_1:= c_0^{k} \bmod N$  &   \tab Compute $ y_2:=c_0^{2^T} \bmod N$ \\
%%\tab return $c_1 \cdot y_1^{-1} \bmod N$ & \tab return $c_2 \cdot y_2^{-1} \bmod N$ \\
%%return $\bot$ & return $\bot$\\
%
%\hline          
%\end{tabular}
%\caption{NY Construction of NITC}
%\label{table:nitc}
%\end{center}
%\end{figure}

% \todo{appendix?}
\begin{proof}
Completeness is implied by the completeness of the NIZK and can be verified by inspection. 

%Our construction is based on the Naor-Yung paradigm where we combine three Paillier-type ciphertext with shared randomness.  


%Similarly to \cite{SCN:BiaMasVen16} we define a PPT algorithm $\rerand$ in Figure~\ref{fig:rerand}, which takes two ciphertexts generated with independent randomness, both public keys, only one secret key (in our case $k$) and randomness which was used to encrypt a message using the public key $g, h_1$. 
%%We assume that modulus $N$ is implicitly known.\todo{Why not explicit?} 
%% which corresponds to the secret key which is given as the input. 
%
%\begin{figure}[tb]
%\centering
%\begin{minipage}{0.75\textwidth}
%$\underline{\rerand(c:= (g^{r}, h_1^{r}\cdot m), c':= (g^{r'}, h_2^{r'}\cdot m), N, h_1, h_2, k, r)}:$
%\vspace{-2mm}
%\begin{itemize}
%\item $c_0:= g^{r}\cdot{g^{r'}} = g^{r+r'} \bmod N$;
%\item $c_1:= (g^{r'})^k \cdot h_1^{r}\cdot m  =  h_1^{r'}\cdot h_1^{r}\cdot m = h_1^{r+r'}\cdot m \bmod N$;
%\item $c_2:=h_2^{r} \cdot h_2^{r'}\cdot m = h_2^{r+r'}\cdot m \bmod N$.
%\end{itemize}
%\end{minipage}
%\caption{\label{fig:rerand}Algorithm $\rerand$.}
%\end{figure}
%
%
%
%
%It is straightforward to see that the ciphertext returned by $\rerand$ is perfectly distributed to the ciphertext produced using a shared randomness where the pair $(c_0, c_1)$ encrypts a message $m$ and the pair $(c_0, c_2)$ encrypts message $m'$.


% We note that if we use value $2^T$ as a secret key, then in order to compute $c_2$ we have to execute $T$ repeated squarings, but since $T$ is polynomial in $\secpar$ this computations is considered to be efficient.

\newsequenceofgames{NITC-LH-ROM}
To prove security we define a sequence of games $\games_0 - \games_{13}$.  For $i \in \{0,1,\dots,13\}$ we denote by $\games_i = 1$ the event that the adversary $\adv = \{(\adv_{1,\secpar}, \adv_{2, \secpar})\}_{\secpar \in \nats}$ outputs $b'$ in the game $\games_i$ such that $b = b'$.
%In individual games we use the algorithm $\decom$ define in \Cref{fig:deco-rom-lh} to answer decommitment queries efficiently. 
\begin{figure}[h!]
\begin{center}
\begin{tabular}{|l|}
\hline
$\underline{\decom(\crs, c, \pi_\com, i)}$\\
Parse $c$ as $(c_0, c_1, c_2, c_3)$\\
if $\nizk.\vrfy((h_1, h_2, c_0, c_1, c_2, c_3), \pi_\com)= 1$\\
\tab Compute $y:= c_0^{k_i} \bmod N$\\
\tab return $\frac{c_i \cdot y^{-N} (\bmod N^2) -1}{N}$\\
return $\bot$\\
\hline          
\end{tabular}
\caption{Decommitment oracle}
\label{fig:deco-rom-lh}
\end{center}
\end{figure}

\nextgame{G0}
Game $\games_\thisgame$ corresponds to the original security experiment where decommitment queries are answered using $\fdecom$.

\nextgame{DecOracle}
In game $\games_\thisgame$ decommitment queries are answered using the algorithm $\decom$ defined in \Cref{fig:deco-rom-lh} with $i:=1$, meaning that secret key $k_1$ and ciphertext $c_1$ are used, to answer decommitment queries efficiently. 


\begin{lemma}\label{nitc-rom-lh:flem}
\[
\left|\Pr[\games_\prevgame = 1] - \Pr[\games_\thisgame = 1]\right| \leq \snd^\nizk_\advB.
\]
\end{lemma}

Notice that if $c_1$ and $c_3$ contain the same message, both oracles answer decommitment queries consistently. Let $\event$ denote the event that the adversary $\adv$ asks a decommitment query $(c, \pi_\com)$ such that its decommitment using the key $k_1$ is different from its decommitment using $\fdecom$. Since $\games_\prevgame$ and $\games_\thisgame$ are identical until $\event$ happens, we bound the probability of $\event$. Concretely, we have

\[
\left|\Pr[\games_\prevgame = 1] - \Pr[\games_\thisgame = 1]\right| \leq \Pr[\event]. 
\]

We construct an adversary $\advB$ that breaks soundness with respect to auxiliary input $\aux:=(p,q)$ of the NIZK. 
The adversary $\advB_{\secpar}(p,q)$ proceeds as follows:
\vspace{-2mm}
\begin{enumerate}
\item Samples $k_1, k_2\rand \smplset$, computes $h_1 := g^{k_1} \bmod N, h_2:=g^{k_2} \bmod N, \allowbreak \varphi(N) :=(p-1)(q-1), t:=2^{T} \bmod \ord$ and sets $\crs:=(N, T, g, h_1, h_2, h_3)$ where $h_3$ is given by $L$. 
\item Then it runs $(m_0, m_1, \st) \leftarrow \adv_{1, \secpar}(\crs)$ and answers decommitment queries using $k_1$.
\item It samples $b \rand \bits, r \rand \smplset$ and computes $c_0^*:=g^r, c_1^*:=h_1^{rN}(1+N)^{m_b}, c_2^*:=h_2^{rN}(1+N)^{m_b}, c_3^*:=h_3^{rN}(1+N)^{m_b}$. It sets $(s:=\mathlist(h_1, h_2, c_0^*, c_1^*, c_2^*, c_3^*), w:=(m,r))$ and runs $\pi^* \leftarrow \nizk.\prove(s,w)$.
\item Next, it runs $b' \leftarrow \adv_{2, \secpar}((c_0^*, c_1^*, c_2^*, c_3^*), \pi^*, \st)$ and answers decommitment queries using $k_1$.
\item Finally, it checks whether there exists a decommitment query $(c, \pi_\com)$ such that $\deco(c, \pi_\com, 1) \neq \dec(\crs,c, \pi_\com, 1)$. If $\event$ occurs, then this is the case, and it returns $((h_1, h_2, c_0, c_1, c_2, c_3), \pi_\com)$. Notice that finding such a query can be done efficiently with the knowledge of $t$ since instead of running $\fdecom$ it is possible to verify the proof and simply compute $\frac{c_3\cdot (c_0^t)^{-N}(\bmod N^2)-1}{N}$ which produce the same result as $\fdecom$.
\end{enumerate}

$\advB$ simulates $\games_\thisgame$ perfectly and if the event $\event$ happens, then it outputs a valid proof for a statement which is not in the specified language $L$. Therefore we get
\[\Pr[\event] \leq \snd^\nizk_\advB.\]

%Let $\gnr$ denote the event that the sampled $g$ in $\kgen$ is a generator of $\qrn$. Recall that $N = pq$ where $p = 2p'+1$ and $q = 2q'+1$. Because $g$ is sampled uniformly at random and $\qrn$ has $\varphi(|\qrn|) = (p'-1)(q'-1)$ generators, this event happens with overwhelming probability. Concretely, $\Pr[\gnr] = 1-\frac{1}{p'}-\frac{1}{q'}+\frac{1}{p'q'}$.
%Therefore the following holds.
%\begin{lemma}\label{tpke3:flem} 
%\begin{align*}
%\Pr[\games_\thisgame = 1] &= \Pr[\games_\thisgame = 1| \gnr]\Pr[\gnr] + \Pr[\games_\thisgame = 1| \overline{\gnr}]\Pr[\overline{\gnr}] \\
%&\leq \Pr[\games_\thisgame = 1| \gnr]\Pr[\gnr] + \Pr[\overline{\gnr}] \\
%&= \Pr[\games_\thisgame = 1| \gnr]\left( 1-\frac{1}{p'}-\frac{1}{q'}+\frac{1}{p'q'} \right) + \frac{1}{p'}+\frac{1}{q'}-\frac{1}{p'q'}.
%\end{align*}
%\end{lemma}




\nextgame{SimulProof}
Game $\games_\thisgame$ proceeds exactly as the previous game but we use the zero-knowledge simulator $(\pi^*, \st) \leftarrow \simul(1, \st, (h_1, h_2, c_0^*,c_1^*,c_2^*,c_3^*))$ to produce a simulated proof for the challenge commitment and $\simul(2, \st, \cdot)$ to answer random oracle queries. By zero-knowledge security of underlying NIZK we directly obtain
\begin{lemma}
\[
\left|\Pr[\games_\prevgame = 1] - \Pr[\games_\thisgame = 1]\right| \leq \zk^\nizk_\advB.
\]
\end{lemma}

We construct an adversary $\advB = \{\advB_\secpar\}_{\secpar \in \N}$ against zero-knowledge security of NIZK as follows:
\vspace{-2mm}
\begin{enumerate}
\item Samples $k_1, k_2 \rand \smplset$, computes $h_1 := g^{k_1} \bmod N, h_2 := g^{k_2} \bmod N$ and sets $\crs:=(N, T(\secpar), g, h_1, h_2, h_3)$. 
\item Runs $(m_0, m_1, \st) \leftarrow \adv_{1, \secpar}(\crs)$ and answers decommitment queries using $k_1$.
\item Samples $b \rand \bits, r \rand \smplset$ and computes $c_0^*:=g^r, c_1^*:=h_1^{rN}(1+N)^{m_b}, c_2^*:=h_2^{rN}(1+N)^{m_b}, c_3^*:=h_3^{rN}(1+N)^{m_b}$. It submits $(s:=\mathlist(h_1,h_2,c_0^*, c_1^*, c_2^*, c_3^*), w:=(m,r))$ to its oracle and obtains proof $\pi^*$ as answer.
\item Runs $b' \leftarrow \adv_{2, \secpar}((c_0^*, c_1^*, c_2^*, c_3^*), \pi^*, \st)$ and answers decommitment queries using $k_1$.
\item Returns the truth value of $b=b'$.
\end{enumerate}
If the proof $\pi^*$ is generated using $\nizk.\prove$, then $\advB$ simulates $\games_\prevgame$ perfectly. Otherwise $\pi^*$ is generated using $\simul_1$ and $\advB$ simulates $\games_\thisgame$ perfectly. This proofs the lemma.


\nextgame{RndExp}
In $\games_\thisgame$ we sample $r$ uniformly at random from $[\varphi(N)/2]$. 

\begin{lemma}
\[
\left|\Pr[\games_\prevgame = 1] - \Pr[\games_\thisgame = 1]\right| \leq \frac{1}{p}+\frac{1}{q}-\frac{1}{N}.
\]
\end{lemma}
%At first we remark that for upper bounding the difference between the games we use a statistical argument. Because $r$ appears only in the exponent of the group generator, we later sample a random element from the group $\qrn$ which can be done efficiently. 
Since the only difference between the two games is in the set from which we sample $r$, to upper bound the advantage of adversary we can use \Cref{sampling-lemma} with $\ell:=2$, which directly yields required upper bound.

%\nextgame{ReRand}
%In Game $\games_\thisgame$ we produce the challenge commitment by encrypting the challenge message using two independent random exponents $r \rand \smplset, r' \rand [\varphi(N)/4]$ to obtain $c:= (g^{r}, h_1^{r}\cdot m_b), c':= (g^{r'}, h_2^{r'}\cdot m_b)$ and then run $\rerand(c,c',N, \allowbreak h_1, h_2,k,r)$ to obtain resulting ciphertext $(c_0^*, c_1^*, c_2^*)$. Since $r'$ is sampled uniformly at random from $[\varphi(N)/4]$ the ciphertext distributions in both games  are the same. Therefore 
%
%\begin{lemma}
%\[
%\Pr[\games_\prevgame = 1] = \Pr[\games_\thisgame = 1].
%\]
%\end{lemma}


\nextgame{SSSA}
In $\games_\thisgame$ we sample $y_3 \rand \Jn$ and compute $c_3^*$ as $y_3^N (1+N)^{m_b}$.

Let $\tilT_\sss(\secpar)$ be the polynomial whose existence is guaranteed by the SSS assumption.
Let $\poly_\advB(\secpar)$ be the fixed polynomial which bounds the time required to execute Steps 1--2 and answer decommitment queries in Step 3 of the adversary $\advB_{2, \secpar}$ defined below. Set $\undT := (\poly_\advB(\secpar))^{1 / \ugap}$.  Set $\tilT_\nitc := \max(\tilT_\sss, \undT)$.
\begin{lemma}
From any polynomial-size adversary $\adv = \{(\adv_{1,\secpar}, \adv_{2, \secpar})\}_{\secpar \in \nats}$, where depth of $\adv_{2, \secpar}$ is at most $T^{\ugap}(\secpar)$ for some $T(\cdot) \geq \undT(\cdot)$ we can construct a polynomial-size adversary $\advB = \{(\advB_{1,\secpar}, \advB_{2, \secpar})\}_{\secpar \in \nats}$ where the depth of $\advB_{2, \secpar}$ is at most $T^{\gap}(\secpar)$ with
\[
\left|\Pr[\games_\prevgame = 1] - \Pr[\games_\thisgame = 1]\right| \leq \advtg_\advB^\sss.
\]
\end{lemma}

The adversary $\advB_{1,\secpar}(N, T(\secpar), g):$
\vspace{-2mm}
\begin{enumerate}
\item Samples $k_1, k_2 \rand \smplset$, computes $h_1 := g^{k_1} \bmod N, h_2 := g^{k_2} \bmod N,  h_3 := g^{2^{T(\secpar)}} \bmod N$ and sets $\crs:=(N, T(\secpar), g, h_1, h_2, h_3)$. Notice that value $h_3$ is computed by repeated squaring.
\item Runs $(m_0, m_1, \st) \leftarrow \adv_{1, \secpar}(\crs)$ and answers decommitment queries using $k_1$.
\item Outputs $(N,g,k_1, k_2, h_1,h_2,h_3, m_0, m_1, \st)$
\end{enumerate}

The adversary $\advB_{2,\secpar}(x,y,(N,g,k_1, k_2, h_1,h_2,h_3, m_0, m_1, \st)):$
\vspace{-2mm}
\begin{enumerate}
\item Samples $b \rand \bits$, computes $c_0^*:=x, c_1^*:=x^{k_1N}(1+N)^{m_b}, c_2^*:=x^{k_2N}(1+N)^{m_b}, c_3^*:=y^{N}(1+N)^{m_b}$.
\item Runs $\pi^* \leftarrow \simul(1, \st', (h_1, h_2, c_0^*, c_1^*, c_2^*, c_3^*))$.
\item Runs $b' \leftarrow \adv_{2, \secpar}((c_0^*, c_1^*, c_2^*, c_3^*), \pi^*, \st)$ and answers decommitment queries using $k_1$.
\item Returns the truth value of $b=b'$.
\end{enumerate}
Since $g$ is a generator of $\Jn$ and $x$ is sampled uniformly at random from $\Jn$ there exists some $r \in [\varphi(N)/2]$ such that $x = g^{r}$. Therefore when $y = x^{2^T} = (g^{2^T})^{r} \bmod N$, then $\advB$ simulates $\games_\prevgame$ perfectly. Otherwise $y$ is random value and $\advB$ simulates $\games_\thisgame$ perfectly. 

Now we analyse the running time of the constructed adversary. Adversary $\advB_1$ computes $h_3$ by $T(\secpar)$ consecutive squarings and because $T(\secpar)$ is polynomial in $\secpar$, $\advB_1$ is efficient. Moreover, $\advB_2$ fulfils the depth constraint:
\[ \dep(\advB_{2,\secpar}) = \poly_\advB(\secpar) + \dep(\adv_{2,\secpar}) \leq \undT^{\ugap}(\secpar) + T^{\ugap}(\secpar) \leq 2 T^{\ugap}(\secpar) = o(T^{\gap}(\secpar)). \] 

Also $T(\cdot) \geq \tilT_\nitc(\cdot) \geq \tilT_\sss(\cdot)$ as required.

%\nextgame{RndExp2}
%In $\games_\thisgame$ we stop to use $\rerand$ algorithm. Concretely, we sample $r \rand [\varphi(N)/4], y_2 \rand \qrn$ and compute challenge ciphertext as $c^*:=(g^r, h_1^r \cdot m_b, y_2 \cdot m_b)$. The ciphertext has the same distribution as in the previous game. Therefore 
%
%\begin{lemma}
%\[
%\Pr[\games_\prevgame = 1] = \Pr[\games_\thisgame = 1].
%\]
%\end{lemma}
%\nextgame{RndExp3}




\nextgame{DCR1}
In $\games_\thisgame$ we sample $y_3 \rand \Zns$ such that it has Jacobi symbol 1 and compute $c_3^*$ as $y_3(1+N)^{m_b}$. 

\begin{lemma}\label{lem:dcr-rom-lh}
\[
\left|\Pr[\games_\prevgame = 1] - \Pr[\games_\thisgame = 1]\right| \leq \advtg_\advB^\dcr.
\]
\end{lemma}
We construct an adversary $\advB = \{\advB_\secpar\}_{\secpar \in \N}$ against DCR.

$\advB_{\secpar}(N,y):$
\vspace{-2mm}
\begin{enumerate}
\item Samples $g, y_3, x \rand \Jn, k_1, k_2 \rand \smplset$, computes $h_1 := g^{k_1} \bmod N, h_2 := g^{k_2} \bmod N,  h_3 := g^{2^{T}} \bmod N$ and sets $\crs:=(N, T, g, h_1, h_2, h_3)$. Notice that value $h_3$ is computed by repeated squaring.
\item Runs $(m_0, m_1, \st) \leftarrow \adv_{1, \secpar}(\crs)$ and answers decommitment queries using $k_1$.
\item Samples $b \rand \bits, w \rand \Zns$ such that $\left( \frac{y}{N} \right)= \left( \frac{w}{N} \right)$. We remark that computing Jacobi symbol can be done efficiently without knowing factorization of N.
\item Computes $c_0^*:=x, , c_1^*:=x^{k_1N}(1+N)^{m_b}, c_2^*:=x^{k_2N}(1+N)^{m_b}, c_3^*:=yw^{N}(1+N)^{m_b}$. Runs $\pi^* \leftarrow \simul(1, \st', (h_1, h_2, c_0^*, c_1^*, c_2^*, c_3^*))$.
\item Runs $b' \leftarrow \adv_{2, \secpar}((c_0^*, c_1^*, c_2^*, c_3^*), \pi^*, \st)$ and answers decommitment queries using $k_1$.
\item Returns the truth value of $b=b'$.
\end{enumerate}
%If $y = v^N \bmod N^2$ then $yw^N = v^N w^N = (vw)^N$ and hence $yw^N$ is $N$-th residue. Moreover, the Jacobi symbol of $yw$ is 1, since the Jacobi symbol is multiplicatively homomorphic. Therefore $\advB$ simulates $\games_\prevgame$ perfectly. Otherwise $y$ is uniform random element in $\Zns$ then $yw^N$ is also uniform in $\Zns$ and $\advB$ simulates $\games_\thisgame$ perfectly. This proofs the lemma. We remark that at this point $c_3^*$ does not reveal any information about $m_b$.

If $y = v^N \bmod N^2$ then $yw^N = v^N w^N = (vw)^N$ and hence $yw^N$ is $N$-th residue. Moreover, the Jacobi symbol of $yw$ is 1, since the Jacobi symbol is multiplicatively homomorphic. Therefore $\advB$ simulates $\games_\prevgame$ perfectly. 

Otherwise, if $y$ is uniform random element in $\Zns$, then $yw^N$ is also uniform among all elements of $\Zns$ that have Jacobi symbol 1 and $\advB$ simulates $\games_\thisgame$ perfectly. This proves the lemma.

We remark that at this point $c_3^*$ does not reveal any information about $b$. Here we use that if $x = y \bmod N$ then $\left( \frac{x}{N} \right)= \left( \frac{y}{N} \right)$ and that there is an isomorphism $f:\Zn^* \times \Zn \rightarrow\Zns$ given by $f(u,v)=u^N(1+N)^v = u^N(1+vN) \bmod N^2$ (see e.g. \cite[Proposition 13.6]{books/crc/KatzLindell2014}).  Since $f(u,v) \bmod N = u^N + u^NvN \bmod N = u^N \bmod N$, that means that Jacobi symbol $\left( \frac{f(u,v)}{N} \right)$ depends only on $u$. Hence if $\left( \frac{f(u,v)}{N} \right) = 1$ then it must hold that $\left( \frac{f(u,r)}{N} \right) = 1$ for all $r \in \Zn$. This implies that a random element $f(u,v)$ in $\Zns$ with $\left( \frac{f(u,v)}{N} \right) = 1$ has a uniformly random distribution of $v$ in $\Zn$. Therefore if $yw^N = u^N(1+N)^v \bmod N^2$ then  $yw^N(1+N)^{m_b}  = u^N(1+N)^{m_b+v} \bmod N^2$. Since $v$ is uniform in $\Zn$, $(m_b + v)$ is also uniform in $\Zn$, which means that ciphertext $c_3^*$ does not reveal any information about $b$. 

\nextgame{RndExp2}
In $\games_\thisgame$ we sample $k_2$ uniformly at random from $[\varphi(N)/2]$. 

\begin{lemma}
\[
\left|\Pr[\games_\prevgame = 1] - \Pr[\games_\thisgame = 1]\right| \leq \frac{1}{p}+\frac{1}{q}-\frac{1}{N}.
\]
\end{lemma}

Again using a statistical argument this lemma directly follows from \Cref{sampling-lemma} with $\ell:=2$.

\nextgame{DDH}
In $\games_\thisgame$ we sample $y_2 \rand \Jn$ and compute $c_2^*$ as  $y_2^N(1+N)^{m_b}$. 

\begin{lemma}\label{lem:ddh-rom-lh}
\[
\left|\Pr[\games_\prevgame = 1] - \Pr[\games_\thisgame = 1]\right| \leq \advtg_\advB^\ddh.
\]
\end{lemma}
We construct an adversary $\advB = \{\advB_\secpar\}_{\secpar \in \N}$ against DDH in the group $\Jn$. %Given \Cref{thm:ddh} this implies an adversary against DDH in large prime-order subgroups of $\Zn^*$.

$\advB_{\secpar}(N,g,g^\alpha, g^\beta, g^\gamma):$
\vspace{-2mm}
\begin{enumerate}
\item Samples $k_1 \rand \smplset$, computes $h_1 := g^{k_1} \bmod N,  h_3 := g^{2^{T}} \bmod N$ and sets $\crs:=(N, T, g, h_1, h_2:=g^\alpha, h_3)$.
\item Runs $(m_0, m_1, \st) \leftarrow \adv_{1, \secpar}(\crs)$ and answers decommitment queries using $k_1$.
\item Samples $b \rand \bits, y_3 \rand \Zns$ such that it has Jacobi symbol 1 and computes $(c_0^*, c_1^*, c_2^*, c_3^*):=(g^\beta, (g^\beta)^{k_1 N}(1+N)^{m_b}, (g^{\gamma})^N(1+N)^{m_b}, y_3(1+N)^{m_b}).$ Runs $\pi^* \leftarrow \simul(1, \st', (h_1, h_2, c_0^*, c_1^*, c_2^*, c_3^*))$.
\item Runs $b' \leftarrow \adv_{2, \secpar}((c_0^*, c_1^*, c_2^*,c_3^*), \pi^*, \st)$ and answers decommitment queries using $k_1$.
\item Returns the truth value of $b=b'$.
\end{enumerate}
If $\gamma = \alpha\beta$, then $\advB$ simulates $\games_\prevgame$ perfectly. Otherwise $g^\gamma$ is uniform random element in $\Jn$ and $\advB$ simulates $\games_\thisgame$ perfectly. This proofs the lemma.

\nextgame{RndExp3}
In $\games_\thisgame$ we sample $k_2$ uniformly at random from $\smplset$. 

\begin{lemma}
\[
\left|\Pr[\games_\prevgame = 1] - \Pr[\games_\thisgame = 1]\right| \leq \frac{1}{p}+\frac{1}{q}-\frac{1}{N}.
\]
\end{lemma}

This lemma directly follows from \Cref{sampling-lemma}.

\nextgame{DCR2}
In $\games_\thisgame$ we sample $y_2 \rand \Zns$ such that it has Jacobi symbol 1 and compute $c_2^*$ as $y_2(1+N)^{m_b}$. 

\begin{lemma}
\[
\left|\Pr[\games_\prevgame = 1] - \Pr[\games_\thisgame = 1]\right| \leq \advtg_\advB^\dcr.
\]
\end{lemma}
This can be proven in similar way as \Cref{lem:dcr-rom-lh}. We remark that at this point $c_2^*$ does not reveal any information about $m_b$.



\nextgame{SimSnd}

In $\games_\thisgame$ we answer decommitment queries using $\dec$ (\Cref{fig:deco-rom-lh}) with $i:=2$ which means that secret key $k_2$ and ciphertext $c_2$ are used. 

\begin{lemma}
\[
\left|\Pr[\games_\prevgame = 1] - \Pr[\games_\thisgame = 1]\right| \leq \simsnd^\nizk_\advB. 
\]
\end{lemma}

Let $\event$ denote the event that adversary $\adv$ asks a decommitment query $(c, \pi_\com)$ such that its decommitment using the key $k_1$ is different from its decommitment using the key $k_2$. Since $\games_\prevgame$ and $\games_\thisgame$ are identical until $\event$ does not happen, by the standard argument it is sufficient to upper bound the probability of happening $\event$. Concretely,  

\[
\left|\Pr[\games_\prevgame = 1] - \Pr[\games_\thisgame = 1]\right| \leq \Pr[\event]. 
\]

We construct an adversary $\advB$ which breaks one-time simulation soundness of the NIZK. 

The adversary $\advB_{\secpar}^{\simul_1, \simul_2}:$
\vspace{-2mm}
\begin{enumerate}
\item Computes $\crs \leftarrow \pgen(\seck, T)$ as defined in the construction where the value $h_3$ is computed using repeated squaring instead.
\item Runs $(m_0, m_1, \st) \leftarrow \adv_{1, \secpar}(\crs)$ and answers decommitment queries using $k_2$.
\item Samples $b \rand \bits, x \rand \Jn, y_2, y_3 \rand \Zns$ and computes $(c_0^*, c_1^*, c_2^*, c_3^*):=(x, x^{k_1 N} (1+N)^{m_b}, y_2 (1+N)^{m_b}, y_3 (1+N)^{m_b})$. Forwards $(h_1, h_2, c_0^*, c_1^*, c_2^*, c_3^*)$ to simulation oracle $\simul_1$ and obtains a proof $\pi^*$.
\item Runs $b' \leftarrow \adv_{2, \secpar}((c_0^*, c_1^*, c_2^*, c_3^*), \pi^*, \st)$ and answers decommitment queries using $k_2$.
\item Find a decommitment query $(c, \pi_\com)$ such that $\dec(\crs, c, \pi_\com, 1) \neq \allowbreak \dec\mathlist(\crs,c, \pi_\com, 2)$ and returns $((h_1, h_2, c_0, c_1, c_2, c_3), \pi_\com)$.
\end{enumerate}

$\advB$ simulates $\games_\thisgame$ perfectly and if the event $\event$ happens, it outputs a valid proof for a statement which is not in the specified language $L$. Therefore
\[\Pr[\event] \leq \simsnd^\nizk_\advB,\]
which concludes the proof of the lemma.  

%\nextgame{ReRand2}
%In $\games_\thisgame$ we use the key $t$ and randomness $r'$ as input for rerandomization. Concretely we compute $\rerand(c,c',h_1,h_2,t,r')$. This is just conceptual change since the ciphertext distributions are the same in both games and therefore 
%
%\begin{lemma}
%\[
%\left|\Pr[\games_\prevgame = 1] = \Pr[\games_\thisgame = 1]\right|.
%\]
%\end{lemma}
%
%\nextgame{RndExp4}
%In $\games_\thisgame$ we sample $r$ uniformly at random from $\varphi(N)$. 
%
%\begin{lemma}
%\[
%\left|\Pr[\games_\prevgame = 1] - \Pr[\games_\thisgame = 1]\right| \leq \frac{1}{N}.
%\]
%\end{lemma}

\nextgame{RndExp4}
In $\games_\thisgame$ we sample $k_1$ uniformly at random from $[\varphi(N)/2]$. 

\begin{lemma}
\[
\left|\Pr[\games_\prevgame = 1] - \Pr[\games_\thisgame = 1]\right| \leq \frac{1}{p}+\frac{1}{q}-\frac{1}{N}.
\]
\end{lemma}

This lemma directly follows from \Cref{sampling-lemma} with $\ell:=2$.

\nextgame{DDH2}
In $\games_\thisgame$ we sample $y_1 \rand \Jn$ and compute $c_1^*$ as  $y_1^{N} (1+N)^{m_b}$. 

\begin{lemma}
\[
\left|\Pr[\games_\prevgame = 1] - \Pr[\games_\thisgame = 1]\right| \leq \advtg_\advB^\ddh.
\]
\end{lemma}
This can be proven in similar way as \Cref{lem:ddh-rom-lh}.

\nextgame{DCR3}
In $\games_\thisgame$ we sample $y_1 \rand \Zns$ such that it has Jacobi symbol 1 and compute $c_1^*$ as $y_1(1+N)^{m_b}$. 

\begin{lemma}
\[
\left|\Pr[\games_\prevgame = 1] - \Pr[\games_\thisgame = 1]\right| \leq \advtg_\advB^\dcr.
\]
\end{lemma}
This can be proven in similar way as \Cref{lem:dcr-rom-lh}. We remark that at this point $c_1^*$ does not reveal any information about $m_b$.

\begin{lemma}\label{nitc-rom-lh:llem}
\[
\Pr[\games_\thisgame = 1] = \half.
\]
\end{lemma}

Clearly, $c_0^*$ is uniform random element in $\Jn$ and hence it does not contain any information about the challenge message. Since $y_1, y_2, y_3$ are sampled uniformly at random from $\Zns$ the ciphertexts $c_1^*, c_2^*, c_3^*$ are also uniform random elements in $\Zns$ and hence do not contain any information about the challenge message $m_b$. Therefore, an adversary can not do better than guessing.

By combining Lemmas \ref{nitc-rom-lh:flem} - \ref{nitc-rom-lh:llem} we obtain the following:
\begin{align*}
\advtg^{\nitc}_{\adv} &= \left| \Pr[\games_0 = 1] - \half \right| \leq \sum_{i=0}^{12} \left|\Pr[\games_i = 1] - \Pr[\games_{i+1} = 1] \right| + \left|\Pr[\games_{13}- \half\right| \\
 &\leq  \snd^\nizk_\advB + \zk^\nizk_\advB + \advtg^{\sss}_{\advB} + \simsnd^{\nizk}_{\advB} + 2 \advtg^{\ddh}_{\advB} + 3 \advtg^{\dcr}_{\advB} \\ &+ 4 \left( \frac{1}{p}+\frac{1}{q}-\frac{1}{N} \right).
\end{align*}
which concludes the proof.
\end{proof}
% \todo{appendix end}



\begin{theorem}
\label{thm:NITC-lin-ROM}
If $\nizk = (\nizk.\prove, \nizk.\vrfy)$ is a one-time simulation-sound non-interactive zero-knowledge proof system for $L$ which is sound with respect to auxiliary input $\aux:=(p,q)$, the strong sequential squaring assumption with gap $\gap$ holds relative to $\genmod$ in $\Jn$, the Decisional Composite Residuosity assumption holds relative to $\genmod$, and the Decisional Diffie-Hellman assumption holds relative to $\genmod$ in $\Jn$, then \mathlist{(\pgen, \com, \cvrfy, \dvrfy, \fdecom)} defined in \Cref{table:nitc-lh-rom} is an IND-CCA-secure non-interactive timed commitment scheme with $\ugap$, for any $\ugap < \gap$. 
\end{theorem}
The proof can be found in \Cref{app:NITC-in-ROM}.

\begin{theorem}
$(\pgen, \com, \cvrfy, \dvrfy, \fdecom)$ defined in \Cref{table:nitc-lh-rom} is a BND-CCA-secure non-interactive timed commitment scheme. 
\end{theorem}
{}
\begin{proof}
This can be proven in the same way as \Cref{bnd-cca-lh}.
\end{proof}

\begin{theorem}
If $\nizk = (\nizk.\prove, \nizk.\vrfy)$ is a non-interactive zero-knowledge proof system for $L$, then \mathlist{(\pgen, \com, \cvrfy, \dvrfy, \fdecom, \fdvrfy)} defined in \Cref{table:nitc-lh-rom} is a publicly verifiable non-interactive timed commitment scheme.
\end{theorem}

\begin{proof}
This can be proven in the same way as \Cref{pv-lh}.
\end{proof}

It is straightforward to verify that considering $\eval$ algorithm, our construction yields linearly homomorphic NITC. 
\begin{theorem}\label{hom-lh-rom}
The NITC \mathlist{(\pgen, \com, \cvrfy, \dvrfy, \fdecom, \fdvrfy, \eval)} defined in \Cref{table:nitc-lh-rom} is a linearly homomorphic non-interactive timed commitment scheme.
\end{theorem}









\subsection{Construction of Multiplicatively Homomorphic Non-Malleable NITC}
\label{sec:mult-ROM}
We define language for our construction of a multiplicatively homomorphic NITC which relies on a Sigma protocol in the following way:
\[
L = \left\{(h_1, h_2, c_0, c_1, c_2)| \exists (m,r):
\begin{aligned}
       (\land_{i=1}^3 c_i = h_i^{r}m \bmod N) \land
       c_0 = g^r \bmod N\\
    \end{aligned}
    \right\}, 
\]
where $g, N$ are parameters specifying the language.

Our construction is given in \Cref{table:nitc-mh-rom} where $(\nizk.\prove, \nizk.\vrfy)$ is the Fiat-Shamir transform of a Sigma protocol for language $L$. Since it is not straightforward to provide a security proof directly with respect to standard definition of one-time simulation sound NIZK, we provide the security proof in the random oracle model relying on the properties of the underlying Sigma protocol.


\begin{figure}[h!]
\begin{center}
\begin{tabular}{|ll|}
\hline
$\underline{\pgen(\seck, T)}$ 							   & $\underline{\com(\crs, m)}$ \\
$(p, q_, N, g) \leftarrow \genmod(\seck)$ & $r \rand \smplset$  \\
$\varphi(N):= (p-1)(q-1)$   & $c_0:= g^r \bmod N$ \\
$k_1\rand \smplsetqrn$ & For $i \in [2]: c_i:= h_i^{r}m \bmod N$\\
$t:= 2^T \bmod \varphi(N)/4$ & $\Phi := (h_1, h_2, c_0, c_1, c_2), w := (m, r)$ \\
$h_1:= g^{k_1} \bmod N$ &  $\pi_\com \leftarrow \nizk.\prove(\Phi, w)$\\
$h_2:=g^{t} \bmod N$ &  $c := (c_0, c_1, c_2)$\\
return $\crs:= (N,T,g,h_1,h_2)$ &  $\pi_\dec: = r$ \\
%$\crs \leftarrow \nizk.\setup(\seck)$ & \\
 & return $(c, \pi_\com, \pi_\dec)$\\
%return $\crs$     & \\
                                             &\\
$\underline{\cvrfy(\crs, c, \pi_\com)}$     & $\underline{\dvrfy(\crs,c, m, \pi_\dec)}$ \\
Parse $c$ as $(c_0, c_1, c_2)$  & Parse $c$ as $(c_0, c_1, c_2)$ \\
return $\nizk.\vrfy((h_1, h_2, c_0, c_1, c_2), \pi_\com)$  & if $ \land_{i=1}^2 c_i = h_i^{\pi_\dec}m  \bmod N \land c_0 = g^{\pi_\dec} \bmod N$\\
 & \tab return 1 \\
& return 0 \\
                                             &\\
$\underline{\fdecom(\crs,c)}$ & $\underline{\eval(\crs,\otimes_N, c_1, \dots, c_n)}$ \\
Parse $c$ as $(c_0, c_1, c_2)$ & Parse $c_i$ as $(c_{i,0}, c_{i,1}, c_{i,2})$\\
 Compute $ y:=c_0^{2^T} \bmod N$  & Compute $c_0 := \prod_{i=1}^n c_{i,0} \bmod N,  c_1:= \bot$\\
$m:=c_2 \cdot y^{-1} \bmod N$  &   Compute $c_2 := \prod_{i=1}^n c_{i,2} \bmod N $\\
return $m$  & return $c := (c_0, c_1, c_2)$\\


%$\underline{\decom(\crs, \sk, c)}$     & $\underline{\fdecom(\crs,c)}$ \\
%Parse $c$ as $(c_0, c_1, c_2, \pi)$  & Parse $c$ as $(c_0, c_1, c_2, \pi)$ \\
%if $\nizk.\vrfy((c_0, c_1, c_2), \pi)= 1$  & if $\nizk.\vrfy((c_0, c_1, c_2), \pi)= 1$\\
%\tab Compute $y_1:= c_0^{k} \bmod N$  &   \tab Compute $ y_2:=c_0^{2^T} \bmod N$ \\
%\tab return $c_1 \cdot y_1^{-1} \bmod N$ & \tab return $c_2 \cdot y_2^{-1} \bmod N$ \\
%return $\bot$ & return $\bot$\\

\hline          
\end{tabular}
\caption{Construction of Multiplicatively Homomorphic NITC in ROM. \\ $\otimes_N$ refers to multiplication $\bmod N$}
\label{table:nitc-mh-rom}
\end{center}
\end{figure}

%Where $(\nizk.\prove, \nizk.\vrfy)$ is the Fiat-Shamir transform of the Sigma protocol for language $L_4$ defined in Cref.... Concretely, $H$ is a hash function modelled as a random oracle, then our $\nizk$ is defined as follows
%\begin{itemize}
%\item $\nizk.\prove(s:=(h_1, h_2, c_0, c_1, c_2), w:=(m, r)):$ Compute $\alpha \rand \smplsetqrn, a_0:=g^\alpha \bmod N, a_1:= (h_1\cdot h_2^{-1})^\alpha \bmod N, v:=H(s,a_0,a_1), z:= \alpha + v \cdot r$ and return $\pi:=(a_0,a_1,z)$.
%\item $\nizk.\vrfy(s:=(h_1, h_2, c_0, c_1, c_2),\pi:=(a_0,a_1,z)):$ Compute $v:=v:=H(s,a_0,a_1)$ and return 1 if and only if $g^z = a_0 \cdot c_0^v \bmod N \land (h_1\cdot h_2^{-1})^z = a_1 \cdot (c_1\cdot c_2^{-1})^v \bmod N \land z \in [\estordqrn + v\estordqrn]$.
%\end{itemize}

%\begin{figure}[h!]
%\begin{center}
%\begin{tabular}{|ll|}
%\hline
%$\underline{\pgen(\seck, T)}$ 							   & $\underline{\com(\pk, m)}$ \\
%$(p, q_, N) \leftarrow \genmod(\seck)$ & $r \rand \smplset$  \\
%$\varphi(N):= (p-1)(q-1)$   & Compute $c_0:= g^r \bmod N$ \\
%Sample random generator $g$ of $\Jn$ & For $i \in [3]: c_i:= h_i^{rN}(1+N)^m \bmod N^2$\\
%$k_1, k_2 \rand \smplset$ & $\Phi := (h_1c_0, c_1, c_2, c_3), w := (m, r)$ \\
%$t:= 2^T \bmod \varphi(N)/2$ &  $\pi \leftarrow \nizk.\prove(\Phi, w)$\\
%For $i \in [2]: h_i:= g^{k_i} \bmod N$ &  $c := (c_0, c_1, c_2, c_3, \pi)$\\
%$h_3:=g^{t} \bmod N$ &  $\pi_\com:= \bot, \pi_\dec: = r$ \\
%%$\crs \leftarrow \nizk.\setup(\seck)$ & \\
%return $\crs:= (N,T,g,h_1,h_2, h_3)$ & return $(c, \pi_\com, \pi_\dec)$\\
%%return $\crs$     & \\
%                                             &\\
%$\underline{\cvrfy(\crs, c, \pi_\com)}$     & $\underline{\dvrfy(\crs,c, m, \pi_\dec)}$ \\
%Parse $c$ as $(c_0, c_1, c_2, c_3 \pi)$  & Parse $c$ as $(c_0, c_1, c_2, c_3 \pi)$ \\
%return $\nizk.\vrfy((c_0, c_1, c_2, c_3,), \pi)$  & if $ \land_{i=1}^3 c_i = h_i^{rN}(1+N)^m  \bmod N^2 \land c_0 = g^r \bmod N$\\
% & \tab return 1 \\
%& return 0 \\
%                                             &\\
%$\underline{\fdecom(\crs,c)}$ & $\underline{\fdvrfy(\crs,c, m, \pi_\fdecom)}$ \\
%Parse $c$ as $(c_0, c_1, c_2, c_3, \pi)$ & Parse $c$ as $(c_0, c_1, c_2, c_3, \pi)$\\
%if $\nizk.\vrfy((c_0, c_1, c_2,c_3), \pi)= 1$& if $\nizk.\vrfy((c_0, c_1, c_2,c_3), \pi)= 1 \land $\\
%\tab Compute $ \pi_\fdecom:=c_0^{2^T} \bmod N$ &  $c_3 = \pi_\fdecom^N (1+N)^m \bmod N^2$\\
%\tab Compute $m:=\frac{c_3 \cdot \pi_\fdecom^{-N} (\bmod N^2) -1}{N}$ &\tab return 1\\
%\tab return $(m,\pi_\fdecom)$ & return 0\\
%return $\bot$ & \\
%
%%$\underline{\decom(\crs, \sk, c)}$     & $\underline{\fdecom(\crs,c)}$ \\
%%Parse $c$ as $(c_0, c_1, c_2, \pi)$  & Parse $c$ as $(c_0, c_1, c_2, \pi)$ \\
%%if $\nizk.\vrfy((c_0, c_1, c_2), \pi)= 1$  & if $\nizk.\vrfy((c_0, c_1, c_2), \pi)= 1$\\
%%\tab Compute $y_1:= c_0^{k} \bmod N$  &   \tab Compute $ y_2:=c_0^{2^T} \bmod N$ \\
%%\tab return $c_1 \cdot y_1^{-1} \bmod N$ & \tab return $c_2 \cdot y_2^{-1} \bmod N$ \\
%%return $\bot$ & return $\bot$\\
%
%\hline          
%\end{tabular}
%\caption{NY Construction of NITC}
%\label{table:nitc}
%\end{center}
%\end{figure}



\begin{theorem}
\label{thm:NITC-mul-ROM}
If $\Sigma = (\prv, \vrf)$ is a Sigma protocol for $L$ with quasi unique responses and $\delta$-unpredictable commitments which is sound with respect to auxiliary input $\aux=(p,q)$ and is honest verifier zero-knowledge, $H: \qrn^7 \rightarrow [2^d]$ is a hash function modelled as a random oracle,  the strong sequential squaring assumption with gap $\gap$ holds relative to $\genmod$ in $\qrn$, and the Decisional Diffie-Hellman assumption holds relative to $\genmod$ in $\qrn$, then \mathlist{(\pgen, \com, \cvrfy, \dvrfy, \fdecom)} defined in \Cref{table:nitc-mh-rom} is an IND-CCA-secure non-interactive timed commitment scheme with $\ugap$, for any $\ugap < \gap$. 
\end{theorem}

% \todo{appendix?}
\begin{proof}
Completeness is implied by the completeness of the NIZK and can be verified by inspection. 

%Our construction is based on the Naor-Yung paradigm where we combine three ElGamal-type ciphertext with shared randomness.  


%Similarly to \cite{SCN:BiaMasVen16} we define a PPT algorithm $\rerand$ in Figure~\ref{fig:rerand}, which takes two ciphertexts generated with independent randomness, both public keys, only one secret key (in our case $k$) and randomness which was used to encrypt a message using the public key $g, h_1$. 
%%We assume that modulus $N$ is implicitly known.\todo{Why not explicit?} 
%% which corresponds to the secret key which is given as the input. 
%
%\begin{figure}[tb]
%\centering
%\begin{minipage}{0.75\textwidth}
%$\underline{\rerand(c:= (g^{r}, h_1^{r}\cdot m), c':= (g^{r'}, h_2^{r'}\cdot m), N, h_1, h_2, k, r)}:$
%\vspace{-2mm}
%\begin{itemize}
%\item $c_0:= g^{r}\cdot{g^{r'}} = g^{r+r'} \bmod N$;
%\item $c_1:= (g^{r'})^k \cdot h_1^{r}\cdot m  =  h_1^{r'}\cdot h_1^{r}\cdot m = h_1^{r+r'}\cdot m \bmod N$;
%\item $c_2:=h_2^{r} \cdot h_2^{r'}\cdot m = h_2^{r+r'}\cdot m \bmod N$.
%\end{itemize}
%\end{minipage}
%\caption{\label{fig:rerand}Algorithm $\rerand$.}
%\end{figure}
%
%
%
%
%It is straightforward to see that the ciphertext returned by $\rerand$ is perfectly distributed to the ciphertext produced using a shared randomness where the pair $(c_0, c_1)$ encrypts a message $m$ and the pair $(c_0, c_2)$ encrypts message $m'$.


% We note that if we use value $2^T$ as a secret key, then in order to compute $c_2$ we have to execute $T$ repeated squarings, but since $T$ is polynomial in $\secpar$ this computations is considered to be efficient.

\newsequenceofgames{NITC-MH-ROM}
To prove security we define a sequence of games $\games_0 - \games_{8}$.  For $i \in \{0,1,\dots,8\}$ we denote by $\games_i = 1$ the event that the adversary $\adv = \{(\adv_{1,\secpar}, \adv_{2, \secpar})\}_{\secpar \in \nats}$ outputs $b'$ in the game $\games_i$ such that $b = b'$.
%In individual games we use the algorithm $\decom$ define in \Cref{fig:deco-rom-mh} to answer decommitment queries efficiently. 
\begin{figure}[h!]
\begin{center}
\begin{tabular}{|l|}
\hline
$\underline{\decom(\crs, c, \pi_\com, i, \sk)}$\\
Parse $c$ as $(c_0, c_1, c_2)$\\
if $\nizk.\vrfy((h_1, h_2, c_0, c_1, c_2), \pi_\com)= 1$\\
\tab Compute $y:= c_0^{\sk} \bmod N$\\
\tab return $c_i \cdot y^{-1} \bmod N$\\
return $\bot$\\
\hline          
\end{tabular}
\caption{Decommitment oracle}
\label{fig:deco-rom-mh}
\end{center}
\end{figure}

In the following we assume that the underlying sigma protocol produces transcripts of the form $(a,v,z)$, where any of these values can be vectors. 
\nextgame{G0}
Game $\games_\thisgame$ corresponds to the original security experiment where decommitment queries are answered using $\fdecom$.

\nextgame{DecOracle}
In game $\games_\thisgame$ decommitment queries are answered using the algorithm $\decom$ defined in \Cref{fig:deco-rom-mh} with $i:=1, \sk:=k_1$ which means that secret key $k_1$ and ciphertext $c_1$ are used, to answer decommitment queries efficiently. 

\begin{lemma}\label{nitc-rom-mh:flem}
\[
\left|\Pr[\games_\prevgame = 1] - \Pr[\games_\thisgame = 1]\right| \leq Q\cdot\snd^\Sigma_\advB.
\]
\end{lemma}

Notice that if $c_1$ and $c_2$ contain the same message, both oracles answer decommitment queries consistently. Let $\event$ denote the event that the adversary $\adv$ asks a decommitment query $(c:=(c_0,c_1,c_2), \pi_\com:=(a,z))$ such that its decommitment using the key $k_1$ is different from its decommitment using $\fdecom$. Since $\games_\prevgame$ and $\games_\thisgame$ are identical until $\event$ happens, we bound the probability of $\event$. Concretely, we have

\[
\left|\Pr[\games_\prevgame = 1] - \Pr[\games_\thisgame = 1]\right| \leq \Pr[\event]. 
\]

We construct an adversary $\advB$ that breaks soundness with respect to auxiliary input $\aux:=(p,q)$ of the Sigma protocol.  W.l.o.g. we assume that whenever event $\event$ happens, $\adv$ previously asked the random oracle $H$ on input $(h_1,h2,c_0,c_1,c_2,a)$. The argument for this is that it is straightforward to transform any adversary that violates this condition into an adversary that makes one additional query to $H$ and wins with the same probability. Let $\adv$ makes at most $Q$ random oracle queries. 
The adversary $\advB_{\secpar}(p,q)$ proceeds as follows:
\vspace{-2mm}
\begin{enumerate}
\item Samples $k_1\rand \smplsetqrn$, computes $h_1 := g^{k_1} \bmod N,\varphi(N):=(p-1)(q-1), t:=2^{T} \bmod \ordqrn, h_2:=g^t \bmod N$ and sets $\crs:=(N, T(\secpar), g, h_1, h_2)$. 
\item Then it runs $(m_0, m_1, \st) \leftarrow \adv_{1, \secpar}(\crs)$ and answers decommitment queries using $k_1$.
\item Samples $i \rand [Q]$ and answers random oracle queries $(h_1^j, h_2^j, c_0^j,c_1^j,c_2^j,a^j)$ in the following way:
\begin{itemize}
\item If $i=j$ it runs the protocol with the honest verifier $\vrf$ for statement $(h_1^j, h_2^j, c_0^j,c_1^j,c_2^j)$ using as a commitment value $a^j$. It obtains challenge $v$ from $\vrf$ and it programs the oracle $H(h_1^j, h_2^j,c_0^j,c_1^j,c_2^j,a^j):=v$ if . Answer the query with $v$.
\item Otherwise, it returns $H(h_1^j, h_2^j,c_0^j,c_1^j,c_2^j,a^j)$ if it is set. If this is not the case samples $v_j$, sets $H(h_1^j, h_2^j,c_0^j,c_1^j,c_2^j,a^j):=v_j$ and returns $v_j$.
\end{itemize}
\item It samples $b \rand \bits, r \rand \smplsetqrn$ and computes $c_0^*:=g^r, c_1^*:=h_1^{r}m_b, c_2^*:=h_2^{r}m_b$. It sets $(s:=\mathlist(h_1, h_2,c_0^*, c_1^*, c_2^*), w:=(m,r))$ and produces honest proof $\pi^*$.
\item Next, it runs $b' \leftarrow \adv_{2, \secpar}((c_0^*, c_1^*, c_2^*), \pi^*, \st)$ and answers decommitment queries using $k_1$.
\item Finally, it checks whether there exists a decommitment query $(c: = \mathlist(c_0, c_1, c_2), \pi_\com=(a,z))$ such that $\deco(\crs, c, \pi_\com) \neq \dec(\crs,c,\pi_\com, 1, k_1)$. This check can be done efficiently with the knowledge of $t$. If $\event$ occurs and $\advB_\secpar$ has guessed index $i$ correctly, then value $z$ allows $\advB_\secpar$ succeed in the attack game. %Notice that this can be done efficiently with the knowledge of $t$.
\end{enumerate}

Suppose that the query $(h_1, h_2,c_0,c_1,c_2,a)$ has been asked to the random oracle as $i^*$-th query and that $i=i^*$. Then it holds
\[\snd^\Sigma_\advB = \Pr[\event \land i=i^*] = \Pr[\event]\Pr[i=i^*] = \frac{1}{Q}\Pr[\event],\]
where the first equality holds since the events are independent.
%$\advB$ simulates $\games_\thisgame$ perfectly and if the event $\event$ happens, then it outputs a valid proof for a statement which is not in the specified language $L$. Therefore we get
%\[\Pr[\event] \leq \snd^\nizk_\advB.\]

%===========================================================Soundness==================================================
%We construct an adversary $\advB$ that breaks soundness with respect to auxiliary input $\aux:=(p,q)$ of the NIZK. 
%The adversary $\advB_{\secpar}(p,q)$ proceeds as follows:
%\vspace{-2mm}
%\begin{enumerate}
%\item Samples $k_1\rand \smplsetqrn$, computes $h_1 := g^{k_1} \bmod N,\varphi(N):=(p-1)(q-1), t:=2^{T} \bmod \ordqrn, h_2:=g^t \bmod N$ and sets $\crs:=(N, T(\secpar), g, h_1, h_2)$. 
%\item Then it runs $(m_0, m_1, \st) \leftarrow \adv_{1, \secpar}(\crs)$ and answers decommitment queries using $k_1$.
% It samples $b \rand \bits, r \rand \smplsetqrn$ and computes $c_0^*:=g^r, c_1^*:=h_1^{r}m_b, c_2^*:=h_2^{r}m_b$. It sets $(s:=\mathlist(h_1, h_2,c_0^*, c_1^*, c_2^*), w:=(m,r))$ and runs $\pi^* \leftarrow \nizk.\prove(s,w)$.
%\item Next, it runs $b' \leftarrow \adv_{2, \secpar}((c_0^*, c_1^*, c_2^*), \pi^*, \st)$ and answers decommitment queries using $k_1$.
%\item Finally, it checks whether there exists a decommitment query $c: = \mathlist(c_0, c_1, c_2, \pi)$ such that $\fdecom(\crs, c) \neq \dec(\crs,c,k_1,1)$. If $\event$ occurs, then this is the case, and it returns $((h_1, h_2, c_0, c_1, c_2), \pi)$. Notice that this can be done efficiently with the knowledge of $t$.
%\end{enumerate}
%
%$\advB$ simulates $\games_\thisgame$ perfectly and if the event $\event$ happens, then it outputs a valid proof for a statement which is not in the specified language $L$. Therefore we get
%\[\Pr[\event] \leq \snd^\nizk_\advB.\]
%====================================================================================================================


%\begin{lemma}\label{nitc-rom-mh:flem}
%\[
%\Pr[\games_\prevgame = 1] = \Pr[\games_\thisgame = 1].
%\]
%\end{lemma}
%
%Notice that both $\fdecom$ and $\decom$ answer decommitment queries in the exactly same way, hence the change is only syntactical. 



%Notice that if $c_1$ and $c_3$ contain the same message, both oracles answer decommitment queries consistently. Let $\event$ denote the event that the adversary $\adv$ asks a decommitment query $c$ such that its decommitment using the key $k_1$ is different from its decommitment using $\fdecom$. Since $\games_\prevgame$ and $\games_\thisgame$ are identical until $\event$ happens, we bound the probability of $\event$. Concretely, we have
%
%\[
%\left|\Pr[\games_\prevgame = 1] - \Pr[\games_\thisgame = 1]\right| \leq \Pr[\event]. 
%\]
%
%We construct an adversary $\advB$ that breaks soundness of the NIZK. 
%The adversary $\advB_{\secpar}$ proceeds as follows:
%\vspace{-2mm}
%\begin{enumerate}
%\item It computes $\crs \leftarrow \pgen(\seck, T)$ as defined in the construction where the value $h_3$ is computed using repeated squaring instead.
%\item Then it runs $(m_0, m_1, \st) \leftarrow \adv_{1, \secpar}(\crs)$ and answers decommitment queries using $k_1$.
% It samples $b \rand \bits, r \rand \smplset$ and computes $c_0^*:=g^r, c_1^*:=h_1^{r}m_b, c_2^*:=h_2^{r}m_b, c_3^*:=h_3^{r}m_b$. It sets $(s:=\mathlist(c_0^*, c_1^*, c_2^*, c_3^*), w:=(m,r))$ and runs $\pi^* \leftarrow \nizk.\prove(s,w)$.
%\item Next, it runs $b' \leftarrow \adv_{2, \secpar}((c_0^*, c_1^*, c_2^*, c_3^*), \pi^*, \st)$ and answers decommitment queries using $k_1$.
%\item Finally, it checks whether there exists a decommitment query $c: = \mathlist(c_0, c_1, c_2, c_3, \pi)$ such that $\fdecom(\crs, c) \neq \dec(\crs,c,1)$. If $\event$ occurs, then this is the case, and it returns $((h_1, h_2, c_0, c_1, c_2, c_3), \pi)$. %Notice that this can be done efficiently with the knowledge of $t$.
%\end{enumerate}
%
%$\advB$ simulates $\games_\thisgame$ perfectly and if the event $\event$ happens, then it outputs a valid proof for a statement which is not in the specified language $L$. Therefore we get
%\[\Pr[\event] \leq \snd^\nizk_\advB.\]

%Let $\gnr$ denote the event that the sampled $g$ in $\kgen$ is a generator of $\qrn$. Recall that $N = pq$ where $p = 2p'+1$ and $q = 2q'+1$. Because $g$ is sampled uniformly at random and $\qrn$ has $\varphi(|\qrn|) = (p'-1)(q'-1)$ generators, this event happens with overwhelming probability. Concretely, $\Pr[\gnr] = 1-\frac{1}{p'}-\frac{1}{q'}+\frac{1}{p'q'}$.
%Therefore the following holds.
%\begin{lemma}\label{tpke3:flem} 
%\begin{align*}
%\Pr[\games_\thisgame = 1] &= \Pr[\games_\thisgame = 1| \gnr]\Pr[\gnr] + \Pr[\games_\thisgame = 1| \overline{\gnr}]\Pr[\overline{\gnr}] \\
%&\leq \Pr[\games_\thisgame = 1| \gnr]\Pr[\gnr] + \Pr[\overline{\gnr}] \\
%&= \Pr[\games_\thisgame = 1| \gnr]\left( 1-\frac{1}{p'}-\frac{1}{q'}+\frac{1}{p'q'} \right) + \frac{1}{p'}+\frac{1}{q'}-\frac{1}{p'q'}.
%\end{align*}
%\end{lemma}




\nextgame{SimulProof}
Game $\games_\thisgame$ proceeds exactly as the previous game but we use the HVZK simulator $\simul$ to produce a simulated proof for the challenge commitment $ (h_1, h_2, c_0^*,c_1^*,c_2^*)$ and upon receiving simulated transcript $(a^*,v^*,z^*)$ from the simulator we try to program the random oracle in the following way: if $H\mathlist(h_1, h_2, c_0^*,c_1^*,c_2^*, a^*)=\bot$ then $H(h_1, h_2, c_0^*,c_1^*,c_2^*, a^*):=v^*$ and setting $\pi^*=(a^*,z^*)$. Now notice that, since the transcripts have the exactly same distributions, the games proceeds exactly the same until our programming of random oracle is successful. Let denote by $\event$ the event that we are not able to set correct value for $H$. $\event$ happens only in the case that adversary already asked a random oracle query at the point $H(h_1, h_2, c_0^*,c_1^*,c_2^*, a^*)$. Let $Q$ is the number of random oracle queries asked by $\adv$. Since the Sigma protocol has $\delta$-unpredictable commitments the probability of this event is by union bound  is less than $Q\delta$. Hence we obtain
\begin{lemma}
\[
\left|\Pr[\games_\prevgame = 1] - \Pr[\games_\thisgame = 1]\right| \leq Q\delta.
\]
\end{lemma}


%We construct an adversary $\advB = \{\advB_\secpar\}_{\secpar \in \N}(p,q)$ against honest verifier zero-knowledge security of the Sigma protocol as follows:
%\vspace{-2mm}
%\begin{enumerate}
%\item Samples $k_1\rand \smplsetqrn$, computes $h_1 := g^{k_1} \bmod N,\varphi(N):=(p-1)(q-1), t:=2^{T} \bmod \ordqrn, h_2:=g^t \bmod N$ and sets $\crs:=(N, T(\secpar), g, h_1, h_2)$. 
%\item Runs $(m_0, m_1, \st) \leftarrow \adv_{1, \secpar}(\crs)$ and answers decommitment queries using $k_1$.
%\item It answer random oracle queries $(h_1^j, h_2^j, c_0^j,c_1^j,c_2^j,a_0^j,a_1^j)$ in the following way:
%if $H(h_1^j, h_2^j,c_0^j,c_1^j,c_2^j,a)\neq \bot$ return $H(h_1^j, h_2^j,c_0^j,c_1^j,c_2^j,a)$ . If this is not the case samples $v_j$ uniformly at random from the challenge space, sets $H(h_1^j, h_2^j,c_0^j,c_1^j,c_2^j,a_0^j,a_1^j):=v_j$ and returns $v_j$.
%\item Samples $b \rand \bits, r \rand \smplset$ and computes $c_0^*:=g^r, c_1^*:=h_1^{r}m_b, c_2^*:=h_2^{r}m_b$. It submits $(s:=\mathlist(h_1,h_2,c_0^*, c_1^*, c_2^*), w:=(m,r))$ to its oracle and obtains transcript  as answer.
%\item Runs $b' \leftarrow \adv_{2, \secpar}((c_0^*, c_1^*, c_2^*), \pi^*, \st)$ and answers decommitment queries using $k_1$.
%\item Returns the truth value of $b=b'$.
%\end{enumerate}
%If the proof $\pi^*$ is generated using $\nizk.\prove$, then $\advB$ simulates $\games_\prevgame$ perfectly. Otherwise $\pi^*$ is generated using $\simul_1$ and $\advB$ simulates $\games_\thisgame$ perfectly. This proofs the lemma.

%===============================================ZK======================================================================
%Game $\games_\thisgame$ proceeds exactly as the previous game but we use the zero-knowledge simulator $(\pi, \st) \leftarrow \simul(1, \st, (h_1, h_2, c_0^*,c_1^*,c_2^*))$ to produce a simulated proof for the challenge commitment and $\simul(2, \st, \cdot)$ to answer random oracle queries. By zero-knowledge security of underlying NIZK we directly obtain
%\begin{lemma}
%\[
%\left|\Pr[\games_\prevgame = 1] - \Pr[\games_\thisgame = 1]\right| \leq \zk^\nizk_\advB.
%\]
%\end{lemma}
%
%
%We construct an adversary $\advB = \{\advB_\secpar\}_{\secpar \in \N}(p,q)$ against zero-knowledge security of NIZK as follows:
%\vspace{-2mm}
%\begin{enumerate}
%\item Samples $k_1\rand \smplsetqrn$, computes $h_1 := g^{k_1} \bmod N,\varphi(N):=(p-1)(q-1), t:=2^{T} \bmod \ordqrn, h_2:=g^t \bmod N$ and sets $\crs:=(N, T(\secpar), g, h_1, h_2)$. 
%\item Runs $(m_0, m_1, \st) \leftarrow \adv_{1, \secpar}(\crs)$ and answers decommitment queries using $k_1$.
%\item Samples $b \rand \bits, r \rand \smplset$ and computes $c_0^*:=g^r, c_1^*:=h_1^{r}m_b, c_2^*:=h_2^{r}m_b$. It submits $(s:=\mathlist(h_1,h_2,c_0^*, c_1^*, c_2^*), w:=(m,r))$ to its oracle and obtains proof $\pi^*$ as answer.
%\item Runs $b' \leftarrow \adv_{2, \secpar}((c_0^*, c_1^*, c_2^*), \pi^*, \st)$ and answers decommitment queries using $k_1$.
%\item Returns the truth value of $b=b'$.
%\end{enumerate}
%If the proof $\pi^*$ is generated using $\nizk.\prove$, then $\advB$ simulates $\games_\prevgame$ perfectly. Otherwise $\pi^*$ is generated using $\simul_1$ and $\advB$ simulates $\games_\thisgame$ perfectly. This proofs the lemma.
%============================================================================================================================

\nextgame{RndExp}
In $\games_\thisgame$ we sample $r$ uniformly at random from $[\ordqrn]$. 
\begin{lemma}
\[
\left|\Pr[\games_\prevgame = 1] - \Pr[\games_\thisgame = 1]\right| \leq \frac{1}{p}+\frac{1}{q}-\frac{1}{N}.
\]
\end{lemma}
Since the only difference between the two games is in the set from which we sample $r$, to upper bound the advantage of adversary we can use \Cref{sampling-lemma} with $\ell:=4$, which directly yields required upper bound.

\nextgame{SSSA}
In $\games_\thisgame$ we sample $y_2 \rand \qrn$ and compute $c_2^*$ as $y_2 m_b$.

Let $\tilT_\sss(\secpar)$ be the polynomial whose existence is guaranteed by the SSS assumption.
Let $\poly_\advB(\secpar)$ be the fixed polynomial which bounds the time required to execute Steps 1--2 and answer decommitment queries in Step 3 of the adversary $\advB_{2, \secpar}$ defined below. Set $\undT := (\poly_\advB(\secpar))^{1 / \ugap}$.  Set $\tilT_\nitc := \max(\tilT_\sss, \undT)$.
\begin{lemma}
From any polynomial-size adversary $\adv = \{(\adv_{1,\secpar}, \adv_{2, \secpar})\}_{\secpar \in \nats}$, where depth of $\adv_{2, \secpar}$ is at most $T^{\ugap}(\secpar)$ for some $T(\cdot) \geq \undT(\cdot)$ we can construct a polynomial-size adversary $\advB = \{(\advB_{1,\secpar}, \advB_{2, \secpar})\}_{\secpar \in \nats}$ where the depth of $\advB_{2, \secpar}$ is at most $T^{\gap}(\secpar)$ with
\[
\left|\Pr[\games_\prevgame = 1] - \Pr[\games_\thisgame = 1]\right| \leq \advtg_\advB^\sss.
\]
\end{lemma}

The adversary $\advB_{1,\secpar}(N, T(\secpar), g):$
\vspace{-2mm}
\begin{enumerate}
\item Samples $k_1 \rand \smplsetqrn$, computes $h_1 := g^{k_1} \bmod N, h_2 := g^{2^{T(\secpar)}} \bmod N$ and sets $\crs:=(N, T(\secpar), g, h_1, h_2)$. Notice that value $h_2$ is computed by repeated squaring.
\item Runs $(m_0, m_1, \st) \leftarrow \adv_{1, \secpar}(\crs)$ and answers decommitment queries using $k_1$. Random oracle queries are answered as before. 
\item Outputs $(N,g,k_1, h_1,h_2, m_0, m_1, \st)$
\end{enumerate}

The adversary $\advB_{2,\secpar}(x,y,(N,g,k_1, h_1,h_2, m_0, m_1, \st)):$
\vspace{-2mm}
\begin{enumerate}
\item Samples $b \rand \bits$, computes $c_0^*:=x, c_1^*:=x^{k_1} m_b, c_2^*:=y m_b$.
\item Runs $(a^*,v^*,z^*) \leftarrow \simul(h_1, h_2, c_0^*, c_1^*, c_2^*)$. If $H(h_1, h_2, c_0^*, c_1^*, c_2^*,a^*) = \bot$ sets $H(h_1, h_2, c_0^*, c_1^*, c_2^*,a^*) = v^*$. Sets $\pi^* = (a^*,z^*)$.
\item Runs $b' \leftarrow \adv_{2, \secpar}((c_0^*, c_1^*, c_2^*), \pi^*, \st)$ and answers decommitment queries using $k_1$. Random oracle queries are answered as before. 
\item Returns the truth value of $b=b'$.
\end{enumerate}
Since $g$ is a generator of $\qrn$ and $x$ is sampled uniformly at random from $\qrn$ there exists some $r \in [\varphi(N)/2]$ such that $x = g^{r}$. Therefore when $y = x^{2^T} = (g^{2^T})^{r} \bmod N$, then $\advB$ simulates $\games_\prevgame$ perfectly. Otherwise $y$ is random value and $\advB$ simulates $\games_\thisgame$ perfectly. We remark that at this point $c_2^*$ does not reveal any information about $m_b$.

Now we analyse the running time of the constructed adversary. Adversary $\advB_1$ computes $h_3$ by $T(\secpar)$ consecutive squarings and because $T(\secpar)$ is polynomial in $\secpar$, $\advB_1$ is efficient. Moreover, $\advB_2$ fulfils the depth constraint:
\[ \dep(\advB_{2,\secpar}) = \poly_\advB(\secpar) + \dep(\adv_{2,\secpar}) \leq \undT^{\ugap}(\secpar) + T^{\ugap}(\secpar) \leq 2 T^{\ugap}(\secpar) = o(T^{\gap}(\secpar)). \] 

Also $T(\cdot) \geq \tilT_\nitc(\cdot) \geq \tilT_\sss(\cdot)$ as required.  


\nextgame{QUR}
Let $(c^*, \pi^*_com=(a^*,z^*))$ is the challenge commitment. In $\games_\thisgame$ we abort experiment if adversary asks any decommitment query $(c,\pi_\com=(a,z))$ such that $c=c^*, a = a^*$ and $z \neq z^*$. We denote this event by $\event$.
\begin{lemma}
\[
\left|\Pr[\games_\prevgame = 1] - \Pr[\games_\thisgame = 1]\right| \leq \qur^\sigma_\advB.
\]
\end{lemma}


Since $\games_\prevgame$ and $\games_\thisgame$ are identical until $\event$ does not happen, by the standard argument it is sufficient to upper bound the probability of happening $\event$. Concretely,  

\[
\left|\Pr[\games_\prevgame = 1] - \Pr[\games_\thisgame = 1]\right| \leq \Pr[\event]. 
\]

We construct an adversary $\advB$ which breaks quasi unique responses property of the Sigma protocol. 
The adversary $\advB_{\secpar}$ proceeds as follows:
\vspace{-2mm}
\begin{enumerate}
\item Samples $k_1\rand \smplsetqrn$, computes $h_1 := g^{k_1} \bmod N, h_2:=g^{2^T} \bmod N$ and sets $\crs:=(N, T(\secpar), g, h_1, h_2)$. Notice that $h_2$ is computed by repeated squaring. 
\item Then it runs $(m_0, m_1, \st) \leftarrow \adv_{1, \secpar}(\crs)$ and answers decommitment queries using $k_1$. Random oracle queries are answered as before. 
\item Samples $b \rand \bits$, computes $c_0^*:=x, c_1^*:=x^{k_1} m_b, c_2^*:=y m_b$.
\item Runs $(a^*,v^*,z^*) \leftarrow \simul(h_1, h_2, c_0^*, c_1^*, c_2^*)$. If $H(h_1, h_2, c_0^*, c_1^*, c_2^*,a^*) = \bot$ sets $H(h_1, h_2, c_0^*, c_1^*, c_2^*,a^*) = v^*$. Sets $\pi^* = (a^*,z^*)$.
\item Runs $b' \leftarrow \adv_{2, \secpar}((c_0^*, c_1^*, c_2^*), \pi^*, \st)$ and answers decommitment queries using $k_1$. Random oracle queries are answered as before. 
\item Returns the truth value of $b=b'$.
\item Finally, it checks whether there is a decommitment query $(c,\pi=(a,z))$ such that $c= c^*, a = a^*$ and $z \neq z^*$. If $\event$ occurs, then this is the case, and it returns $((h_1, h_2, c_0^*, c_1^*, c_2^*),a^*,v^*,z^*,z)$ where $v^* = H(h_1, h_2, c_0^*, c_1^*, c_2^*,a^*)$.
\end{enumerate}

Notice, that if $\event$ happens that $\advB_\secpar$ indeed provides two transcripts which has different responses which yields
\[\Pr[\event] \leq \qur^\Sigma_\advB.\]

\nextgame{Snd2}
In game $\games_\thisgame$ decommitment queries are answered using the algorithm $\decom$ defined in \Cref{fig:deco-rom-mh} with $i:=2, \sk:=t$ which means that secret key $t$ and ciphertext $c_2$ are used, to answer decommitment queries efficiently. 

\begin{lemma}
\[
\left|\Pr[\games_\prevgame = 1] - \Pr[\games_\thisgame = 1]\right| \leq Q\cdot\snd^\Sigma_\advB.
\]
\end{lemma}

Let $\event$ denote the event that adversary $\adv$ asks a decommitment query $(c, \pi_\com=(a_0,a_1,z))$ such that its decommitment using the key $k_1$ is different from its decommitment using the key $t$. Since $\games_\prevgame$ and $\games_\thisgame$ are identical until $\event$ does not happen, by the standard argument it is sufficient to upper bound the probability of happening $\event$. Concretely,  

\[
\left|\Pr[\games_\prevgame = 1] - \Pr[\games_\thisgame = 1]\right| \leq \Pr[\event]. 
\]


We construct an adversary $\advB$ that breaks soundness with respect to auxiliary input $\aux:=(p,q)$ of the Sigma protocol. W.l.o.g. we assume that whenever event $\event$ happens, $\adv$ previously asked the random oracle $H$ on input $(h_1,h_2,c_0,c_1,c_2,a)$. The argument for this is that it is straightforward to transform any adversary that violates this condition into an adversary that makes one additional query to $H$ and wins with the same probability. Let $\adv$ makes at most $Q$ random oracle queries. 
The adversary $\advB_{\secpar}(p,q)$ proceeds as follows:
\vspace{-2mm}
\begin{enumerate}
\item Samples $k_1\rand \smplsetqrn$, computes $h_1 := g^{k_1} \bmod N,\varphi(N):=(p-1)(q-1), t:=2^{T} \bmod \ordqrn, h_2:=g^t \bmod N$ and sets $\crs:=(N, T(\secpar), g, h_1, h_2)$. 
\item Then it runs $(m_0, m_1, \st) \leftarrow \adv_{1, \secpar}(\crs)$ and answers decommitment queries using $k_1$.
\item Samples $i \rand [Q]$ and answers random oracle queries $(h_1^j, h_2^j, c_0^j,c_1^j,c_2^j,a^j,)$ in the following way:
\begin{itemize}
\item If $i=j$ it runs the protocol with the honest verifier $\vrf$ for statement $(h_1^j, h_2^j, c_0^j,c_1^j,c_2^j)$ using as a commitment value $a^j$. It obtains challenge $v$ from $\vrf$ and it programs the oracle $H(h_1^j, h_2^j,c_0^j,c_1^j,c_2^j,a^j):=v$. Answer the query with $v$.
\item Otherwise, if $H(h_1^j, h_2^j,c_0^j,c_1^j,c_2^j,a^j) \neq \bot$ returns $H(h_1^j, h_2^j,c_0^j,c_1^j,c_2^j,a^j)$. Else it samples $v_j$, sets $H(h_1^j, h_2^j,c_0^j,c_1^j,c_2^j,a^j):=v_j$ and returns $v_j$.
\end{itemize}
\item Samples $b \rand \bits, x, y_2 \rand \qrn$ and computes $(c_0^*, c_1^*, c_2^*):=(x, x^{k_1} m_b,\allowbreak y_2 m_b)$. 
\item Runs $(a^*,v^*,z^*) \leftarrow \simul(h_1, h_2, c_0^*, c_1^*, c_2^*)$. If $H(h_1, h_2, c_0^*, c_1^*, c_2^*,a^*) = \bot$ sets $H(h_1, h_2, c_0^*, c_1^*, c_2^*,a^*) = v^*$. Sets $\pi^* = (a^*,z^*)$.
%Run the HVZK simulator on $(h_1, h_2, c_0^*, c_1^*, c_2^*)$ to to obtain an accepting transcript $(a,v,z)$ and in case that $H(h_1, h_2, c_0^*, c_1^*, c_2^*,a)$ is already defined with value different to $v$ it aborts the experiment, otherwise it programs the oracle $H(h_1, h_2, c_0^*, c_1^*, c_2^*,a):=v$ and outputs $(a,z)$.
\item Runs $b' \leftarrow \adv_{2, \secpar}((c_0^*, c_1^*, c_2^*), \pi^*, \st)$ and answers decommitment queries using $k_1$ and random oracle queries in the same way as before.
\item Finally, it checks whether there exists a decommitment query $(c, \pi_\com=(a,z))$ such that $\dec(\crs, c, \pi_\com, 1, k_1) \neq \dec(\crs,c, \pi_\com, 2 ,t)$. If $\event$ occurs, then this is the case and it returns $z$ as a response to the challenge $v$. 
\end{enumerate}

Suppose that the query $(h_1, h_2,c_0,c_1,c_2,a)$ has been asked to the random oracle as $i^*$-th query and that $i=i^*$. Then $\advB_\secpar$ guesses a correct index for which $\adv$ asks decommitment query for a statement which is not in the specified language $L$. Therefore we get
\[\snd^\Sigma_\advB = \Pr[\event \land i=i^*] = \Pr[\event]\Pr[i=i^*] = \frac{1}{Q}\Pr[\event],\]
where the first equality holds since the events are independent.


\nextgame{RndExp4}
In $\games_\thisgame$ we sample $k_1$ uniformly at random from $[\ordqrn]$.

\begin{lemma}
\[
\left|\Pr[\games_\prevgame = 1] - \Pr[\games_\thisgame = 1]\right| \leq \frac{1}{p}+\frac{1}{q}-\frac{1}{N}.
\]
\end{lemma}

This lemma directly follows from \Cref{sampling-lemma} with $\ell:=4$.

\nextgame{DDH1}
In $\games_\thisgame$ we sample $y_1 \rand \qrn$ and compute $c_1^*$ as  $y_1 m_b$. 

\begin{lemma}
\[
\left|\Pr[\games_\prevgame = 1] - \Pr[\games_\thisgame = 1]\right| \leq \advtg_\advB^\ddh.
\]
\end{lemma}
We construct an adversary $\advB = \{\advB_\secpar\}_{\secpar \in \N}$ against DDH in the group $\qrn$. %Given \Cref{thm:ddh} this implies an adversary against DDH in large prime-order subgroups of $\Zn^*$.

$\advB_{\secpar}(N,p,q,g,g^\alpha, g^\beta, g^\gamma):$
\vspace{-2mm}
\begin{enumerate}
\item Computes $\varphi(N):=(p-1)(q-1), t:=2^{T} \bmod \varphi(N)/4, h_2:=g^t \bmod N$ and sets $\crs:=(N, T, g, h_1: = g^\alpha, h_2)$.
\item Runs $(m_0, m_1, \st) \leftarrow \adv_{1, \secpar}(\crs)$ and answers decommitment queries using $t$. Random oracle queries are answered as before.
\item Samples $b \rand \bits$ and computes $(c_0^*, c_1^*, c_2^*):=(g^\beta, g^\gamma \cdot m_b, (g^\beta)^t \cdot m_b).$ 
\item Runs $(a^*,v^*,z^*) \leftarrow \simul(h_1, h_2, c_0^*, c_1^*, c_2^*)$. If $H(h_1, h_2, c_0^*, c_1^*, c_2^*,a^*) = \bot$ sets $H(h_1, h_2, c_0^*, c_1^*, c_2^*,a^*) = v^*$. Sets $\pi^* = (a^*,z^*)$.
\item Runs $b' \leftarrow \adv_{2, \secpar}((c_0^*, c_1^*, c_2^*), \pi^*, \st)$ and answers decommitment queries using $t$. Random oracle queries are answered as before.
\item Returns the truth value of $b=b'$. We remark that at this point $c_1^*$ does not reveal any information about $m_b$.
\end{enumerate}
If $\gamma = \alpha\beta$ then $\advB$ simulates $\games_\prevgame$ perfectly. Otherwise $g^\gamma$ is uniform random element in $\qrn$ and $\advB$ simulates $\games_\thisgame$ perfectly. This proofs the lemma. We remark that at this point $c_1^*$ does not reveal any information about $m_b$.


\begin{lemma}\label{nitc-rom-mh:llem}
\[
\Pr[\games_\thisgame = 1] = \half.
\]
\end{lemma}

Clearly, $c_0^*$ is uniform random element in $\qrn$ and hence it does not contain any information about the challenge message. Since $y_1, y_2$ are sampled uniformly at random from $\qrn$ the ciphertexts $c_1^*, c_2^*$ are also uniform random elements in $\qrn$ and hence do not contain any information about the challenge message $m_b$. Therefore, an adversary can not do better than guessing.

By combining Lemmas \ref{nitc-rom-mh:flem} - \ref{nitc-rom-mh:llem} we obtain the following:
\begin{align*}
&\advtg^{\nitc}_{\adv} = \left| \Pr[\games_0 = 1] - \half \right| \leq \sum_{i=0}^7 \left|\Pr[\games_i = 1] - \Pr[\games_{i+1} = 1] \right| + \left|\Pr[\games_{8}- \half\right| \\
 &\leq 2Q \cdot \snd^\Sigma_\advB + Q\delta + \advtg^{\sss}_{\advB} + \qur^\Sigma_\advB + \advtg^{\ddh}_{\advB} + 2 \left( \frac{1}{p}+\frac{1}{q}-\frac{1}{N} \right),
\end{align*}
which concludes the proof.
\end{proof}
% \todo{appendix end}

\begin{theorem}
$(\pgen, \com, \cvrfy, \dvrfy, \fdecom)$ defined in \Cref{table:nitc-mh-rom} is a BND-CCA-secure non-interactive timed commitment scheme. 
\end{theorem}

\begin{proof}
This can be proven in the same way as \Cref{bnd-cca-mh}.
\end{proof}

It is straightforward to verify that considering $\eval$ algorithm, our construction yields multiplicatively homomorphic NITC. 
\begin{theorem}\label{hom-mh-rom}
The NITC \mathlist{(\pgen, \com, \cvrfy, \dvrfy, \fdecom, \fdvrfy, \eval)} defined in \Cref{table:nitc-mh-rom} is a multiplicatively homomorphic non-interactive timed commitment scheme.
\end{theorem}


\begin{remark}[Public Verifiability]
The construction can be made publicly verifiable in the same way as suggested in \Cref{rem:pv}. 
 \end{remark}

%\begin{figure}[h!]
%\begin{center}
%\begin{tabular}{|ll|}
%\hline
%$\underline{\kgen(\seck, T)}$ 							   & $\underline{\decf(\sk, c)}$\\
%$(p, q_, N) \leftarrow \genmod(\seck)$ &  Parse $c$ as $(c_0, c_1, c_2, \pi)$\\
%$\varphi(N):= (p-1)(q-1)$   & if $\nizk.\vrfy((c_0, c_1, c_2), \pi)= 1$ \\
%$g\rand \qrn$ & \tab Compute $y_1:= c_0^{k} \bmod N$\\
%$k \rand \varphi(N)/4$ & \tab return $c_1 \cdot H(y_1)^{-1} \bmod N$ \\
%$t:= 2^T \bmod \varphi(N)/4$ & return $\bot$\\
%$h_1:= g^k \bmod N$ & \\
%$h_2:=g^{t} \bmod N$ & \\
%%$\crs \leftarrow \nizk.\setup(\seck)$ & \\
%$\pk:= (N,T,g,h_1,h_2), \sk:= (N, k)$ & \\
%return $(\pk, \sk)$       & \\
%                                             &\\
%$\underline{\enc(\pk, m)}$           & $\underline{\decs(\pk,c)}$ \\
%$r \rand [\floor{N/4}]$     & Parse $c$ as $(c_0, c_1, c_2, \pi)$ \\
%Compute $c_0:= g^r \bmod N$ & if $\nizk.\vrfy((c_0, c_1, c_2), \pi)= 1$\\
%For $i \in [2]: y_i:= h_i^r \bmod N$   &   \tab Compute $ y_2:=c_0^{2^T} \bmod N$ \\
%For $i \in [2]: c_i:= H(y_i) \cdot m \bmod N$ & \tab return $c_2 \cdot H(y_2)^{-1} \bmod N$ \\
%$\Phi := (c_0, c_1, c_2), w := (m, r)$& return $\bot$\\
%$\pi \leftarrow \nizk.\prove(\Phi, w)$  &  \\
%return $c \leftarrow (c_0, c_1, c_2, \pi)$ &  \\
%
%\hline          
%\end{tabular}
%\caption{NY Construction of TPKE from SSSA}
%\label{table:tpke-elgamal}
%\end{center}
%\end{figure}

 

%\begin{figure}[h!]
%\begin{center}
%\begin{tabular}{|ll|}
%\hline
%$\underline{\kgen(\seck, T)}$ 							   & $\underline{\decf(\sk, c)}$\\
%$(p, q_, N) \leftarrow \genmod(\seck)$ &  Parse $c$ as $(c_0, c_1, c_2, \pi)$\\
%$\varphi(N):= (p-1)(q-1)$   & if $\nizk.\vrfy((c_0, c_1, c_2), \pi)= 1$ \\
%$g\rand \qrn$ & \tab Compute $y_1:= c_0^{k} \bmod N$\\
%$k \rand \varphi(N)/4$ & \tab return $c_1 \cdot H(y_1)^{-1} \bmod N$ \\
%$t:= 2^T \bmod \varphi(N)/4$ & return $\bot$\\
%$h_1:= g^k \bmod N$ & \\
%$h_2:=g^{t} \bmod N$ & \\
%%$\crs \leftarrow \nizk.\setup(\seck)$ & \\
%$\pk:= (N,T,g,h_1,h_2), \sk:= (N, k)$ & \\
%return $(\pk, \sk)$       & \\
%                                             &\\
%$\underline{\enc(\pk, m)}$           & $\underline{\decs(\pk,c)}$ \\
%$r \rand [\floor{N/4}]$     & Parse $c$ as $(c_0, c_1, c_2, \pi)$ \\
%Compute $c_0:= g^r \bmod N$ & if $\nizk.\vrfy((c_0, c_1, c_2), \pi)= 1$\\
%For $i \in [2]: y_i:= h_i^r \bmod N$   &   \tab Compute $ y_2:=c_0^{2^T} \bmod N$ \\
%For $i \in [2]: c_i:= H(y_i) \cdot m \bmod N$ & \tab return $c_2 \cdot H(y_2)^{-1} \bmod N$ \\
%$\Phi := (c_0, c_1, c_2), w := (m, r)$& return $\bot$\\
%$\pi \leftarrow \nizk.\prove(\Phi, w)$  &  \\
%return $c \leftarrow (c_0, c_1, c_2, \pi)$ &  \\
%
%\hline          
%\end{tabular}
%\caption{NY Construction of TPKE from SSSA}
%\label{table:tpke-elgamal}
%\end{center}
%\end{figure}




%%% Local Variables:
%%% mode: latex
%%% TeX-master: "main"
%%% End:



%%% Local Variables:
%%% mode: latex
%%% TeX-master: "main"
%%% End:






% To avoid hard rejects


% --- -----------------------------------------------------------------
% --- The Bibliography.
% --- -----------------------------------------------------------------
\bibliographystyle{alpha}
\bibliography{cryptobib/abbrev3,cryptobib/crypto,extrarefs}

% --- -----------------------------------------------------------------
% --- The Appendix.
% --- -----------------------------------------------------------------

\begin{appendix}

%%!TEX root=./main.tex
\section{Non-interactive zero-knowledge proofs in the Random Oracle Model}
Since we can build a very efficient NIZK proof system for our constructions in the random oracle model (ROM), we recall the definition of NIZKs in the ROM.  

\begin{definition} 
A \emph{non-interactive proof system} for an NP language $L$ with relation $\rel$ is a pair of algorithms $(\prove, \vrfy)$, which work as follows:
\begin{itemize}
\item $\pi \leftarrow \prove(s,w)$ is a PPT algorithm which takes as input a statement $s$ and a witness $w$ such that $(s,w) \in \rel$ and outputs a proof $\pi$.
\item $\vrfy(s, \pi) \in \{0,1\}$ is a deterministic algorithm which takes as input a statement $s$ and a proof $\pi$ and outputs either 1 or 0, where 1 means that the proof is ``accepted'' and 0 means it is ``rejected''.
\end{itemize}
We say that a non-interactive proof system is \emph{complete}, if for all $(s, w) \in \rel$ holds:
\[\Pr[\vrfy(s,\pi)=1:\pi \leftarrow \prove(s,w)] =1.\] 
\end{definition}

Next we define \emph{zero-knowledge} property for non-interactive proof system in the random oracle model. The simulator $\simul$ of a non-interactive zero-knowledge proof system is modelled as a stateful algorithm which has two modes: $(\pi, \st) \leftarrow \simul(1, \st, s)$  for answering proof queries and $(v, \st) \leftarrow \simul(2, \st, u)$ for answering random oracle queries. The common state $\st$ is updated after each operation.



\begin{definition}[Zero-Knowledge in the ROM]
Let $(\prove, \allowbreak \vrfy)$ be a non-interactive proof system for a relation $\rel$ which may make use of a hash function $H : \hdom \rightarrow \himg$. Let $\funs$ be the set of all functions from the set $\hdom$ to the set $\himg$. We say that $(\prove, \vrfy)$ is \emph{non-interactive zero-knowledge proof in the random oracle model (NIZK)}, if there exists an efficient simulator $\simul$ such that for all non-uniform polynomial-size adversaries $\adv = \{\adv_\secpar\}_{\secpar \in \nats}$ there exists a negligible function $\negl(\cdot)$ such that for all $\secpar \in \nats$ 
\[\zk_\adv^\nizk = 
\left| \Pr\left[ \adv_\secpar^{\prove^H(\cdot, \cdot), H(\cdot)} = 1 \right] -  \Pr\left[\adv_\secpar^{\simul_1(\cdot, \cdot), \simul_2(\cdot)} \right] = 1 \right|
\leq \negl(\secpar),
\]
where 
\begin{itemize}
\item $H$ is a function sampled uniformly at random from $\funs$,
\item $\prove^H$ corresponds to the $\prove$ algorithm, having oracle access to $H$,
\item $\pi \leftarrow \simul_1(s, w)$ takes as input $(s, w) \in \rel$, and outputs the first output of $(\pi, \st) \leftarrow \simul(1, \st, s)$,
\item $v \leftarrow \simul_2(u)$ takes as input $u \in \hdom$ and outputs the first output of $(v, \st) \leftarrow \simul(2, \st, u)$.
\end{itemize}
\end{definition}

\begin{definition}[One-Time Simulation Soundness]
Let $(\prove, \vrfy)$ be a non-interactive proof system for an NP language $L$ with zero-knowledge simulator $\simul$. We say that $(\prove, \vrfy)$ is \emph{one-time simulation sound} in the random oracle model, if for all non-uniform polynomial-size adversaries $\adv = \{\adv_\secpar\}_{\secpar \in \nats}$ there exists a negligible function $\negl(\cdot)$ such that for all $\secpar \in \nats$ 
\[
\simsnd_\adv^\nizk = \Pr\left[
\begin{aligned}
s \notin L \land (s, \pi) \neq (s', \pi') \\
\land \vrfy^{\simul_2(\cdot)}(s, \pi) = 1
\end{aligned}
:(s, \pi) \leftarrow \adv_\secpar^{\simul_1(\cdot), \simul_2(\cdot)} \right] \leq \negl(\secpar),
\]
where $\simul_1(\cdot)$ is a single query oracle which on input $s'$ returns the first output of $(\pi', \st) \leftarrow \simul(1, \st, s')$ and $\simul_2(u)$ returns the first output of $(v, \st) \leftarrow \simul(2, \st, u)$.
\end{definition}













%%% Local Variables:
%%% mode: latex
%%% TeX-master: "main"
%%% End:

%!TEX root=./main.tex
\section{Proof of  \Cref{sampling-lemma}}\label{proof:smpl-lemma}
\primelemma*


\begin{proof}
The following computation proves the lemma:
\begin{align*}
\sd(X, Y) = \half \sum_{r \in [\floor{N/\ell}]} \abs{\Pr[X = r] - \Pr[Y = r]} = \\ 
\half \left( \sum_{r = 1}^{\varphi(N)/\ell} \abs{\Pr[X = r] - \Pr[Y = r]} + \sum_{r = \varphi(N)/\ell+1}^{\floor{N/\ell}} \abs{\Pr[X = r] - \Pr[Y = r]} \right) = 
\\
\half \left( \sum_{r = 1}^{\varphi(N)/\ell} \abs{\frac{1}{\floor{N/\ell}} - \frac{\ell}{\varphi(N)}} + \sum_{r = \varphi(N)/\ell+1}^{\floor{N/\ell}} \abs{\frac{1}{\floor{N/\ell}} - 0} \right) \leq \\
\half \left( \sum_{r = 1}^{\varphi(N)/\ell} \abs{\frac{\ell}{N} - \frac{\ell}{\varphi(N)}} + \sum_{r = \varphi(N)/\ell+1}^{\floor{N/\ell}} \abs{\frac{1}{\floor{N/\ell}} - 0} \right) = \\
\half \left( \varphi(N)/\ell \abs{\frac{\ell(\varphi(N) - N)}{\varphi(N)N}} + (\floor{N/\ell}-\varphi(N)/\ell) \frac{1}{\floor{N/\ell}} \right) = \\
 \half \left(\frac{(N-\varphi(N))}{N} + 1 - \frac{\varphi(N)/\ell}{{\floor{N/\ell}}} \right) \leq 
\half \left(\frac{(N-\varphi(N))}{N} + 1 - \frac{\varphi(N)/\ell}{{N/\ell}} \right) = \\
 \half\frac{2(N-\varphi(N))}{N} = 
\frac{(N-(N-p-q+1))}{N} = \frac{p+q-1}{N} = \frac{1}{p}+\frac{1}{q}-\frac{1}{N}.
\end{align*}
\end{proof}



\section{Proof of Theorem \ref{thm:NITC-IND-Mul-Std}} % (fold)
\label{app:NITC-IND-Mul-Std}


%\todo{Put the proof into an appendix}
%\begin{proof}
Completeness is implied by the completeness of the NIZK and can be verified by inspection. 

%Our construction is based on the Naor-Yung paradigm where we combine three ElGamal-type ciphertext with shared randomness.  


%Similarly to \cite{SCN:BiaMasVen16} we define a PPT algorithm $\rerand$ in Figure~\ref{fig:rerand}, which takes two ciphertexts generated with independent randomness, both public keys, only one secret key (in our case $k$) and randomness which was used to encrypt a message using the public key $g, h_1$. 
%%We assume that modulus $N$ is implicitly known.\todo{Why not explicit?} 
%% which corresponds to the secret key which is given as the input. 
%
%\begin{figure}[tb]
%\centering
%\begin{minipage}{0.75\textwidth}
%$\underline{\rerand(c:= (g^{r}, h_1^{r}\cdot m), c':= (g^{r'}, h_2^{r'}\cdot m), N, h_1, h_2, k, r)}:$
%\vspace{-2mm}
%\begin{itemize}
%\item $c_0:= g^{r}\cdot{g^{r'}} = g^{r+r'} \bmod N$;
%\item $c_1:= (g^{r'})^k \cdot h_1^{r}\cdot m  =  h_1^{r'}\cdot h_1^{r}\cdot m = h_1^{r+r'}\cdot m \bmod N$;
%\item $c_2:=h_2^{r} \cdot h_2^{r'}\cdot m = h_2^{r+r'}\cdot m \bmod N$.
%\end{itemize}
%\end{minipage}
%\caption{\label{fig:rerand}Algorithm $\rerand$.}
%\end{figure}
%
%
%
%
%It is straightforward to see that the ciphertext returned by $\rerand$ is perfectly distributed to the ciphertext produced using a shared randomness where the pair $(c_0, c_1)$ encrypts a message $m$ and the pair $(c_0, c_2)$ encrypts message $m'$.


% We note that if we use value $2^T$ as a secret key, then in order to compute $c_2$ we have to execute $T$ repeated squarings, but since $T$ is polynomial in $\secpar$ this computations is considered to be efficient.

\newsequenceofgames{NITC-MH}
To prove security we define a sequence of games $\games_0 - \games_{8}$.  For $i \in \{0,1,\dots,8\}$ we denote by $\games_i = 1$ the event that the adversary $\adv = \{(\adv_{1,\secpar}, \adv_{2, \secpar})\}_{\secpar \in \nats}$ outputs $b'$ in the game $\games_i$ such that $b = b'$.
%In individual games we use the algorithm $\decom$ define in \Cref{fig:deco} to answer decommitment queries efficiently. 
\begin{figure}[h!]
\begin{center}
\begin{tabular}{|l|}
\hline
$\underline{\decom(\crs, c, \sk)}$\\
Parse $c$ as $(c_0, c_1, c_2, \pi)$\\
if $\nizk.\vrfy(\crs_\nizk,(c_0, c_1, c_2), \pi)= 1$\\
\tab Compute $y:= c_0^{\sk} \bmod N$\\
\tab return $c_i \cdot y^{-1} \bmod N$\\
return $\bot$\\
\hline          
\end{tabular}
\caption{Decommitment oracle}
\label{fig:deco-mh}
\end{center}
\end{figure}

\nextgame{G0}
Game $\games_\thisgame$ corresponds to the original security experiment where decommitment queries are answered using $\fdecom$.

\nextgame{DecOracle}
In game $\games_\thisgame$ decommitment queries are answered using the algorithm $\decom$ defined in \Cref{fig:deco-mh} with $i:=2, \sk:=t$ which means that secret key $t$ and ciphertext $c_2$ are used, to answer decommitment queries efficiently. 


\begin{lemma}\label{nitc-mh:flem}
\[
\Pr[\games_\prevgame = 1] = \Pr[\games_\thisgame = 1].
\]
\end{lemma}

Notice that both $\fdecom$ and $\decom$ answer decommitment queries in the exactly same way, hence the change is only syntactical. 


% if $c_1$ and $c_2$ contain the same message, both oracles answer decommitment queries consistently. Let $\event$ denote the event that the adversary $\adv$ asks a decommitment query $c$ such that its decommitment using the key $k_1$ is different from its decommitment using $\fdecom$. Since $\games_\prevgame$ and $\games_\thisgame$ are identical until $\event$ happens, we bound the probability of $\event$. Concretely, we have
%
%\[
%\left|\Pr[\games_\prevgame = 1] - \Pr[\games_\thisgame = 1]\right| \leq \Pr[\event]. 
%\]
%
%We construct an adversary $\advB$ that breaks soundness of the NIZK. It is given as input $\crs_\nizk$ together with a membership testing trapdoor $\tau_L:=(k_1, t)$ where $t:=2^T \bmod \varphi(N)/4$. 
%The adversary $\advB_{\secpar}(\crs_\nizk, \tau_L)$ proceeds as follows:
%\vspace{-2mm}
%\begin{enumerate}
%\item It computes $h_1:= g^{k_1} \bmod N, h_2:= g^{t} \bmod N$ using the membership testing trapdoor $\tau_L:=(k_1, t)$ and sets $\crs:=\mathlist(N, T, g, h_1, h_2, \crs_\nizk)$.
%\item Then it runs $(m_0, m_1, \st) \leftarrow \adv_{1, \secpar}(\crs)$ and answers decommitment queries using $k_1$.
%\item It samples $b \rand \bits, r \rand \smplset$ and computes $c_0^*:=g^r, c_1^*:=h_1^{r}m_b, c_2^*:=h_2^{r}m_b$. It sets $(s:=\mathlist(c_0^*, c_1^*, c_2^*), w:=(m,r))$ and runs $\pi^* \leftarrow \nizk.\prove(s,w)$.
%\item Next, it runs $b' \leftarrow \adv_{2, \secpar}((c_0^*, c_1^*, c_2^*), \pi^*, \st)$ and answers decommitment queries using $k_1$.
%\item Finally, it checks whether there exists a decommitment query $c: = (c_0, c_1, c_2, \pi)$ such that $\fdecom(\crs, c) \neq \dec(\crs,c,2)$. If $\event$ occurs, then this is the case, and it returns $((c_0, c_1, c_2), \pi)$. Notice that this can be done efficiently with the knowledge of $t$.
%\end{enumerate}
%
%$\advB$ simulates $\games_\thisgame$ perfectly and if the event $\event$ happens, then it outputs a valid proof for a statement which is not in the specified language $L$. Therefore we get
%\[\Pr[\event] \leq \snd^\nizk_\advB.\]

%Let $\gnr$ denote the event that the sampled $g$ in $\kgen$ is a generator of $\qrn$. Recall that $N = pq$ where $p = 2p'+1$ and $q = 2q'+1$. Because $g$ is sampled uniformly at random and $\qrn$ has $\varphi(|\qrn|) = (p'-1)(q'-1)$ generators, this event happens with overwhelming probability. Concretely, $\Pr[\gnr] = 1-\frac{1}{p'}-\frac{1}{q'}+\frac{1}{p'q'}$.
%Therefore the following holds.
%\begin{lemma}\label{tpke3:flem} 
%\begin{align*}
%\Pr[\games_\thisgame = 1] &= \Pr[\games_\thisgame = 1| \gnr]\Pr[\gnr] + \Pr[\games_\thisgame = 1| \overline{\gnr}]\Pr[\overline{\gnr}] \\
%&\leq \Pr[\games_\thisgame = 1| \gnr]\Pr[\gnr] + \Pr[\overline{\gnr}] \\
%&= \Pr[\games_\thisgame = 1| \gnr]\left( 1-\frac{1}{p'}-\frac{1}{q'}+\frac{1}{p'q'} \right) + \frac{1}{p'}+\frac{1}{q'}-\frac{1}{p'q'}.
%\end{align*}
%\end{lemma}




\nextgame{SimulProof}
Game $\games_\thisgame$ proceeds exactly as the previous game but we run the zero-knowledge simulator $(\crs, \tau) \leftarrow \simul_1(\seck, L)$ in $\pgen$ and produce a simulated proof for the challenge commitment as $\pi \leftarrow \simul_2(\crs, \tau, (c_0^*,c_1^*,c_2^*))$. By the zero-knowledge security of the NIZK we directly obtain
\begin{lemma}
\[
\left|\Pr[\games_\prevgame = 1] - \Pr[\games_\thisgame = 1]\right| \leq \zk^\nizk_\advB.
\]
\end{lemma}

We construct an adversary $\advB = \{\advB_\secpar\}_{\secpar \in \N}$ against the zero-knowledge security of NIZK as follows:
$\advB_\secpar(\crs_\nizk, \tau_L):$
\vspace{-2mm}
\begin{enumerate}
\item Set $\crs:=\mathlist(N, T(\secpar), g, h_1, h_2, \crs_\nizk)$, run $(m_0, m_1, \st) \leftarrow \adv_{1, \secpar}(\crs)$, and answer decommitment queries using $t$ which is included in $\tau_L = (k_1, t)$.
\item Sample $b \rand \bits, r \rand \smplsetqrn$ and compute $c_0^*:=g^r, c_1^*:=h_1^{r}m_b, c_2^*:=h_2^{r}m_b$. Then submit $(s:=\mathlist(h_1,h_2,c_0^*, c_1^*, c_2^*), w:=(m,r))$ to the oracle to obtain proof $\pi^*$.
\item Run $b' \leftarrow \adv_{2, \secpar}((c_0^*, c_1^*, c_2^*), \pi^*, \st)$, answering decommitment queries using $t$.
\item Return the truth value of $b=b'$.
\end{enumerate}
If the proof $\pi^*$ is generated using $\nizk.\prove$, then $\advB$ simulates $\games_\prevgame$ perfectly. Otherwise, $\pi^*$ is generated using $\simul_1$ and $\advB$ simulates $\games_\thisgame$ perfectly. This proves the lemma.

\nextgame{RndExp4}
In $\games_\thisgame$ we sample $k_1$ uniformly at random from $[\varphi(N)/4]$. 

\begin{lemma}
\[
\left|\Pr[\games_\prevgame = 1] - \Pr[\games_\thisgame = 1]\right| \leq \frac{1}{p}+\frac{1}{q}-\frac{1}{N}.
\]
\end{lemma}

This lemma directly follows from \Cref{sampling-lemma}.

\nextgame{DDH1}
In $\games_\thisgame$ we sample $y_1 \rand \qrn$ and compute $c_1^*$ as  $y_1 m_b$. 

\begin{lemma}
\[
\left|\Pr[\games_\prevgame = 1] - \Pr[\games_\thisgame = 1]\right| \leq \advtg_\advB^\ddh.
\]
\end{lemma}
We construct an adversary $\advB = \{\advB_\secpar\}_{\secpar \in \N}$ against DDH in the group $\qrn$. %Given \Cref{thm:ddh} this implies an adversary against DDH in large prime-order subgroups of $\Zn^*$.

$\advB_{\secpar}(N,p,q,g,g^\alpha, g^\beta, g^\gamma):$
\vspace{-2mm}
\begin{enumerate}
\item Computes $\varphi(N):=(p-1)(q-1), t:=2^{T} \bmod \varphi(N)/4, h_2:=g^t \bmod N$ and sets $\crs:=(N, T, g, h_1: = g^\alpha, h_2)$.
\item Runs $(m_0, m_1, \st) \leftarrow \adv_{1, \secpar}(\crs)$ and answers decommitment queries using $t$.
\item Samples $b \rand \bits$ and computes $(c_0^*, c_1^*, c_2^*):=(g^\beta, g^\gamma \cdot m_b, (g^\beta)^t \cdot m_b).$ Runs $\pi^* \leftarrow \simul(1, \st', (c_0^*, c_1^*, c_2^*))$.
\item Runs $b' \leftarrow \adv_{2, \secpar}(c_0^*, c_1^*, c_2^*, \pi^*), \st)$ and answers decommitment queries using $t$.
\item Returns the truth value of $b=b'$. We remark that at this point $c_1^*$ does not reveal any information about $m_b$.
\end{enumerate}
If $\gamma = \alpha\beta$ then $\advB$ simulates $\games_\prevgame$ perfectly. Otherwise $g^\gamma$ is uniform random element in $\qrn$ and $\advB$ simulates $\games_\thisgame$ perfectly. This proofs the lemma. We remark that at this point $c_1^*$ does not reveal any information about $m_b$.


\nextgame{RndExp2}
In $\games_\thisgame$ we sample $k_1$ uniformly at random from $\smplsetqrn$. 

\begin{lemma}
\[
\left|\Pr[\games_\prevgame = 1] - \Pr[\games_\thisgame = 1]\right| \leq \frac{1}{p}+\frac{1}{q}-\frac{1}{N}.
\]
\end{lemma}

This lemma directly follows from \Cref{sampling-lemma}.


\nextgame{SimSnd}

In $\games_\thisgame$ we answer decommitment queries using $\dec$ (\Cref{fig:deco-mh}) with $i:=1, \sk:=k_1$ which means that secret key $k_1$ and ciphertext $c_1$ are used. 

\begin{lemma}
\[
\left|\Pr[\games_\prevgame = 1] - \Pr[\games_\thisgame = 1]\right| \leq \simsnd^\nizk_\advB. 
\]
\end{lemma}

Let $\event$ denote the event that adversary $\adv$ asks a decommitment query $c$ such that its decommitment using the key $k_1$ is different from its decommitment using the key $t$. Since $\games_\prevgame$ and $\games_\thisgame$ are identical until $\event$ does not happen, by the standard argument it is sufficient to upper bound the probability of happening $\event$. Concretely,  

\[
\left|\Pr[\games_\prevgame = 1] - \Pr[\games_\thisgame = 1]\right| \leq \Pr[\event]. 
\]

We construct an adversary $\advB$ that breaks one-time simulation soundness of the NIZK and it is given as input $\crs_\nizk$ together with a membership testing trapdoor $\tau_L:=(k_1, t)$ where $t:=2^T \bmod \varphi(N)/4$. 

The adversary $\advB_{\secpar}^{\simul_2}(\crs_\nizk, \tau_L):$
\vspace{-2mm}
\begin{enumerate}
\item Computes $h_1:= g^{k_1} \bmod N, h_2:= g^{t} \bmod N$ using the membership testing trapdoor $\tau_L$ and sets $\crs:=(N, T, g, h_1, h_2, \crs_\nizk)$.
\item Runs $(m_0, m_1, \st) \leftarrow \adv_{1, \secpar}(\crs)$ and answers decommitment queries using $k_1$.
\item Samples $b \rand \bits, x, y_1\rand \qrn$ and computes $(c_0^*, c_1^*, c_2^*):=(x, y_1 m_b,\allowbreak x^t m_b)$. Forwards $(c_0^*, c_1^*, c_2^*)$ to simulation oracle $\simul_2$ and obtains a proof $\pi^*$.
\item Runs $b' \leftarrow \adv_{2, \secpar}((c_0^*, c_1^*, c_2^*), \pi^*, \st)$ and answers decommitment queries using $k_1$.
\item Find a decommitment query $c: = (c_0, c_1, c_2, \pi)$ such that $\dec(\crs, c, k_1, 1) \neq \dec(\crs,c,t,2)$ and returns $((c_0, c_1, c_2), \pi)$
\end{enumerate}

$\advB$ simulates $\games_\thisgame$ perfectly and if the event $\event$ happens, it outputs a valid proof for a statement which is not in the specified language $L$. Therefore
\[\Pr[\event] \leq \simsnd^\nizk_\advB,\]
which concludes the proof of the lemma.  


\nextgame{RndExp}
In $\games_\thisgame$ we sample $r$ uniformly at random from $[\varphi(N)/4]$. 

\begin{lemma}
\[
\left|\Pr[\games_\prevgame = 1] - \Pr[\games_\thisgame = 1]\right| \leq \frac{1}{p}+\frac{1}{q}-\frac{1}{N}.
\]
\end{lemma}
%At first we remark that for upper bounding the difference between the games we use a statistical argument. Because $r$ appears only in the exponent of the group generator, we later sample a random element from the group $\qrn$ which can be done efficiently. 
Since the only difference between the two games is in the set from which we sample $r$, to upper bound the advantage of adversary we can use \Cref{sampling-lemma}, which directly yields required upper bound.

%\nextgame{ReRand}
%In Game $\games_\thisgame$ we produce the challenge commitment by encrypting the challenge message using two independent random exponents $r \rand \smplset, r' \rand [\varphi(N)/4]$ to obtain $c:= (g^{r}, h_1^{r}\cdot m_b), c':= (g^{r'}, h_2^{r'}\cdot m_b)$ and then run $\rerand(c,c',N, \allowbreak h_1, h_2,k,r)$ to obtain resulting ciphertext $(c_0^*, c_1^*, c_2^*)$. Since $r'$ is sampled uniformly at random from $[\varphi(N)/4]$ the ciphertext distributions in both games  are the same. Therefore 
%
%\begin{lemma}
%\[
%\Pr[\games_\prevgame = 1] = \Pr[\games_\thisgame = 1].
%\]
%\end{lemma}


\nextgame{SSSA}
In $\games_\thisgame$ we sample $y_2 \rand \qrn$ and compute $c_2^*$ as $y_2 m_b$.

Let $\tilT_\sss(\secpar)$ be the polynomial whose existence is guaranteed by the SSS assumption.
Let $\poly_\advB(\secpar)$ be the fixed polynomial which bounds the time required to execute Steps 1--2 and answer decommitment queries in Step 3 of the adversary $\advB_{2, \secpar}$ defined below. Set $\undT := (\poly_\advB(\secpar))^{1 / \ugap}$.  Set $\tilT_\nitc := \max(\tilT_\sss, \undT)$.
\begin{lemma}
From any polynomial-size adversary $\adv = \{(\adv_{1,\secpar}, \adv_{2, \secpar})\}_{\secpar \in \nats}$, where depth of $\adv_{2, \secpar}$ is at most $T^{\ugap}(\secpar)$ for some $T(\cdot) \geq \undT(\cdot)$ we can construct a polynomial-size adversary $\advB = \{(\advB_{1,\secpar}, \advB_{2, \secpar})\}_{\secpar \in \nats}$ where the depth of $\advB_{2, \secpar}$ is at most $T^{\gap}(\secpar)$ with
\[
\left|\Pr[\games_\prevgame = 1] - \Pr[\games_\thisgame = 1]\right| \leq \advtg_\advB^\sss.
\]
\end{lemma}

The adversary $\advB_{1,\secpar}(N, T(\secpar), g):$
\vspace{-2mm}
\begin{enumerate}
\item Samples $k_1\rand \smplsetqrn$, computes $h_1 := g^{k_1} \bmod N, h_2 := g^{2^{T(\secpar)}} \bmod N$, runs $(\crs_\nizk, \tau) \leftarrow \nizk.\simul_1(\seck, L)$ and sets $\crs:=\mathlist(N, T(\secpar), g, h_1, h_2, \crs_\nizk)$. Notice that value $h_2$ is computed by repeated squaring.
\item Runs $(m_0, m_1, \st) \leftarrow \adv_{1, \secpar}(\crs)$ and answers decommitment queries using $k_1$.
\item Outputs $(N,g,k_1, h_1,h_2,\crs_\nizk, \tau, m_0, m_1, \st)$
\end{enumerate}

The adversary $\advB_{2,\secpar}(x,y,(N,g,k_1, h_1,h_2, \crs_\nizk, \tau, m_0, m_1, \st)):$
\vspace{-2mm}
\begin{enumerate}
\item Samples $b \rand \bits, y_1 \rand \qrn$, computes $c_0^*:=x, c_1^*:=y_1 m_b,  c_2^*:=y m_b$.
\item Runs $\pi^* \leftarrow \simul(\crs_\nizk, \tau, (c_0^*, c_1^*, c_2^*))$.
\item Runs $b' \leftarrow \adv_{2, \secpar}((c_0^*, c_1^*, c_2^*), \pi^*), \st)$ and answers decommitment queries using $k_1$.
\item Returns the truth value of $b=b'$.
\end{enumerate}
Since $g$ is a generator of $\qrn$ and $x$ is sampled uniformly at random from $\qrn$ there exists some $r \in [\varphi(N)/4]$ such that $x = g^{r}$. Therefore when $y = x^{2^T} = (g^{2^T})^{r} \bmod N$, then $\advB$ simulates $\games_\prevgame$ perfectly. Otherwise $y$ is random value and $\advB$ simulates $\games_\thisgame$ perfectly. We remark that at this point $c_2^*$ does not reveal any information about $m_b$.

Now we analyse the running time of the constructed adversary. Adversary $\advB_1$ computes $h_2$ by $T(\secpar)$ consecutive squarings and because $T(\secpar)$ is polynomial in $\secpar$, $\advB_1$ is efficient. Moreover, $\advB_2$ fulfils the depth constraint:
\[ \dep(\advB_{2,\secpar}) = \poly_\advB(\secpar) + \dep(\adv_{2,\secpar}) \leq \undT^{\ugap}(\secpar) + T^{\ugap}(\secpar) \leq 2 T^{\ugap}(\secpar) = o(T^{\gap}(\secpar)). \] 

Also $T(\cdot) \geq \tilT_\nitc(\cdot) \geq \tilT_\sss(\cdot)$ as required.  

%\nextgame{RndExp2}
%In $\games_\thisgame$ we stop to use $\rerand$ algorithm. Concretely, we sample $r \rand [\varphi(N)/4], y_2 \rand \qrn$ and compute challenge ciphertext as $c^*:=(g^r, h_1^r \cdot m_b, y_2 \cdot m_b)$. The ciphertext has the same distribution as in the previous game. Therefore 
%
%\begin{lemma}
%\[
%\Pr[\games_\prevgame = 1] = \Pr[\games_\thisgame = 1].
%\]
%\end{lemma}
%\nextgame{RndExp3}







%\nextgame{ReRand2}
%In $\games_\thisgame$ we use the key $t$ and randomness $r'$ as input for rerandomization. Concretely we compute $\rerand(c,c',h_1,h_2,t,r')$. This is just conceptual change since the ciphertext distributions are the same in both games and therefore 
%
%\begin{lemma}
%\[
%\left|\Pr[\games_\prevgame = 1] = \Pr[\games_\thisgame = 1]\right|.
%\]
%\end{lemma}
%
%\nextgame{RndExp4}
%In $\games_\thisgame$ we sample $r$ uniformly at random from $\varphi(N)$. 
%
%\begin{lemma}
%\[
%\left|\Pr[\games_\prevgame = 1] - \Pr[\games_\thisgame = 1]\right| \leq \frac{1}{N}.
%\]
%\end{lemma}



\begin{lemma}\label{nitc-mh:llem}
\[
\Pr[\games_\thisgame = 1] = \half.
\]
\end{lemma}

Clearly, $c_0^*$ is uniform random element in $\qrn$ and hence it does not contain any information about the challenge message. Since $y_1, y_2$ are sampled uniformly at random from $\qrn$ the ciphertexts $c_1^*, c_2^*$ are also uniform random elements in $\qrn$ and hence do not contain any information about the challenge message $m_b$. Therefore, an adversary can not do better than guessing.

By combining Lemmas \ref{nitc-mh:flem} - \ref{nitc-mh:llem} we obtain the following:
\begin{align*}
&\advtg^{\nitc}_{\adv} = \left| \Pr[\games_0 = 1] - \half \right| \leq \sum_{i=0}^7 \left|\Pr[\games_i = 1] - \Pr[\games_{i+1} = 1] \right| + \left|\Pr[\games_{8}- \half\right| \\
 &\leq \zk^\nizk_\advB + \advtg^{\sss}_{\advB} + \simsnd^{\nizk}_{\advB} + \advtg^{\ddh}_{\advB} + 3 \left( \frac{1}{p}+\frac{1}{q}-\frac{1}{N} \right),
\end{align*}
which concludes the proof.
%\end{proof}

% section proof_of_theorem_thm:nitc-ind-mul-std (end)


\section{Proof of Theorem \ref{thm:NITC-lin-ROM}} % (fold)
\label{app:NITC-lin-ROM}

% \todo{appendix?}
% \begin{proof}
Completeness is implied by the completeness of the NIZK and can be verified by inspection. 

%Our construction is based on the Naor-Yung paradigm where we combine three Paillier-type ciphertext with shared randomness.  


%Similarly to \cite{SCN:BiaMasVen16} we define a PPT algorithm $\rerand$ in Figure~\ref{fig:rerand}, which takes two ciphertexts generated with independent randomness, both public keys, only one secret key (in our case $k$) and randomness which was used to encrypt a message using the public key $g, h_1$. 
%%We assume that modulus $N$ is implicitly known.\todo{Why not explicit?} 
%% which corresponds to the secret key which is given as the input. 
%
%\begin{figure}[tb]
%\centering
%\begin{minipage}{0.75\textwidth}
%$\underline{\rerand(c:= (g^{r}, h_1^{r}\cdot m), c':= (g^{r'}, h_2^{r'}\cdot m), N, h_1, h_2, k, r)}:$
%\vspace{-2mm}
%\begin{itemize}
%\item $c_0:= g^{r}\cdot{g^{r'}} = g^{r+r'} \bmod N$;
%\item $c_1:= (g^{r'})^k \cdot h_1^{r}\cdot m  =  h_1^{r'}\cdot h_1^{r}\cdot m = h_1^{r+r'}\cdot m \bmod N$;
%\item $c_2:=h_2^{r} \cdot h_2^{r'}\cdot m = h_2^{r+r'}\cdot m \bmod N$.
%\end{itemize}
%\end{minipage}
%\caption{\label{fig:rerand}Algorithm $\rerand$.}
%\end{figure}
%
%
%
%
%It is straightforward to see that the ciphertext returned by $\rerand$ is perfectly distributed to the ciphertext produced using a shared randomness where the pair $(c_0, c_1)$ encrypts a message $m$ and the pair $(c_0, c_2)$ encrypts message $m'$.


% We note that if we use value $2^T$ as a secret key, then in order to compute $c_2$ we have to execute $T$ repeated squarings, but since $T$ is polynomial in $\secpar$ this computations is considered to be efficient.

\newsequenceofgames{NITC-LH-ROM}
To prove security we define a sequence of games $\games_0 - \games_{13}$.  For $i \in \{0,1,\dots,13\}$ we denote by $\games_i = 1$ the event that the adversary $\adv = \{(\adv_{1,\secpar}, \adv_{2, \secpar})\}_{\secpar \in \nats}$ outputs $b'$ in the game $\games_i$ such that $b = b'$.
%In individual games we use the algorithm $\decom$ define in \Cref{fig:deco-rom-lh} to answer decommitment queries efficiently. 
\begin{figure}[h!]
\begin{center}
\begin{tabular}{|l|}
\hline
$\underline{\decom(\crs, c, i)}$\\
Parse $c$ as $(c_0, c_1, c_2,, c_3, \pi)$\\
if $\nizk.\vrfy((h_1, h_2, c_0, c_1, c_2, c_3), \pi)= 1$\\
\tab Compute $y:= c_0^{k_i} \bmod N$\\
\tab return $\frac{c_i \cdot y^{-N} (\bmod N^2) -1}{N}$\\
return $\bot$\\
\hline          
\end{tabular}
\caption{Decommitment oracle}
\label{fig:deco-rom-lh}
\end{center}
\end{figure}

\nextgame{G0}
Game $\games_\thisgame$ corresponds to the original security experiment where decommitment queries are answered using $\fdecom$.

\nextgame{DecOracle}
In game $\games_\thisgame$ decommitment queries are answered using the algorithm $\decom$ defined in \Cref{fig:deco-rom-lh} with $i:=1$, meaning that secret key $k_1$ and ciphertext $c_1$ are used, to answer decommitment queries efficiently. 


\begin{lemma}\label{nitc-lh:flem}
\[
\left|\Pr[\games_\prevgame = 1] - \Pr[\games_\thisgame = 1]\right| \leq \snd^\nizk_\advB.
\]
\end{lemma}

Notice that if $c_1$ and $c_3$ contain the same message, both oracles answer decommitment queries consistently. Let $\event$ denote the event that the adversary $\adv$ asks a decommitment query $c$ such that its decommitment using the key $k_1$ is different from its decommitment using $\fdecom$. Since $\games_\prevgame$ and $\games_\thisgame$ are identical until $\event$ happens, we bound the probability of $\event$. Concretely, we have

\[
\left|\Pr[\games_\prevgame = 1] - \Pr[\games_\thisgame = 1]\right| \leq \Pr[\event]. 
\]

We construct an adversary $\advB$ that breaks soundness of the NIZK. 
The adversary $\advB_{\secpar}$ proceeds as follows:
\vspace{-2mm}
\begin{enumerate}
\item It computes $\crs \leftarrow \pgen(\seck, T)$ as defined in the construction where the value $h_3$ is computed using repeated squaring instead.
\item Then it runs $(m_0, m_1, \st) \leftarrow \adv_{1, \secpar}(\crs)$ and answers decommitment queries using $k_1$.
\item It samples $b \rand \bits, r \rand \smplset$ and computes $c_0^*:=g^r, c_1^*:=h_1^{rN}(1+N)^{m_b}, c_2^*:=h_2^{rN}(1+N)^{m_b}, c_3^*:=h_3^{rN}(1+N)^{m_b}$. It sets $(s:=\mathlist(h_1, h_2, c_0^*, c_1^*, c_2^*, c_3^*), w:=(m,r))$ and runs $\pi^* \leftarrow \nizk.\prove(s,w)$.
\item Next, it runs $b' \leftarrow \adv_{2, \secpar}((c_0^*, c_1^*, c_2^*, c_3^*), \pi^*, \st)$ and answers decommitment queries using $k_1$.
\item Finally, it checks whether there exists a decommitment query $c: = \mathlist(c_0, c_1, c_2, c_3, \pi)$ such that $\fdecom(\crs, c, 1) \neq \dec(\crs,c,1)$. If $\event$ occurs, then this is the case, and it returns $((h_1, h_2, c_0, c_1, c_2, c_3), \pi)$. %Notice that this can be done efficiently with the knowledge of $t$.
\end{enumerate}

$\advB$ simulates $\games_\thisgame$ perfectly and if the event $\event$ happens, then it outputs a valid proof for a statement which is not in the specified language $L$. Therefore we get
\[\Pr[\event] \leq \snd^\nizk_\advB.\]

%Let $\gnr$ denote the event that the sampled $g$ in $\kgen$ is a generator of $\qrn$. Recall that $N = pq$ where $p = 2p'+1$ and $q = 2q'+1$. Because $g$ is sampled uniformly at random and $\qrn$ has $\varphi(|\qrn|) = (p'-1)(q'-1)$ generators, this event happens with overwhelming probability. Concretely, $\Pr[\gnr] = 1-\frac{1}{p'}-\frac{1}{q'}+\frac{1}{p'q'}$.
%Therefore the following holds.
%\begin{lemma}\label{tpke3:flem} 
%\begin{align*}
%\Pr[\games_\thisgame = 1] &= \Pr[\games_\thisgame = 1| \gnr]\Pr[\gnr] + \Pr[\games_\thisgame = 1| \overline{\gnr}]\Pr[\overline{\gnr}] \\
%&\leq \Pr[\games_\thisgame = 1| \gnr]\Pr[\gnr] + \Pr[\overline{\gnr}] \\
%&= \Pr[\games_\thisgame = 1| \gnr]\left( 1-\frac{1}{p'}-\frac{1}{q'}+\frac{1}{p'q'} \right) + \frac{1}{p'}+\frac{1}{q'}-\frac{1}{p'q'}.
%\end{align*}
%\end{lemma}




\nextgame{SimulProof}
Game $\games_\thisgame$ proceeds exactly as the previous game but we use the zero-knowledge simulator $(\pi, \st) \leftarrow \simul(1, \st, (h_1, h_2, c_0^*,c_1^*,c_2^*,c_3^*))$ to produce a simulated proof for the challenge commitment and $\simul(2, \st, \cdot)$ to answer random oracle queries. By zero-knowledge security of underlying NIZK we directly obtain
\begin{lemma}\label{nitc-rom-lh:flem}
\[
\left|\Pr[\games_\prevgame = 1] - \Pr[\games_\thisgame = 1]\right| \leq \zk^\nizk_\advB.
\]
\end{lemma}

We construct an adversary $\advB = \{\advB_\secpar\}_{\secpar \in \N}$ against zero-knowledge security of NIZK as follows:
\vspace{-2mm}
\begin{enumerate}
\item Samples $k_1, k_2 \rand \smplset$, computes $h_1 := g^{k_1} \bmod N, h_2 := g^{k_2} \bmod N$ and sets $\crs:=(N, T(\secpar), g, h_1, h_2, h_3)$. 
\item Runs $(m_0, m_1, \st) \leftarrow \adv_{1, \secpar}(\crs)$ and answers decommitment queries using $k_1$.
\item Samples $b \rand \bits, r \rand \smplset$ and computes $c_0^*:=g^r, c_1^*:=h_1^{rN}(1+N)^{m_b}, c_2^*:=h_2^{rN}(1+N)^{m_b}, c_3^*:=h_3^{rN}(1+N)^{m_b}$. It submits $(s:=\mathlist(h_1,h_2,c_0^*, c_1^*, c_2^*, c_3^*), w:=(m,r))$ to its oracle and obtains proof $\pi^*$ as answer.
\item Runs $b' \leftarrow \adv_{2, \secpar}((c_0^*, c_1^*, c_2^*, c_3^*), \pi^*, \st)$ and answers decommitment queries using $k_1$.
\item Returns the truth value of $b=b'$.
\end{enumerate}
If the proof $\pi^*$ is generated using $\nizk.\prove$, then $\advB$ simulates $\games_\prevgame$ perfectly. Otherwise $\pi^*$ is generated using $\simul_1$ and $\advB$ simulates $\games_\thisgame$ perfectly. This proofs the lemma.


\nextgame{RndExp}
In $\games_\thisgame$ we sample $r$ uniformly at random from $[\varphi(N)/2]$. 

\begin{lemma}
\[
\left|\Pr[\games_\prevgame = 1] - \Pr[\games_\thisgame = 1]\right| \leq \frac{1}{p}+\frac{1}{q}-\frac{1}{N}.
\]
\end{lemma}
%At first we remark that for upper bounding the difference between the games we use a statistical argument. Because $r$ appears only in the exponent of the group generator, we later sample a random element from the group $\qrn$ which can be done efficiently. 
Since the only difference between the two games is in the set from which we sample $r$, to upper bound the advantage of adversary we can use \Cref{sampling-lemma}, which directly yields required upper bound.

%\nextgame{ReRand}
%In Game $\games_\thisgame$ we produce the challenge commitment by encrypting the challenge message using two independent random exponents $r \rand \smplset, r' \rand [\varphi(N)/4]$ to obtain $c:= (g^{r}, h_1^{r}\cdot m_b), c':= (g^{r'}, h_2^{r'}\cdot m_b)$ and then run $\rerand(c,c',N, \allowbreak h_1, h_2,k,r)$ to obtain resulting ciphertext $(c_0^*, c_1^*, c_2^*)$. Since $r'$ is sampled uniformly at random from $[\varphi(N)/4]$ the ciphertext distributions in both games  are the same. Therefore 
%
%\begin{lemma}
%\[
%\Pr[\games_\prevgame = 1] = \Pr[\games_\thisgame = 1].
%\]
%\end{lemma}


\nextgame{SSSA}
In $\games_\thisgame$ we sample $y_3 \rand \Jn$ and compute $c_3^*$ as $y_3^N (1+N)^{m_b}$.

Let $\tilT_\sss(\secpar)$ be the polynomial whose existence is guaranteed by the SSS assumption.
Let $\poly_\advB(\secpar)$ be the fixed polynomial which bounds the time required to execute Steps 1--2 and answer decommitment queries in Step 3 of the adversary $\advB_{2, \secpar}$ defined below. Set $\undT := (\poly_\advB(\secpar))^{1 / \ugap}$.  Set $\tilT_\nitc := \max(\tilT_\sss, \undT)$.
\begin{lemma}
From any polynomial-size adversary $\adv = \{(\adv_{1,\secpar}, \adv_{2, \secpar})\}_{\secpar \in \nats}$, where depth of $\adv_{2, \secpar}$ is at most $T^{\ugap}(\secpar)$ for some $T(\cdot) \geq \undT(\cdot)$ we can construct a polynomial-size adversary $\advB = \{(\advB_{1,\secpar}, \advB_{2, \secpar})\}_{\secpar \in \nats}$ where the depth of $\advB_{2, \secpar}$ is at most $T^{\gap}(\secpar)$ with
\[
\left|\Pr[\games_\prevgame = 1] - \Pr[\games_\thisgame = 1]\right| \leq \advtg_\advB^\sss.
\]
\end{lemma}

The adversary $\advB_{1,\secpar}(N, T(\secpar), g):$
\vspace{-2mm}
\begin{enumerate}
\item Samples $k_1, k_2 \rand \smplset$, computes $h_1 := g^{k_1} \bmod N, h_2 := g^{k_2} \bmod N,  h_3 := g^{2^{T(\secpar)}} \bmod N$ and sets $\crs:=(N, T(\secpar), g, h_1, h_2, h_3)$. Notice that value $h_3$ is computed by repeated squaring.
\item Runs $(m_0, m_1, \st) \leftarrow \adv_{1, \secpar}(\crs)$ and answers decommitment queries using $k_1$.
\item Outputs $(N,g,k_1, k_2, h_1,h_2,h_3, m_0, m_1, \st)$
\end{enumerate}

The adversary $\advB_{2,\secpar}(x,y,(N,g,k_1, k_2, h_1,h_2,h_3, m_0, m_1, \st)):$
\vspace{-2mm}
\begin{enumerate}
\item Samples $b \rand \bits$, computes $c_0^*:=x, c_1^*:=x^{k_1N}(1+N)^{m_b}, c_2^*:=x^{k_2N}(1+N)^{m_b}, c_3^*:=y^{N}(1+N)^{m_b}$.
\item Runs $\pi^* \leftarrow \simul(1, \st', (h_1, h_2, c_0^*, c_1^*, c_2^*, c_3^*))$.
\item Runs $b' \leftarrow \adv_{2, \secpar}((c_0^*, c_1^*, c_2^*, c_3^*), \pi^*), \st)$ and answers decommitment queries using $k_1$.
\item Returns the truth value of $b=b'$.
\end{enumerate}
Since $g$ is a generator of $\Jn$ and $x$ is sampled uniformly at random from $\Jn$ there exists some $r \in [\varphi(N)/2]$ such that $x = g^{r}$. Therefore when $y = x^{2^T} = (g^{2^T})^{r} \bmod N$, then $\advB$ simulates $\games_\prevgame$ perfectly. Otherwise $y$ is random value and $\advB$ simulates $\games_\thisgame$ perfectly. 

Now we analyse the running time of the constructed adversary. Adversary $\advB_1$ computes $h_3$ by $T(\secpar)$ consecutive squarings and because $T(\secpar)$ is polynomial in $\secpar$, $\advB_1$ is efficient. Moreover, $\advB_2$ fulfils the depth constraint:
\[ \dep(\advB_{2,\secpar}) = \poly_\advB(\secpar) + \dep(\adv_{2,\secpar}) \leq \undT^{\ugap}(\secpar) + T^{\ugap}(\secpar) \leq 2 T^{\ugap}(\secpar) = o(T^{\gap}(\secpar)). \] 

Also $T(\cdot) \geq \tilT_\nitc(\cdot) \geq \tilT_\sss(\cdot)$ as required.

%\nextgame{RndExp2}
%In $\games_\thisgame$ we stop to use $\rerand$ algorithm. Concretely, we sample $r \rand [\varphi(N)/4], y_2 \rand \qrn$ and compute challenge ciphertext as $c^*:=(g^r, h_1^r \cdot m_b, y_2 \cdot m_b)$. The ciphertext has the same distribution as in the previous game. Therefore 
%
%\begin{lemma}
%\[
%\Pr[\games_\prevgame = 1] = \Pr[\games_\thisgame = 1].
%\]
%\end{lemma}
%\nextgame{RndExp3}




\nextgame{DCR1}
In $\games_\thisgame$ we sample $y_3 \rand \Zns$ such that it has Jacobi symbol 1 and compute $c_3^*$ as $y_3(1+N)^{m_b}$. 

\begin{lemma}\label{lem:dcr-rom-lh}
\[
\left|\Pr[\games_\prevgame = 1] - \Pr[\games_\thisgame = 1]\right| \leq \advtg_\advB^\dcr.
\]
\end{lemma}
We construct an adversary $\advB = \{\advB_\secpar\}_{\secpar \in \N}$ against DCR.

$\advB_{\secpar}(N,y):$
\vspace{-2mm}
\begin{enumerate}
\item Samples $g, y_3, x \rand \Jn, k_1, k_2 \rand \smplset$, computes $h_1 := g^{k_1} \bmod N, h_2 := g^{k_2} \bmod N,  h_3 := g^{2^{T}} \bmod N$ and sets $\crs:=(N, T, g, h_1, h_2, h_3)$. Notice that value $h_3$ is computed by repeated squaring.
\item Runs $(m_0, m_1, \st) \leftarrow \adv_{1, \secpar}(\crs)$ and answers decommitment queries using $k_1$.
\item Samples $b \rand \bits, w \rand \Zns$ such that $\left( \frac{y}{N} \right)= \left( \frac{w}{N} \right)$. We remark that computing Jacobi symbol can be done efficiently without knowing factorization of N.
\item Computes $c_0^*:=x, , c_1^*:=x^{k_1N}(1+N)^{m_b}, c_2^*:=x^{k_2N}(1+N)^{m_b}, c_3^*:=yw^{N}(1+N)^{m_b}$. Runs $\pi^* \leftarrow \simul(1, \st', (h_1, h_2, c_0^*, c_1^*, c_2^*, c_3^*))$.
\item Runs $b' \leftarrow \adv_{2, \secpar}((c_0^*, c_1^*, c_2^*, c_3^*), \pi^*, \st)$ and answers decommitment queries using $k_1$.
\item Returns the truth value of $b=b'$.
\end{enumerate}
%If $y = v^N \bmod N^2$ then $yw^N = v^N w^N = (vw)^N$ and hence $yw^N$ is $N$-th residue. Moreover, the Jacobi symbol of $yw$ is 1, since the Jacobi symbol is multiplicatively homomorphic. Therefore $\advB$ simulates $\games_\prevgame$ perfectly. Otherwise $y$ is uniform random element in $\Zns$ then $yw^N$ is also uniform in $\Zns$ and $\advB$ simulates $\games_\thisgame$ perfectly. This proofs the lemma. We remark that at this point $c_3^*$ does not reveal any information about $m_b$.

If $y = v^N \bmod N^2$ then $yw^N = v^N w^N = (vw)^N$ and hence $yw^N$ is $N$-th residue. Moreover, the Jacobi symbol of $yw$ is 1, since the Jacobi symbol is multiplicatively homomorphic. Therefore $\advB$ simulates $\games_\prevgame$ perfectly. 

Otherwise, if $y$ is uniform random element in $\Zns$, then $yw^N$ is also uniform among all elements of $\Zns$ that have Jacobi symbol 1 and $\advB$ simulates $\games_\thisgame$ perfectly. This proves the lemma.

We remark that at this point $c_3^*$ does not reveal any information about $b$. Here we use that if $x = y \bmod N$ then $\left( \frac{x}{N} \right)= \left( \frac{y}{N} \right)$ and that there is an isomorphism $f:\Zn^* \times \Zn \rightarrow\Zns$ given by $f(u,v)=u^N(1+N)^v = u^N(1+vN) \bmod N^2$ (see e.g. \cite[Proposition 13.6]{books/crc/KatzLindell2014}).  Since $f(u,v) \bmod N = u^N + u^NvN \bmod N = u^N \bmod N$, that means that Jacobi symbol $\left( \frac{f(u,v)}{N} \right)$ depends only on $u$. Hence if $\left( \frac{f(u,v)}{N} \right) = 1$ then it must hold that $\left( \frac{f(u,r)}{N} \right) = 1$ for all $r \in \Zn$. This implies that a random element $f(u,v)$ in $\Zns$ with $\left( \frac{f(u,v)}{N} \right) = 1$ has a uniformly random distribution of $v$ in $\Zn$. Therefore if $yw^N = u^N(1+N)^v \bmod N^2$ then  $yw^N(1+N)^{m_b}  = u^N(1+N)^{m_b+v} \bmod N^2$. Since $v$ is uniform in $\Zn$, $(m_b + v)$ is also uniform in $\Zn$, which means that ciphertext $c_3^*$ does not reveal any information about $b$. 

\nextgame{RndExp2}
In $\games_\thisgame$ we sample $k_2$ uniformly at random from $[\varphi(N)/2]$. 

\begin{lemma}
\[
\left|\Pr[\games_\prevgame = 1] - \Pr[\games_\thisgame = 1]\right| \leq \frac{1}{p}+\frac{1}{q}-\frac{1}{N}.
\]
\end{lemma}

Again using a statistical argument this lemma directly follows from \Cref{sampling-lemma}.

\nextgame{DDH}
In $\games_\thisgame$ we sample $y_2 \rand \Jn$ and compute $c_2^*$ as  $y_2^N(1+N)^{m_b}$. 

\begin{lemma}\label{lem:ddh-rom-lh}
\[
\left|\Pr[\games_\prevgame = 1] - \Pr[\games_\thisgame = 1]\right| \leq \advtg_\advB^\ddh.
\]
\end{lemma}
We construct an adversary $\advB = \{\advB_\secpar\}_{\secpar \in \N}$ against DDH in the group $\Jn$. %Given \Cref{thm:ddh} this implies an adversary against DDH in large prime-order subgroups of $\Zn^*$.

$\advB_{\secpar}(N,g,g^\alpha, g^\beta, g^\gamma):$
\vspace{-2mm}
\begin{enumerate}
\item Samples $k_1 \rand \smplset$, computes $h_1 := g^{k_1} \bmod N,  h_3 := g^{2^{T}} \bmod N$ and sets $\crs:=(N, T, g, h_1, h_2:=g^\alpha, h_3)$.
\item Runs $(m_0, m_1, \st) \leftarrow \adv_{1, \secpar}(\crs)$ and answers decommitment queries using $k_1$.
\item Samples $b \rand \bits, y_3 \rand \Zns$ such that it has Jacobi symbol 1 and computes $(c_0^*, c_1^*, c_2^*, c_3^*):=(g^\beta, (g^\beta)^{k_1 N}(1+N)^{m_b}, (g^{\gamma})^N(1+N)^{m_b}, y_3(1+N)^{m_b}).$ Runs $\pi^* \leftarrow \simul(1, \st', (h_1, h_2, c_0^*, c_1^*, c_2^*, c_3^*))$.
\item Runs $b' \leftarrow \adv_{2, \secpar}((c_0^*, c_1^*, c_2^*,c_3^*), \pi^*, \st)$ and answers decommitment queries using $k_1$.
\item Returns the truth value of $b=b'$.
\end{enumerate}
If $\gamma = \alpha\beta$, then $\advB$ simulates $\games_\prevgame$ perfectly. Otherwise $g^\gamma$ is uniform random element in $\Jn$ and $\advB$ simulates $\games_\thisgame$ perfectly. This proofs the lemma.

\nextgame{RndExp3}
In $\games_\thisgame$ we sample $k_2$ uniformly at random from $\smplset$. 

\begin{lemma}
\[
\left|\Pr[\games_\prevgame = 1] - \Pr[\games_\thisgame = 1]\right| \leq \frac{1}{p}+\frac{1}{q}-\frac{1}{N}.
\]
\end{lemma}

This lemma directly follows from \Cref{sampling-lemma}.

\nextgame{DCR2}
In $\games_\thisgame$ we sample $y_2 \rand \Zns$ such that it has Jacobi symbol 1 and compute $c_2^*$ as $y_2(1+N)^{m_b}$. 

\begin{lemma}
\[
\left|\Pr[\games_\prevgame = 1] - \Pr[\games_\thisgame = 1]\right| \leq \advtg_\advB^\dcr.
\]
\end{lemma}
This can be proven in similar way as \Cref{lem:dcr-rom-lh}. We remark that at this point $c_2^*$ does not reveal any information about $m_b$.



\nextgame{SimSnd}

In $\games_\thisgame$ we answer decommitment queries using $\dec$ (\Cref{fig:deco-rom-lh}) with $i:=2$ which means that secret key $k_2$ and ciphertext $c_2$ are used. 

\begin{lemma}
\[
\left|\Pr[\games_\prevgame = 1] - \Pr[\games_\thisgame = 1]\right| \leq \simsnd^\nizk_\advB. 
\]
\end{lemma}

Let $\event$ denote the event that adversary $\adv$ asks a decommitment query $c$ such that its decommitment using the key $k_1$ is different from its decommitment using the key $k_2$. Since $\games_\prevgame$ and $\games_\thisgame$ are identical until $\event$ does not happen, by the standard argument it is sufficient to upper bound the probability of happening $\event$. Concretely,  

\[
\left|\Pr[\games_\prevgame = 1] - \Pr[\games_\thisgame = 1]\right| \leq \Pr[\event]. 
\]

We construct an adversary $\advB$ which breaks one-time simulation soundness of the NIZK. 

The adversary $\advB_{\secpar}^{\simul_1, \simul_2}:$
\vspace{-2mm}
\begin{enumerate}
\item Computes $\crs \leftarrow \pgen(\seck, T)$ as defined in the construction where the value $h_3$ is computed using repeated squaring instead.
\item Runs $(m_0, m_1, \st) \leftarrow \adv_{1, \secpar}(\crs)$ and answers decommitment queries using $k_2$.
\item Samples $b \rand \bits, x \rand \Jn, y_2, y_3 \rand \Zns$ and computes $(c_0^*, c_1^*, c_2^*, c_3^*):=(x, x^{k_1 N} (1+N)^{m_b}, y_2 (1+N)^{m_b}, y_3 (1+N)^{m_b})$. Forwards $(h_1, h_2, c_0^*, c_1^*, c_2^*, c_3^*)$ to simulation oracle $\simul_1$ and obtains a proof $\pi^*$.
\item Runs $b' \leftarrow \adv_{2, \secpar}((c_0^*, c_1^*, c_2^*, c_3^*), \pi^*, \st)$ and answers decommitment queries using $k_2$.
\item Find a decommitment query $c: = (c_0, c_1, c_2, c_3, \pi)$ such that $\dec(\crs, c, 1) \neq \dec(\crs,c,2)$ and returns $((h_1, h_2, c_0, c_1, c_2, c_3), \pi)$.
\end{enumerate}

$\advB$ simulates $\games_\thisgame$ perfectly and if the event $\event$ happens, it outputs a valid proof for a statement which is not in the specified language $L$. Therefore
\[\Pr[\event] \leq \simsnd^\nizk_\advB,\]
which concludes the proof of the lemma.  

%\nextgame{ReRand2}
%In $\games_\thisgame$ we use the key $t$ and randomness $r'$ as input for rerandomization. Concretely we compute $\rerand(c,c',h_1,h_2,t,r')$. This is just conceptual change since the ciphertext distributions are the same in both games and therefore 
%
%\begin{lemma}
%\[
%\left|\Pr[\games_\prevgame = 1] = \Pr[\games_\thisgame = 1]\right|.
%\]
%\end{lemma}
%
%\nextgame{RndExp4}
%In $\games_\thisgame$ we sample $r$ uniformly at random from $\varphi(N)$. 
%
%\begin{lemma}
%\[
%\left|\Pr[\games_\prevgame = 1] - \Pr[\games_\thisgame = 1]\right| \leq \frac{1}{N}.
%\]
%\end{lemma}

\nextgame{RndExp4}
In $\games_\thisgame$ we sample $k_1$ uniformly at random from $[\varphi(N)/2]$. 

\begin{lemma}
\[
\left|\Pr[\games_\prevgame = 1] - \Pr[\games_\thisgame = 1]\right| \leq \frac{1}{p}+\frac{1}{q}-\frac{1}{N}.
\]
\end{lemma}

This lemma directly follows from \Cref{sampling-lemma}.

\nextgame{DDH2}
In $\games_\thisgame$ we sample $y_1 \rand \Jn$ and compute $c_1^*$ as  $y_1^{N} (1+N)^{m_b}$. 

\begin{lemma}
\[
\left|\Pr[\games_\prevgame = 1] - \Pr[\games_\thisgame = 1]\right| \leq \advtg_\advB^\ddh.
\]
\end{lemma}
This can be proven in similar way as \Cref{lem:ddh-rom-lh}.

\nextgame{DCR3}
In $\games_\thisgame$ we sample $y_1 \rand \Zns$ such that it has Jacobi symbol 1 and compute $c_1^*$ as $y_1(1+N)^{m_b}$. 

\begin{lemma}
\[
\left|\Pr[\games_\prevgame = 1] - \Pr[\games_\thisgame = 1]\right| \leq \advtg_\advB^\dcr.
\]
\end{lemma}
This can be proven in similar way as \Cref{lem:dcr-rom-lh}. We remark that at this point $c_1^*$ does not reveal any information about $m_b$.

\begin{lemma}\label{nitc-rom-lh:llem}
\[
\Pr[\games_\thisgame = 1] = \half.
\]
\end{lemma}

Clearly, $c_0^*$ is uniform random element in $\Jn$ and hence it does not contain any information about the challenge message. Since $y_1, y_2, y_3$ are sampled uniformly at random from $\Zns$ the ciphertexts $c_1^*, c_2^*, c_3^*$ are also uniform random elements in $\Zns$ and hence do not contain any information about the challenge message $m_b$. Therefore, an adversary can not do better than guessing.

By combining Lemmas \ref{nitc-rom-lh:flem} - \ref{nitc-rom-lh:llem} we obtain the following:
\begin{align*}
\advtg^{\nitc}_{\adv} &= \left| \Pr[\games_0 = 1] - \half \right| \leq \sum_{i=0}^{12} \left|\Pr[\games_i = 1] - \Pr[\games_{i+1} = 1] \right| + \left|\Pr[\games_{13}- \half\right| \\
 &\leq  \snd^\nizk_\advB + \zk^\nizk_\advB + \advtg^{\sss}_{\advB} + \simsnd^{\nizk}_{\advB} + 2 \advtg^{\ddh}_{\advB} + 3 \advtg^{\dcr}_{\advB} \\ &+ 4 \left( \frac{1}{p}+\frac{1}{q}-\frac{1}{N} \right).
\end{align*}
which concludes the proof.
% \end{proof}
% \todo{appendix end}


% section proof_of_theorem_nitc-lin-rom (end)

\section{Proof of Theorem \ref{thm:NITC-mul-ROM}} % (fold)
\label{app:NITC-mul-ROM}

% \todo{appendix?}
% \begin{proof}
Completeness is implied by the completeness of the NIZK and can be verified by inspection. 

%Our construction is based on the Naor-Yung paradigm where we combine three ElGamal-type ciphertext with shared randomness.  


%Similarly to \cite{SCN:BiaMasVen16} we define a PPT algorithm $\rerand$ in Figure~\ref{fig:rerand}, which takes two ciphertexts generated with independent randomness, both public keys, only one secret key (in our case $k$) and randomness which was used to encrypt a message using the public key $g, h_1$. 
%%We assume that modulus $N$ is implicitly known.\todo{Why not explicit?} 
%% which corresponds to the secret key which is given as the input. 
%
%\begin{figure}[tb]
%\centering
%\begin{minipage}{0.75\textwidth}
%$\underline{\rerand(c:= (g^{r}, h_1^{r}\cdot m), c':= (g^{r'}, h_2^{r'}\cdot m), N, h_1, h_2, k, r)}:$
%\vspace{-2mm}
%\begin{itemize}
%\item $c_0:= g^{r}\cdot{g^{r'}} = g^{r+r'} \bmod N$;
%\item $c_1:= (g^{r'})^k \cdot h_1^{r}\cdot m  =  h_1^{r'}\cdot h_1^{r}\cdot m = h_1^{r+r'}\cdot m \bmod N$;
%\item $c_2:=h_2^{r} \cdot h_2^{r'}\cdot m = h_2^{r+r'}\cdot m \bmod N$.
%\end{itemize}
%\end{minipage}
%\caption{\label{fig:rerand}Algorithm $\rerand$.}
%\end{figure}
%
%
%
%
%It is straightforward to see that the ciphertext returned by $\rerand$ is perfectly distributed to the ciphertext produced using a shared randomness where the pair $(c_0, c_1)$ encrypts a message $m$ and the pair $(c_0, c_2)$ encrypts message $m'$.


% We note that if we use value $2^T$ as a secret key, then in order to compute $c_2$ we have to execute $T$ repeated squarings, but since $T$ is polynomial in $\secpar$ this computations is considered to be efficient.

\newsequenceofgames{NITC-MH-ROM}
To prove security we define a sequence of games $\games_0 - \games_{10}$.  For $i \in \{0,1,\dots,10\}$ we denote by $\games_i = 1$ the event that the adversary $\adv = \{(\adv_{1,\secpar}, \adv_{2, \secpar})\}_{\secpar \in \nats}$ outputs $b'$ in the game $\games_i$ such that $b = b'$.
%In individual games we use the algorithm $\decom$ define in \Cref{fig:deco-rom-mh} to answer decommitment queries efficiently. 
\begin{figure}[h!]
\begin{center}
\begin{tabular}{|l|}
\hline
$\underline{\decom(\crs, c, \sk, i)}$\\
Parse $c$ as $(c_0, c_1, c_2, \pi)$\\
if $\nizk.\vrfy((h_1, h_2, c_0, c_1, c_2, c_3), \pi)= 1$\\
\tab Compute $y:= c_0^{\sk} \bmod N$\\
\tab return $c_i \cdot y^{-1} \bmod N$\\
return $\bot$\\
\hline          
\end{tabular}
\caption{Decommitment oracle}
\label{fig:deco-rom-mh}
\end{center}
\end{figure}

\nextgame{G0}
Game $\games_\thisgame$ corresponds to the original security experiment where decommitment queries are answered using $\fdecom$.

\nextgame{DecOracle}
In game $\games_\thisgame$ decommitment queries are answered using the algorithm $\decom$ defined in \Cref{fig:deco-rom-mh} with $i:=1, \sk:=k_1$ which means that secret key $k_1$ and ciphertext $c_1$ are used, to answer decommitment queries efficiently. 

\begin{lemma}\label{nitc-rom-mh:flem}
\[
\left|\Pr[\games_\prevgame = 1] - \Pr[\games_\thisgame = 1]\right| \leq \snd^\nizk_\advB.
\]
\end{lemma}

Notice that if $c_1$ and $c_2$ contain the same message, both oracles answer decommitment queries consistently. Let $\event$ denote the event that the adversary $\adv$ asks a decommitment query $c$ such that its decommitment using the key $k_1$ is different from its decommitment using $\fdecom$. Since $\games_\prevgame$ and $\games_\thisgame$ are identical until $\event$ happens, we bound the probability of $\event$. Concretely, we have

\[
\left|\Pr[\games_\prevgame = 1] - \Pr[\games_\thisgame = 1]\right| \leq \Pr[\event]. 
\]

We construct an adversary $\advB$ that breaks soundness of the NIZK. 
The adversary $\advB_{\secpar}(p,q)$ proceeds as follows:
\vspace{-2mm}
\begin{enumerate}
\item Samples $k_1\rand \smplsetqrn$, computes $h_1 := g^{k_1} \bmod N,\varphi(N):=(p-1)(q-1), t:=2^{T} \bmod \ordqrn, h_2:=g^t \bmod N$ and sets $\crs:=(N, T(\secpar), g, h_1, h_2)$. 
\item Then it runs $(m_0, m_1, \st) \leftarrow \adv_{1, \secpar}(\crs)$ and answers decommitment queries using $k_1$.
 It samples $b \rand \bits, r \rand \smplsetqrn$ and computes $c_0^*:=g^r, c_1^*:=h_1^{r}m_b, c_2^*:=h_2^{r}m_b$. It sets $(s:=\mathlist(h_1, h_2,c_0^*, c_1^*, c_2^*), w:=(m,r))$ and runs $\pi^* \leftarrow \nizk.\prove(s,w)$.
\item Next, it runs $b' \leftarrow \adv_{2, \secpar}((c_0^*, c_1^*, c_2^*), \pi^*, \st)$ and answers decommitment queries using $k_1$.
\item Finally, it checks whether there exists a decommitment query $c: = \mathlist(c_0, c_1, c_2, \pi)$ such that $\fdecom(\crs, c) \neq \dec(\crs,c,k_1,1)$. If $\event$ occurs, then this is the case, and it returns $((h_1, h_2, c_0, c_1, c_2), \pi)$. Notice that this can be done efficiently with the knowledge of $t$.
\end{enumerate}

$\advB$ simulates $\games_\thisgame$ perfectly and if the event $\event$ happens, then it outputs a valid proof for a statement which is not in the specified language $L$. Therefore we get
\[\Pr[\event] \leq \snd^\nizk_\advB.\]



%\begin{lemma}\label{nitc-rom-mh:flem}
%\[
%\Pr[\games_\prevgame = 1] = \Pr[\games_\thisgame = 1].
%\]
%\end{lemma}
%
%Notice that both $\fdecom$ and $\decom$ answer decommitment queries in the exactly same way, hence the change is only syntactical. 



%Notice that if $c_1$ and $c_3$ contain the same message, both oracles answer decommitment queries consistently. Let $\event$ denote the event that the adversary $\adv$ asks a decommitment query $c$ such that its decommitment using the key $k_1$ is different from its decommitment using $\fdecom$. Since $\games_\prevgame$ and $\games_\thisgame$ are identical until $\event$ happens, we bound the probability of $\event$. Concretely, we have
%
%\[
%\left|\Pr[\games_\prevgame = 1] - \Pr[\games_\thisgame = 1]\right| \leq \Pr[\event]. 
%\]
%
%We construct an adversary $\advB$ that breaks soundness of the NIZK. 
%The adversary $\advB_{\secpar}$ proceeds as follows:
%\vspace{-2mm}
%\begin{enumerate}
%\item It computes $\crs \leftarrow \pgen(\seck, T)$ as defined in the construction where the value $h_3$ is computed using repeated squaring instead.
%\item Then it runs $(m_0, m_1, \st) \leftarrow \adv_{1, \secpar}(\crs)$ and answers decommitment queries using $k_1$.
% It samples $b \rand \bits, r \rand \smplset$ and computes $c_0^*:=g^r, c_1^*:=h_1^{r}m_b, c_2^*:=h_2^{r}m_b, c_3^*:=h_3^{r}m_b$. It sets $(s:=\mathlist(c_0^*, c_1^*, c_2^*, c_3^*), w:=(m,r))$ and runs $\pi^* \leftarrow \nizk.\prove(s,w)$.
%\item Next, it runs $b' \leftarrow \adv_{2, \secpar}((c_0^*, c_1^*, c_2^*, c_3^*), \pi^*, \st)$ and answers decommitment queries using $k_1$.
%\item Finally, it checks whether there exists a decommitment query $c: = \mathlist(c_0, c_1, c_2, c_3, \pi)$ such that $\fdecom(\crs, c) \neq \dec(\crs,c,1)$. If $\event$ occurs, then this is the case, and it returns $((h_1, h_2, c_0, c_1, c_2, c_3), \pi)$. %Notice that this can be done efficiently with the knowledge of $t$.
%\end{enumerate}
%
%$\advB$ simulates $\games_\thisgame$ perfectly and if the event $\event$ happens, then it outputs a valid proof for a statement which is not in the specified language $L$. Therefore we get
%\[\Pr[\event] \leq \snd^\nizk_\advB.\]

%Let $\gnr$ denote the event that the sampled $g$ in $\kgen$ is a generator of $\qrn$. Recall that $N = pq$ where $p = 2p'+1$ and $q = 2q'+1$. Because $g$ is sampled uniformly at random and $\qrn$ has $\varphi(|\qrn|) = (p'-1)(q'-1)$ generators, this event happens with overwhelming probability. Concretely, $\Pr[\gnr] = 1-\frac{1}{p'}-\frac{1}{q'}+\frac{1}{p'q'}$.
%Therefore the following holds.
%\begin{lemma}\label{tpke3:flem} 
%\begin{align*}
%\Pr[\games_\thisgame = 1] &= \Pr[\games_\thisgame = 1| \gnr]\Pr[\gnr] + \Pr[\games_\thisgame = 1| \overline{\gnr}]\Pr[\overline{\gnr}] \\
%&\leq \Pr[\games_\thisgame = 1| \gnr]\Pr[\gnr] + \Pr[\overline{\gnr}] \\
%&= \Pr[\games_\thisgame = 1| \gnr]\left( 1-\frac{1}{p'}-\frac{1}{q'}+\frac{1}{p'q'} \right) + \frac{1}{p'}+\frac{1}{q'}-\frac{1}{p'q'}.
%\end{align*}
%\end{lemma}




\nextgame{SimulProof}
Game $\games_\thisgame$ proceeds exactly as the previous game but we use the zero-knowledge simulator $(\pi, \st) \leftarrow \simul(1, \st, (h_1, h_2, c_0^*,c_1^*,c_2^*))$ to produce a simulated proof for the challenge commitment and $\simul(2, \st, \cdot)$ to answer random oracle queries. By zero-knowledge security of underlying NIZK we directly obtain
\begin{lemma}
\[
\left|\Pr[\games_\prevgame = 1] - \Pr[\games_\thisgame = 1]\right| \leq \zk^\nizk_\advB.
\]
\end{lemma}


We construct an adversary $\advB = \{\advB_\secpar\}_{\secpar \in \N}(p,q)$ against zero-knowledge security of NIZK as follows:
\vspace{-2mm}
\begin{enumerate}
\item Samples $k_1\rand \smplsetqrn$, computes $h_1 := g^{k_1} \bmod N,\varphi(N):=(p-1)(q-1), t:=2^{T} \bmod \ordqrn, h_2:=g^t \bmod N$ and sets $\crs:=(N, T(\secpar), g, h_1, h_2)$. 
\item Runs $(m_0, m_1, \st) \leftarrow \adv_{1, \secpar}(\crs)$ and answers decommitment queries using $k_1$.
\item Samples $b \rand \bits, r \rand \smplset$ and computes $c_0^*:=g^r, c_1^*:=h_1^{r}m_b, c_2^*:=h_2^{r}m_b$. It submits $(s:=\mathlist(h_1,h_2,c_0^*, c_1^*, c_2^*), w:=(m,r))$ to its oracle and obtains proof $\pi^*$ as answer.
\item Runs $b' \leftarrow \adv_{2, \secpar}((c_0^*, c_1^*, c_2^*), \pi^*, \st)$ and answers decommitment queries using $k_1$.
\item Returns the truth value of $b=b'$.
\end{enumerate}
If the proof $\pi^*$ is generated using $\nizk.\prove$, then $\advB$ simulates $\games_\prevgame$ perfectly. Otherwise $\pi^*$ is generated using $\simul_1$ and $\advB$ simulates $\games_\thisgame$ perfectly. This proofs the lemma.

\nextgame{RndExp}
In $\games_\thisgame$ we sample $r$ uniformly at random from $[\varphi(N)/2]$. 
\begin{lemma}
\[
\left|\Pr[\games_\prevgame = 1] - \Pr[\games_\thisgame = 1]\right| \leq \frac{1}{p}+\frac{1}{q}-\frac{1}{N}.
\]
\end{lemma}
Since the only difference between the two games is in the set from which we sample $r$, to upper bound the advantage of adversary we can use \Cref{sampling-lemma}, which directly yields required upper bound.

\nextgame{SSSA}
In $\games_\thisgame$ we sample $y_2 \rand \qrn$ and compute $c_2^*$ as $y_2 m_b$.

Let $\tilT_\sss(\secpar)$ be the polynomial whose existence is guaranteed by the SSS assumption.
Let $\poly_\advB(\secpar)$ be the fixed polynomial which bounds the time required to execute Steps 1--2 and answer decommitment queries in Step 3 of the adversary $\advB_{2, \secpar}$ defined below. Set $\undT := (\poly_\advB(\secpar))^{1 / \ugap}$.  Set $\tilT_\nitc := \max(\tilT_\sss, \undT)$.
\begin{lemma}
From any polynomial-size adversary $\adv = \{(\adv_{1,\secpar}, \adv_{2, \secpar})\}_{\secpar \in \nats}$, where depth of $\adv_{2, \secpar}$ is at most $T^{\ugap}(\secpar)$ for some $T(\cdot) \geq \undT(\cdot)$ we can construct a polynomial-size adversary $\advB = \{(\advB_{1,\secpar}, \advB_{2, \secpar})\}_{\secpar \in \nats}$ where the depth of $\advB_{2, \secpar}$ is at most $T^{\gap}(\secpar)$ with
\[
\left|\Pr[\games_\prevgame = 1] - \Pr[\games_\thisgame = 1]\right| \leq \advtg_\advB^\sss.
\]
\end{lemma}

The adversary $\advB_{1,\secpar}(N, T(\secpar), g):$
\vspace{-2mm}
\begin{enumerate}
\item Samples $k_1 \rand \smplsetqrn$, computes $h_1 := g^{k_1} \bmod N, h_2 := g^{2^{T(\secpar)}} \bmod N$ and sets $\crs:=(N, T(\secpar), g, h_1, h_2)$. Notice that value $h_2$ is computed by repeated squaring.
\item Runs $(m_0, m_1, \st) \leftarrow \adv_{1, \secpar}(\crs)$ and answers decommitment queries using $k_1$.
\item Outputs $(N,g,k_1, h_1,h_2, m_0, m_1, \st)$
\end{enumerate}

The adversary $\advB_{2,\secpar}(x,y,(N,g,k_1, h_1,h_2, m_0, m_1, \st)):$
\vspace{-2mm}
\begin{enumerate}
\item Samples $b \rand \bits$, computes $c_0^*:=x, c_1^*:=x^{k_1} m_b, c_2^*:=y m_b$.
\item Runs $\pi^* \leftarrow \simul(1, \st', (h_1, h_2, c_0^*, c_1^*, c_2^*))$.
\item Runs $b' \leftarrow \adv_{2, \secpar}((c_0^*, c_1^*, c_2^*), \pi^*), \st)$ and answers decommitment queries using $k_1$.
\item Returns the truth value of $b=b'$.
\end{enumerate}
Since $g$ is a generator of $\qrn$ and $x$ is sampled uniformly at random from $\qrn$ there exists some $r \in [\varphi(N)/2]$ such that $x = g^{r}$. Therefore when $y = x^{2^T} = (g^{2^T})^{r} \bmod N$, then $\advB$ simulates $\games_\prevgame$ perfectly. Otherwise $y$ is random value and $\advB$ simulates $\games_\thisgame$ perfectly. We remark that at this point $c_2^*$ does not reveal any information about $m_b$.

Now we analyse the running time of the constructed adversary. Adversary $\advB_1$ computes $h_3$ by $T(\secpar)$ consecutive squarings and because $T(\secpar)$ is polynomial in $\secpar$, $\advB_1$ is efficient. Moreover, $\advB_2$ fulfils the depth constraint:
\[ \dep(\advB_{2,\secpar}) = \poly_\advB(\secpar) + \dep(\adv_{2,\secpar}) \leq \undT^{\ugap}(\secpar) + T^{\ugap}(\secpar) \leq 2 T^{\ugap}(\secpar) = o(T^{\gap}(\secpar)). \] 

Also $T(\cdot) \geq \tilT_\nitc(\cdot) \geq \tilT_\sss(\cdot)$ as required.  


\nextgame{QUR}
Let $(c_0^*, c_1^*, c_2^*), \pi^*=(a_0^*,a_1^*,z^*))$ is the challenge commitment. In $\games_\thisgame$ we abort experiment if adversary asks any decommitment query $(c_0, c_1,c_2,\pi=(a_0,a_1,z))$ such that $c_0 = c_0^*, c_1 = c_1^*, c_2 = c_2^*, a_0 = a_0^*, a_1 = a_1^*$ and $z \neq z^*$. 

\nextgame{Snd2}
In game $\games_\thisgame$ decommitment queries are answered using the algorithm $\decom$ defined in \Cref{fig:deco-rom-mh} with $i:=2, \sk:=t$ which means that secret key $t$ and ciphertext $c_2$ are used, to answer decommitment queries efficiently. 

Let $\event$ denote the event that adversary $\adv$ asks a decommitment query $c$ such that its decommitment using the key $k_1$ is different from its decommitment using the key $t$. Since $\games_\prevgame$ and $\games_\thisgame$ are identical until $\event$ does not happen, by the standard argument it is sufficient to upper bound the probability of happening $\event$. Concretely,  

\[
\left|\Pr[\games_\prevgame = 1] - \Pr[\games_\thisgame = 1]\right| \leq \Pr[\event]. 
\]




\nextgame{RndExp4}
In $\games_\thisgame$ we sample $k_1$ uniformly at random from $[\ordqrn]$.

\begin{lemma}
\[
\left|\Pr[\games_\prevgame = 1] - \Pr[\games_\thisgame = 1]\right| \leq \frac{1}{p}+\frac{1}{q}-\frac{1}{N}.
\]
\end{lemma}

This lemma directly follows from \Cref{sampling-lemma}.

\nextgame{DDH1}
In $\games_\thisgame$ we sample $y_1 \rand \qrn$ and compute $c_1^*$ as  $y_1 m_b$. 

\begin{lemma}
\[
\left|\Pr[\games_\prevgame = 1] - \Pr[\games_\thisgame = 1]\right| \leq \advtg_\advB^\ddh.
\]
\end{lemma}
We construct an adversary $\advB = \{\advB_\secpar\}_{\secpar \in \N}$ against DDH in the group $\qrn$. %Given \Cref{thm:ddh} this implies an adversary against DDH in large prime-order subgroups of $\Zn^*$.

$\advB_{\secpar}(N,p,q,g,g^\alpha, g^\beta, g^\gamma):$
\vspace{-2mm}
\begin{enumerate}
\item Computes $\varphi(N):=(p-1)(q-1), t:=2^{T} \bmod \varphi(N)/4, h_2:=g^t \bmod N$ and sets $\crs:=(N, T, g, h_1: = g^\alpha, h_2)$.
\item Runs $(m_0, m_1, \st) \leftarrow \adv_{1, \secpar}(\crs)$ and answers decommitment queries using $t$.
\item Samples $b \rand \bits$ and computes $(c_0^*, c_1^*, c_2^*):=(g^\beta, g^\gamma \cdot m_b, (g^\beta)^t \cdot m_b).$ Runs $\pi^* \leftarrow \simul(1, \st', (h_1, h_2, c_0^*, c_1^*, c_2^*))$.
\item Runs $b' \leftarrow \adv_{2, \secpar}(c_0^*, c_1^*, c_2^*, \pi^*), \st)$ and answers decommitment queries using $t$.
\item Returns the truth value of $b=b'$. We remark that at this point $c_1^*$ does not reveal any information about $m_b$.
\end{enumerate}
If $\gamma = \alpha\beta$ then $\advB$ simulates $\games_\prevgame$ perfectly. Otherwise $g^\gamma$ is uniform random element in $\qrn$ and $\advB$ simulates $\games_\thisgame$ perfectly. This proofs the lemma. We remark that at this point $c_1^*$ does not reveal any information about $m_b$.

\nextgame{RndExp2}
In $\games_\thisgame$ we sample $k_1$ uniformly at random from $\smplsetqrn$. 

\begin{lemma}
\[
\left|\Pr[\games_\prevgame = 1] - \Pr[\games_\thisgame = 1]\right| \leq \frac{1}{p}+\frac{1}{q}-\frac{1}{N}.
\]
\end{lemma}

This lemma directly follows from \Cref{sampling-lemma}.











\nextgame{RndExp2}
In $\games_\thisgame$ we sample $k_2$ uniformly at random from $[\varphi(N)/2]$. 

\begin{lemma}
\[
\left|\Pr[\games_\prevgame = 1] - \Pr[\games_\thisgame = 1]\right| \leq \frac{1}{p}+\frac{1}{q}-\frac{1}{N}.
\]
\end{lemma}

Again using a statistical argument this lemma directly follows from \Cref{sampling-lemma}.

\nextgame{DDH}
In $\games_\thisgame$ we sample $y_2 \rand \Jn$ and compute $c_2^*$ as  $y_2 m_b$. 

\begin{lemma}\label{lem:ddh-rom-mh}
\[
\left|\Pr[\games_\prevgame = 1] - \Pr[\games_\thisgame = 1]\right| \leq \advtg_\advB^\ddh.
\]
\end{lemma}
We construct an adversary $\advB = \{\advB_\secpar\}_{\secpar \in \N}$ against DDH in the group $\Jn$. %Given \Cref{thm:ddh} this implies an adversary against DDH in large prime-order subgroups of $\Zn^*$.

$\advB_{\secpar}(N,g,g^\alpha, g^\beta, g^\gamma):$
\vspace{-2mm}
\begin{enumerate}
\item Samples $k_1 \rand \smplset$, computes $h_1 := g^{k_1} \bmod N,  h_3 := g^{2^{T}} \bmod N$ and sets $\crs:=(N, T, g, h_1, h_2:=g^\alpha, h_3)$.
\item Runs $(m_0, m_1, \st) \leftarrow \adv_{1, \secpar}(\crs)$ and answers decommitment queries using $k_1$.
\item Samples $b \rand \bits, y_3 \rand \Jn$ and computes $(c_0^*, c_1^*, c_2^*, c_3^*):=\mathlist(g^\beta, (g^\beta)^{k_1} m_b, (g^{\gamma}) m_b, y_3 m_b).$ Runs $\pi^* \leftarrow \simul(1, \st', (h_1, h_2, c_0^*, c_1^*, c_2^*, c_3^*))$.
\item Runs $b' \leftarrow \adv_{2, \secpar}((c_0^*, c_1^*, c_2^*,c_3^*), \pi^*, \st)$ and answers decommitment queries using $k_1$.
\item Returns the truth value of $b=b'$.
\end{enumerate}
If $\gamma = \alpha\beta$, then $\advB$ simulates $\games_\prevgame$ perfectly. Otherwise $g^\gamma$ is uniform random element in $\Jn$ and $\advB$ simulates $\games_\thisgame$ perfectly. This proofs the lemma. We remark that at this point $c_2^*$ does not reveal any information about $m_b$.

\nextgame{RndExp3}
In $\games_\thisgame$ we sample $k_2$ uniformly at random from $\smplset$. 

\begin{lemma}
\[
\left|\Pr[\games_\prevgame = 1] - \Pr[\games_\thisgame = 1]\right| \leq \frac{1}{p}+\frac{1}{q}-\frac{1}{N}.
\]
\end{lemma}

This lemma directly follows from \Cref{sampling-lemma}.

\nextgame{SimSnd}

In $\games_\thisgame$ we answer decommitment queries using $\dec$ (\Cref{fig:deco-rom-mh}) with $i:=2$ which means that secret key $k_2$ and ciphertext $c_2$ are used. 

\begin{lemma}
\[
\left|\Pr[\games_\prevgame = 1] - \Pr[\games_\thisgame = 1]\right| \leq \simsnd^\nizk_\advB. 
\]
\end{lemma}

Let $\event$ denote the event that adversary $\adv$ asks a decommitment query $c$ such that its decommitment using the key $k_1$ is different from its decommitment using the key $k_2$. Since $\games_\prevgame$ and $\games_\thisgame$ are identical until $\event$ does not happen, by the standard argument it is sufficient to upper bound the probability of happening $\event$. Concretely,  

\[
\left|\Pr[\games_\prevgame = 1] - \Pr[\games_\thisgame = 1]\right| \leq \Pr[\event]. 
\]

We construct an adversary $\advB$ which breaks one-time simulation soundness of the NIZK. 

The adversary $\advB_{\secpar}^{\simul_1, \simul_2}:$
\vspace{-2mm}
\begin{enumerate}
\item Computes $\crs \leftarrow \pgen(\seck, T)$ as defined in the construction where the value $h_3$ is computed using repeated squaring instead.
\item Runs $(m_0, m_1, \st) \leftarrow \adv_{1, \secpar}(\crs)$ and answers decommitment queries using $k_2$.
\item Samples $b \rand \bits, x, y_2, y_3 \rand \Jn$ and computes $(c_0^*, c_1^*, c_2^*, c_3^*):=(x, x^{k_1} m_b,\allowbreak y_2 m_b, y_3 m_b)$. Forwards $(h_1, h_2, c_0^*, c_1^*, c_2^*, c_3^*)$ to simulation oracle $\simul_1$ and obtains a proof $\pi^*$.
\item Runs $b' \leftarrow \adv_{2, \secpar}((c_0^*, c_1^*, c_2^*, c_3^*), \pi^*, \st)$ and answers decryption queries using $k_2$.
\item Find a decommitment query $c: = (c_0, c_1, c_2, c_3, \pi)$ such that $\dec(\crs, c, 1) \neq \dec(\crs,c,2)$ and returns $((h_1, h_2, c_0, c_1, c_2, c_3), \pi)$.
\end{enumerate}

$\advB$ simulates $\games_\thisgame$ perfectly and if the event $\event$ happens, it outputs a valid proof for a statement which is not in the specified language $L$. Therefore
\[\Pr[\event] \leq \simsnd^\nizk_\advB,\]
which concludes the proof of the lemma.  

%\nextgame{ReRand2}
%In $\games_\thisgame$ we use the key $t$ and randomness $r'$ as input for rerandomization. Concretely we compute $\rerand(c,c',h_1,h_2,t,r')$. This is just conceptual change since the ciphertext distributions are the same in both games and therefore 
%
%\begin{lemma}
%\[
%\left|\Pr[\games_\prevgame = 1] = \Pr[\games_\thisgame = 1]\right|.
%\]
%\end{lemma}
%
%\nextgame{RndExp4}
%In $\games_\thisgame$ we sample $r$ uniformly at random from $\varphi(N)$. 
%
%\begin{lemma}
%\[
%\left|\Pr[\games_\prevgame = 1] - \Pr[\games_\thisgame = 1]\right| \leq \frac{1}{N}.
%\]
%\end{lemma}

\nextgame{RndExp4}
In $\games_\thisgame$ we sample $k_1$ uniformly at random from $[\varphi(N)/2]$. 

\begin{lemma}
\[
\left|\Pr[\games_\prevgame = 1] - \Pr[\games_\thisgame = 1]\right| \leq \frac{1}{p}+\frac{1}{q}-\frac{1}{N}.
\]
\end{lemma}

This lemma directly follows from \Cref{sampling-lemma}.

\nextgame{DDH2}
In $\games_\thisgame$ we sample $y_1 \rand \Jn$ and compute $c_1^*$ as  $y_1 m_b$. 

\begin{lemma}
\[
\left|\Pr[\games_\prevgame = 1] - \Pr[\games_\thisgame = 1]\right| \leq \advtg_\advB^\ddh.
\]
\end{lemma}
This can be proven in similar way as \Cref{lem:ddh-rom-mh}. We remark that at this point $c_1^*$ does not reveal any information about $m_b$.

\begin{lemma}\label{nitc-rom-mh:llem}
\[
\Pr[\games_\thisgame = 1] = \half.
\]
\end{lemma}

Clearly, $c_0^*$ is uniform random element in $\Jn$ and hence it does not contain any information about the challenge message. Since $y_1, y_2, y_3$ are sampled uniformly at random from $\Jn$ the ciphertexts $c_1^*, c_2^*, c_3^*$ are also uniform random elements in $\Jn$ and hence do not contain any information about the challenge message $m_b$. Therefore, an adversary can not do better than guessing.

By combining Lemmas \ref{nitc-rom-mh:flem} - \ref{nitc-rom-mh:llem} we obtain the following:
\begin{align*}
&\advtg^{\nitc}_{\adv} = \left| \Pr[\games_0 = 1] - \half \right| \leq \sum_{i=0}^9 \left|\Pr[\games_i = 1] - \Pr[\games_{i+1} = 1] \right| + \left|\Pr[\games_{10}- \half\right| \\
 &\leq \snd^\nizk_\advB + \zk^\nizk_\advB + \advtg^{\sss}_{\advB} + \simsnd^{\nizk}_{\advB} + 2 \advtg^{\ddh}_{\advB} + 4 \left( \frac{1}{p}+\frac{1}{q}-\frac{1}{N} \right),
\end{align*}
which concludes the proof.
% \end{proof}
% \todo{appendix end}

% section proof_of_theorem_thm:nitc-mul-rom (end)


%%% Local Variables:
%%% mode: latex
%%% TeX-master: "main"
%%% End:


\end{appendix}

\end{document}



%%% Local Variables:
%%% mode: latex
%%% TeX-master: t
%%% End:
