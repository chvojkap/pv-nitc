%!TEX root=main.tex

\subsection{Non-Interactive Zero-Knowledge Proofs}
We recall the definition of a simulation-sound non-interactive proof system (SS-NIZK) that we take from Libert~\etal \cite{Libert2021OneShotFN}. 

\begin{definition} 
A \emph{non-interactive zero-knowledge proof system} $\Pi$ for an NP language $L$ associated with a relation $\rel$ is a tuple of four PPT algorithms $(\gen_\param, \gen_L, \prove, \vrfy)$, which work as follows:
\begin{itemize}
%\item $\param \leftarrow \gen_\param(\seck)$ takes as input binary representation of a security parameter $\seck$ and outputs public parameters $\param$.
\item $\crs \leftarrow \setup(\seck, L)$ takes a security parameter $\seck$ and the description of a language $L$.  It outputs a a common reference string $\crs$. 
%\item $\crs \leftarrow \gen_L(\seck, L)$ takes as input binary representation of a security parameter $\seck$ and the description of language $L$ which specifies length of statements $n$.  It outputs a language dependent part $\crs_L$ of the common reference string $\crs:=(\param, \crs_L)$. 
%\item $\crs \leftarrow \gen_L(\seck, L, \tau_L)$ takes as input binary representation of a security parameter $\seck$, the description of language $L$ which specifies length of statements $n$ and a membership testing trapdoor $\tau_L$ for $L$.  It outputs a language dependent part $\crs_L$ of the common reference string $\crs:=(\param, \crs_L)$. 
\item $\pi \leftarrow \prove(\crs,s,w)$ is a PPT algorithm which takes as input the common reference string $\crs$, a statement $s$, and a witness $w$ such that $(s,w) \in \rel$ and outputs a proof $\pi$.
\item $0/1 \leftarrow \vrfy(\crs, s, \pi)$ is a deterministic algorithm which takes as input the common reference string $\crs$, a statement $s$ and a proof $\pi$ and outputs either 1 or 0, where 1 means that the proof is ``accepted'' and 0 means it is ``rejected''.
\end{itemize}
Moreover, $\Pi$ should satisfy the following properties. 
%For simplification we denote below by $\setup$ an algorithm that runs successively $\gen_\param$ and $\gen_L$ to generate a common reference string. 
\begin{itemize}
\item \emph{Completeness:}  for all $(s, w) \in \rel$ holds:
\[\Pr[\vrfy(\crs, s,\pi)=1:\crs \leftarrow \setup(\seck, L), \pi \leftarrow \prove(\crs, s,w)] =1.\] 
\item \emph{Soundness:} for all non-uniform polynomial-size adversaries $\adv = \{\adv_\secpar\}_{\secpar \in \nats}$ there exists a negligible function $\negl(\cdot)$ such that for all $\secpar \in N$
\[
\snd_\adv^\nizk = \Pr\left[
\begin{aligned}
s \notin L \land \\
\vrfy(\crs, s, \pi) = 1
\end{aligned}
:
\begin{aligned}
(\crs \leftarrow \setup(\seck, L) \\
(\pi, s) \leftarrow \adv_\secpar(\crs, \tau_L)
\end{aligned} \right] \leq \negl(\secpar),
\]
where $\tau_L$ is membership testing trapdoor. 
\item \emph{Zero-Knowledge:} there is a PPT simulator $(\simul_1, \simul_2)$, such that for all non-uniform polynomial-size adversaries $\adv = \{\adv_\secpar\}_{\secpar \in \nats}$ there exists a negligible function $\negl(\cdot)$ such that for all $\secpar \in \nats$:
\begin{align*}
\zk_\adv^\nizk =& 
\left| \Pr\left[ \adv_\secpar^{\prove(\crs, \cdot, \cdot),}(\crs, \tau_L) = 1:\crs \leftarrow \setup(\seck, L) \right] \right. \\
&\left. - \Pr\left[ \adv_\secpar^{\oracle(\crs, \tau, \cdot, \cdot),}(\crs, \tau_L) = 1: (\crs, \tau) \leftarrow \simul_1(\seck, L) \right] \right|
\leq \negl(\secpar).
\end{align*}
Here $\tau_L$ is a membership testing trapdoor for language $L$; $\prove(\crs, \cdot, \cdot)$ is an oracle that outputs $\bot$ on input $(s,w) \notin \rel$ and outputs a valid proof $\pi \leftarrow \prove(\crs, s, w)$ otherwise; $\oracle(\crs, \tau, \cdot, \cdot)$ is an oracle that outputs $\bot$ on input $(s,w) \notin \rel$ and outputs a simulated proof $\pi \leftarrow \simul_2(\crs, \tau, s)$ on input $(s,w) \in \rel$. Note that the simulated proof is generated independently of the witness $w$.
\end{itemize}

\end{definition}

\begin{remark} 
We have slightly modified the soundness and zero-knowledge definitions compared to \cite{Libert2021OneShotFN}. Our soundness definition is adaptive and an adversary is given as input also a membership testing trapdoor $\tau_L$. This notion is implied by the simulation-soundness as defined in \Cref{def:simsnd}. Our zero-knowledge definition provides a membership testing trapdoor $\tau_L$ as an input for an adversary, whereas the definition of \cite{Libert2021OneShotFN} lets an adversary generate the language $L$ itself. The definition of \cite{Libert2021OneShotFN} works in our constructions too, but we prefer to base our constructions on a slightly weaker definition. 
\end{remark}

\begin{definition}[One-Time Simulation Soundness]\label{def:simsnd}
A NIZK for an NP language $L$ with zero-knowledge simulator $\simul = (\simul_0, \simul_1)$ is \emph{one-time simulation sound}, if for all non-uniform polynomial-size adversaries $\adv = \{\adv_\secpar\}_{\secpar \in \nats}$ there exists a negligible function $\negl(\cdot)$ such that for all $\secpar \in \nats$ 
\[
\simsnd_\adv^\nizk = \Pr\left[
\begin{aligned}
s \notin L \land \\
(s, \pi) \neq (s', \pi') \land \\
\vrfy(\crs, s, \pi) = 1
\end{aligned}
:
\begin{aligned}
(\crs, \tau) \leftarrow \simul_1(\seck, L) \\
(s, \pi) \leftarrow \adv_\secpar^{\simul_2(\crs, \tau, \cdot)}(\crs, \tau_L)
\end{aligned} \right] \leq \negl(\secpar),
\]
where $\tau_L$ is a membership testing trapdoor for language $L$ and $\simul_2(\crs, \tau, \cdot)$ is a single query oracle which on input $s'$ returns $\pi' \leftarrow \simul(\crs, \tau, s')$.
\end{definition}

Libert~\etal \cite{Libert2021OneShotFN} show that given an additively homomorphic encryption scheme, one can build a trapdoor Sigma protocol for the language defined below. Moreover, any trapdoor Sigma protocol can be turned into an unbounded simulation sound NIZK which directly implies existence of a one-time simulation sound NIZK. Since we use the term \emph{trapdoor Sigma protocol} only as intermediate notion and never instantiate it, we do not state formal definition and only reference it for brevity. For more details about trapdoor Sigma protocols see e.g. \cite{Libert2021OneShotFN}. 

\begin{lemma}[Lemma D.1~\cite{Libert2021OneShotFN}]\label{lem:tsp}
Let $(\gen, \enc, \dec)$ be an additively homomorphic encryption scheme where the message space $M$, randomness space $R$ and the ciphertext space $C$ form groups $(M, +), (R,+)$ and $(C, \cdot)$. Let the encryption scheme be such that for any public key $\pk$ generated using $(\pk, \sk) \leftarrow \gen(\seck)$, any messages $m_1, m_2 \in M$ and randomness $r_1, r_2 \in R$ holds
\[\enc(\pk, m_1;r_1) \cdot \enc(\pk, m_2;r_2) = \enc(\pk, m_1+m_2; r_1+r_2).\]
Let $S$ be a finite set of public cardinality such that uniform sampling from $S$ is computationally indistinguishable from uniform sampling from $R$. 
Then there is an trapdoor Sigma protocol for the language $L:= \{c \in C| \exists r \in R: c = \enc(\pk, 0; r)\}$ of encryptions of zero, where $\pk$ is fixed by the language.  
\end{lemma}

\begin{remark}
We note that Libert~\etal required that the order of the group $(R,+)$ is public and that this group is efficiently samplable, which is used in their proof of the zero-knowledge property. This is however, not necessary, since it is sufficient to be able to sample from a distribution which is computationally indistinguishable from the uniform distribution. This results in computational indistinguishability of real and simulated transcripts. In case of our constructions, we will sample randomness from a distribution which is statistically close and hence indistinguishable from the uniform distribution over $R$, which yields that the real and the simulated transcripts are statistically indistinguishable. 
\end{remark}

Additionally, Libert~\etal construct a simulation sound non-interactive argument system from any trapdoor Sigma protocol relying on a strongly unforgeable one-time signature, a lossy public-key encryption scheme, an admissible hash function and a correlation intractable hash function. 

\begin{theorem}[Thm B.1, Thm. B.2~\cite{Libert2021OneShotFN}]\label{thm:nizk}
Let $(\gen_\param, \gen_L, \prove, \vrfy)$ be a trapdoor Sigma protocol for an NP language $L$. Then given a strongly unforgeable one-time signature scheme, $\rel$-lossy public-key encryption scheme, a correlation intractable hash function and an admissible hash function, there is an unbounded simulation sound non-interactive zero-knowledge proof system for the language $L$. 
\end{theorem}

We note that in order to achieve negligible soundness error, it is needed to run the underlying trapdoor Sigma protocol $\mathcal{O}(\log \secpar)$ times in parallel. One run of the trapdoor Sigma protocol of Libert~\etal for $L$, as defined above, corresponds to sending one ciphertext of the homomorphic encryption scheme and one random element $r \in R$.

\subsection{Standard-Model Instantiation of SS-NIZKs}\label{sec:nizk-crs}
In this section we provide simulation sound NIZK proof systems for languages $L_1$ and $L_2$ that are used in our constructions. The languages are defined in the following way:  
\[
L_1 = \left\{(c_0, c_1, c_2, c_3)| \exists (m,r):
\begin{aligned}
       (\land_{i=1}^3 c_i = h_i^{rN}(1+N)^m \bmod N^2) \land \\
       c_0 = g^r \bmod N\\
    \end{aligned}
    \right\} \text{ and } 
\]
\[
L_2 = \left\{(c_0, c_1, c_2)| \exists (m,r):
\begin{aligned}
       (\land_{i=1}^2 c_i = h_i^{r}m \bmod N) \land
       c_0 = g^r \bmod N\\
    \end{aligned}
    \right\},   
\]
where $g, h_1, h_2, h_3, N$ are parameters defining the languages. 

Note that $L_1$ consists of all ciphertexts $(c_0, c_1\cdot (c_2)^{-1}, c_3\cdot (c_2)^{-1})$ that are encryptions of zero, where the corresponding public key is defined as $\pk:=(g, (h_1\cdot (h_2)^{-1}), (h_3\cdot (h_2)^{-1}), N)$ and encryption is defined as $\enc(\pk:=(g,h,h'),m): c:=g^r \bmod N, c':=h^{rN}(1+N)^m \bmod N^2, c':=h'^{rN}(1+N)^m \bmod N^2$. $L_2$ consists of all ciphertexts $(c_0, c_1\cdot (c_2)^{-1})$ that are encryptions of zero, where the corresponding public key is defined as $\pk:=(g, (h_1\cdot (h_2)^{-1})), N)$ and encryption is defined as $\enc(\pk:=(g,h),m): c:=g^r, c':=hg^m \bmod N$. Hence, both encryption schemes are additively homomorphic and by \Cref{lem:tsp} we obtain a trapdoor Sigma protocol for the languages $L_1, L_2$. By \Cref{thm:nizk} this yields unbounded simulation-sound NIZKs for these languages.

%Note that these languages consists of all ciphertexts $(c_0, c_1\cdot (c_2)^{-1}, c_3\cdot (c_2)^{-1})$ that are encryptions of zero, where the corresponding public key is defined as $\pk:=(g, (h_1\cdot (h_2)^{-1}), (h_3\cdot (h_2)^{-1}), N)$. For $L_1$ encryption is defined as $\enc(\pk:=(g,h,h'),m): c:=g^r \bmod N, c':=h^{rN}(1+N)^m \bmod N^2, c':=h'^{rN}(1+N)^m \bmod N^2$. For $L_2$ encryption is defined as $\enc(\pk:=(g,h,h'),m): c:=g^r, c':=hg^m, c':=h'g^m \bmod N$. Hence, both encryption schemes are additively homomorphic and by \Cref{lem:tsp} we obtain a trapdoor Sigma protocol for the languages $L_1, L_2$. By \Cref{thm:nizk} this yields unbounded simulation-sound NIZKs for these languages.








%%% Local Variables:
%%% mode: latex
%%% TeX-master: "main"
%%% End:
