%!TEX root=./main.tex
\section{Proof of  \Cref{sampling-lemma}}\label{proof:smpl-lemma}
\primelemma*


\begin{proof}
The following computation proves the lemma:
\begin{align*}
\sd(X, Y) = \half \sum_{r \in [\floor{N/2}]} \abs{\Pr[X = r] - \Pr[Y = r]} = \\ 
\half \left( \sum_{r = 1}^{\varphi(N)/2} \abs{\Pr[X = r] - \Pr[Y = r]} + \sum_{r = \varphi(N)/2+1}^{\floor{N/2}} \abs{\Pr[X = r] - \Pr[Y = r]} \right) = 
\\
\half \left( \sum_{r = 1}^{\varphi(N)/2} \abs{\frac{1}{\floor{N/2}} - \frac{2}{\varphi(N)}} + \sum_{r = \varphi(N)/2+1}^{\floor{N/2}} \abs{\frac{1}{\floor{N/2}} - 0} \right) \leq \\
\half \left( \sum_{r = 1}^{\varphi(N)/2} \abs{\frac{2}{N} - \frac{2}{\varphi(N)}} + \sum_{r = \varphi(N)/2+1}^{\floor{N/2}} \abs{\frac{1}{\floor{N/2}} - 0} \right) = \\
\half \left( \varphi(N)/2 \abs{\frac{2(\varphi(N) - N)}{\varphi(N)N}} + (\floor{N/2}-\varphi(N)/2) \frac{1}{\floor{N/2}} \right) = \\
 \half \left(\frac{(N-\varphi(N))}{N} + 1 - \frac{\varphi(N)/2}{{\floor{N/2}}} \right) \leq 
\half \left(\frac{(N-\varphi(N))}{N} + 1 - \frac{\varphi(N)/2}{{N/2}} \right) = \\
 \half\frac{2(N-\varphi(N))}{N} = 
\frac{(N-(N-p-q+1))}{N} = \frac{p+q-1}{N} = \frac{1}{p}+\frac{1}{q}-\frac{1}{N}.
\end{align*}
\end{proof}



\section{Proof of Theorem \ref{thm:NITC-IND-Mul-Std}} % (fold)
\label{app:NITC-IND-Mul-Std}


%\todo{Put the proof into an appendix}
%\begin{proof}
Completeness is implied by the completeness of the NIZK and can be verified by inspection. 

%Our construction is based on the Naor-Yung paradigm where we combine three ElGamal-type ciphertext with shared randomness.  


%Similarly to \cite{SCN:BiaMasVen16} we define a PPT algorithm $\rerand$ in Figure~\ref{fig:rerand}, which takes two ciphertexts generated with independent randomness, both public keys, only one secret key (in our case $k$) and randomness which was used to encrypt a message using the public key $g, h_1$. 
%%We assume that modulus $N$ is implicitly known.\todo{Why not explicit?} 
%% which corresponds to the secret key which is given as the input. 
%
%\begin{figure}[tb]
%\centering
%\begin{minipage}{0.75\textwidth}
%$\underline{\rerand(c:= (g^{r}, h_1^{r}\cdot m), c':= (g^{r'}, h_2^{r'}\cdot m), N, h_1, h_2, k, r)}:$
%\vspace{-2mm}
%\begin{itemize}
%\item $c_0:= g^{r}\cdot{g^{r'}} = g^{r+r'} \bmod N$;
%\item $c_1:= (g^{r'})^k \cdot h_1^{r}\cdot m  =  h_1^{r'}\cdot h_1^{r}\cdot m = h_1^{r+r'}\cdot m \bmod N$;
%\item $c_2:=h_2^{r} \cdot h_2^{r'}\cdot m = h_2^{r+r'}\cdot m \bmod N$.
%\end{itemize}
%\end{minipage}
%\caption{\label{fig:rerand}Algorithm $\rerand$.}
%\end{figure}
%
%
%
%
%It is straightforward to see that the ciphertext returned by $\rerand$ is perfectly distributed to the ciphertext produced using a shared randomness where the pair $(c_0, c_1)$ encrypts a message $m$ and the pair $(c_0, c_2)$ encrypts message $m'$.


% We note that if we use value $2^T$ as a secret key, then in order to compute $c_2$ we have to execute $T$ repeated squarings, but since $T$ is polynomial in $\secpar$ this computations is considered to be efficient.

\newsequenceofgames{NITC-MH}
To prove security we define a sequence of games $\games_0 - \games_{10}$.  For $i \in \{0,1,\dots,9\}$ we denote by $\games_i = 1$ the event that the adversary $\adv = \{(\adv_{1,\secpar}, \adv_{2, \secpar})\}_{\secpar \in \nats}$ outputs $b'$ in the game $\games_i$ such that $b = b'$.
%In individual games we use the algorithm $\decom$ define in \Cref{fig:deco} to answer decommitment queries efficiently. 
\begin{figure}[h!]
\begin{center}
\begin{tabular}{|l|}
\hline
$\underline{\decom(\crs, c, i)}$\\
Parse $c$ as $(c_0, c_1, c_2,, c_3, \pi)$\\
if $\nizk.\vrfy(\crs_\nizk,(c_0, c_1, c_2, c_3), \pi)= 1$\\
\tab Compute $y:= c_0^{k_i} \bmod N$\\
\tab return $c_i \cdot y^{-1} \bmod N$\\
return $\bot$\\
\hline          
\end{tabular}
\caption{Decommitment oracle}
\label{fig:deco-mh}
\end{center}
\end{figure}

\nextgame{G0}
Game $\games_\thisgame$ corresponds to the original security experiment where decommitment queries are answered using $\fdecom$.

\nextgame{DecOracle}
In game $\games_\thisgame$ decommitment queries are answered using the algorithm $\decom$ defined in \Cref{fig:deco-mh} with $i:=1$, meaning that secret key $k_1$ and ciphertext $c_1$ are used, to answer decommitment queries efficiently. 


\begin{lemma}\label{nitc-mh:flem}
\[
\left|\Pr[\games_\prevgame = 1] - \Pr[\games_\thisgame = 1]\right| \leq \snd^\nizk_\advB.
\]
\end{lemma}

Notice that if $c_1$ and $c_3$ contain the same message, both oracles answer decommitment queries consistently. Let $\event$ denote the event that the adversary $\adv$ asks a decommitment query $c$ such that its decommitment using the key $k_1$ is different from its decommitment using $\fdecom$. Since $\games_\prevgame$ and $\games_\thisgame$ are identical until $\event$ happens, we bound the probability of $\event$. Concretely, we have

\[
\left|\Pr[\games_\prevgame = 1] - \Pr[\games_\thisgame = 1]\right| \leq \Pr[\event]. 
\]

We construct an adversary $\advB$ that breaks soundness of the NIZK. It is given as input $\crs_\nizk$ together with a membership testing trapdoor $\tau_L:=(k_1, k_2, t)$ where $t:=2^T \bmod \varphi(N)/2$. 
The adversary $\advB_{\secpar}(\crs_\nizk, \tau_L)$ proceeds as follows:
\vspace{-2mm}
\begin{enumerate}
\item It computes $h_1:= g^{k_1} \bmod N, h_2:= g^{k_2} \bmod N, h_3:= g^{t} \bmod N$ using the membership testing trapdoor $\tau_L:=(k_1, k_2, t)$ and sets $\crs:=\mathlist(N, T, g, h_1, h_2, h_3, \crs_\nizk)$.
\item Then it runs $(m_0, m_1, \st) \leftarrow \adv_{1, \secpar}(\crs)$ and answers decommitment queries using $k_1$.
\item It samples $b \rand \bits, r \rand \smplset$ and computes $c_0^*:=g^r, c_1^*:=h_1^{r}m_b, c_2^*:=h_2^{r}m_b, c_3^*:=h_3^{r}m_b$. It sets $(s:=\mathlist(c_0^*, c_1^*, c_2^*, c_3^*), w:=(m,r))$ and runs $\pi^* \leftarrow \nizk.\prove(s,w)$.
\item Next, it runs $b' \leftarrow \adv_{2, \secpar}((c_0^*, c_1^*, c_2^*, c_3^*), \pi^*, \st)$ and answers decommitment queries using $k_1$.
\item Finally, it checks whether there exists a decommitment query $c: = (c_0, c_1, c_2, c_3, \pi)$ such that $\fdecom(\crs, c) \neq \dec(\crs,c,2)$. If $\event$ occurs, then this is the case, and it returns $((c_0, c_1, c_2, c_3), \pi)$. Notice that this can be done efficiently with the knowledge of $t$.
\end{enumerate}

$\advB$ simulates $\games_\thisgame$ perfectly and if the event $\event$ happens, then it outputs a valid proof for a statement which is not in the specified language $L$. Therefore we get
\[\Pr[\event] \leq \snd^\nizk_\advB.\]

%Let $\gnr$ denote the event that the sampled $g$ in $\kgen$ is a generator of $\qrn$. Recall that $N = pq$ where $p = 2p'+1$ and $q = 2q'+1$. Because $g$ is sampled uniformly at random and $\qrn$ has $\varphi(|\qrn|) = (p'-1)(q'-1)$ generators, this event happens with overwhelming probability. Concretely, $\Pr[\gnr] = 1-\frac{1}{p'}-\frac{1}{q'}+\frac{1}{p'q'}$.
%Therefore the following holds.
%\begin{lemma}\label{tpke3:flem} 
%\begin{align*}
%\Pr[\games_\thisgame = 1] &= \Pr[\games_\thisgame = 1| \gnr]\Pr[\gnr] + \Pr[\games_\thisgame = 1| \overline{\gnr}]\Pr[\overline{\gnr}] \\
%&\leq \Pr[\games_\thisgame = 1| \gnr]\Pr[\gnr] + \Pr[\overline{\gnr}] \\
%&= \Pr[\games_\thisgame = 1| \gnr]\left( 1-\frac{1}{p'}-\frac{1}{q'}+\frac{1}{p'q'} \right) + \frac{1}{p'}+\frac{1}{q'}-\frac{1}{p'q'}.
%\end{align*}
%\end{lemma}




\nextgame{SimulProof}
Game $\games_\thisgame$ proceeds exactly as the previous game but we run the zero-knowledge simulator $(\crs, \tau) \leftarrow \simul_1(\seck, L)$ in $\pgen$ and produce a simulated proof for the challenge commitment as $\pi \leftarrow \simul_2(\crs, \tau, (c_0^*,c_1^*,c_2^*,c_3^*))$. By the zero-knowledge security of the NIZK we directly obtain
\begin{lemma}
\[
\left|\Pr[\games_\prevgame = 1] - \Pr[\games_\thisgame = 1]\right| \leq \zk^\nizk_\advB.
\]
\end{lemma}

We construct an adversary $\advB = \{\advB_\secpar\}_{\secpar \in \N}$ against the zero-knowledge security of NIZK as follows:
$\advB_\secpar(\crs_\nizk, \tau_L):$
\vspace{-2mm}
\begin{enumerate}
\item Set $\crs:=\mathlist(N, T(\secpar), g, h_1, h_2, h_3, \crs_\nizk)$, run $(m_0, m_1, \st) \leftarrow \adv_{1, \secpar}(\crs)$, and answer decommitment queries using $k_1$ (which is included in $\tau_L$.
\item Sample $b \rand \bits, r \rand \smplset$ and compute $c_0^*:=g^r, c_1^*:=h_1^{r}m_b, c_2^*:=h_2^{r}m_b, c_3^*:=h_3^{r}m_b$. Then submit $(s:=\mathlist(h_1,h_2,c_0^*, c_1^*, c_2^*, c_3^*), w:=(m,r))$ to the oracle to obtain proof $\pi^*$.
\item Run $b' \leftarrow \adv_{2, \secpar}((c_0^*, c_1^*, c_2^*, c_3^*), \pi^*, \st)$, answering decommitment queries using $k_1$.
\item Return the truth value of $b=b'$.
\end{enumerate}
If the proof $\pi^*$ is generated using $\nizk.\prove$, then $\advB$ simulates $\games_\prevgame$ perfectly. Otherwise, $\pi^*$ is generated using $\simul_1$ and $\advB$ simulates $\games_\thisgame$ perfectly. This proves the lemma.

\nextgame{RndExp}
In $\games_\thisgame$ we sample $r$ uniformly at random from $[\varphi(N)/2]$. 

\begin{lemma}
\[
\left|\Pr[\games_\prevgame = 1] - \Pr[\games_\thisgame = 1]\right| \leq \frac{1}{p}+\frac{1}{q}-\frac{1}{N}.
\]
\end{lemma}
%At first we remark that for upper bounding the difference between the games we use a statistical argument. Because $r$ appears only in the exponent of the group generator, we later sample a random element from the group $\qrn$ which can be done efficiently. 
Since the only difference between the two games is in the set from which we sample $r$, to upper bound the advantage of adversary we can use \Cref{sampling-lemma}, which directly yields required upper bound.

%\nextgame{ReRand}
%In Game $\games_\thisgame$ we produce the challenge commitment by encrypting the challenge message using two independent random exponents $r \rand \smplset, r' \rand [\varphi(N)/4]$ to obtain $c:= (g^{r}, h_1^{r}\cdot m_b), c':= (g^{r'}, h_2^{r'}\cdot m_b)$ and then run $\rerand(c,c',N, \allowbreak h_1, h_2,k,r)$ to obtain resulting ciphertext $(c_0^*, c_1^*, c_2^*)$. Since $r'$ is sampled uniformly at random from $[\varphi(N)/4]$ the ciphertext distributions in both games  are the same. Therefore 
%
%\begin{lemma}
%\[
%\Pr[\games_\prevgame = 1] = \Pr[\games_\thisgame = 1].
%\]
%\end{lemma}


\nextgame{SSSA}
In $\games_\thisgame$ we sample $y_3 \rand \Jn$ and compute $c_3^*$ as $y_3 m_b$.

Let $\tilT_\sss(\secpar)$ be the polynomial whose existence is guaranteed by the SSS assumption.
Let $\poly_\advB(\secpar)$ be the fixed polynomial which bounds the time required to execute Steps 1--2 and answer decommitment queries in Step 3 of the adversary $\advB_{2, \secpar}$ defined below. Set $\undT := (\poly_\advB(\secpar))^{1 / \ugap}$.  Set $\tilT_\nitc := \max(\tilT_\sss, \undT)$.
\begin{lemma}
From any polynomial-size adversary $\adv = \{(\adv_{1,\secpar}, \adv_{2, \secpar})\}_{\secpar \in \nats}$, where depth of $\adv_{2, \secpar}$ is at most $T^{\ugap}(\secpar)$ for some $T(\cdot) \geq \undT(\cdot)$ we can construct a polynomial-size adversary $\advB = \{(\advB_{1,\secpar}, \advB_{2, \secpar})\}_{\secpar \in \nats}$ where the depth of $\advB_{2, \secpar}$ is at most $T^{\gap}(\secpar)$ with
\[
\left|\Pr[\games_\prevgame = 1] - \Pr[\games_\thisgame = 1]\right| \leq \advtg_\advB^\sss.
\]
\end{lemma}

The adversary $\advB_{1,\secpar}(N, T(\secpar), g):$
\vspace{-2mm}
\begin{enumerate}
\item Samples $k_1, k_2 \rand \smplset$, computes $h_1 := g^{k_1} \bmod N, h_2 := g^{k_2} \bmod N,  h_3 := g^{2^{T(\secpar)}} \bmod N$, runs $(\crs_\nizk, \tau) \leftarrow \nizk.\simul_1(\seck, L)$ and sets $\crs:=\mathlist(N, T(\secpar), g, h_1, h_2, h_3, \crs_\nizk)$. Notice that value $h_3$ is computed by repeated squaring.
\item Runs $(m_0, m_1, \st) \leftarrow \adv_{1, \secpar}(\crs)$ and answers decommitment queries using $k_1$.
\item Outputs $(N,g,k_1, k_2, h_1,h_2,h_3,\crs_\nizk, \tau, m_0, m_1, \st)$
\end{enumerate}

The adversary $\advB_{2,\secpar}(x,y,(N,g,k_1, k_2, h_1,h_2,h_3, \crs_\nizk, \tau, m_0, m_1, \st)):$
\vspace{-2mm}
\begin{enumerate}
\item Samples $b \rand \bits$, computes $c_0^*:=x, c_1^*:=x^{k_1} m_b, c_2^*:=x^{k_2}m_b, c_3^*:=y m_b$.
\item Runs $\pi^* \leftarrow \simul(\crs_\nizk, \tau, (c_0^*, c_1^*, c_2^*, c_3^*))$.
\item Runs $b' \leftarrow \adv_{2, \secpar}((c_0^*, c_1^*, c_2^*, c_3^*), \pi^*), \st)$ and answers decommitment queries using $k_1$.
\item Returns the truth value of $b=b'$.
\end{enumerate}
Since $g$ is a generator of $\Jn$ and $x$ is sampled uniformly at random from $\Jn$ there exists some $r \in [\varphi(N)/2]$ such that $x = g^{r}$. Therefore when $y = x^{2^T} = (g^{2^T})^{r} \bmod N$, then $\advB$ simulates $\games_\prevgame$ perfectly. Otherwise $y$ is random value and $\advB$ simulates $\games_\thisgame$ perfectly. We remark that at this point $c_3^*$ does not reveal any information about $m_b$.

Now we analyse the running time of the constructed adversary. Adversary $\advB_1$ computes $h_3$ by $T(\secpar)$ consecutive squarings and because $T(\secpar)$ is polynomial in $\secpar$, $\advB_1$ is efficient. Moreover, $\advB_2$ fulfils the depth constraint:
\[ \dep(\advB_{2,\secpar}) = \poly_\advB(\secpar) + \dep(\adv_{2,\secpar}) \leq \undT^{\ugap}(\secpar) + T^{\ugap}(\secpar) \leq 2 T^{\ugap}(\secpar) = o(T^{\gap}(\secpar)). \] 

Also $T(\cdot) \geq \tilT_\nitc(\cdot) \geq \tilT_\sss(\cdot)$ as required.  

%\nextgame{RndExp2}
%In $\games_\thisgame$ we stop to use $\rerand$ algorithm. Concretely, we sample $r \rand [\varphi(N)/4], y_2 \rand \qrn$ and compute challenge ciphertext as $c^*:=(g^r, h_1^r \cdot m_b, y_2 \cdot m_b)$. The ciphertext has the same distribution as in the previous game. Therefore 
%
%\begin{lemma}
%\[
%\Pr[\games_\prevgame = 1] = \Pr[\games_\thisgame = 1].
%\]
%\end{lemma}
%\nextgame{RndExp3}


\nextgame{RndExp2}
In $\games_\thisgame$ we sample $k_2$ uniformly at random from $[\varphi(N)/2]$. 

\begin{lemma}
\[
\left|\Pr[\games_\prevgame = 1] - \Pr[\games_\thisgame = 1]\right| \leq \frac{1}{p}+\frac{1}{q}-\frac{1}{N}.
\]
\end{lemma}

Again using a statistical argument this lemma directly follows from \Cref{sampling-lemma}.

\nextgame{DDH}
In $\games_\thisgame$ we sample $y_2 \rand \Jn$ and compute $c_2^*$ as  $y_2 m_b$. 

\begin{lemma}\label{lem-mh:ddh}
\[
\left|\Pr[\games_\prevgame = 1] - \Pr[\games_\thisgame = 1]\right| \leq \advtg_\advB^\ddh.
\]
\end{lemma}
We construct an adversary $\advB = \{\advB_\secpar\}_{\secpar \in \N}$ against DDH in the group $\Jn$. %Given \Cref{thm:ddh} this implies an adversary against DDH in large prime-order subgroups of $\Zn^*$.

$\advB_{\secpar}(N,g,g^\alpha, g^\beta, g^\gamma):$
\vspace{-2mm}
\begin{enumerate}
\item Samples $k_1 \rand \smplset$, computes $h_1 := g^{k_1} \bmod N,  h_3 := g^{2^{T}} \bmod N$, runs $(\crs_\nizk, \tau) \leftarrow \nizk.\simul_1(\seck, L)$ and sets $\crs:=\mathlist(N, T(\secpar), g, h_1, h_2: = g^\alpha, h_3, \crs_\nizk)$.
\item Runs $(m_0, m_1, \st) \leftarrow \adv_{1, \secpar}(\crs)$ and answers decommitment queries using $k_1$.
\item Samples $b \rand \bits, y_3 \rand \Jn$ and computes $(c_0^*, c_1^*, c_2^*, c_3^*):=(g^\beta, (g^\beta)^{k_1} m_b, (g^{\gamma}) m_b, y_3 m_b).$ Runs $\pi^* \leftarrow \simul(1, \st', (h_1, h_2, c_0^*, c_1^*, c_2^*, c_3^*))$.
\item Runs $b' \leftarrow \adv_{2, \secpar}((c_0^*, c_1^*, c_2^*,c_3^*), \pi^*, \st)$ and answers decommitment queries using $k_1$.
\item Returns the truth value of $b=b'$.
\end{enumerate}
If $\gamma = \alpha\beta$, then $\advB$ simulates $\games_\prevgame$ perfectly. Otherwise $g^\gamma$ is uniform random element in $\Jn$ and $\advB$ simulates $\games_\thisgame$ perfectly. This proofs the lemma. We remark that at this point $c_2^*$ does not reveal any information about $m_b$.

\nextgame{RndExp3}
In $\games_\thisgame$ we sample $k_2$ uniformly at random from $\smplset$. 

\begin{lemma}
\[
\left|\Pr[\games_\prevgame = 1] - \Pr[\games_\thisgame = 1]\right| \leq \frac{1}{p}+\frac{1}{q}-\frac{1}{N}.
\]
\end{lemma}

This lemma directly follows from \Cref{sampling-lemma}.

\nextgame{SimSnd}

In $\games_\thisgame$ we answer decommitment queries using $\dec$ (\Cref{fig:deco-mh}) with $i:=2$ which means that secret key $k_2$ and ciphertext $c_2$ are used. 

\begin{lemma}
\[
\left|\Pr[\games_\prevgame = 1] - \Pr[\games_\thisgame = 1]\right| \leq \simsnd^\nizk_\advB. 
\]
\end{lemma}

Let $\event$ denote the event that adversary $\adv$ asks a decommitment query $c$ such that its decommitment using the key $k_1$ is different from its decommitment using the key $k_2$. Since $\games_\prevgame$ and $\games_\thisgame$ are identical until $\event$ does not happen, by the standard argument it is sufficient to upper bound the probability of happening $\event$. Concretely,  

\[
\left|\Pr[\games_\prevgame = 1] - \Pr[\games_\thisgame = 1]\right| \leq \Pr[\event]. 
\]

We construct an adversary $\advB$ that breaks one-time simulation soundness of the NIZK and it is given as input $\crs_\nizk$ together with a membership testing trapdoor $\tau_L:=(k_1, k_2, t)$ where $t:=2^T \bmod \varphi(N)/2$. 

The adversary $\advB_{\secpar}^{\simul_2}(\crs_\nizk, \tau_L):$
\vspace{-2mm}
\begin{enumerate}
\item Computes $h_1:= g^{k_1} \bmod N, h_2:= g^{k_2} \bmod N, h_3:= g^{t} \bmod N$ using the membership testing trapdoor $\tau_L$ and sets $\crs:=(N, T, g, h_1, h_2, h_3, \crs_\nizk)$.
\item Runs $(m_0, m_1, \st) \leftarrow \adv_{1, \secpar}(\crs)$ and answers decommitment queries using $k_2$.
\item Samples $b \rand \bits, x, y_2, y_3 \rand \Jn$ and computes $(c_0^*, c_1^*, c_2^*, c_3^*):=(x, x^{k_1} m_b,\allowbreak y_2 m_b, y_3 m_b)$. Forwards $(c_0^*, c_1^*, c_2^*, c_3^*)$ to simulation oracle $\simul_2$ and obtains a proof $\pi^*$.
\item Runs $b' \leftarrow \adv_{2, \secpar}((c_0^*, c_1^*, c_2^*, c_3^*), \pi^*, \st)$ and answers decommitment queries using $k_2$.
\item Find a decommitment query $c: = (c_0, c_1, c_2, c_3, \pi)$ such that $\dec(\crs, c, 1) \neq \dec(\crs,c,2)$ and returns $((c_0, c_1, c_2, c_3), \pi)$
\end{enumerate}

$\advB$ simulates $\games_\thisgame$ perfectly and if the event $\event$ happens, it outputs a valid proof for a statement which is not in the specified language $L$. Therefore
\[\Pr[\event] \leq \simsnd^\nizk_\advB,\]
which concludes the proof of the lemma.  

%\nextgame{ReRand2}
%In $\games_\thisgame$ we use the key $t$ and randomness $r'$ as input for rerandomization. Concretely we compute $\rerand(c,c',h_1,h_2,t,r')$. This is just conceptual change since the ciphertext distributions are the same in both games and therefore 
%
%\begin{lemma}
%\[
%\left|\Pr[\games_\prevgame = 1] = \Pr[\games_\thisgame = 1]\right|.
%\]
%\end{lemma}
%
%\nextgame{RndExp4}
%In $\games_\thisgame$ we sample $r$ uniformly at random from $\varphi(N)$. 
%
%\begin{lemma}
%\[
%\left|\Pr[\games_\prevgame = 1] - \Pr[\games_\thisgame = 1]\right| \leq \frac{1}{N}.
%\]
%\end{lemma}

\nextgame{RndExp4}
In $\games_\thisgame$ we sample $k_1$ uniformly at random from $[\varphi(N)/2]$. 

\begin{lemma}
\[
\left|\Pr[\games_\prevgame = 1] - \Pr[\games_\thisgame = 1]\right| \leq \frac{1}{p}+\frac{1}{q}-\frac{1}{N}.
\]
\end{lemma}

This lemma directly follows from \Cref{sampling-lemma}.

\nextgame{DDH2}
In $\games_\thisgame$ we sample $y_1 \rand \Jn$ and compute $c_1^*$ as  $y_1 m_b$. 

\begin{lemma}
\[
\left|\Pr[\games_\prevgame = 1] - \Pr[\games_\thisgame = 1]\right| \leq \advtg_\advB^\ddh.
\]
\end{lemma}
This can be proven in similar way as \Cref{lem-mh:ddh}. We remark that at this point $c_1^*$ does not reveal any information about $m_b$.

\begin{lemma}\label{nitc-mh:llem}
\[
\Pr[\games_\thisgame = 1] = \half.
\]
\end{lemma}

Clearly, $c_0^*$ is uniform random element in $\Jn$ and hence it does not contain any information about the challenge message. Since $y_1, y_2, y_3$ are sampled uniformly at random from $\Jn$ the ciphertexts $c_1^*, c_2^*, c_3^*$ are also uniform random elements in $\Jn$ and hence do not contain any information about the challenge message $m_b$. Therefore, an adversary can not do better than guessing.

By combining Lemmas \ref{nitc-mh:flem} - \ref{nitc-mh:llem} we obtain the following:
\begin{align*}
&\advtg^{\nitc}_{\adv} = \left| \Pr[\games_0 = 1] - \half \right| \leq \sum_{i=0}^9 \left|\Pr[\games_i = 1] - \Pr[\games_{i+1} = 1] \right| + \left|\Pr[\games_{10}- \half\right| \\
 &\leq \snd^\nizk_\advB + \zk^\nizk_\advB + \advtg^{\sss}_{\advB} + \simsnd^{\nizk}_{\advB} + 2 \advtg^{\ddh}_{\advB} + 4 \left( \frac{1}{p}+\frac{1}{q}-\frac{1}{N} \right),
\end{align*}
which concludes the proof.
%\end{proof}

% section proof_of_theorem_thm:nitc-ind-mul-std (end)


\section{Proof of Theorem \ref{thm:NITC-lin-ROM}} % (fold)
\label{app:NITC-lin-ROM}

% \todo{appendix?}
% \begin{proof}
Completeness is implied by the completeness of the NIZK and can be verified by inspection. 

%Our construction is based on the Naor-Yung paradigm where we combine three Paillier-type ciphertext with shared randomness.  


%Similarly to \cite{SCN:BiaMasVen16} we define a PPT algorithm $\rerand$ in Figure~\ref{fig:rerand}, which takes two ciphertexts generated with independent randomness, both public keys, only one secret key (in our case $k$) and randomness which was used to encrypt a message using the public key $g, h_1$. 
%%We assume that modulus $N$ is implicitly known.\todo{Why not explicit?} 
%% which corresponds to the secret key which is given as the input. 
%
%\begin{figure}[tb]
%\centering
%\begin{minipage}{0.75\textwidth}
%$\underline{\rerand(c:= (g^{r}, h_1^{r}\cdot m), c':= (g^{r'}, h_2^{r'}\cdot m), N, h_1, h_2, k, r)}:$
%\vspace{-2mm}
%\begin{itemize}
%\item $c_0:= g^{r}\cdot{g^{r'}} = g^{r+r'} \bmod N$;
%\item $c_1:= (g^{r'})^k \cdot h_1^{r}\cdot m  =  h_1^{r'}\cdot h_1^{r}\cdot m = h_1^{r+r'}\cdot m \bmod N$;
%\item $c_2:=h_2^{r} \cdot h_2^{r'}\cdot m = h_2^{r+r'}\cdot m \bmod N$.
%\end{itemize}
%\end{minipage}
%\caption{\label{fig:rerand}Algorithm $\rerand$.}
%\end{figure}
%
%
%
%
%It is straightforward to see that the ciphertext returned by $\rerand$ is perfectly distributed to the ciphertext produced using a shared randomness where the pair $(c_0, c_1)$ encrypts a message $m$ and the pair $(c_0, c_2)$ encrypts message $m'$.


% We note that if we use value $2^T$ as a secret key, then in order to compute $c_2$ we have to execute $T$ repeated squarings, but since $T$ is polynomial in $\secpar$ this computations is considered to be efficient.

\newsequenceofgames{NITC-LH-ROM}
To prove security we define a sequence of games $\games_0 - \games_{13}$.  For $i \in \{0,1,\dots,13\}$ we denote by $\games_i = 1$ the event that the adversary $\adv = \{(\adv_{1,\secpar}, \adv_{2, \secpar})\}_{\secpar \in \nats}$ outputs $b'$ in the game $\games_i$ such that $b = b'$.
%In individual games we use the algorithm $\decom$ define in \Cref{fig:deco-rom-lh} to answer decommitment queries efficiently. 
\begin{figure}[h!]
\begin{center}
\begin{tabular}{|l|}
\hline
$\underline{\decom(\crs, c, i)}$\\
Parse $c$ as $(c_0, c_1, c_2,, c_3, \pi)$\\
if $\nizk.\vrfy((h_1, h_2, c_0, c_1, c_2, c_3), \pi)= 1$\\
\tab Compute $y:= c_0^{k_i} \bmod N$\\
\tab return $\frac{c_i \cdot y^{-N} (\bmod N^2) -1}{N}$\\
return $\bot$\\
\hline          
\end{tabular}
\caption{Decommitment oracle}
\label{fig:deco-rom-lh}
\end{center}
\end{figure}

\nextgame{G0}
Game $\games_\thisgame$ corresponds to the original security experiment where decommitment queries are answered using $\fdecom$.

\nextgame{DecOracle}
In game $\games_\thisgame$ decommitment queries are answered using the algorithm $\decom$ defined in \Cref{fig:deco-rom-lh} with $i:=1$, meaning that secret key $k_1$ and ciphertext $c_1$ are used, to answer decommitment queries efficiently. 


\begin{lemma}\label{nitc-lh:flem}
\[
\left|\Pr[\games_\prevgame = 1] - \Pr[\games_\thisgame = 1]\right| \leq \snd^\nizk_\advB.
\]
\end{lemma}

Notice that if $c_1$ and $c_3$ contain the same message, both oracles answer decommitment queries consistently. Let $\event$ denote the event that the adversary $\adv$ asks a decommitment query $c$ such that its decommitment using the key $k_1$ is different from its decommitment using $\fdecom$. Since $\games_\prevgame$ and $\games_\thisgame$ are identical until $\event$ happens, we bound the probability of $\event$. Concretely, we have

\[
\left|\Pr[\games_\prevgame = 1] - \Pr[\games_\thisgame = 1]\right| \leq \Pr[\event]. 
\]

We construct an adversary $\advB$ that breaks soundness of the NIZK. 
The adversary $\advB_{\secpar}$ proceeds as follows:
\vspace{-2mm}
\begin{enumerate}
\item It computes $\crs \leftarrow \pgen(\seck, T)$ as defined in the construction where the value $h_3$ is computed using repeated squaring instead.
\item Then it runs $(m_0, m_1, \st) \leftarrow \adv_{1, \secpar}(\crs)$ and answers decommitment queries using $k_1$.
\item It samples $b \rand \bits, r \rand \smplset$ and computes $c_0^*:=g^r, c_1^*:=h_1^{rN}(1+N)^{m_b}, c_2^*:=h_2^{rN}(1+N)^{m_b}, c_3^*:=h_3^{rN}(1+N)^{m_b}$. It sets $(s:=\mathlist(h_1, h_2, c_0^*, c_1^*, c_2^*, c_3^*), w:=(m,r))$ and runs $\pi^* \leftarrow \nizk.\prove(s,w)$.
\item Next, it runs $b' \leftarrow \adv_{2, \secpar}((c_0^*, c_1^*, c_2^*, c_3^*), \pi^*, \st)$ and answers decommitment queries using $k_1$.
\item Finally, it checks whether there exists a decommitment query $c: = \mathlist(c_0, c_1, c_2, c_3, \pi)$ such that $\fdecom(\crs, c, 1) \neq \dec(\crs,c,1)$. If $\event$ occurs, then this is the case, and it returns $((h_1, h_2, c_0, c_1, c_2, c_3), \pi)$. %Notice that this can be done efficiently with the knowledge of $t$.
\end{enumerate}

$\advB$ simulates $\games_\thisgame$ perfectly and if the event $\event$ happens, then it outputs a valid proof for a statement which is not in the specified language $L$. Therefore we get
\[\Pr[\event] \leq \snd^\nizk_\advB.\]

%Let $\gnr$ denote the event that the sampled $g$ in $\kgen$ is a generator of $\qrn$. Recall that $N = pq$ where $p = 2p'+1$ and $q = 2q'+1$. Because $g$ is sampled uniformly at random and $\qrn$ has $\varphi(|\qrn|) = (p'-1)(q'-1)$ generators, this event happens with overwhelming probability. Concretely, $\Pr[\gnr] = 1-\frac{1}{p'}-\frac{1}{q'}+\frac{1}{p'q'}$.
%Therefore the following holds.
%\begin{lemma}\label{tpke3:flem} 
%\begin{align*}
%\Pr[\games_\thisgame = 1] &= \Pr[\games_\thisgame = 1| \gnr]\Pr[\gnr] + \Pr[\games_\thisgame = 1| \overline{\gnr}]\Pr[\overline{\gnr}] \\
%&\leq \Pr[\games_\thisgame = 1| \gnr]\Pr[\gnr] + \Pr[\overline{\gnr}] \\
%&= \Pr[\games_\thisgame = 1| \gnr]\left( 1-\frac{1}{p'}-\frac{1}{q'}+\frac{1}{p'q'} \right) + \frac{1}{p'}+\frac{1}{q'}-\frac{1}{p'q'}.
%\end{align*}
%\end{lemma}




\nextgame{SimulProof}
Game $\games_\thisgame$ proceeds exactly as the previous game but we use the zero-knowledge simulator $(\pi, \st) \leftarrow \simul(1, \st, (h_1, h_2, c_0^*,c_1^*,c_2^*,c_3^*))$ to produce a simulated proof for the challenge commitment and $\simul(2, \st, \cdot)$ to answer random oracle queries. By zero-knowledge security of underlying NIZK we directly obtain
\begin{lemma}\label{nitc-rom-lh:flem}
\[
\left|\Pr[\games_\prevgame = 1] - \Pr[\games_\thisgame = 1]\right| \leq \zk^\nizk_\advB.
\]
\end{lemma}

We construct an adversary $\advB = \{\advB_\secpar\}_{\secpar \in \N}$ against zero-knowledge security of NIZK as follows:
\vspace{-2mm}
\begin{enumerate}
\item Samples $k_1, k_2 \rand \smplset$, computes $h_1 := g^{k_1} \bmod N, h_2 := g^{k_2} \bmod N$ and sets $\crs:=(N, T(\secpar), g, h_1, h_2, h_3)$. 
\item Runs $(m_0, m_1, \st) \leftarrow \adv_{1, \secpar}(\crs)$ and answers decommitment queries using $k_1$.
\item Samples $b \rand \bits, r \rand \smplset$ and computes $c_0^*:=g^r, c_1^*:=h_1^{rN}(1+N)^{m_b}, c_2^*:=h_2^{rN}(1+N)^{m_b}, c_3^*:=h_3^{rN}(1+N)^{m_b}$. It submits $(s:=\mathlist(h_1,h_2,c_0^*, c_1^*, c_2^*, c_3^*), w:=(m,r))$ to its oracle and obtains proof $\pi^*$ as answer.
\item Runs $b' \leftarrow \adv_{2, \secpar}((c_0^*, c_1^*, c_2^*, c_3^*), \pi^*, \st)$ and answers decommitment queries using $k_1$.
\item Returns the truth value of $b=b'$.
\end{enumerate}
If the proof $\pi^*$ is generated using $\nizk.\prove$, then $\advB$ simulates $\games_\prevgame$ perfectly. Otherwise $\pi^*$ is generated using $\simul_1$ and $\advB$ simulates $\games_\thisgame$ perfectly. This proofs the lemma.


\nextgame{RndExp}
In $\games_\thisgame$ we sample $r$ uniformly at random from $[\varphi(N)/2]$. 

\begin{lemma}
\[
\left|\Pr[\games_\prevgame = 1] - \Pr[\games_\thisgame = 1]\right| \leq \frac{1}{p}+\frac{1}{q}-\frac{1}{N}.
\]
\end{lemma}
%At first we remark that for upper bounding the difference between the games we use a statistical argument. Because $r$ appears only in the exponent of the group generator, we later sample a random element from the group $\qrn$ which can be done efficiently. 
Since the only difference between the two games is in the set from which we sample $r$, to upper bound the advantage of adversary we can use \Cref{sampling-lemma}, which directly yields required upper bound.

%\nextgame{ReRand}
%In Game $\games_\thisgame$ we produce the challenge commitment by encrypting the challenge message using two independent random exponents $r \rand \smplset, r' \rand [\varphi(N)/4]$ to obtain $c:= (g^{r}, h_1^{r}\cdot m_b), c':= (g^{r'}, h_2^{r'}\cdot m_b)$ and then run $\rerand(c,c',N, \allowbreak h_1, h_2,k,r)$ to obtain resulting ciphertext $(c_0^*, c_1^*, c_2^*)$. Since $r'$ is sampled uniformly at random from $[\varphi(N)/4]$ the ciphertext distributions in both games  are the same. Therefore 
%
%\begin{lemma}
%\[
%\Pr[\games_\prevgame = 1] = \Pr[\games_\thisgame = 1].
%\]
%\end{lemma}


\nextgame{SSSA}
In $\games_\thisgame$ we sample $y_3 \rand \Jn$ and compute $c_3^*$ as $y_3^N (1+N)^{m_b}$.

Let $\tilT_\sss(\secpar)$ be the polynomial whose existence is guaranteed by the SSS assumption.
Let $\poly_\advB(\secpar)$ be the fixed polynomial which bounds the time required to execute Steps 1--2 and answer decommitment queries in Step 3 of the adversary $\advB_{2, \secpar}$ defined below. Set $\undT := (\poly_\advB(\secpar))^{1 / \ugap}$.  Set $\tilT_\nitc := \max(\tilT_\sss, \undT)$.
\begin{lemma}
From any polynomial-size adversary $\adv = \{(\adv_{1,\secpar}, \adv_{2, \secpar})\}_{\secpar \in \nats}$, where depth of $\adv_{2, \secpar}$ is at most $T^{\ugap}(\secpar)$ for some $T(\cdot) \geq \undT(\cdot)$ we can construct a polynomial-size adversary $\advB = \{(\advB_{1,\secpar}, \advB_{2, \secpar})\}_{\secpar \in \nats}$ where the depth of $\advB_{2, \secpar}$ is at most $T^{\gap}(\secpar)$ with
\[
\left|\Pr[\games_\prevgame = 1] - \Pr[\games_\thisgame = 1]\right| \leq \advtg_\advB^\sss.
\]
\end{lemma}

The adversary $\advB_{1,\secpar}(N, T(\secpar), g):$
\vspace{-2mm}
\begin{enumerate}
\item Samples $k_1, k_2 \rand \smplset$, computes $h_1 := g^{k_1} \bmod N, h_2 := g^{k_2} \bmod N,  h_3 := g^{2^{T(\secpar)}} \bmod N$ and sets $\crs:=(N, T(\secpar), g, h_1, h_2, h_3)$. Notice that value $h_3$ is computed by repeated squaring.
\item Runs $(m_0, m_1, \st) \leftarrow \adv_{1, \secpar}(\crs)$ and answers decommitment queries using $k_1$.
\item Outputs $(N,g,k_1, k_2, h_1,h_2,h_3, m_0, m_1, \st)$
\end{enumerate}

The adversary $\advB_{2,\secpar}(x,y,(N,g,k_1, k_2, h_1,h_2,h_3, m_0, m_1, \st)):$
\vspace{-2mm}
\begin{enumerate}
\item Samples $b \rand \bits$, computes $c_0^*:=x, c_1^*:=x^{k_1N}(1+N)^{m_b}, c_2^*:=x^{k_2N}(1+N)^{m_b}, c_3^*:=y^{N}(1+N)^{m_b}$.
\item Runs $\pi^* \leftarrow \simul(1, \st', (h_1, h_2, c_0^*, c_1^*, c_2^*, c_3^*))$.
\item Runs $b' \leftarrow \adv_{2, \secpar}((c_0^*, c_1^*, c_2^*, c_3^*), \pi^*), \st)$ and answers decommitment queries using $k_1$.
\item Returns the truth value of $b=b'$.
\end{enumerate}
Since $g$ is a generator of $\Jn$ and $x$ is sampled uniformly at random from $\Jn$ there exists some $r \in [\varphi(N)/2]$ such that $x = g^{r}$. Therefore when $y = x^{2^T} = (g^{2^T})^{r} \bmod N$, then $\advB$ simulates $\games_\prevgame$ perfectly. Otherwise $y$ is random value and $\advB$ simulates $\games_\thisgame$ perfectly. 

Now we analyse the running time of the constructed adversary. Adversary $\advB_1$ computes $h_3$ by $T(\secpar)$ consecutive squarings and because $T(\secpar)$ is polynomial in $\secpar$, $\advB_1$ is efficient. Moreover, $\advB_2$ fulfils the depth constraint:
\[ \dep(\advB_{2,\secpar}) = \poly_\advB(\secpar) + \dep(\adv_{2,\secpar}) \leq \undT^{\ugap}(\secpar) + T^{\ugap}(\secpar) \leq 2 T^{\ugap}(\secpar) = o(T^{\gap}(\secpar)). \] 

Also $T(\cdot) \geq \tilT_\nitc(\cdot) \geq \tilT_\sss(\cdot)$ as required.

%\nextgame{RndExp2}
%In $\games_\thisgame$ we stop to use $\rerand$ algorithm. Concretely, we sample $r \rand [\varphi(N)/4], y_2 \rand \qrn$ and compute challenge ciphertext as $c^*:=(g^r, h_1^r \cdot m_b, y_2 \cdot m_b)$. The ciphertext has the same distribution as in the previous game. Therefore 
%
%\begin{lemma}
%\[
%\Pr[\games_\prevgame = 1] = \Pr[\games_\thisgame = 1].
%\]
%\end{lemma}
%\nextgame{RndExp3}




\nextgame{DCR1}
In $\games_\thisgame$ we sample $y_3 \rand \Zns$ such that it has Jacobi symbol 1 and compute $c_3^*$ as $y_3(1+N)^{m_b}$. 

\begin{lemma}\label{lem:dcr-rom-lh}
\[
\left|\Pr[\games_\prevgame = 1] - \Pr[\games_\thisgame = 1]\right| \leq \advtg_\advB^\dcr.
\]
\end{lemma}
We construct an adversary $\advB = \{\advB_\secpar\}_{\secpar \in \N}$ against DCR.

$\advB_{\secpar}(N,y):$
\vspace{-2mm}
\begin{enumerate}
\item Samples $g, y_3, x \rand \Jn, k_1, k_2 \rand \smplset$, computes $h_1 := g^{k_1} \bmod N, h_2 := g^{k_2} \bmod N,  h_3 := g^{2^{T}} \bmod N$ and sets $\crs:=(N, T, g, h_1, h_2, h_3)$. Notice that value $h_3$ is computed by repeated squaring.
\item Runs $(m_0, m_1, \st) \leftarrow \adv_{1, \secpar}(\crs)$ and answers decommitment queries using $k_1$.
\item Samples $b \rand \bits, w \rand \Zns$ such that $\left( \frac{y}{N} \right)= \left( \frac{w}{N} \right)$. We remark that computing Jacobi symbol can be done efficiently without knowing factorization of N.
\item Computes $c_0^*:=x, , c_1^*:=x^{k_1N}(1+N)^{m_b}, c_2^*:=x^{k_2N}(1+N)^{m_b}, c_3^*:=yw^{N}(1+N)^{m_b}$. Runs $\pi^* \leftarrow \simul(1, \st', (h_1, h_2, c_0^*, c_1^*, c_2^*, c_3^*))$.
\item Runs $b' \leftarrow \adv_{2, \secpar}((c_0^*, c_1^*, c_2^*, c_3^*), \pi^*, \st)$ and answers decommitment queries using $k_1$.
\item Returns the truth value of $b=b'$.
\end{enumerate}
%If $y = v^N \bmod N^2$ then $yw^N = v^N w^N = (vw)^N$ and hence $yw^N$ is $N$-th residue. Moreover, the Jacobi symbol of $yw$ is 1, since the Jacobi symbol is multiplicatively homomorphic. Therefore $\advB$ simulates $\games_\prevgame$ perfectly. Otherwise $y$ is uniform random element in $\Zns$ then $yw^N$ is also uniform in $\Zns$ and $\advB$ simulates $\games_\thisgame$ perfectly. This proofs the lemma. We remark that at this point $c_3^*$ does not reveal any information about $m_b$.

If $y = v^N \bmod N^2$ then $yw^N = v^N w^N = (vw)^N$ and hence $yw^N$ is $N$-th residue. Moreover, the Jacobi symbol of $yw$ is 1, since the Jacobi symbol is multiplicatively homomorphic. Therefore $\advB$ simulates $\games_\prevgame$ perfectly. 

Otherwise, if $y$ is uniform random element in $\Zns$, then $yw^N$ is also uniform among all elements of $\Zns$ that have Jacobi symbol 1 and $\advB$ simulates $\games_\thisgame$ perfectly. This proves the lemma.

We remark that at this point $c_3^*$ does not reveal any information about $b$. Here we use that if $x = y \bmod N$ then $\left( \frac{x}{N} \right)= \left( \frac{y}{N} \right)$ and that there is an isomorphism $f:\Zn^* \times \Zn \rightarrow\Zns$ given by $f(u,v)=u^N(1+N)^v = u^N(1+vN) \bmod N^2$ (see e.g. \cite[Proposition 13.6]{books/crc/KatzLindell2014}).  Since $f(u,v) \bmod N = u^N + u^NvN \bmod N = u^N \bmod N$, that means that Jacobi symbol $\left( \frac{f(u,v)}{N} \right)$ depends only on $u$. Hence if $\left( \frac{f(u,v)}{N} \right) = 1$ then it must hold that $\left( \frac{f(u,r)}{N} \right) = 1$ for all $r \in \Zn$. This implies that a random element $f(u,v)$ in $\Zns$ with $\left( \frac{f(u,v)}{N} \right) = 1$ has a uniformly random distribution of $v$ in $\Zn$. Therefore if $yw^N = u^N(1+N)^v \bmod N^2$ then  $yw^N(1+N)^{m_b}  = u^N(1+N)^{m_b+v} \bmod N^2$. Since $v$ is uniform in $\Zn$, $(m_b + v)$ is also uniform in $\Zn$, which means that ciphertext $c_3^*$ does not reveal any information about $b$. 

\nextgame{RndExp2}
In $\games_\thisgame$ we sample $k_2$ uniformly at random from $[\varphi(N)/2]$. 

\begin{lemma}
\[
\left|\Pr[\games_\prevgame = 1] - \Pr[\games_\thisgame = 1]\right| \leq \frac{1}{p}+\frac{1}{q}-\frac{1}{N}.
\]
\end{lemma}

Again using a statistical argument this lemma directly follows from \Cref{sampling-lemma}.

\nextgame{DDH}
In $\games_\thisgame$ we sample $y_2 \rand \Jn$ and compute $c_2^*$ as  $y_2^N(1+N)^{m_b}$. 

\begin{lemma}\label{lem:ddh-rom-lh}
\[
\left|\Pr[\games_\prevgame = 1] - \Pr[\games_\thisgame = 1]\right| \leq \advtg_\advB^\ddh.
\]
\end{lemma}
We construct an adversary $\advB = \{\advB_\secpar\}_{\secpar \in \N}$ against DDH in the group $\Jn$. %Given \Cref{thm:ddh} this implies an adversary against DDH in large prime-order subgroups of $\Zn^*$.

$\advB_{\secpar}(N,g,g^\alpha, g^\beta, g^\gamma):$
\vspace{-2mm}
\begin{enumerate}
\item Samples $k_1 \rand \smplset$, computes $h_1 := g^{k_1} \bmod N,  h_3 := g^{2^{T}} \bmod N$ and sets $\crs:=(N, T, g, h_1, h_2:=g^\alpha, h_3)$.
\item Runs $(m_0, m_1, \st) \leftarrow \adv_{1, \secpar}(\crs)$ and answers decommitment queries using $k_1$.
\item Samples $b \rand \bits, y_3 \rand \Zns$ such that it has Jacobi symbol 1 and computes $(c_0^*, c_1^*, c_2^*, c_3^*):=(g^\beta, (g^\beta)^{k_1 N}(1+N)^{m_b}, (g^{\gamma})^N(1+N)^{m_b}, y_3(1+N)^{m_b}).$ Runs $\pi^* \leftarrow \simul(1, \st', (h_1, h_2, c_0^*, c_1^*, c_2^*, c_3^*))$.
\item Runs $b' \leftarrow \adv_{2, \secpar}((c_0^*, c_1^*, c_2^*,c_3^*), \pi^*, \st)$ and answers decommitment queries using $k_1$.
\item Returns the truth value of $b=b'$.
\end{enumerate}
If $\gamma = \alpha\beta$, then $\advB$ simulates $\games_\prevgame$ perfectly. Otherwise $g^\gamma$ is uniform random element in $\Jn$ and $\advB$ simulates $\games_\thisgame$ perfectly. This proofs the lemma.

\nextgame{RndExp3}
In $\games_\thisgame$ we sample $k_2$ uniformly at random from $\smplset$. 

\begin{lemma}
\[
\left|\Pr[\games_\prevgame = 1] - \Pr[\games_\thisgame = 1]\right| \leq \frac{1}{p}+\frac{1}{q}-\frac{1}{N}.
\]
\end{lemma}

This lemma directly follows from \Cref{sampling-lemma}.

\nextgame{DCR2}
In $\games_\thisgame$ we sample $y_2 \rand \Zns$ such that it has Jacobi symbol 1 and compute $c_2^*$ as $y_2(1+N)^{m_b}$. 

\begin{lemma}
\[
\left|\Pr[\games_\prevgame = 1] - \Pr[\games_\thisgame = 1]\right| \leq \advtg_\advB^\dcr.
\]
\end{lemma}
This can be proven in similar way as \Cref{lem:dcr-rom-lh}. We remark that at this point $c_2^*$ does not reveal any information about $m_b$.



\nextgame{SimSnd}

In $\games_\thisgame$ we answer decommitment queries using $\dec$ (\Cref{fig:deco-rom-lh}) with $i:=2$ which means that secret key $k_2$ and ciphertext $c_2$ are used. 

\begin{lemma}
\[
\left|\Pr[\games_\prevgame = 1] - \Pr[\games_\thisgame = 1]\right| \leq \simsnd^\nizk_\advB. 
\]
\end{lemma}

Let $\event$ denote the event that adversary $\adv$ asks a decommitment query $c$ such that its decommitment using the key $k_1$ is different from its decommitment using the key $k_2$. Since $\games_\prevgame$ and $\games_\thisgame$ are identical until $\event$ does not happen, by the standard argument it is sufficient to upper bound the probability of happening $\event$. Concretely,  

\[
\left|\Pr[\games_\prevgame = 1] - \Pr[\games_\thisgame = 1]\right| \leq \Pr[\event]. 
\]

We construct an adversary $\advB$ which breaks one-time simulation soundness of the NIZK. 

The adversary $\advB_{\secpar}^{\simul_1, \simul_2}:$
\vspace{-2mm}
\begin{enumerate}
\item Computes $\crs \leftarrow \pgen(\seck, T)$ as defined in the construction where the value $h_3$ is computed using repeated squaring instead.
\item Runs $(m_0, m_1, \st) \leftarrow \adv_{1, \secpar}(\crs)$ and answers decommitment queries using $k_2$.
\item Samples $b \rand \bits, x \rand \Jn, y_2, y_3 \rand \Zns$ and computes $(c_0^*, c_1^*, c_2^*, c_3^*):=(x, x^{k_1 N} (1+N)^{m_b}, y_2 (1+N)^{m_b}, y_3 (1+N)^{m_b})$. Forwards $(h_1, h_2, c_0^*, c_1^*, c_2^*, c_3^*)$ to simulation oracle $\simul_1$ and obtains a proof $\pi^*$.
\item Runs $b' \leftarrow \adv_{2, \secpar}((c_0^*, c_1^*, c_2^*, c_3^*), \pi^*, \st)$ and answers decommitment queries using $k_2$.
\item Find a decommitment query $c: = (c_0, c_1, c_2, c_3, \pi)$ such that $\dec(\crs, c, 1) \neq \dec(\crs,c,2)$ and returns $((h_1, h_2, c_0, c_1, c_2, c_3), \pi)$.
\end{enumerate}

$\advB$ simulates $\games_\thisgame$ perfectly and if the event $\event$ happens, it outputs a valid proof for a statement which is not in the specified language $L$. Therefore
\[\Pr[\event] \leq \simsnd^\nizk_\advB,\]
which concludes the proof of the lemma.  

%\nextgame{ReRand2}
%In $\games_\thisgame$ we use the key $t$ and randomness $r'$ as input for rerandomization. Concretely we compute $\rerand(c,c',h_1,h_2,t,r')$. This is just conceptual change since the ciphertext distributions are the same in both games and therefore 
%
%\begin{lemma}
%\[
%\left|\Pr[\games_\prevgame = 1] = \Pr[\games_\thisgame = 1]\right|.
%\]
%\end{lemma}
%
%\nextgame{RndExp4}
%In $\games_\thisgame$ we sample $r$ uniformly at random from $\varphi(N)$. 
%
%\begin{lemma}
%\[
%\left|\Pr[\games_\prevgame = 1] - \Pr[\games_\thisgame = 1]\right| \leq \frac{1}{N}.
%\]
%\end{lemma}

\nextgame{RndExp4}
In $\games_\thisgame$ we sample $k_1$ uniformly at random from $[\varphi(N)/2]$. 

\begin{lemma}
\[
\left|\Pr[\games_\prevgame = 1] - \Pr[\games_\thisgame = 1]\right| \leq \frac{1}{p}+\frac{1}{q}-\frac{1}{N}.
\]
\end{lemma}

This lemma directly follows from \Cref{sampling-lemma}.

\nextgame{DDH2}
In $\games_\thisgame$ we sample $y_1 \rand \Jn$ and compute $c_1^*$ as  $y_1^{N} (1+N)^{m_b}$. 

\begin{lemma}
\[
\left|\Pr[\games_\prevgame = 1] - \Pr[\games_\thisgame = 1]\right| \leq \advtg_\advB^\ddh.
\]
\end{lemma}
This can be proven in similar way as \Cref{lem:ddh-rom-lh}.

\nextgame{DCR3}
In $\games_\thisgame$ we sample $y_1 \rand \Zns$ such that it has Jacobi symbol 1 and compute $c_1^*$ as $y_1(1+N)^{m_b}$. 

\begin{lemma}
\[
\left|\Pr[\games_\prevgame = 1] - \Pr[\games_\thisgame = 1]\right| \leq \advtg_\advB^\dcr.
\]
\end{lemma}
This can be proven in similar way as \Cref{lem:dcr-rom-lh}. We remark that at this point $c_1^*$ does not reveal any information about $m_b$.

\begin{lemma}\label{nitc-rom-lh:llem}
\[
\Pr[\games_\thisgame = 1] = \half.
\]
\end{lemma}

Clearly, $c_0^*$ is uniform random element in $\Jn$ and hence it does not contain any information about the challenge message. Since $y_1, y_2, y_3$ are sampled uniformly at random from $\Zns$ the ciphertexts $c_1^*, c_2^*, c_3^*$ are also uniform random elements in $\Zns$ and hence do not contain any information about the challenge message $m_b$. Therefore, an adversary can not do better than guessing.

By combining Lemmas \ref{nitc-rom-lh:flem} - \ref{nitc-rom-lh:llem} we obtain the following:
\begin{align*}
\advtg^{\nitc}_{\adv} &= \left| \Pr[\games_0 = 1] - \half \right| \leq \sum_{i=0}^{12} \left|\Pr[\games_i = 1] - \Pr[\games_{i+1} = 1] \right| + \left|\Pr[\games_{13}- \half\right| \\
 &\leq  \snd^\nizk_\advB + \zk^\nizk_\advB + \advtg^{\sss}_{\advB} + \simsnd^{\nizk}_{\advB} + 2 \advtg^{\ddh}_{\advB} + 3 \advtg^{\dcr}_{\advB} \\ &+ 4 \left( \frac{1}{p}+\frac{1}{q}-\frac{1}{N} \right).
\end{align*}
which concludes the proof.
% \end{proof}
% \todo{appendix end}


% section proof_of_theorem_nitc-lin-rom (end)

\section{Proof of Theorem \ref{thm:NITC-mul-ROM}} % (fold)
\label{app:NITC-mul-ROM}

% \todo{appendix?}
% \begin{proof}
Completeness is implied by the completeness of the NIZK and can be verified by inspection. 

%Our construction is based on the Naor-Yung paradigm where we combine three ElGamal-type ciphertext with shared randomness.  


%Similarly to \cite{SCN:BiaMasVen16} we define a PPT algorithm $\rerand$ in Figure~\ref{fig:rerand}, which takes two ciphertexts generated with independent randomness, both public keys, only one secret key (in our case $k$) and randomness which was used to encrypt a message using the public key $g, h_1$. 
%%We assume that modulus $N$ is implicitly known.\todo{Why not explicit?} 
%% which corresponds to the secret key which is given as the input. 
%
%\begin{figure}[tb]
%\centering
%\begin{minipage}{0.75\textwidth}
%$\underline{\rerand(c:= (g^{r}, h_1^{r}\cdot m), c':= (g^{r'}, h_2^{r'}\cdot m), N, h_1, h_2, k, r)}:$
%\vspace{-2mm}
%\begin{itemize}
%\item $c_0:= g^{r}\cdot{g^{r'}} = g^{r+r'} \bmod N$;
%\item $c_1:= (g^{r'})^k \cdot h_1^{r}\cdot m  =  h_1^{r'}\cdot h_1^{r}\cdot m = h_1^{r+r'}\cdot m \bmod N$;
%\item $c_2:=h_2^{r} \cdot h_2^{r'}\cdot m = h_2^{r+r'}\cdot m \bmod N$.
%\end{itemize}
%\end{minipage}
%\caption{\label{fig:rerand}Algorithm $\rerand$.}
%\end{figure}
%
%
%
%
%It is straightforward to see that the ciphertext returned by $\rerand$ is perfectly distributed to the ciphertext produced using a shared randomness where the pair $(c_0, c_1)$ encrypts a message $m$ and the pair $(c_0, c_2)$ encrypts message $m'$.


% We note that if we use value $2^T$ as a secret key, then in order to compute $c_2$ we have to execute $T$ repeated squarings, but since $T$ is polynomial in $\secpar$ this computations is considered to be efficient.

\newsequenceofgames{NITC-MH-ROM}
To prove security we define a sequence of games $\games_0 - \games_{10}$.  For $i \in \{0,1,\dots,10\}$ we denote by $\games_i = 1$ the event that the adversary $\adv = \{(\adv_{1,\secpar}, \adv_{2, \secpar})\}_{\secpar \in \nats}$ outputs $b'$ in the game $\games_i$ such that $b = b'$.
%In individual games we use the algorithm $\decom$ define in \Cref{fig:deco-rom-mh} to answer decommitment queries efficiently. 
\begin{figure}[h!]
\begin{center}
\begin{tabular}{|l|}
\hline
$\underline{\decom(\crs, c, i)}$\\
Parse $c$ as $(c_0, c_1, c_2,, c_3, \pi)$\\
if $\nizk.\vrfy((h_1, h_2, c_0, c_1, c_2, c_3), \pi)= 1$\\
\tab Compute $y:= c_0^{k_i} \bmod N$\\
\tab return $c_i \cdot y^{-1} \bmod N$\\
return $\bot$\\
\hline          
\end{tabular}
\caption{Decommitment oracle}
\label{fig:deco-rom-mh}
\end{center}
\end{figure}

\nextgame{G0}
Game $\games_\thisgame$ corresponds to the original security experiment where decommitment queries are answered using $\fdecom$.

\nextgame{DecOracle}
In game $\games_\thisgame$ decommitment queries are answered using the algorithm $\decom$ defined in \Cref{fig:deco-rom-mh} with $i:=1$, meaning that secret key $k_1$ and ciphertext $c_1$ are used, to answer decommitment queries efficiently. 


\begin{lemma}\label{nitc-rom-mh:flem}
\[
\left|\Pr[\games_\prevgame = 1] - \Pr[\games_\thisgame = 1]\right| \leq \snd^\nizk_\advB.
\]
\end{lemma}

Notice that if $c_1$ and $c_3$ contain the same message, both oracles answer decommitment queries consistently. Let $\event$ denote the event that the adversary $\adv$ asks a decommitment query $c$ such that its decommitment using the key $k_1$ is different from its decommitment using $\fdecom$. Since $\games_\prevgame$ and $\games_\thisgame$ are identical until $\event$ happens, we bound the probability of $\event$. Concretely, we have

\[
\left|\Pr[\games_\prevgame = 1] - \Pr[\games_\thisgame = 1]\right| \leq \Pr[\event]. 
\]

We construct an adversary $\advB$ that breaks soundness of the NIZK. 
The adversary $\advB_{\secpar}$ proceeds as follows:
\vspace{-2mm}
\begin{enumerate}
\item It computes $\crs \leftarrow \pgen(\seck, T)$ as defined in the construction where the value $h_3$ is computed using repeated squaring instead.
\item Then it runs $(m_0, m_1, \st) \leftarrow \adv_{1, \secpar}(\crs)$ and answers decommitment queries using $k_1$.
 It samples $b \rand \bits, r \rand \smplset$ and computes $c_0^*:=g^r, c_1^*:=h_1^{r}m_b, c_2^*:=h_2^{r}m_b, c_3^*:=h_3^{r}m_b$. It sets $(s:=\mathlist(c_0^*, c_1^*, c_2^*, c_3^*), w:=(m,r))$ and runs $\pi^* \leftarrow \nizk.\prove(s,w)$.
\item Next, it runs $b' \leftarrow \adv_{2, \secpar}((c_0^*, c_1^*, c_2^*, c_3^*), \pi^*, \st)$ and answers decommitment queries using $k_1$.
\item Finally, it checks whether there exists a decommitment query $c: = \mathlist(c_0, c_1, c_2, c_3, \pi)$ such that $\fdecom(\crs, c) \neq \dec(\crs,c,1)$. If $\event$ occurs, then this is the case, and it returns $((h_1, h_2, c_0, c_1, c_2, c_3), \pi)$. %Notice that this can be done efficiently with the knowledge of $t$.
\end{enumerate}

$\advB$ simulates $\games_\thisgame$ perfectly and if the event $\event$ happens, then it outputs a valid proof for a statement which is not in the specified language $L$. Therefore we get
\[\Pr[\event] \leq \snd^\nizk_\advB.\]

%Let $\gnr$ denote the event that the sampled $g$ in $\kgen$ is a generator of $\qrn$. Recall that $N = pq$ where $p = 2p'+1$ and $q = 2q'+1$. Because $g$ is sampled uniformly at random and $\qrn$ has $\varphi(|\qrn|) = (p'-1)(q'-1)$ generators, this event happens with overwhelming probability. Concretely, $\Pr[\gnr] = 1-\frac{1}{p'}-\frac{1}{q'}+\frac{1}{p'q'}$.
%Therefore the following holds.
%\begin{lemma}\label{tpke3:flem} 
%\begin{align*}
%\Pr[\games_\thisgame = 1] &= \Pr[\games_\thisgame = 1| \gnr]\Pr[\gnr] + \Pr[\games_\thisgame = 1| \overline{\gnr}]\Pr[\overline{\gnr}] \\
%&\leq \Pr[\games_\thisgame = 1| \gnr]\Pr[\gnr] + \Pr[\overline{\gnr}] \\
%&= \Pr[\games_\thisgame = 1| \gnr]\left( 1-\frac{1}{p'}-\frac{1}{q'}+\frac{1}{p'q'} \right) + \frac{1}{p'}+\frac{1}{q'}-\frac{1}{p'q'}.
%\end{align*}
%\end{lemma}




\nextgame{SimulProof}
Game $\games_\thisgame$ proceeds exactly as the previous game but we use the zero-knowledge simulator $(\pi, \st) \leftarrow \simul(1, \st, (h_1, h_2, c_0^*,c_1^*,c_2^*,c_3^*))$ to produce a simulated proof for the challenge commitment and $\simul(2, \st, \cdot)$ to answer random oracle queries. By zero-knowledge security of underlying NIZK we directly obtain
\begin{lemma}
\[
\left|\Pr[\games_\prevgame = 1] - \Pr[\games_\thisgame = 1]\right| \leq \zk^\nizk_\advB.
\]
\end{lemma}

\nextgame{RndExp}
In $\games_\thisgame$ we sample $r$ uniformly at random from $[\varphi(N)/2]$. 

We construct an adversary $\advB = \{\advB_\secpar\}_{\secpar \in \N}$ against zero-knowledge security of NIZK as follows:
\vspace{-2mm}
\begin{enumerate}
\item Samples $k_1, k_2 \rand \smplset$, computes $h_1 := g^{k_1} \bmod N, h_2 := g^{k_2} \bmod N$ and sets $\crs:=(N, T(\secpar), g, h_1, h_2, h_3)$. 
\item Runs $(m_0, m_1, \st) \leftarrow \adv_{1, \secpar}(\crs)$ and answers decommitment queries using $k_1$.
\item Samples $b \rand \bits, r \rand \smplset$ and computes $c_0^*:=g^r, c_1^*:=h_1^{r}m_b, c_2^*:=h_2^{r}m_b, c_3^*:=h_3^{r}m_b$. It submits $(s:=\mathlist(h_1,h_2,c_0^*, c_1^*, c_2^*, c_3^*), w:=(m,r))$ to its oracle and obtains proof $\pi^*$ as answer.
\item Runs $b' \leftarrow \adv_{2, \secpar}((c_0^*, c_1^*, c_2^*, c_3^*), \pi^*, \st)$ and answers decommitment queries using $k_1$.
\item Returns the truth value of $b=b'$.
\end{enumerate}
If the proof $\pi^*$ is generated using $\nizk.\prove$, then $\advB$ simulates $\games_\prevgame$ perfectly. Otherwise $\pi^*$ is generated using $\simul_1$ and $\advB$ simulates $\games_\thisgame$ perfectly. This proofs the lemma.


\begin{lemma}
\[
\left|\Pr[\games_\prevgame = 1] - \Pr[\games_\thisgame = 1]\right| \leq \frac{1}{p}+\frac{1}{q}-\frac{1}{N}.
\]
\end{lemma}
%At first we remark that for upper bounding the difference between the games we use a statistical argument. Because $r$ appears only in the exponent of the group generator, we later sample a random element from the group $\qrn$ which can be done efficiently. 
Since the only difference between the two games is in the set from which we sample $r$, to upper bound the advantage of adversary we can use \Cref{sampling-lemma}, which directly yields required upper bound.

%\nextgame{ReRand}
%In Game $\games_\thisgame$ we produce the challenge commitment by encrypting the challenge message using two independent random exponents $r \rand \smplset, r' \rand [\varphi(N)/4]$ to obtain $c:= (g^{r}, h_1^{r}\cdot m_b), c':= (g^{r'}, h_2^{r'}\cdot m_b)$ and then run $\rerand(c,c',N, \allowbreak h_1, h_2,k,r)$ to obtain resulting ciphertext $(c_0^*, c_1^*, c_2^*)$. Since $r'$ is sampled uniformly at random from $[\varphi(N)/4]$ the ciphertext distributions in both games  are the same. Therefore 
%
%\begin{lemma}
%\[
%\Pr[\games_\prevgame = 1] = \Pr[\games_\thisgame = 1].
%\]
%\end{lemma}


\nextgame{SSSA}
In $\games_\thisgame$ we sample $y_3 \rand \Jn$ and compute $c_3^*$ as $y_3 m_b$.

Let $\tilT_\sss(\secpar)$ be the polynomial whose existence is guaranteed by the SSS assumption.
Let $\poly_\advB(\secpar)$ be the fixed polynomial which bounds the time required to execute Steps 1--2 and answer decommitment queries in Step 3 of the adversary $\advB_{2, \secpar}$ defined below. Set $\undT := (\poly_\advB(\secpar))^{1 / \ugap}$.  Set $\tilT_\nitc := \max(\tilT_\sss, \undT)$.
\begin{lemma}
From any polynomial-size adversary $\adv = \{(\adv_{1,\secpar}, \adv_{2, \secpar})\}_{\secpar \in \nats}$, where depth of $\adv_{2, \secpar}$ is at most $T^{\ugap}(\secpar)$ for some $T(\cdot) \geq \undT(\cdot)$ we can construct a polynomial-size adversary $\advB = \{(\advB_{1,\secpar}, \advB_{2, \secpar})\}_{\secpar \in \nats}$ where the depth of $\advB_{2, \secpar}$ is at most $T^{\gap}(\secpar)$ with
\[
\left|\Pr[\games_\prevgame = 1] - \Pr[\games_\thisgame = 1]\right| \leq \advtg_\advB^\sss.
\]
\end{lemma}

The adversary $\advB_{1,\secpar}(N, T(\secpar), g):$
\vspace{-2mm}
\begin{enumerate}
\item Samples $k_1, k_2 \rand \smplset$, computes $h_1 := g^{k_1} \bmod N, h_2 := g^{k_2} \bmod N,  h_3 := g^{2^{T(\secpar)}} \bmod N$ and sets $\crs:=(N, T(\secpar), g, h_1, h_2, h_3)$. Notice that value $h_3$ is computed by repeated squaring.
\item Runs $(m_0, m_1, \st) \leftarrow \adv_{1, \secpar}(\crs)$ and answers decommitment queries using $k_1$.
\item Outputs $(N,g,k_1, k_2, h_1,h_2,h_3, m_0, m_1, \st)$
\end{enumerate}

The adversary $\advB_{2,\secpar}(x,y,(N,g,k_1, k_2, h_1,h_2,h_3, m_0, m_1, \st)):$
\vspace{-2mm}
\begin{enumerate}
\item Samples $b \rand \bits$, computes $c_0^*:=x, c_1^*:=x^{k_1} m_b, c_2^*:=x^{k_2}m_b, c_3^*:=y m_b$.
\item Runs $\pi^* \leftarrow \simul(1, \st', (h_1, h_2, c_0^*, c_1^*, c_2^*, c_3^*))$.
\item Runs $b' \leftarrow \adv_{2, \secpar}((c_0^*, c_1^*, c_2^*, c_3^*), \pi^*), \st)$ and answers decommitment queries using $k_1$.
\item Returns the truth value of $b=b'$.
\end{enumerate}
Since $g$ is a generator of $\Jn$ and $x$ is sampled uniformly at random from $\Jn$ there exists some $r \in [\varphi(N)/2]$ such that $x = g^{r}$. Therefore when $y = x^{2^T} = (g^{2^T})^{r} \bmod N$, then $\advB$ simulates $\games_\prevgame$ perfectly. Otherwise $y$ is random value and $\advB$ simulates $\games_\thisgame$ perfectly. We remark that at this point $c_3^*$ does not reveal any information about $m_b$.

Now we analyse the running time of the constructed adversary. Adversary $\advB_1$ computes $h_3$ by $T(\secpar)$ consecutive squarings and because $T(\secpar)$ is polynomial in $\secpar$, $\advB_1$ is efficient. Moreover, $\advB_2$ fulfils the depth constraint:
\[ \dep(\advB_{2,\secpar}) = \poly_\advB(\secpar) + \dep(\adv_{2,\secpar}) \leq \undT^{\ugap}(\secpar) + T^{\ugap}(\secpar) \leq 2 T^{\ugap}(\secpar) = o(T^{\gap}(\secpar)). \] 

Also $T(\cdot) \geq \tilT_\nitc(\cdot) \geq \tilT_\sss(\cdot)$ as required.  

%\nextgame{RndExp2}
%In $\games_\thisgame$ we stop to use $\rerand$ algorithm. Concretely, we sample $r \rand [\varphi(N)/4], y_2 \rand \qrn$ and compute challenge ciphertext as $c^*:=(g^r, h_1^r \cdot m_b, y_2 \cdot m_b)$. The ciphertext has the same distribution as in the previous game. Therefore 
%
%\begin{lemma}
%\[
%\Pr[\games_\prevgame = 1] = \Pr[\games_\thisgame = 1].
%\]
%\end{lemma}
%\nextgame{RndExp3}


\nextgame{RndExp2}
In $\games_\thisgame$ we sample $k_2$ uniformly at random from $[\varphi(N)/2]$. 

\begin{lemma}
\[
\left|\Pr[\games_\prevgame = 1] - \Pr[\games_\thisgame = 1]\right| \leq \frac{1}{p}+\frac{1}{q}-\frac{1}{N}.
\]
\end{lemma}

Again using a statistical argument this lemma directly follows from \Cref{sampling-lemma}.

\nextgame{DDH}
In $\games_\thisgame$ we sample $y_2 \rand \Jn$ and compute $c_2^*$ as  $y_2 m_b$. 

\begin{lemma}\label{lem:ddh-rom-mh}
\[
\left|\Pr[\games_\prevgame = 1] - \Pr[\games_\thisgame = 1]\right| \leq \advtg_\advB^\ddh.
\]
\end{lemma}
We construct an adversary $\advB = \{\advB_\secpar\}_{\secpar \in \N}$ against DDH in the group $\Jn$. %Given \Cref{thm:ddh} this implies an adversary against DDH in large prime-order subgroups of $\Zn^*$.

$\advB_{\secpar}(N,g,g^\alpha, g^\beta, g^\gamma):$
\vspace{-2mm}
\begin{enumerate}
\item Samples $k_1 \rand \smplset$, computes $h_1 := g^{k_1} \bmod N,  h_3 := g^{2^{T}} \bmod N$ and sets $\crs:=(N, T, g, h_1, h_2:=g^\alpha, h_3)$.
\item Runs $(m_0, m_1, \st) \leftarrow \adv_{1, \secpar}(\crs)$ and answers decommitment queries using $k_1$.
\item Samples $b \rand \bits, y_3 \rand \Jn$ and computes $(c_0^*, c_1^*, c_2^*, c_3^*):=\mathlist(g^\beta, (g^\beta)^{k_1} m_b, (g^{\gamma}) m_b, y_3 m_b).$ Runs $\pi^* \leftarrow \simul(1, \st', (h_1, h_2, c_0^*, c_1^*, c_2^*, c_3^*))$.
\item Runs $b' \leftarrow \adv_{2, \secpar}((c_0^*, c_1^*, c_2^*,c_3^*), \pi^*, \st)$ and answers decommitment queries using $k_1$.
\item Returns the truth value of $b=b'$.
\end{enumerate}
If $\gamma = \alpha\beta$, then $\advB$ simulates $\games_\prevgame$ perfectly. Otherwise $g^\gamma$ is uniform random element in $\Jn$ and $\advB$ simulates $\games_\thisgame$ perfectly. This proofs the lemma. We remark that at this point $c_2^*$ does not reveal any information about $m_b$.

\nextgame{RndExp3}
In $\games_\thisgame$ we sample $k_2$ uniformly at random from $\smplset$. 

\begin{lemma}
\[
\left|\Pr[\games_\prevgame = 1] - \Pr[\games_\thisgame = 1]\right| \leq \frac{1}{p}+\frac{1}{q}-\frac{1}{N}.
\]
\end{lemma}

This lemma directly follows from \Cref{sampling-lemma}.

\nextgame{SimSnd}

In $\games_\thisgame$ we answer decommitment queries using $\dec$ (\Cref{fig:deco-rom-mh}) with $i:=2$ which means that secret key $k_2$ and ciphertext $c_2$ are used. 

\begin{lemma}
\[
\left|\Pr[\games_\prevgame = 1] - \Pr[\games_\thisgame = 1]\right| \leq \simsnd^\nizk_\advB. 
\]
\end{lemma}

Let $\event$ denote the event that adversary $\adv$ asks a decommitment query $c$ such that its decommitment using the key $k_1$ is different from its decommitment using the key $k_2$. Since $\games_\prevgame$ and $\games_\thisgame$ are identical until $\event$ does not happen, by the standard argument it is sufficient to upper bound the probability of happening $\event$. Concretely,  

\[
\left|\Pr[\games_\prevgame = 1] - \Pr[\games_\thisgame = 1]\right| \leq \Pr[\event]. 
\]

We construct an adversary $\advB$ which breaks one-time simulation soundness of the NIZK. 

The adversary $\advB_{\secpar}^{\simul_1, \simul_2}:$
\vspace{-2mm}
\begin{enumerate}
\item Computes $\crs \leftarrow \pgen(\seck, T)$ as defined in the construction where the value $h_3$ is computed using repeated squaring instead.
\item Runs $(m_0, m_1, \st) \leftarrow \adv_{1, \secpar}(\crs)$ and answers decommitment queries using $k_2$.
\item Samples $b \rand \bits, x, y_2, y_3 \rand \Jn$ and computes $(c_0^*, c_1^*, c_2^*, c_3^*):=(x, x^{k_1} m_b,\allowbreak y_2 m_b, y_3 m_b)$. Forwards $(h_1, h_2, c_0^*, c_1^*, c_2^*, c_3^*)$ to simulation oracle $\simul_1$ and obtains a proof $\pi^*$.
\item Runs $b' \leftarrow \adv_{2, \secpar}((c_0^*, c_1^*, c_2^*, c_3^*), \pi^*, \st)$ and answers decryption queries using $k_2$.
\item Find a decommitment query $c: = (c_0, c_1, c_2, c_3, \pi)$ such that $\dec(\crs, c, 1) \neq \dec(\crs,c,2)$ and returns $((h_1, h_2, c_0, c_1, c_2, c_3), \pi)$.
\end{enumerate}

$\advB$ simulates $\games_\thisgame$ perfectly and if the event $\event$ happens, it outputs a valid proof for a statement which is not in the specified language $L$. Therefore
\[\Pr[\event] \leq \simsnd^\nizk_\advB,\]
which concludes the proof of the lemma.  

%\nextgame{ReRand2}
%In $\games_\thisgame$ we use the key $t$ and randomness $r'$ as input for rerandomization. Concretely we compute $\rerand(c,c',h_1,h_2,t,r')$. This is just conceptual change since the ciphertext distributions are the same in both games and therefore 
%
%\begin{lemma}
%\[
%\left|\Pr[\games_\prevgame = 1] = \Pr[\games_\thisgame = 1]\right|.
%\]
%\end{lemma}
%
%\nextgame{RndExp4}
%In $\games_\thisgame$ we sample $r$ uniformly at random from $\varphi(N)$. 
%
%\begin{lemma}
%\[
%\left|\Pr[\games_\prevgame = 1] - \Pr[\games_\thisgame = 1]\right| \leq \frac{1}{N}.
%\]
%\end{lemma}

\nextgame{RndExp4}
In $\games_\thisgame$ we sample $k_1$ uniformly at random from $[\varphi(N)/2]$. 

\begin{lemma}
\[
\left|\Pr[\games_\prevgame = 1] - \Pr[\games_\thisgame = 1]\right| \leq \frac{1}{p}+\frac{1}{q}-\frac{1}{N}.
\]
\end{lemma}

This lemma directly follows from \Cref{sampling-lemma}.

\nextgame{DDH2}
In $\games_\thisgame$ we sample $y_1 \rand \Jn$ and compute $c_1^*$ as  $y_1 m_b$. 

\begin{lemma}
\[
\left|\Pr[\games_\prevgame = 1] - \Pr[\games_\thisgame = 1]\right| \leq \advtg_\advB^\ddh.
\]
\end{lemma}
This can be proven in similar way as \Cref{lem:ddh-rom-mh}. We remark that at this point $c_1^*$ does not reveal any information about $m_b$.

\begin{lemma}\label{nitc-rom-mh:llem}
\[
\Pr[\games_\thisgame = 1] = \half.
\]
\end{lemma}

Clearly, $c_0^*$ is uniform random element in $\Jn$ and hence it does not contain any information about the challenge message. Since $y_1, y_2, y_3$ are sampled uniformly at random from $\Jn$ the ciphertexts $c_1^*, c_2^*, c_3^*$ are also uniform random elements in $\Jn$ and hence do not contain any information about the challenge message $m_b$. Therefore, an adversary can not do better than guessing.

By combining Lemmas \ref{nitc-rom-mh:flem} - \ref{nitc-rom-mh:llem} we obtain the following:
\begin{align*}
&\advtg^{\nitc}_{\adv} = \left| \Pr[\games_0 = 1] - \half \right| \leq \sum_{i=0}^9 \left|\Pr[\games_i = 1] - \Pr[\games_{i+1} = 1] \right| + \left|\Pr[\games_{10}- \half\right| \\
 &\leq \snd^\nizk_\advB + \zk^\nizk_\advB + \advtg^{\sss}_{\advB} + \simsnd^{\nizk}_{\advB} + 2 \advtg^{\ddh}_{\advB} + 4 \left( \frac{1}{p}+\frac{1}{q}-\frac{1}{N} \right),
\end{align*}
which concludes the proof.
% \end{proof}
% \todo{appendix end}

% section proof_of_theorem_thm:nitc-mul-rom (end)


%%% Local Variables:
%%% mode: latex
%%% TeX-master: "main"
%%% End:
