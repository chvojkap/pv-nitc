\usepackage{amsmath,amsfonts,amssymb,amstext}
\usepackage{latexsym,ifthen,url}
\usepackage[pdftex,usenames,dvipsnames]{color}


% \usepackage{amsthm}
% \theoremstyle{plain}
% \newtheorem{theorem}{Theorem}
% \newtheorem{lemma}{Lemma}
% \newtheorem{claim}{Claim}
% \newtheorem{corollary}{Corollary}

% \theoremstyle{definition}
% \newtheorem{definition}{Definition}

% \theoremstyle{remark}
% \newtheorem{remark}{Remark}
% \newtheorem{example}{Example}

% \renewenvironment{proof}{\vspace{1mm}\noindent{\sc Proof.}}{\hfill$\square$\newline\vspace{1mm}}
\DeclareMathAlphabet{\mathscript}{OT1}{pzc}{m}{it} 

\usepackage[pdftex,colorlinks,backref]{hyperref}
\hypersetup{colorlinks=false, linkcolor=blue, menucolor=blue, urlcolor=blue, citecolor=blue}
\renewcommand*{\backref}[1]{}
\def\sectionautorefname{Section}
\def\subsectionautorefname{Section}
\def\subsubsectionautorefname{Section}
\def\appendixautorefname{Appendix}
\def\definitionautorefname{Definition}
\def\claimautorefname{Claim}
\def\lemmaautorefname{Lemma}
\def\theoremautorefname{Theorem}

%%%%%%%%%%%%%%%%%%%%%%%%%%%%%%%%%%%%%%%%%%%%%%%%%%%%%%%%%%%%%%%%%%%%%%%%%%%%%%%%
%\newcommand{\newsequenceofgames}[1]{
%  \newcounter{#1}
%  \setcounter{#1}{-1}
%
%  \ifx \GameID \undefined
%    \newcommand{\GameID}{#1}
%  \else
%    \renewcommand{\GameID}{#1}
%  \fi
%
%  \ifx \PrevLabel \undefined
%    \newcommand{\PrevLabel}{\GameID.NULL}
%  \else
%    \renewcommand{\PrevLabel}{\GameID.NULL}
%  \fi
%
%  \ifx \ThisLabel \undefined
%    \newcommand{\ThisLabel}{\GameID.NULL}
%  \else
%    \renewcommand{\ThisLabel}{\GameID.NULL}
%  \fi
%}
%
%
%\newcommand{\nextgame}[1]{
%  \let\PrevLabel\ThisLabel
%  \renewcommand{\ThisLabel}{\GameID.#1}
%  \refstepcounter{\GameID}\label{\GameID.#1}
%  \paragraph{Game~\arabic{\GameID}.}
%}
%
%\newcommand{\thisgame}{{\ref{\ThisLabel}}\xspace}
%\newcommand{\prevgame}{{\ref{\PrevLabel}}\xspace}
%%%%%%%%%%%%%%%%%%%%%%%%%%%%%%%%%%%%%%%%%%%%%%%%%%%%%%%%%%%%%%%%%%%%%%%%%%%%%%%%

\usepackage{color}
\usepackage{marginnote}
%\newcommand{\todo}[1]{{\marginnote{\textcolor{red}{TODO}}\textcolor{red}{(todo: #1)}}}
%
%\newcommand{\heading}[1]{\paragraph{\sc #1}}

\usepackage{xspace}
%\newcommand{\x}{\ensuremath{\cdot}}
%\newcommand{\Z}{\ensuremath{\mathbb{Z}}\xspace}
%\newcommand{\N}{\ensuremath{\mathbb{N}}\xspace}
%\newcommand{\Q}{\ensuremath{\mathbb{Q}}\xspace}
%\newcommand{\R}{\ensuremath{\mathbb{R}}\xspace}
%\newcommand{\F}{\ensuremath{\mathbb{F}}\xspace}
%\newcommand{\G}{\ensuremath{\mathbb{G}}\xspace}
%
%\newcommand{\ceil}[1]{\left\lceil #1 \right\rceil}
%\newcommand{\floor}[1]{\left\lfloor #1 \right\rfloor}
%\newcommand{\abs}[1]{\left| #1 \right|}
%
%\newcommand{\rand}{\stackrel{{\scriptscriptstyle\$}}{\leftarrow}}
%\newcommand{\testequal}{\stackrel{{\scriptscriptstyle ?}}{=}}
%
%\newcommand{\Adversary}{\ensuremath{\mathcal{A}}\xspace}
%\newcommand{\AdversaryB}{\ensuremath{\mathcal{B}}\xspace}
%\newcommand{\Algo}{\ensuremath{\mathcal{A}}\xspace}
%\newcommand{\AlgoB}{\ensuremath{\mathcal{B}}\xspace}
%\newcommand{\Challenger}{\ensuremath{\mathcal{C}}\xspace}
%\newcommand{\Oracle}{\ensuremath{\mathcal{O}}\xspace}

\usepackage{cleveref}

%%% Local Variables:
%%% mode: latex
%%% TeX-master: "main"
%%% End:
