%!TEX root=./main.tex
\section{Construction of Non-Malleable NITC}
In our construction we rely on SS-NIZK for the following language: 

\[
L = \left\{(c_0, c_1, c_2) \mid \exists (m,r):
\begin{aligned}
      c_0 = g^r \land c_1 = h_1^r \cdot m \land c_2 = h_2^r \cdot m  \bmod N\\
    \end{aligned}
    \right\}, 
\]
where $g, h_1, h_2, N$ are fixed.




\begin{figure}[h!]
\begin{center}
\begin{tabular}{|ll|}
\hline
$\underline{\pgen(\seck, T)}$ 							   & $\underline{\com(\pk, m)}$ \\
$(p, q_, N) \leftarrow \genmod(\seck)$ & $r \rand \smplset$  \\
$\varphi(N):= (p-1)(q-1)$   & Compute $c_0:= g^r \bmod N$ \\
Sample random generator $g$ of $\qrn$ & For $i \in [2]: y_i:= h_i^r \bmod N$\\
$k \rand \smplset$ & $ c_1:= y_1 \cdot m \bmod N$ \\
$t:= 2^T \bmod \varphi(N)/4$ &  $c_2:= y_2^e \cdot m \bmod N$ \\
$h_1:= g^k \bmod N$ &  $c_3:=H(y_2)$\\
$h_2:=g^{t} \bmod N$ & $\Phi := (c_0, c_1, c_2,c_3), w := (m, r)$\\
%$\crs \leftarrow \nizk.\setup(\seck)$ & \\
Choose $e$ s.t. $gcd(e, \varphi(N)) =1$, e.g. $e:=3$ & $\pi \leftarrow \nizk.\prove(\Phi, w)$\\
return $\crs:= (N,T,g,h_1,h_2, e, H)$ & $c := (c_0, c_1, c_2,c_3, \pi), \pi_\com:= \bot, \pi_\dec: = r$\\
																			& return $(c, \pi_\com, \pi_\dec)$\\
%return $\crs$     & \\
                                             &\\
$\underline{\cvrfy(\crs, c, \pi_\com)}$     & $\underline{\dvrfy(\crs,c, m, \pi_\dec)}$ \\
Parse $c$ as $(c_0, c_1, c_2, \pi)$  & Parse $c$ as $(c_0, c_1, c_2, c_3, \pi)$ \\
return $\nizk.\vrfy((c_0, c_1, c_2), \pi)$  & if $c_0 = g^r \land c_1 = h_1^r \cdot m \land c_2 = h_2^{er} \cdot m \bmod N$\\
& $\land c_3=H(h_2^r)$ not necessary, guaranteed by NIZK\\
 & \tab return 1 \\
& return 0 \\
                                             &\\
$\underline{\fdecom(\crs,c)}$ & $\underline{\fdecom\vrfy(\crs,c,m, \pi)}$\\
Parse $c$ as $(c_0, c_1, c_2, \pi)$ & if $c_2 = \pi^e\cdot  m \land c_3 = H(pi)$\\
if $\nizk.\vrfy((c_0, c_1, c_2), \pi)= 1$&\tab return 1\\
\tab Compute $ y_2:=c_0^{2^T} \bmod N$ & return 0\\
\tab return $m:=c_2 \cdot y_2^{-e} \bmod N, \pi:=y_2$ &\\
return $\bot$ & \\

%$\underline{\decom(\crs, \sk, c)}$     & $\underline{\fdecom(\crs,c)}$ \\
%Parse $c$ as $(c_0, c_1, c_2, \pi)$  & Parse $c$ as $(c_0, c_1, c_2, \pi)$ \\
%if $\nizk.\vrfy((c_0, c_1, c_2), \pi)= 1$  & if $\nizk.\vrfy((c_0, c_1, c_2), \pi)= 1$\\
%\tab Compute $y_1:= c_0^{k} \bmod N$  &   \tab Compute $ y_2:=c_0^{2^T} \bmod N$ \\
%\tab return $c_1 \cdot y_1^{-1} \bmod N$ & \tab return $c_2 \cdot y_2^{-1} \bmod N$ \\
%return $\bot$ & return $\bot$\\

\hline          
\end{tabular}
\caption{NY Construction of NITC}
\label{table:nitc}
\end{center}
\end{figure}



\begin{theorem}
If $\nizk = (\nizk.\prove, \nizk.\vrfy)$ is a one-time simulation-sound non-interactive zero-knowledge proof system for $L$, the strong sequential squaring assumption with gap $\gap$ holds relative to $\genmod$, and the Decisional Diffie-Hellman assumption holds relative to $\genmod$ in large prime order subgroups of $\qrn$, then $(\pgen, \com, \cvrfy, \dvrfy, \fdecom)$ defined in \Cref{table:nitc} is an IND-CCA-secure non-interactive timed commitment scheme with $\ugap$, for any $\ugap < \gap$. 
\end{theorem}

\begin{proof}
Completeness is implied by the completeness of the NIZK and can be verified by inspection. 

Before we prove security, we define an additional algorithm which makes the explanation simpler. Our construction is based on the Naor-Yung paradigm where we combine two ElGamal ciphertext with shared randomness. 
Similarly to \cite{SCN:BiaMasVen16} we define a PPT algorithm $\rerand$ in Figure~\ref{fig:rerand}, which takes two ciphertexts generated with independent randomness, both public keys, only one secret key (in our case $k$) and randomness which was used to encrypt a message using the public key $g, h_1$. 
%We assume that modulus $N$ is implicitly known.\todo{Why not explicit?} 
% which corresponds to the secret key which is given as the input. 

\begin{figure}[tb]
\centering
\begin{minipage}{0.75\textwidth}
$\underline{\rerand(c:= (g^{r}, h_1^{r}\cdot m), c':= (g^{r'}, h_2^{r'}\cdot m), N, h_1, h_2, k, r)}:$
\vspace{-2mm}
\begin{itemize}
\item $c_0:= g^{r}\cdot{g^{r'}} = g^{r+r'} \bmod N$;
\item $c_1:= (g^{r'})^k \cdot h_1^{r}\cdot m  =  h_1^{r'}\cdot h_1^{r}\cdot m = h_1^{r+r'}\cdot m \bmod N$;
\item $c_2:=h_2^{r} \cdot h_2^{r'}\cdot m = h_2^{r+r'}\cdot m \bmod N$.
\end{itemize}
\end{minipage}
\caption{\label{fig:rerand}Algorithm $\rerand$.}
\end{figure}




It is straightforward to see that the ciphertext returned by $\rerand$ is perfectly distributed to the ciphertext produced using a shared randomness where the pair $(c_0, c_1)$ encrypts a message $m$ and the pair $(c_0, c_2)$ encrypts message $m'$.
% We note that if we use value $2^T$ as a secret key, then in order to compute $c_2$ we have to execute $T$ repeated squarings, but since $T$ is polynomial in $\secpar$ this computations is considered to be efficient.

\newsequenceofgames{TPKE}
To prove security we define a sequence of games $\games_0 - \games_8$.  For $i \in \{0,1,\dots,8\}$ we denote by $\games_i = 1$ the event that the adversary $\adv = \{(\adv_{1,\secpar}, \adv_{2, \secpar})\}_{\secpar \in \nats}$ outputs $b'$ in the game $\games_i$ such that $b = b'$.
In individual games we use the algorithm $\decom$ define in \Cref{fig:deco} to answer decommitment queries efficiently. 
\begin{figure}[h!]
\begin{center}
\begin{tabular}{|l|}
\hline
$\underline{\decom(\crs, \sk, c)}$\\
Parse $c$ as $(c_0, c_1, c_2, \pi)$\\
if $\nizk.\vrfy((c_0, c_1, c_2), \pi)= 1$\\
\tab Compute $y_1:= c_0^{k} \bmod N$\\
\tab return $c_1 \cdot y_1^{-1} \bmod N$\\
return $\bot$\\
\hline          
\end{tabular}
\caption{Decommitment oracle}
\label{fig:deco}
\end{center}
\end{figure}

\nextgame{G0}
Game $\games_\thisgame$ corresponds to the original security experiment where decommitment queries are answered using $\decom$ with $\sk:=k$.

%Let $\gnr$ denote the event that the sampled $g$ in $\kgen$ is a generator of $\qrn$. Recall that $N = pq$ where $p = 2p'+1$ and $q = 2q'+1$. Because $g$ is sampled uniformly at random and $\qrn$ has $\varphi(|\qrn|) = (p'-1)(q'-1)$ generators, this event happens with overwhelming probability. Concretely, $\Pr[\gnr] = 1-\frac{1}{p'}-\frac{1}{q'}+\frac{1}{p'q'}$.
%Therefore the following holds.
%\begin{lemma}\label{tpke3:flem} 
%\begin{align*}
%\Pr[\games_\thisgame = 1] &= \Pr[\games_\thisgame = 1| \gnr]\Pr[\gnr] + \Pr[\games_\thisgame = 1| \overline{\gnr}]\Pr[\overline{\gnr}] \\
%&\leq \Pr[\games_\thisgame = 1| \gnr]\Pr[\gnr] + \Pr[\overline{\gnr}] \\
%&= \Pr[\games_\thisgame = 1| \gnr]\left( 1-\frac{1}{p'}-\frac{1}{q'}+\frac{1}{p'q'} \right) + \frac{1}{p'}+\frac{1}{q'}-\frac{1}{p'q'}.
%\end{align*}
%\end{lemma}




\nextgame{SimulProof}
Game $\games_\thisgame$ proceeds exactly as the previous game but we use the zero-knowledge simulator $(\pi, \st) \leftarrow \simul(1, \st, (c_0^*,c_1^*,c_2^*))$ to produce a simulated proof for the challenge commitment and $\simul(2, \st, \cdot)$ to answer random oracle queries. By zero-knowledge security of underlying NIZK we directly obtain
\begin{lemma}\label{nitc:flem}
\[
\left|\Pr[\games_\prevgame = 1] - \Pr[\games_\thisgame = 1]\right| \leq \zk^\nizk_\advB.
\]
\end{lemma}

\nextgame{RndExp}
In $\games_\thisgame$ we sample $r$ uniformly at random from $[\varphi(N)/4]$. 

\begin{lemma}
\[
\left|\Pr[\games_\prevgame = 1] - \Pr[\games_\thisgame = 1]\right| \leq \frac{1}{p}+\frac{1}{q}-\frac{1}{N}.
\]
\end{lemma}
%At first we remark that for upper bounding the difference between the games we use a statistical argument. Because $r$ appears only in the exponent of the group generator, we later sample a random element from the group $\qrn$ which can be done efficiently. 
Since the only difference between the two games is in the set from which we sample $r$, to upper bound the advantage of adversary we can use \Cref{sampling-lemma}, which directly yields required upper bound.

\nextgame{ReRand}
In Game $\games_\thisgame$ we produce the challenge commitment by encrypting the challenge message using two independent random exponents $r \rand \smplset, r' \rand [\varphi(N)/4]$ to obtain $c:= (g^{r}, h_1^{r}\cdot m_b), c':= (g^{r'}, h_2^{r'}\cdot m_b)$ and then run $\rerand(c,c',N, \allowbreak h_1, h_2,k,r)$ to obtain resulting ciphertext $(c_0^*, c_1^*, c_2^*)$. Since $r'$ is sampled uniformly at random from $[\varphi(N)/4]$ the ciphertext distributions in both games  are the same. Therefore 

\begin{lemma}
\[
\Pr[\games_\prevgame = 1] = \Pr[\games_\thisgame = 1].
\]
\end{lemma}


\nextgame{SSSA}
In $\games_\thisgame$ we sample $y_2 \rand \qrn$ and compute ciphertext $c':= (g^{r'}, y_2 \cdot m_b)$.

Let $\tilT_\sss(\secpar)$ be the polynomial whose existence is guaranteed by the SSS assumption.
Let $\poly_\advB(\secpar)$ be the fixed polynomial which bounds the time required to execute Steps 1--2 and answer decommitment queries in Step 3 of the adversary $\advB_{2, \secpar}$ defined below. Set $\undT := (\poly_\advB(\secpar))^{1 / \ugap}$.  Set $\tilT_\nitc := \max(\tilT_\sss, \undT)$.
\begin{lemma}
From any polynomial-size adversary $\adv = \{(\adv_{1,\secpar}, \adv_{2, \secpar})\}_{\secpar \in \nats}$, where depth of $\adv_{2, \secpar}$ is at most $T^{\ugap}(\secpar)$ for some $T(\cdot) \geq \undT(\cdot)$ we can construct a polynomial-size adversary $\advB = \{(\advB_{1,\secpar}, \advB_{2, \secpar})\}_{\secpar \in \nats}$ where the depth of $\advB_{2, \secpar}$ is at most $T^{\gap}(\secpar)$ with
\[
\left|\Pr[\games_\prevgame = 1] - \Pr[\games_\thisgame = 1]\right| \leq \advtg_\advB^\sss.
\]
\end{lemma}

The adversary $\advB_{1,\secpar}(N, T(\secpar), g):$
\vspace{-2mm}
\begin{enumerate}
\item Samples $k \rand \smplset$, computes $h_1 := g^k \bmod N, h_2 := g^{2^{T(\secpar)}} \bmod N$ and sets $\pk:=(N, T(\secpar), g, h_1, h_2)$.
\item Runs $(m_0, m_1, \st) \leftarrow \adv_{1, \secpar}(\pk)$ and answers decommitment queries using $k$.
\item Outputs $(N,g,k,h_1,h_2, m_0, m_1, \st)$
\end{enumerate}

The adversary $\advB_{2,\secpar}(x,y,(N,g,k,h_1,h_2, m_0, m_1, \st)):$
\vspace{-2mm}
\begin{enumerate}
\item Samples $b \rand \bits, r \rand \smplset$, computes $c:=(g^r, h_1^{r}\cdot m_b)), c':= (x, y \cdot m_b)$, and runs $(c_0^*, c_1^*, c_2^*)\leftarrow \rerand(c, c', h_1, h_2, k, r)$.
\item Runs $\pi^* \leftarrow \simul(1, \st', (c_0^*, c_1^*, c_2^*))$.
\item Runs $b' \leftarrow \adv_{2, \secpar}((c_0^*, c_1^*, c_2^*, \pi^*), \st)$ and answers decommitment queries using $k$.
\item Returns the truth value of $b=b'$.
\end{enumerate}
Since $g$ is a generator of $\qrn$ and $x$ is sampled uniformly at random from $\qrn$ there exists some $r' \in [\varphi(N)/4]$ such that $x = g^{r'}$. Therefore when $y = x^{2^T} = (g^{2^T})^{r'} \bmod N$, then $\advB$ simulates $\games_\prevgame$ perfectly. Otherwise $y$ is random value and $\advB$ simulates $\games_\thisgame$ perfectly. We remark that since $y$ is a random element, the ciphertext $c^*_2$ does not reveal any information about $b$.

Now we analyse the running time of the constructed adversary. Adversary $\advB_1$ computes $h_2$ by $T(\secpar)$ consecutive squarings and because $T(\secpar)$ is polynomial in $\secpar$, $\advB_1$ is efficient. Moreover, $\advB_2$ fulfils the depth constraint:
\[ \dep(\advB_{2,\secpar}) = \poly_\advB(\secpar) + \dep(\adv_{2,\secpar}) \leq \undT^{\ugap}(\secpar) + T^{\ugap}(\secpar) \leq 2 T^{\ugap}(\secpar) = o(T^{\gap}(\secpar)). \] 

Also $T(\cdot) \geq \tilT_\nitc(\cdot) \geq \tilT_\sss(\cdot)$ as required.

\nextgame{RndExp2}
In $\games_\thisgame$ we stop to use $\rerand$ algorithm. Concretely, we sample $r \rand [\varphi(N)/4], y_2 \rand \qrn$ and compute challenge ciphertext as $c^*:=(g^r, h_1^r \cdot m_b, y_2 \cdot m_b)$. The ciphertext has the same distribution as in the previous game. Therefore 

\begin{lemma}
\[
\Pr[\games_\prevgame = 1] = \Pr[\games_\thisgame = 1].
\]
\end{lemma}
\nextgame{RndExp3}
In $\games_\thisgame$ we use the secret key $\sk:=t$ to answer decommitment queries. 

\begin{lemma}
\[
\left|\Pr[\games_\prevgame = 1] - \Pr[\games_\thisgame = 1]\right| \leq \simsnd^\nizk_\advB. 
\]
\end{lemma}

Let $\event$ denote the event that adversary $\adv$ asks a decommitment query $c$ such that its decommitment using the key $k$ is different from its decommitment using the key $t$. Since $\games_\prevgame$ and $\games_\thisgame$ are identical until $\event$ does not happen, by the standard argument it is sufficient to upper bound the probability of happening $\event$. Concretely,  

\[
\left|\Pr[\games_\prevgame = 1] - \Pr[\games_\thisgame = 1]\right| \leq \Pr[\event]. 
\]

We construct an adversary $\advB$ which breaks one-time simulation soundness of the NIZK. 

The adversary $\advB_{\secpar}^{\simul_1, \simul_2}:$
\vspace{-2mm}
\begin{enumerate}
\item Computes $\pgen$ as defined in the construction.
\item Runs $(m_0, m_1, \st) \leftarrow \adv_{1, \secpar}(\pk)$ and answers decommitment queries using $t$.
\item Samples $b \rand \bits, r \rand [\varphi(N)/4], y_2 \rand \qrn$ and computes $(c_0^*, c_1^*, c_2^*):=(g^r, h_1^r \cdot m_b, y_2 \cdot m_b)$. Forwards $(c_0^*, c_1^*, c_2^*)$ to simulation oracle $\simul_1$ and obtains a proof $\pi^*$.
\item Runs $b' \leftarrow \adv_{2, \secpar}(c_0^*, c_1^*, c_2^*, \pi^*), \st)$ and answers decryption queries using $t$.
\item Find a decommitment query $c: = (c_0, c_1, c_2, \pi)$ such that $\dec(\crs, k, c) \neq \dec(\crs,t,c)$ and returns $((c_0, c_1, c_2), \pi)$
\end{enumerate}

$\advB$ simulates $\games_\thisgame$ perfectly and if the event $\event$ happens, it outputs a valid proof for a statement which is not in the specified language $L$. Therefore
\[\Pr[\event] \leq \simsnd^\nizk_\advB,\]
which concludes the proof of the lemma.  

%\nextgame{ReRand2}
%In $\games_\thisgame$ we use the key $t$ and randomness $r'$ as input for rerandomization. Concretely we compute $\rerand(c,c',h_1,h_2,t,r')$. This is just conceptual change since the ciphertext distributions are the same in both games and therefore 
%
%\begin{lemma}
%\[
%\left|\Pr[\games_\prevgame = 1] = \Pr[\games_\thisgame = 1]\right|.
%\]
%\end{lemma}
%
%\nextgame{RndExp4}
%In $\games_\thisgame$ we sample $r$ uniformly at random from $\varphi(N)$. 
%
%\begin{lemma}
%\[
%\left|\Pr[\games_\prevgame = 1] - \Pr[\games_\thisgame = 1]\right| \leq \frac{1}{N}.
%\]
%\end{lemma}

\nextgame{RndExp}
In $\games_\thisgame$ we sample $k$ uniformly at random from $[\varphi(N)/4]$. 

\begin{lemma}
\[
\left|\Pr[\games_\prevgame = 1] - \Pr[\games_\thisgame = 1]\right| \leq \frac{1}{p}+\frac{1}{q}-\frac{1}{N}.
\]
\end{lemma}

Again using a statistical argument this lemma directly follows from \Cref{sampling-lemma}.

\nextgame{DDH}
In $\games_\thisgame$ we sample $y_1 \rand \qrn$ and compute the challenge ciphertext $c^*:=(g^r, y_1 \cdot m_b, y_2 \cdot m_b)$. 

\begin{lemma}
\[
\left|\Pr[\games_\prevgame = 1] - \Pr[\games_\thisgame = 1]\right| \leq \advtg_\advB^\ddh.
\]
\end{lemma}
We construct an adversary $\advB = \{\advB_\secpar\}_{\secpar \in \N}$ against DDH in the group $\qrn$. Given \Cref{thm:ddh} this implies an adversary against DDH in large prime-order subgroups of $\Zn^*$.

$\advB_{\secpar}(N,p,q,g,g^\alpha, g^\beta, g^\gamma):$
\vspace{-2mm}
\begin{enumerate}
\item Computes $\varphi(N):=(p-1)(q-1), t:=2^{T} \bmod \varphi(N)/4, h_2:=g^t \bmod N$ and sets $\pk:=(N, T(\secpar), g, h_1: = g^\alpha, h_2)$
\item Runs $(m_0, m_1, \st) \leftarrow \adv_{1, \secpar}(\pk)$ and answers decommitment queries using $t$.
\item Samples $b \rand \bits, y_2 \rand \qrn$ and computes $(c_0^*, c_1^*, c_2^*):=(g^\beta, g^\gamma \cdot m_b, y_2 \cdot m_b).$ Runs $\pi^* \leftarrow \simul(1, \st', (c_0^*, c_1^*, c_2^*))$.
\item Runs $b' \leftarrow \adv_{2, \secpar}(c_0^*, c_1^*, c_2^*, \pi^*), \st)$ and answers decommitment queries using $t$.
\item Returns the truth value of $b=b'$.
\end{enumerate}
If $\gamma = \alpha\beta$ then $\advB$ simulates $\games_\prevgame$ perfectly. Otherwise $g^\gamma$ is uniform random element in $\qrn$ and $\advB$ simulates $\games_\thisgame$ perfectly. This proofs the lemma.

\begin{lemma}\label{nitc:llem}
\[
\Pr[\games_\thisgame = 1] = \half.
\]
\end{lemma}

Clearly, $c_0^*$ is uniform random element in $\qrn$ it does not contain any information about the challenge message. Since $y_1, y_2$ are sampled uniformly at random from $\qrn$ the ciphertexts $c_1^*, c_2^*$ are also uniform random element in $\qrn$ and hence do not contain any information about the challenge bit $b$. Therefore, an adversary can not do better than guessing.

By combining Lemmas \ref{nitc:flem} - \ref{nitc:llem} we obtain the following:
\begin{align*}
&\advtg^{\nitc}_{\adv} = \left| \Pr[\games_0 = 1] - \half \right| \leq \sum_{i=0}^7 \left|\Pr[\games_i = 1] - \Pr[\games_{i+1} = 1] \right| + \left|\Pr[\games_{8}- \half\right| \\
 &\leq \zk^\nizk_\advB + \advtg^{\sss}_{\advB} + \simsnd^{\nizk}_{\advB} + \advtg^{\ddh}_{\advB}  + 2 \left( \frac{1}{p}+\frac{1}{q}-\frac{1}{N} \right),
\end{align*}
which concludes the proof.
\end{proof}

\begin{theorem}
$(\pgen, \com, \cvrfy, \dvrfy, \fdecom)$ defined in \Cref{table:nitc} is a BND-CCA-secure non-interactive timed commitment scheme. 
\end{theorem}

\begin{proof}
We show that the construction is actually perfectly binding. This is straightforward to show since ElGamal encryption is perfectly binding. Therefore there is exactly one message/randomness pair $(m,r)$ which can pass the check in $\dvrfy$. Therefore the first winning condition of BND-CCA experiment happens with probability 0. Moreover, since the value $\pgen$ is executed by the challenger and hence the value $h_2$ is computed correctly, $\fdecom$ reconstructs the correct message $m$. Therefore the second winning condition of BND-CCA experiment happens with probability 0 as well.
\end{proof}



 

%\begin{figure}[h!]
%\begin{center}
%\begin{tabular}{|ll|}
%\hline
%$\underline{\kgen(\seck, T)}$ 							   & $\underline{\decf(\sk, c)}$\\
%$(p, q_, N) \leftarrow \genmod(\seck)$ &  Parse $c$ as $(c_0, c_1, c_2, \pi)$\\
%$\varphi(N):= (p-1)(q-1)$   & if $\nizk.\vrfy((c_0, c_1, c_2), \pi)= 1$ \\
%$g\rand \qrn$ & \tab Compute $y_1:= c_0^{k} \bmod N$\\
%$k \rand \varphi(N)/4$ & \tab return $c_1 \cdot H(y_1)^{-1} \bmod N$ \\
%$t:= 2^T \bmod \varphi(N)/4$ & return $\bot$\\
%$h_1:= g^k \bmod N$ & \\
%$h_2:=g^{t} \bmod N$ & \\
%%$\crs \leftarrow \nizk.\setup(\seck)$ & \\
%$\pk:= (N,T,g,h_1,h_2), \sk:= (N, k)$ & \\
%return $(\pk, \sk)$       & \\
%                                             &\\
%$\underline{\enc(\pk, m)}$           & $\underline{\decs(\pk,c)}$ \\
%$r \rand [\floor{N/4}]$     & Parse $c$ as $(c_0, c_1, c_2, \pi)$ \\
%Compute $c_0:= g^r \bmod N$ & if $\nizk.\vrfy((c_0, c_1, c_2), \pi)= 1$\\
%For $i \in [2]: y_i:= h_i^r \bmod N$   &   \tab Compute $ y_2:=c_0^{2^T} \bmod N$ \\
%For $i \in [2]: c_i:= H(y_i) \cdot m \bmod N$ & \tab return $c_2 \cdot H(y_2)^{-1} \bmod N$ \\
%$\Phi := (c_0, c_1, c_2), w := (m, r)$& return $\bot$\\
%$\pi \leftarrow \nizk.\prove(\Phi, w)$  &  \\
%return $c \leftarrow (c_0, c_1, c_2, \pi)$ &  \\
%
%\hline          
%\end{tabular}
%\caption{NY Construction of TPKE from SSSA}
%\label{table:tpke-elgamal}
%\end{center}
%\end{figure}




%%% Local Variables:
%%% mode: latex
%%% TeX-master: "main"
%%% End:
