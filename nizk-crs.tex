%!TEX root=main.tex
\section{Efficient Instantiation of SS-NIZK}\label{sec:nizk-crs}
In this section we provide an efficient NIZK proof systems for the languages  
\[
L_1 = \left\{(c_0, c_1, c_2)| \exists (m,r):
\begin{aligned}
       (\land_{i=1}^3 c_i = h_i^{rN}(1+N)^m \bmod N^2) \land \\
       c_0 = g^r \bmod N\\
    \end{aligned}
    \right\} \text{ and } 
\]
\[
L_2 = \left\{(c_0, c_1, c_2)| \exists (m,r):
\begin{aligned}
       (\land_{i=1}^3 c_i = h_i^{r}m \bmod N) \land
       c_0 = g^r \bmod N\\
    \end{aligned}
    \right\},   
\]
where $g, h_1, h_2, h_3, N$ are parameters defining the language. One can observe that this is equivalent to proving that the ciphertext defined as $(c_0, c_1\cdot (c_2)^{-1}, c_3\cdot (c_2)^{-1})$ is encryption of zero for public-key encryption scheme where a public key is defined as $\pk:=(g, (h_1\cdot (h_2)^{-1}), (h_3\cdot (h_2)^{-1}), N)$ and encryption is defined for $L_1$ as $\enc(\pk:=(g,h,h'),m): c:=g^r \bmod N, c':=h^{rN}(1+N)^m \bmod N^2, c':=h'^{rN}(1+N)^m \bmod N^2$ and for $L_2$ as $\enc(\pk:=(g,h,h'),m): c:=g^r, c':=hg^m, c':=h'g^m \bmod N$. Hence, both encryption schemes are additively homomorphic and by \Cref{lem:tsp} we obtain a trapdoor sigma protocol for the languages $L_1, L_2$ which in turn by \Cref{thm:nizk} yields an unbounded simulation-sound NIZKs for the same languages.  








%%% Local Variables:
%%% mode: latex
%%% TeX-master: "main"
%%% End:
