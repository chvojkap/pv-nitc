%!TEX root=./main.tex
%\section{Constructions of Non-Malleable NITC using NIZK in ROM}
\subsection{Construction of Linearly Homomorphic Non-Malleable NITC}
We define language for our construction of a linearly homomorphic NITC which relies on NIZKs in the ROM in the following way:

\[
L = \left\{(h_1, h_2, c_0, c_1, c_2, c_3)| \exists (m,r):
\begin{aligned}
       (\land_{i=1}^3 c_i = h_i^{rN}(1+N)^m \bmod N^2) \land \\
       c_0 = g^r \bmod N\\
    \end{aligned}
    \right\}, 
\]
where $g, h_3, N$ are parameters specifying the language.




\begin{figure}[h!]
\begin{center}
\begin{tabular}{|ll|}
\hline
$\underline{\pgen(\seck, T)}$ 							   & $\underline{\com(\crs, m)}$ \\
$(p, q_, N, g) \leftarrow \genmod(\seck)$ & $r \rand \smplset$  \\
$\varphi(N):= (p-1)(q-1)$   & $c_0:= g^r \bmod N$ \\
$k_1, k_2 \rand \smplset$ & For $i \in [3]: c_i:= h_i^{rN}(1+N)^m \bmod N^2$\\
$t:= 2^T \bmod \varphi(N)/2$ & $\Phi := (h_1, h_2, c_0, c_1, c_2, c_3), w := (m, r)$ \\
For $i \in [2]: h_i:= g^{k_i} \bmod N$ &  $\pi \leftarrow \nizk.\prove(\Phi, w)$\\
$h_3:=g^{t} \bmod N$ &  $c := (c_0, c_1, c_2, c_3, \pi)$\\
return $\crs:= (N,T,g,h_1,h_2, h_3)$ &  $\pi_\com:= \bot, \pi_\dec: = r$ \\
%$\crs \leftarrow \nizk.\setup(\seck)$ & \\
 & return $(c, \pi_\com, \pi_\dec)$\\
%return $\crs$     & \\
                                             &\\
$\underline{\cvrfy(\crs, c, \pi_\com)}$     & $\underline{\dvrfy(\crs,c, m, \pi_\dec)}$ \\
Parse $c$ as $(c_0, c_1, c_2, c_3, \pi)$  & Parse $c$ as $(c_0, c_1, c_2, c_3, \pi)$ \\
return $\nizk.\vrfy((h_1, h_2, c_0, c_1, c_2, c_3), \pi)$  & if $ \land_{i=1}^3 c_i = h_i^{\pi_\dec N}(1+N)^m  \bmod N^2$ \\
 & $\land c_0 = g^{\pi_\dec} \bmod N$\\
 & \tab return 1 \\
& return 0 \\
                                             &\\
$\underline{\fdecom(\crs,c)}$ & $\underline{\fdvrfy(\crs,c, m, \pi_\fdecom)}$ \\
Parse $c$ as $(c_0, c_1, c_2, c_3, \pi)$ & Parse $c$ as $(c_0, c_1, c_2, c_3, \pi)$\\
if $\nizk.\vrfy((h_1, h_2, c_0, c_1, c_2, c_3), \pi)$ & if $c_3 = \pi_\fdecom^N (1+N)^m \bmod N^2$\\
\tab Compute $ \pi_\fdecom:=c_0^{2^T} \bmod N$ &  \tab return 1\\
\tab $m:=\frac{c_3 \cdot \pi_\fdecom^{-N} (\bmod N^2) -1}{N}$ & return 0\\
\tab return $(m,\pi_\fdecom)$ & \\
return $\bot$ & \\
                                             &\\
$\underline{\eval(\crs,\oplus_N, c_1, \dots, c_n)}$ &  \\
Parse $c_i$ as $(c_{i,0}, c_{i,1}, c_{i,2}, c_{i,3}, \pi_i)$ & \\
\multicolumn{2}{|l|}{Compute $c_0 := \prod_{i=1}^n c_{i,0} \bmod N, c_1:= \bot, c_2:=\bot, c_3 := \prod_{i=1}^n c_{i,3} \bmod N^2, \pi:= \bot$} \\
return $c := (c_0, c_1, c_2, c_3, \pi)$ & \\
%$\underline{\decom(\crs, \sk, c)}$     & $\underline{\fdecom(\crs,c)}$ \\
%Parse $c$ as $(c_0, c_1, c_2, \pi)$  & Parse $c$ as $(c_0, c_1, c_2, \pi)$ \\
%if $\nizk.\vrfy((c_0, c_1, c_2), \pi)= 1$  & if $\nizk.\vrfy((c_0, c_1, c_2), \pi)= 1$\\
%\tab Compute $y_1:= c_0^{k} \bmod N$  &   \tab Compute $ y_2:=c_0^{2^T} \bmod N$ \\
%\tab return $c_1 \cdot y_1^{-1} \bmod N$ & \tab return $c_2 \cdot y_2^{-1} \bmod N$ \\
%return $\bot$ & return $\bot$\\
\hline          
\end{tabular}
\caption{Construction of Linearly Homomorphic NITC in ROM. \\ $\oplus_N$ refers to addition $\bmod N$}
\label{table:nitc-lh-rom}
\end{center}
\end{figure}

%\begin{figure}[h!]
%\begin{center}
%\begin{tabular}{|ll|}
%\hline
%$\underline{\pgen(\seck, T)}$ 							   & $\underline{\com(\pk, m)}$ \\
%$(p, q_, N) \leftarrow \genmod(\seck)$ & $r \rand \smplset$  \\
%$\varphi(N):= (p-1)(q-1)$   & Compute $c_0:= g^r \bmod N$ \\
%Sample random generator $g$ of $\Jn$ & For $i \in [3]: c_i:= h_i^{rN}(1+N)^m \bmod N^2$\\
%$k_1, k_2 \rand \smplset$ & $\Phi := (h_1c_0, c_1, c_2, c_3), w := (m, r)$ \\
%$t:= 2^T \bmod \varphi(N)/2$ &  $\pi \leftarrow \nizk.\prove(\Phi, w)$\\
%For $i \in [2]: h_i:= g^{k_i} \bmod N$ &  $c := (c_0, c_1, c_2, c_3, \pi)$\\
%$h_3:=g^{t} \bmod N$ &  $\pi_\com:= \bot, \pi_\dec: = r$ \\
%%$\crs \leftarrow \nizk.\setup(\seck)$ & \\
%return $\crs:= (N,T,g,h_1,h_2, h_3)$ & return $(c, \pi_\com, \pi_\dec)$\\
%%return $\crs$     & \\
%                                             &\\
%$\underline{\cvrfy(\crs, c, \pi_\com)}$     & $\underline{\dvrfy(\crs,c, m, \pi_\dec)}$ \\
%Parse $c$ as $(c_0, c_1, c_2, c_3 \pi)$  & Parse $c$ as $(c_0, c_1, c_2, c_3 \pi)$ \\
%return $\nizk.\vrfy((c_0, c_1, c_2, c_3,), \pi)$  & if $ \land_{i=1}^3 c_i = h_i^{rN}(1+N)^m  \bmod N^2 \land c_0 = g^r \bmod N$\\
% & \tab return 1 \\
%& return 0 \\
%                                             &\\
%$\underline{\fdecom(\crs,c)}$ & $\underline{\fdvrfy(\crs,c, m, \pi_\fdecom)}$ \\
%Parse $c$ as $(c_0, c_1, c_2, c_3, \pi)$ & Parse $c$ as $(c_0, c_1, c_2, c_3, \pi)$\\
%if $\nizk.\vrfy((c_0, c_1, c_2,c_3), \pi)= 1$& if $\nizk.\vrfy((c_0, c_1, c_2,c_3), \pi)= 1 \land $\\
%\tab Compute $ \pi_\fdecom:=c_0^{2^T} \bmod N$ &  $c_3 = \pi_\fdecom^N (1+N)^m \bmod N^2$\\
%\tab Compute $m:=\frac{c_3 \cdot \pi_\fdecom^{-N} (\bmod N^2) -1}{N}$ &\tab return 1\\
%\tab return $(m,\pi_\fdecom)$ & return 0\\
%return $\bot$ & \\
%
%%$\underline{\decom(\crs, \sk, c)}$     & $\underline{\fdecom(\crs,c)}$ \\
%%Parse $c$ as $(c_0, c_1, c_2, \pi)$  & Parse $c$ as $(c_0, c_1, c_2, \pi)$ \\
%%if $\nizk.\vrfy((c_0, c_1, c_2), \pi)= 1$  & if $\nizk.\vrfy((c_0, c_1, c_2), \pi)= 1$\\
%%\tab Compute $y_1:= c_0^{k} \bmod N$  &   \tab Compute $ y_2:=c_0^{2^T} \bmod N$ \\
%%\tab return $c_1 \cdot y_1^{-1} \bmod N$ & \tab return $c_2 \cdot y_2^{-1} \bmod N$ \\
%%return $\bot$ & return $\bot$\\
%
%\hline          
%\end{tabular}
%\caption{NY Construction of NITC}
%\label{table:nitc}
%\end{center}
%\end{figure}



\begin{theorem}
\label{thm:NITC-lin-ROM}
If $\nizk = (\nizk.\prove, \nizk.\vrfy)$ is a one-time simulation-sound non-interactive zero-knowledge proof system for $L$, the strong sequential squaring assumption with gap $\gap$ holds relative to $\genmod$, the Decisional Composite Residuosity assumption holds relative to $\genmod$, and the Decisional Diffie-Hellman assumption holds relative to $\genmod$ in $\Jn$, then \mathlist{(\pgen, \com, \cvrfy, \dvrfy, \fdecom)} defined in \Cref{table:nitc-lh-rom} is an IND-CCA-secure non-interactive timed commitment scheme with $\ugap$, for any $\ugap < \gap$. 
\end{theorem}
The proof can be found in \Cref{app:NITC-lin-ROM}.

\begin{theorem}
$(\pgen, \com, \cvrfy, \dvrfy, \fdecom)$ defined in \Cref{table:nitc-lh-rom} is a BND-CCA-secure non-interactive timed commitment scheme. 
\end{theorem}
{}
\begin{proof}
This can be proven in the same way as \Cref{bnd-cca-lh}.
\end{proof}

\begin{theorem}
If $\nizk = (\nizk.\prove, \nizk.\vrfy)$ is a non-interactive zero-knowledge proof system for $L$, then \mathlist{(\pgen, \com, \cvrfy, \dvrfy, \fdecom, \fdvrfy)} defined in \Cref{table:nitc-lh-rom} is a publicly verifiable non-interactive timed commitment scheme.
\end{theorem}

\begin{proof}
This can be proven in the same way as \Cref{pv-lh}.
\end{proof}

It is straightforward to verify that considering $\eval$ algorithm, our construction yields linearly homomorphic NITC. 
\begin{theorem}\label{hom-lh-rom}
The NITC \mathlist{(\pgen, \com, \cvrfy, \dvrfy, \fdecom, \fdvrfy, \eval)} defined in \Cref{table:nitc-lh-rom} is a linearly homomorphic non-interactive timed commitment scheme.
\end{theorem}









\subsection{Construction of Multiplicatively Homomorphic Non-Malleable NITC}
We define language for our construction of a linearly homomorphic NITC which relies on NIZKs in the ROM in the following way:
\[
L = \left\{(h_1, h_2, c_0, c_1, c_2)| \exists (m,r):
\begin{aligned}
       (\land_{i=1}^3 c_i = h_i^{r}m \bmod N) \land
       c_0 = g^r \bmod N\\
    \end{aligned}
    \right\}, 
\]
where $g, N$ are parameters specifying the language.




\begin{figure}[h!]
\begin{center}
\begin{tabular}{|ll|}
\hline
$\underline{\pgen(\seck, T)}$ 							   & $\underline{\com(\crs, m)}$ \\
$(p, q_, N, g) \leftarrow \genmod(\seck)$ & $r \rand \smplset$  \\
$\varphi(N):= (p-1)(q-1)$   & $c_0:= g^r \bmod N$ \\
$k_1\rand \smplsetqrn$ & For $i \in [2]: c_i:= h_i^{r}m \bmod N$\\
$t:= 2^T \bmod \varphi(N)/4$ & $\Phi := (h_1, h_2, c_0, c_1, c_2), w := (m, r)$ \\
$h_1:= g^{k_1} \bmod N$ &  $\pi \leftarrow \nizk.\prove(\Phi, w)$\\
$h_2:=g^{t} \bmod N$ &  $c := (c_0, c_1, c_2, \pi)$\\
return $\crs:= (N,T,g,h_1,h_2)$ &  $\pi_\com:= \bot, \pi_\dec: = r$ \\
%$\crs \leftarrow \nizk.\setup(\seck)$ & \\
 & return $(c, \pi_\com, \pi_\dec)$\\
%return $\crs$     & \\
                                             &\\
$\underline{\cvrfy(\crs, c, \pi_\com)}$     & $\underline{\dvrfy(\crs,c, m, \pi_\dec)}$ \\
Parse $c$ as $(c_0, c_1, c_2, \pi)$  & Parse $c$ as $(c_0, c_1, c_2 \pi)$ \\
return $\nizk.\vrfy((h_1, h_2, c_0, c_1, c_2), \pi)$  & if $ \land_{i=1}^2 c_i = h_i^{\pi_\dec}m  \bmod N \land c_0 = g^{\pi_\dec} \bmod N$\\
 & \tab return 1 \\
& return 0 \\
                                             &\\
$\underline{\fdecom(\crs,c)}$ & $\underline{\eval(\crs,\otimes_N, c_1, \dots, c_n)}$ \\
Parse $c$ as $(c_0, c_1, c_2, \pi)$ & Parse $c_i$ as $(c_{i,0}, c_{i,1}, c_{i,2}, \pi_i)$\\
if $\nizk.\vrfy((h_1, h_2, c_0, c_1, c_2), \pi)$ & Compute $c_0 := \prod_{i=1}^n c_{i,0} \bmod N$\\
\tab Compute $ y:=c_0^{2^T} \bmod N$ &   Compute $c_3 := \prod_{i=1}^n c_{i,3} \bmod N $\\
\tab $m:=c_2 \cdot y^{-1} \bmod N$ & $c_1:= \bot, \pi:= \bot$\\
\tab return $m$ & return $c := (c_0, c_1, c_2, \pi)$\\
return $\bot$ & \\

%$\underline{\decom(\crs, \sk, c)}$     & $\underline{\fdecom(\crs,c)}$ \\
%Parse $c$ as $(c_0, c_1, c_2, \pi)$  & Parse $c$ as $(c_0, c_1, c_2, \pi)$ \\
%if $\nizk.\vrfy((c_0, c_1, c_2), \pi)= 1$  & if $\nizk.\vrfy((c_0, c_1, c_2), \pi)= 1$\\
%\tab Compute $y_1:= c_0^{k} \bmod N$  &   \tab Compute $ y_2:=c_0^{2^T} \bmod N$ \\
%\tab return $c_1 \cdot y_1^{-1} \bmod N$ & \tab return $c_2 \cdot y_2^{-1} \bmod N$ \\
%return $\bot$ & return $\bot$\\

\hline          
\end{tabular}
\caption{Construction of Multiplicatively Homomorphic NITC in ROM. \\ $\otimes_N$ refers to multiplication $\bmod N$}
\label{table:nitc-mh-rom}
\end{center}
\end{figure}

Where $(\nizk.\prove, \nizk.\vrfy)$ is the Fiat-Shamir transform of the Sigma protocol for language $L_4$ defined in Cref.... Concretely, $H$ is a hash function modelled as a random oracle, then our $\nizk$ is defined as follows
\begin{itemize}
\item $\nizk.\prove(s:=(h_1, h_2, c_0, c_1, c_2), w:=(m, r)):$ Compute $\alpha \rand \smplsetqrn, a_0:=g^\alpha \bmod N, a_1:= (h_1\cdot h_2^{-1})^\alpha \bmod N, v:=H(s,a_0,a_1), z:= \alpha + v \cdot r$ and return $\pi:=(a_0,a_1,z)$.
\item $\nizk.\vrfy(s:=(h_1, h_2, c_0, c_1, c_2),\pi:=(a_0,a_1,z)):$ Compute $v:=v:=H(s,a_0,a_1)$ and return 1 if and only if $g^z = a_0 \cdot c_0^v \bmod N \land (h_1\cdot h_2^{-1})^z = a_1 \cdot (c_1\cdot c_2^{-1})^v \bmod N \land z \in [\estordqrn + v\estordqrn]$.
\end{itemize}

%\begin{figure}[h!]
%\begin{center}
%\begin{tabular}{|ll|}
%\hline
%$\underline{\pgen(\seck, T)}$ 							   & $\underline{\com(\pk, m)}$ \\
%$(p, q_, N) \leftarrow \genmod(\seck)$ & $r \rand \smplset$  \\
%$\varphi(N):= (p-1)(q-1)$   & Compute $c_0:= g^r \bmod N$ \\
%Sample random generator $g$ of $\Jn$ & For $i \in [3]: c_i:= h_i^{rN}(1+N)^m \bmod N^2$\\
%$k_1, k_2 \rand \smplset$ & $\Phi := (h_1c_0, c_1, c_2, c_3), w := (m, r)$ \\
%$t:= 2^T \bmod \varphi(N)/2$ &  $\pi \leftarrow \nizk.\prove(\Phi, w)$\\
%For $i \in [2]: h_i:= g^{k_i} \bmod N$ &  $c := (c_0, c_1, c_2, c_3, \pi)$\\
%$h_3:=g^{t} \bmod N$ &  $\pi_\com:= \bot, \pi_\dec: = r$ \\
%%$\crs \leftarrow \nizk.\setup(\seck)$ & \\
%return $\crs:= (N,T,g,h_1,h_2, h_3)$ & return $(c, \pi_\com, \pi_\dec)$\\
%%return $\crs$     & \\
%                                             &\\
%$\underline{\cvrfy(\crs, c, \pi_\com)}$     & $\underline{\dvrfy(\crs,c, m, \pi_\dec)}$ \\
%Parse $c$ as $(c_0, c_1, c_2, c_3 \pi)$  & Parse $c$ as $(c_0, c_1, c_2, c_3 \pi)$ \\
%return $\nizk.\vrfy((c_0, c_1, c_2, c_3,), \pi)$  & if $ \land_{i=1}^3 c_i = h_i^{rN}(1+N)^m  \bmod N^2 \land c_0 = g^r \bmod N$\\
% & \tab return 1 \\
%& return 0 \\
%                                             &\\
%$\underline{\fdecom(\crs,c)}$ & $\underline{\fdvrfy(\crs,c, m, \pi_\fdecom)}$ \\
%Parse $c$ as $(c_0, c_1, c_2, c_3, \pi)$ & Parse $c$ as $(c_0, c_1, c_2, c_3, \pi)$\\
%if $\nizk.\vrfy((c_0, c_1, c_2,c_3), \pi)= 1$& if $\nizk.\vrfy((c_0, c_1, c_2,c_3), \pi)= 1 \land $\\
%\tab Compute $ \pi_\fdecom:=c_0^{2^T} \bmod N$ &  $c_3 = \pi_\fdecom^N (1+N)^m \bmod N^2$\\
%\tab Compute $m:=\frac{c_3 \cdot \pi_\fdecom^{-N} (\bmod N^2) -1}{N}$ &\tab return 1\\
%\tab return $(m,\pi_\fdecom)$ & return 0\\
%return $\bot$ & \\
%
%%$\underline{\decom(\crs, \sk, c)}$     & $\underline{\fdecom(\crs,c)}$ \\
%%Parse $c$ as $(c_0, c_1, c_2, \pi)$  & Parse $c$ as $(c_0, c_1, c_2, \pi)$ \\
%%if $\nizk.\vrfy((c_0, c_1, c_2), \pi)= 1$  & if $\nizk.\vrfy((c_0, c_1, c_2), \pi)= 1$\\
%%\tab Compute $y_1:= c_0^{k} \bmod N$  &   \tab Compute $ y_2:=c_0^{2^T} \bmod N$ \\
%%\tab return $c_1 \cdot y_1^{-1} \bmod N$ & \tab return $c_2 \cdot y_2^{-1} \bmod N$ \\
%%return $\bot$ & return $\bot$\\
%
%\hline          
%\end{tabular}
%\caption{NY Construction of NITC}
%\label{table:nitc}
%\end{center}
%\end{figure}



\begin{theorem}
\label{thm:NITC-mul-ROM}
If $\nizk = (\nizk.\prove, \nizk.\vrfy)$ is a one-time simulation-sound non-interactive zero-knowledge proof system for $L$, the strong sequential squaring assumption with gap $\gap$ holds relative to $\genmod$, and the Decisional Diffie-Hellman assumption holds relative to $\genmod$ in $\Jn$, then \mathlist{(\pgen, \com, \cvrfy, \dvrfy, \fdecom)} defined in \Cref{table:nitc-mh-rom} is an IND-CCA-secure non-interactive timed commitment scheme with $\ugap$, for any $\ugap < \gap$. 
\end{theorem}
The proof can be found in \Cref{app:NITC-mul-ROM}.

\begin{theorem}
$(\pgen, \com, \cvrfy, \dvrfy, \fdecom)$ defined in \Cref{table:nitc-mh-rom} is a BND-CCA-secure non-interactive timed commitment scheme. 
\end{theorem}

\begin{proof}
This can be proven in the same way as \Cref{bnd-cca-mh}.
\end{proof}

It is straightforward to verify that considering $\eval$ algorithm, our construction yields multiplicatively homomorphic NITC. 
\begin{theorem}\label{hom-mh-rom}
The NITC \mathlist{(\pgen, \com, \cvrfy, \dvrfy, \fdecom, \fdvrfy, \eval)} defined in \Cref{table:nitc-mh-rom} is a multiplicatively homomorphic non-interactive timed commitment scheme.
\end{theorem}


\begin{remark}[Public Verifiability]
The construction can be made publicly verifiable in the same way as suggested in \Cref{rem:pv}. 
 \end{remark}

%\begin{figure}[h!]
%\begin{center}
%\begin{tabular}{|ll|}
%\hline
%$\underline{\kgen(\seck, T)}$ 							   & $\underline{\decf(\sk, c)}$\\
%$(p, q_, N) \leftarrow \genmod(\seck)$ &  Parse $c$ as $(c_0, c_1, c_2, \pi)$\\
%$\varphi(N):= (p-1)(q-1)$   & if $\nizk.\vrfy((c_0, c_1, c_2), \pi)= 1$ \\
%$g\rand \qrn$ & \tab Compute $y_1:= c_0^{k} \bmod N$\\
%$k \rand \varphi(N)/4$ & \tab return $c_1 \cdot H(y_1)^{-1} \bmod N$ \\
%$t:= 2^T \bmod \varphi(N)/4$ & return $\bot$\\
%$h_1:= g^k \bmod N$ & \\
%$h_2:=g^{t} \bmod N$ & \\
%%$\crs \leftarrow \nizk.\setup(\seck)$ & \\
%$\pk:= (N,T,g,h_1,h_2), \sk:= (N, k)$ & \\
%return $(\pk, \sk)$       & \\
%                                             &\\
%$\underline{\enc(\pk, m)}$           & $\underline{\decs(\pk,c)}$ \\
%$r \rand [\floor{N/4}]$     & Parse $c$ as $(c_0, c_1, c_2, \pi)$ \\
%Compute $c_0:= g^r \bmod N$ & if $\nizk.\vrfy((c_0, c_1, c_2), \pi)= 1$\\
%For $i \in [2]: y_i:= h_i^r \bmod N$   &   \tab Compute $ y_2:=c_0^{2^T} \bmod N$ \\
%For $i \in [2]: c_i:= H(y_i) \cdot m \bmod N$ & \tab return $c_2 \cdot H(y_2)^{-1} \bmod N$ \\
%$\Phi := (c_0, c_1, c_2), w := (m, r)$& return $\bot$\\
%$\pi \leftarrow \nizk.\prove(\Phi, w)$  &  \\
%return $c \leftarrow (c_0, c_1, c_2, \pi)$ &  \\
%
%\hline          
%\end{tabular}
%\caption{NY Construction of TPKE from SSSA}
%\label{table:tpke-elgamal}
%\end{center}
%\end{figure}

 

%\begin{figure}[h!]
%\begin{center}
%\begin{tabular}{|ll|}
%\hline
%$\underline{\kgen(\seck, T)}$ 							   & $\underline{\decf(\sk, c)}$\\
%$(p, q_, N) \leftarrow \genmod(\seck)$ &  Parse $c$ as $(c_0, c_1, c_2, \pi)$\\
%$\varphi(N):= (p-1)(q-1)$   & if $\nizk.\vrfy((c_0, c_1, c_2), \pi)= 1$ \\
%$g\rand \qrn$ & \tab Compute $y_1:= c_0^{k} \bmod N$\\
%$k \rand \varphi(N)/4$ & \tab return $c_1 \cdot H(y_1)^{-1} \bmod N$ \\
%$t:= 2^T \bmod \varphi(N)/4$ & return $\bot$\\
%$h_1:= g^k \bmod N$ & \\
%$h_2:=g^{t} \bmod N$ & \\
%%$\crs \leftarrow \nizk.\setup(\seck)$ & \\
%$\pk:= (N,T,g,h_1,h_2), \sk:= (N, k)$ & \\
%return $(\pk, \sk)$       & \\
%                                             &\\
%$\underline{\enc(\pk, m)}$           & $\underline{\decs(\pk,c)}$ \\
%$r \rand [\floor{N/4}]$     & Parse $c$ as $(c_0, c_1, c_2, \pi)$ \\
%Compute $c_0:= g^r \bmod N$ & if $\nizk.\vrfy((c_0, c_1, c_2), \pi)= 1$\\
%For $i \in [2]: y_i:= h_i^r \bmod N$   &   \tab Compute $ y_2:=c_0^{2^T} \bmod N$ \\
%For $i \in [2]: c_i:= H(y_i) \cdot m \bmod N$ & \tab return $c_2 \cdot H(y_2)^{-1} \bmod N$ \\
%$\Phi := (c_0, c_1, c_2), w := (m, r)$& return $\bot$\\
%$\pi \leftarrow \nizk.\prove(\Phi, w)$  &  \\
%return $c \leftarrow (c_0, c_1, c_2, \pi)$ &  \\
%
%\hline          
%\end{tabular}
%\caption{NY Construction of TPKE from SSSA}
%\label{table:tpke-elgamal}
%\end{center}
%\end{figure}




%%% Local Variables:
%%% mode: latex
%%% TeX-master: "main"
%%% End:
