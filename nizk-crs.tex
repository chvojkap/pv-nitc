%!TEX root=main.tex

\subsection{Non-Interactive Zero-Knowledge Proofs}
We recall definition of a simulation-sound non-interactive proof system (SS-NIZK). 

\begin{definition} 
A \emph{non-interactive zero-knowledge argument system} $\Pi$ for an NP language $L$ associated with a relation $\rel$ is a tuple of four PPT algorithms $(\gen_\param, \gen_L, \prove, \vrfy)$, which work as follows:
\begin{itemize}
\item $\param \leftarrow \gen_\param(\seck)$ takes as input binary representation of a security parameter $\seck$ and outputs public parameters $\param$.
\item $\crs \leftarrow \gen_L(\seck, L)$ takes as input binary representation of a security parameter $\seck$ and the description of language $L$ which specifies length of statements $n$.  It outputs a language dependent part $\crs_L$ of the common reference string $\crs:=(\param, \crs_L)$. 
%\item $\crs \leftarrow \gen_L(\seck, L, \tau_L)$ takes as input binary representation of a security parameter $\seck$, the description of language $L$ which specifies length of statements $n$ and a membership testing trapdoor $\tau_L$ for $L$.  It outputs a language dependent part $\crs_L$ of the common reference string $\crs:=(\param, \crs_L)$. 
\item $\pi \leftarrow \prove(\crs,s,w)$ is a PPT algorithm which takes as input the common reference sting $\crs$, a statement $s \in \{0,1\}^n$, and a witness $w$ such that $(s,w) \in \rel$ and outputs a proof $\pi$.
\item $0/1 \leftarrow \vrfy(\crs, s, \pi)$ is a deterministic algorithm which takes as input the common reference string $\crs$, a statement $s\in \{0,1\}^n$ and a proof $\pi$ and outputs either 1 or 0, where 1 means that the proof is ``accepted'' and 0 means it is ``rejected''.
\end{itemize}
Moreover, $\Pi$ should satisfy the following properties. For simplification we denote below by $\setup$ an algorithm that runs successively $\gen_\param$ and $\gen_L$ to generate a common reference string. 
\begin{itemize}
\item \emph{Completeness:}  for all $(s, w) \in \rel$ holds:
\[\Pr[\vrfy(\crs, s,\pi)=1:\crs \leftarrow \setup(\seck, L), \pi \leftarrow \prove(\crs, s,w)] =1.\] 
\item \emph{Soundness:} for all $x \notin L$ and all non-uniform polynomial-size adversaries $\adv = \{\adv_\secpar\}_{\secpar \in \nats}$, we have:
\[\Pr[\vrfy(\crs, s,\pi)=1:\crs \leftarrow \setup(\seck, L), \pi \leftarrow \adv_\secpar(\crs, s)] \leq \negl(\secpar).\] 
\item \emph{Zero-Knowledge:} there is a PPT simulator $(\simul_1, \simul_2)$, such that for all non-uniform polynomial-size adversaries $\adv = \{\adv_\secpar\}_{\secpar \in \nats}$ there exists a negligible function $\negl(\cdot)$ such that for all $\secpar \in \nats$ 
\begin{align*}
\zk_\adv^\nizk =& 
\left| \Pr\left[ \adv_\secpar^{\prove(\crs, \cdot, \cdot),}(\crs) = 1:\crs \leftarrow \setup(\seck, L) \right] \right. \\
&\left. - \Pr\left[ \adv_\secpar^{\oracle(\crs, \tau, \cdot, \cdot),}(\crs) = 1: (\crs, \tau) \leftarrow \simul_1(\seck, L) \right] \right|
\leq \negl(\secpar).
\end{align*}
Here $\prove(\crs, \cdot, \cdot)$ is an oracle that outputs $\bot$ on input $(s,w) \notin \rel$ and outputs a valid proof $\pi \leftarrow \prove(\crs, s, w)$ otherwise; $\oracle(\crs, \tau, \cdot, \cdot)$ is an oracle that outputs $\bot$ on input $(s,w) \notin \rel$ and outputs simulated proof $\pi \leftarrow \simul_2(\crs, \tau, s)$ on input $(s,w) \in \rel$. Note that the simulated proof is generated independently of the witness $w$.
\end{itemize}

\end{definition}

\begin{definition}[One-Time Simulation Soundness]
A NIZK for an NP language $L$ with zero-knowledge simulator $\simul = (\simul_0, \simul_1)$ is \emph{one-time simulation sound}, if for all non-uniform polynomial-size adversaries $\adv = \{\adv_\secpar\}_{\secpar \in \nats}$ there exists a negligible function $\negl(\cdot)$ such that for all $\secpar \in \nats$ 
\[
\simsnd_\adv^\nizk = \Pr\left[
\begin{aligned}
s \notin L \land \\
(s, \pi) \neq (s', \pi') \land \\
\vrfy(\crs, s, \pi) = 1
\end{aligned}
:
\begin{aligned}
(\crs, \tau) \leftarrow \simul_1(\seck, L) \\
(s, \pi) \leftarrow \adv_\secpar^{\simul_2(\crs, \tau, \cdot)}(\crs, \tau_L)
\end{aligned} \right] \leq \negl(\secpar),
\]
where $\tau_L$ is a membership testing trapdoor for language $L$ chosen by challenger and $\simul_2(\crs, \tau, \cdot)$ is a single query oracle which on input $s'$ returns $\pi' \leftarrow \simul(\crs, \tau, s')$.
\end{definition}

Liber~\etal \cite{Libert2021OneShotFN} show that given additively homomorphic encryption scheme, one can build a trapdoor sigma protocol for a language defined below. Moreover, any trapdoor sigma protocol can be turned into unbounded simulation sound NIZK which directly implies existence of a one-time simulation sound NIZK. 

\begin{lemma}[Lemma D.1~\cite{Libert2021OneShotFN}]\label{lem:tsp}
Let $(\gen, \enc, \dec)$ be an additively homomorphic encryption scheme where the message space $M$, randomness space $R$ and the ciphertext space $C$ form groups $(M, +), (R,+)$ and $(C, \cdot)$. Let the encryption scheme is such that for any public key $\pk$ generated using $(\pk, \sk) \leftarrow \gen(\seck)$, any messages $m_1, m_2 \in M$ and randomness $r_1, r_2 \in R$ holds
\[\enc(\pk, m_1;r_1) \cdot \enc(\pk, m_2;r_2) = \enc(\pk, m_1+m_2; r_1+r_2).\]
Let $S$ be a finite set of public cardinality such that uniform sampling from $S$ is computationally indistinguishable from uniform sampling from $R$. 
Then there is an trapdoor sigma protocol for the language $L:= \{c \in C| \exists r \in R: c = \enc(\pk, 0; r)\}$, where $\pk$ is fixed by the language.  
\end{lemma}

\begin{remark}
We note that Liber~\etal required that the order of the group $(R,+)$ is public and that this group is efficiently samplable which is used in the arguing of zero-knowledge property. This is however, not necessary, since it is sufficient to be able to sample from distribution which is computationally indistinguishable from the real uniform distribution. This results in computational indistinguishability of the real and simulated transcript. In case of our constructions, we are able to sample randomness from the distribution which is statistically indistinguishable from the uniform distribution over $R$ and hence the real and the simulated transcript are statistically indistinguishable. 
\end{remark}

Additionally, Liber~\etal construct a simulation sound non-interactive argument system from any trapdoor sigma protocol relying on a strongly unforgeable one-time signature, a lossy public-key encryption scheme and a correlation intractable hash function. 

\begin{theorem}[Thm B.1, Thm. B.2~\cite{Libert2021OneShotFN}]\label{thm:nizk}
Let $(\gen_\param, \gen_L, \prove, \vrfy)$ be a trapdoor sigma protocol for an NP language $L$. Then given a strongly unforgeable one-time signature scheme, $\rel$-lossy public-key encryption scheme, a correlation intractable hash function and an admissible hash function, there is an unbounded simulation sound non-interactive zero-knowledge argument system for the language $L$. 
\end{theorem}

\subsection{Efficient Instantiation of SS-NIZK}\label{sec:nizk-crs}
In this section we provide simulation sound NIZK proof systems for languages $L_1$ and $L_2$ that are used in our constructions. The languages are defined in the following way:  
\[
L_1 = \left\{(c_0, c_1, c_2)| \exists (m,r):
\begin{aligned}
       (\land_{i=1}^3 c_i = h_i^{rN}(1+N)^m \bmod N^2) \land \\
       c_0 = g^r \bmod N\\
    \end{aligned}
    \right\} \text{ and } 
\]
\[
L_2 = \left\{(c_0, c_1, c_2)| \exists (m,r):
\begin{aligned}
       (\land_{i=1}^3 c_i = h_i^{r}m \bmod N) \land
       c_0 = g^r \bmod N\\
    \end{aligned}
    \right\},   
\]
where $g, h_1, h_2, h_3, N$ are parameters defining the language. One can observe that this is equivalent to proving that the ciphertext defined as $(c_0, c_1\cdot (c_2)^{-1}, c_3\cdot (c_2)^{-1})$ is encryption of zero for public-key encryption scheme where a public key is defined as $\pk:=(g, (h_1\cdot (h_2)^{-1}), (h_3\cdot (h_2)^{-1}), N)$ and encryption is defined for $L_1$ as $\enc(\pk:=(g,h,h'),m): c:=g^r \bmod N, c':=h^{rN}(1+N)^m \bmod N^2, c':=h'^{rN}(1+N)^m \bmod N^2$ and for $L_2$ as $\enc(\pk:=(g,h,h'),m): c:=g^r, c':=hg^m, c':=h'g^m \bmod N$. Hence, both encryption schemes are additively homomorphic and by \Cref{lem:tsp} we obtain a trapdoor sigma protocol for the languages $L_1, L_2$ which in turn by \Cref{thm:nizk} yields an unbounded simulation-sound NIZKs for the same languages.  








%%% Local Variables:
%%% mode: latex
%%% TeX-master: "main"
%%% End:
