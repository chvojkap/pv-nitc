%!TEX root=main.tex
\section{Introduction}\label{sec:intro}
\emph{Time-locked puzzles}, introduced by Rivest, Shamir, and Wagner, allow a party $A$ to `commit a message $m$ to the future' such that 1) the commitment hides $m$ until some time $T$ and 2) the commitment can be forcibly opened after $T$. This powerful concept can be applied in a variety of applications, such as fair multiparty computation, second-prize auctions, and eVoting. 
As a simple example, consider the following two party coin flipping protocol: In the first phase, $A$ and $B$ sample random bits $b_A$ and $b_B$, and exchange commitments $c_A$ and $c_B$ to these values. In the second phase, $A$ and $B$ open their respective commitments, by sending each other the opening information. Each party then locally computes the output as $b=b_A\oplus b_B$ (assume that commitments are perfectly binding). This protocol is clearly not fair: a cheating $B$ can easily bias the outcome of the coin flip by refusing to send the opening to $c_B$ in the second phase after learning $b_A$ from $A$.
This problem can be overcome by using time-locked puzzles (and running both phases for a fixed number of time): instead of the parties sending each other their opening information $A$ can simply force $c_B$ open to recover $b_B$ (and vice versa for $B$). On the other hand, the timing parameter can be set so that $c_A$ hides $b_A$ until after the first phase of the protocol has passed.
Unfortunately, this solution by itself does not solve all problems with the above protocol. Namely, without any further assumptions on the commitment scheme, $B$ can create $c_B$ as a tampered version of $c_A$, e.g. such that $c_B$ is a commitment to $1\oplus b_A$. Now, $A$ deems the protocol successful, but the coin flip always evaluates to $1$. (To avoid a trivial replay attack where $B$ echoes $c_A$ as $c_B$, the parties can simply abort the protocol after step 1, in case $c_A=c_B$). To overcome such issues, a recent line of works has explored the idea of \emph{non-malleable time-locked puzzles} \cite{TCC:KatLosXu20,EPRINT:EFKP20a,EC:BDDNO21}.
These are essentially tamper-proof versions of time-locked puzzles which do not admit changing the opening of the value hidden inside the puzzle by applying some transform to the outer puzzle. While this type of puzzle overcomes the above attack, the resulting protocol becomes very computationally expensive: even if both parties behave honestly, both sides have to invest significant computational effort to force open the others commitment. To overcome such issues, the beautiful work of Boneh and Naor~\cite{C:BonNao00} introduced \emph{timed commitments}, which give the same guarantees as explained above, but have the additional feature that an opening to the commitment can be efficiently verified (and thus the commitment can be opened efficiently). This, however, introduces further subtleties: since there are now two paths of opening the commitment (fast and slow), there needs to be a means of convincing the receiving party that both paths lead to the \emph{same opening}. Otherwise, it would be easy to introduce a bias in the above protocol when using timed commitments and sending each other the opening information in the second phase rather than always opening it by brute force. To do so, $B$ can choose to withhold the opening information in the second phase \emph{selectively}.
Existing constructions of timed commitments are either malleable, rely on the random oracle model, or require the sender of the commitment to invest as much effort to commit to a value as for the receiver to forcibly open the commitment. On top of this, the only known standard model construction due to Katz et al. \cite{TCC:KatLosXu20} relies on of zero-knowledge proofs (NIZK) for arbitrary NP relations with very specific properties. The goal of this paper is to provide simpler constructions and new design paradigms for timed commitments which allow for efficient committing in the standard model and reduce the reliance on NIZK as much as possible. 
\subsection{Our Contribution}
We give an overview of our contributions. 
\begin{enumerate}
\item We begin by extending the formal definitions of prior work to cover additional properties which might be useful for applications. Concretely, we introduce public verifiability of forced opening and meaningful homomorphic properties in the setting of non-malleable non-interactive timed commitments. Public verifiability allows to anyone verify that the claimed opening computed by force is indeed value which is contained in the given commitment. This is useful in settings where forced opening is executed by some third party which is not necessarily trusted. Moreover, if we are only interested in the result of some computation on a set of NITCs, one can instead of force opening all unopened commitments, simply homomorphically combine commitments and then force open the resulting commitment. We remark, that for these to work, one must be sure that all commitments are properly generated, which is easy to check using $\cvrfy$ algorithm of NITC. Hence, one must at first check that all commitments are properly generated, then homomorphically combine commitments to obtain a final commitments, which is then opened by force. If some untrusted party is executing these steps, then everybody is able to verify that computation was executed properly, by verifying that all commitments are generated properly, homomorphically combining commitments and then checking a proof of forced opening that was provided by untrusted party with respect to the final commitment which was computed by us. 
\item Equipped with the necessary formalism, we then give four constructions of non-interactive non-malleable timed commitment. Our construction relies on the variation of double encryption paradigm due to Naor and Yung \cite{STOC:NaoYun90}, which was also leveraged by Katz et al. in their construction. Compared to the generic construction of Katz et al. we do not start from a timed public key encryption scheme, but build our timed commitment from scratch. This lets us avoid two out of the three NIZKs in their construction and lets us replace the third by a simple NIZK proof for the variation of DDH relation over groups of unknown order. Interestingly, we are able to instantiate the given NIZK both in standard model and the random oracle model resulting in different efficiency gains. As for their construction, we support fast opening and verification of commitments (given the opening information) and fast verification of commitments (i.e., a means to check whether they can be forced open to the correct message). Another important benefit of our constructions over that of Katz et al. is that it allows for fast commitment. 
\item\todo{some of the informations in this paragraph are not totally correct} We can also make this construction non-interactive by using a single simulation sound NIZK. This is a far simpler requirement than what is needed for the construction of Katz et al., who need three NIZKs and also require that the simulated prover runs in time independently of the size of a witness. We remark that the (non-interactive) construction of David et al.~\cite{EC:BDDNO21} is in the programmable random oracle model, while ours is in the standard model. By comparison, the work of Ephraim et al. \cite{EPRINT:EFKP20a} (also non-interactive) \emph{does} support fast opening and verification of commitments. On the other hand, their construction is in the auxiliary non-programmable random oracle model (which is slightly stronger than the standard NPROM) and does not allow for fast commitments.
\end{enumerate}

We give a comparison of our schemes with these works in ....




%%% Local Variables:
%%% mode: latex
%%% TeX-master: "main"
%%% End:
