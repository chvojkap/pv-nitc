%!TEX root=main.tex
\section{Introduction}\label{sec:intro}

Timed commitments make it possible to commit to a message with respect to some time parameter $T \in \N$, such that (1) the commitment is \emph{binding} for the committing party, (2) it is \emph{hiding} the committed message for $T$ units of time (\eg, seconds, minutes, days), but (3) it can also forcibly be opened after time $T$ in case the committing party refuses to open the commitment or becomes unavailable. This idea goes back to a seminal work by Rivest, Shamir, and Wagner \cite{RSW96} introducing the strongly related notion of \emph{time-lock puzzles}, and Boneh and Naor~\cite{C:BonNao00} extended this idea to \emph{timed commitments}, which have the additional feature that an opening to the commitment can be efficiently verified (and thus the commitment can be opened efficiently).

\paragraph{Achieving fairness via timed commitments.}
One prime application of timed commitmens is to achieve \emph{fairness} in secure two- or multi-party protocols. For instance, consider a simple sealed-bid auction protocol with $n$ bidders $B_1, \ldots, B_n$, where every bidder $B_i$ commits to its bid $x_i$ and publishes the commitment $c_i = \com(x_i,r_i)$ using randomness $r_i$. When all bidders have published their commitments, everyone reveals their bid $x_i$ along with $r_i$, such that everyone can publicly verify that the claimed bid $x_i$ is indeed consistent with the initial commitment $c_i$. The bidder with the maximal bid wins the auction.
For this to be most practical, we want commitments to be \emph{non-interactive}.

Now suppose that after the first $(n-1)$ bidders $B_1, \ldots, B_{n-1}$ have opened their commitments $(x_i, r_i)$, the last bidder $B_n$ claims that it has ``lost'' its randomness $r_{i^*}$, \eg, by accidentally deleting it. However, $B_n$ also argues strongly and quite plausibly that it has made the highest bid $x_{i^*}$. This is a difficult situation to resolve in practice:
\begin{itemize}
	\item \emph{$B_n$ might indeed be honest.} In this case, it would be fair to accept its highest bid $x_{i^*}$. One could argue that it is $B_n$'s own fault and thus it should not win the auction, but at the same time a seller might strongly argue to accept the bid, as it is interested in maximising the price, and if $B_n$'s claim is indeed true, then discarding the real highest bit could be considerd unfair by the seller.
	\item \emph{However, $B_n$ might also be cheating.} Maybe it didn't commit to the highest bid, and now $B_n$ tries to ``win'' the auction in an unfair way.
\end{itemize}

Timed commitments can resolve this situation very elegantly and without the need to resort to a third party that might collude with bidders, and thus needs to be trusted, or which might not even be available in certain settings, \eg, in fully decentralized protocols, such as blockchain-based applications. 
In a timed commitment scheme, the parties create their commitments $c_i = \com(x_i,r_i, T)$ with respect to a suitable time parameter $T$ for the given application. In case one party is not able to or refuses to open its commitment, the other parties can force the commitment open in time $T$ and thus resolve a potential dispute. 


\subsection{Requirements on Practical Timed Commitments}
Several challenges arise when constructing timed commitments that can be used in practical applications.

\begin{description}
	\item[Consistency of standard and forced opening.] A first challenge to resolve when constructing a timed commitment scheme is to guarantee that the availability of an alternative way to open a commitment, by using the forced decommitment procedure, does not break the \emph{binding} property. Standard and forced opening must be guaranteed to reveal the same message. Otherwise, a malicious party could create a commitment where standard and forced openings yields different values. Then it could decide in the opening phase whether it provide the ``real'' opening, or whether it refuses to open, such that the other parties will perform the forced opening.
	
	\item[Non-interactivity.] Having non-interactive commitments is generally desirable to obtain protocols that do not require all parties to be online at the same time. Furthermore, certain applications inherently require the commitment scheme to be non-interactive. This includes, for example, protocols where the commitments are published in a public ledger (\eg, a decentralized blockchain). Several examples of such applications are described in \cite{C:MalThy19}. Non-interactivity also avoids concurrent executions of the commitment protocol, which simplifies the security model significantly.


	\item[Non-malleability.] 
	Non-malleability of a commitment guarantees that no party can turn a given commitment $c$ that decommits to some value $x$ into another commitment $c'$ which decommits to a different value $x'$, such that $x$ and $x'$ are related in some meaningful way.
	%
	For instance, in the above example of an auction, a malicious party $B_n$ could first wait for all other parties to publish their commitments. Then it would select the commitment $c_i$ which most likely contains the highest bid $x_i$, and exploit the malleability of to create a new commitment $c_n$, which is derived from $c_i$ and opens to $x_i + 1$. Hence, $B_n$ would be able win the auction with a bid that is only slightly larger than the 2\textsuperscript{nd} highest bit, which does not meet the intuitive security expectations on a secure auctioning protocol.

	In order to achieve non-malleability for timed commitments, a recent line of works has explored the idea of \emph{non-malleable time-locked commitments} and \emph{puzzles} \cite{TCC:KatLosXu20,EPRINT:EFKP20a,EC:BDDNO21}.
%
	% Ephraim is FO with TLPs, not homomorphic
	% EC21 TLPs in UC, automatically achieve NM, RO, no public verifiability, not homomorphic
%
	Existing constructions of timed commitments are either malleable, rely on the random oracle model, or require the sender of the commitment to invest as much effort to commit to a value as for the receiver to forcibly open the commitment. 
%
	The only known standard model construction by Katz \etal \cite{TCC:KatLosXu20} relies on non-interactive zero-knowledge proofs (NIZKs) for \emph{general} NP relations with very specific properties. 



	

	\item[Force opening many commitments at once via homomorphism.] 
	Yet another interesting property that can make timed commitments more practical is a possibility to aggregate multiple commitments into a \emph{single} one, such that it is sufficient to force open only this commitment. The idea of homomorphic time-lock \emph{puzzles} was introduced by Malavolta and Thyagarajan~\cite{C:MalThy19}. We consider the adoption of this idea to timed commitments. 

	A homomorphic timed commitment scheme allows to efficiently evaluate a circuit $C$ over a set of commitments $c_1, \ldots, c_n$, where $c_i$ is a commitment to some value $x_i$ for all $i$, to obtain a commitment $c$ to $C(x_1, \ldots, x_n)$. 
	%
	If there are multiple parties $B_{i_1}, \ldots, B_{i_z}$ that refuse to open their commitments and it is not necessary to recover the full committed messages $x_{i_1}, \ldots, x_{i_z}$, but recovering $C(x_{i_1}, \ldots, x_{i_z})$ is sufficient, then one can use the homomorphism to compute a signle commitment $c$ that needs to be opened. Malavolta and Thyagarajan~\cite{C:MalThy19} describe several interesting applications, includings e-voting and sealed-bid auctions over blockchains, multi-party coin flipping, and multi-party contract signing.

	\item[Public verifiability of commitments.] 
	Another property is \emph{public verifiability} of a timed commitment, which requires that one can efficiently check whether a commitment is well-formed, such that a forced decommitment will yield a correct result. 
%This property was first suggested for time-lock puzzles by \cite{EPRINT:EFKP20a}.

	Without public verifiability, timed commitments might not provide practical solutions for certain applications. For instance, a malicious party could output a malformed commitment that cannot be opened in time $T$, such that a protocol would fail again in case the malicious party refuses to open the commitment. This could pose a problem in time-sensitive applications, in particular if a large time parameter $T$ is used, and also give rise do Denial-of-Service attacks.
	Note that public verifiability is particularly relevant for homomorphic commitments. When many commitments are aggregated into a single one, then it is essentiall that all these commitments are well-formed, as otherwise the forced opening may fail. Public verifiability allows to efficiently decide which subset of commitments is well-formed, and thus to include only these in the homomorphic aggregate.

	Note that the requirement of public verifiability rules out several natural ways to achieve non-malleability, such as the Fujisaki-Okamoto transform \cite{C:FujOka99,JC:FujOka13} used by Ephraim \etal \cite{EPRINT:EFKP20a}. It seems that even in the random oracle model ZK proofs are required.
	% However, even the construction by Katz \etal \cite{TCC:KatLosXu20}, which intensively uses ZK proofs, does not yet achieve public verifiability. \todo{is this true? or only not yet defined?}
	\item[Public verifiability of forced opening.] In scenarios when the forced opening is executed by untrusted party, it is desirable to be able efficiently check that forced opening has been executed properly without redoing an expensive sequential computation. This particularly useful when the forced opening computation is outsourced to untrusted server. This property was first suggested for time-lock puzzles by \cite{EPRINT:EFKP20a}. 
\end{description}









\subsection{Our Contributions}

We provide a simpler and more efficient approach to construct practical non-malleable timed commitments. We give the first constructions that simultaneously achieve non-interactivity, non-malleability, linear (\ie, additive) or multiplicative homomorphism, public verifiability of commitments and public verifiability of forced opening. 
Instead of relying on expensive ZK proofs for general NP languages as prior work, we show how to use Fiat-Shamir \cite{C:FiaSha86} NIZKs derived from Sigma protocols for simple algebraic languages. Our constructions can be instantiated in the standard model by leveraging techniques from Libert \etal \cite{Libert2021OneShotFN} and more efficiently in the random oracle model.

In more detail, we make the following contributions.
\begin{enumerate}
\item We begin by extending the formal definitions of prior work to cover public verifiability of forced opening and homomorphic properties in the setting of non-malleable non-interactive timed commitments. 
%Public verifiability allows to anyone verify that the claimed opening computed by force is indeed value which is contained in the given commitment. This is useful in settings where forced opening is executed by some third party which is not necessarily trusted. Moreover, if we are only interested in the result of some computation on a set of NITCs, one can instead of force opening all unopened commitments, simply homomorphically combine commitments and then force open the resulting commitment. We remark, that for these to work, one must be sure that all commitments are properly generated, which is easy to check using $\cvrfy$ algorithm of NITC. Hence, one must at first check that all commitments are properly generated, then homomorphically combine commitments to obtain a final commitments, which is then opened by force. If some untrusted party is executing these steps, then everybody is able to verify that computation was executed properly, by verifying that all commitments are generated properly, homomorphically combining commitments and then checking a proof of forced opening that was provided by untrusted party with respect to the final commitment which was computed by us. 
\item We then give four constructions of non-interactive non-malleable timed commitments. All our constructions rely on a variation of the double encryption paradigm by Naor and Yung \cite{STOC:NaoYun90}, which was also used by Katz \etal \cite{TCC:KatLosXu20}. 

However, in contrast to \cite{TCC:KatLosXu20}, we do not start from a timed public key encryption scheme, but build our timed commitment from scratch. This enables us avoid two out of the three NIZK proofs in their construction, and lets us replace the third by a proof for a variation of the DDH relation over groups of unknown order. We are able to instantiate the given NIZK both in the standard model and in the random oracle model \cite{CCS:BelRog93}. Like the construction from \cite{TCC:KatLosXu20} we support public verifiability of commitments. Another important advantage of our constructions over that of Katz \etal \cite{TCC:KatLosXu20} is that it allows for fast commitment, whereas \cite{TCC:KatLosXu20} requires solving a puzzle for commitments.Additionally we achieve, public verifiability of forced opening and homomorphic properties. 
\end{enumerate}

%We give a comparison of our schemes with these works in ....


%\item\todo{some of the informations in this paragraph are not totally correct} We can also make this construction non-interactive by using a single simulation sound NIZK. This is a far simpler requirement than what is needed for the construction of Katz et al., who need three NIZKs and also require that the simulated prover runs in time independently of the size of a witness. 

In comparison, the non-interactive construction of David \etal~\cite{EC:BDDNO21} is in the programmable random oracle model, while ours can also be instantiated in the standard model. David \etal achieve fast commitments, however the construction does not provide public verifiability of commitments, public verifiability of forced opening nor homomorphic properties. The work of Ephraim et al. \cite{EPRINT:EFKP20a} does support fast commitments and public verifiability of forced opening, but is also in the (auxiliary non-programmable) random oracle model and does not support public verifiability of commitments and homomorphic properties.


\todo{Connect table to rest of intro.}

\begin{table}
\begin{center}
\begin{tabular}{llclcclllc}
\textbf{Construction}         & \textbf{Hom.} & \textbf{Std.} & \textbf{Setup} & \textbf{Com?} & \textbf{FDec?} & $|\textbf{Com}|$ & \textbf{$|\pi_\com|$} & $t_{Com}$   & \textbf{Tight} \\
\hline
\cite{EPRINT:EFKP20a}         & ---           & \xmark        & ---            & \xmark        & \cmark         & $O(1)$           &    ---     & $O(\log T)$ & \cmark         \\
\cite{TCC:KatLosXu20}         & ---           & \cmark        & priv.          & \cmark        & \xmark         & $O(1)$           &    $O(1)$        & $O(T)$      & \cmark         \\
\cite{CCS:TCLM21}             & linear        & \xmark        & pub.            & \cmark        & \xmark         & $O(\secpar)$     &    $O(\secpar)$      & $O(1)$      & \xmark         \\
\hline
\Cref{sec:linearNITCstdmodel} & linear        & \cmark        & priv.          & \cmark        & \cmark         & $O(1)$ &                 $O(\log\secpar)$ & $O(1)$      & \cmark         \\
\Cref{sec:multNITCstdmodel}   & mult.         & \cmark        & priv.          & \cmark        & \cmark         & $O(1)$ &      $O(\log\secpar)$  & $O(1)$      & \cmark         \\
\Cref{sec:linear-ROM}         & linear        & \xmark        & priv.          & \cmark        & \cmark         & $O(1)$           &    $O(1)$     & $O(1)$      & \cmark         \\
\Cref{sec:mult-ROM}           & mult.         & \xmark        & priv.          & \cmark        & \cmark         & $O(1)$           &  $O(1)$    & $O(1)$      & \cmark         \\
\hline
\end{tabular}
\caption{\label{tab:comparison-related-work}Comparison of our constructions with related work. Column \textbf{Hom.} indicates whether the construction provides a linear/multiplicative homomorphism, \textbf{Std.} whether the construction has a standard-model proof, \textbf{Com?} whether it is publicly verifiable that commitments are well-formed, \textbf{FDec?} efficient public verifiability of forced decommitments, $|\textbf{Com}|$ is the size of commitments, \textbf{$|\pi_\com|$} the size of proofs,  $t_{Com}$ the running time of the commitment algorithm, and \textbf{Tight} whether the proof avoids running the forced decommitment algorithm to respond to CCA queries.}
\end{center}
\end{table}






%%% Local Variables:
%%% mode: latex
%%% TeX-master: "main"
%%% End:
