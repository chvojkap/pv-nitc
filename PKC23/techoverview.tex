%!TEX root=main.tex

\subsection{Technical Overview}\label{sec:techoverview}
The \emph{binding} property of our commitment scheme will be relatively easy to argue, therefore let us focus on the \emph{hiding} property and non-malleability. Like in \cite{TCC:KatLosXu20}, we prove this by considering an IND-CCA security experiment, where the adversary has access to a forced decommitment oracle. 
Even though the forced decommitment can be performed in polynomial time, this polynomial may be very large, if the time parameter $T$ is large. Since the experiment needs to be perform a forced decommitment for \emph{every} decommitment query of the adversary, this would incur a very significant overhead and a highly lossy reduction.
Hence, following Katz \etal \cite{TCC:KatLosXu20}, we aim to build commitment schemes where a reduction can perform a fast decommitment. 

Recall that a classical approach to achieve IND-CCA security is to apply the Naor-Yung paradigm \cite{STOC:NaoYun90}. A natural approach to construct non-malleable timed commitments is therefore to apply this paradigm as follows. A commitment $c = (c_1, c_2, \pi)$ to a message $m$ consists of a time-lock puzzle $c_1$ opening to $m$, a public key encryption of $m$, and a simulation-sound zero knowledge proof $\pi$ that both contain the same message $m$, everything with respect to public parameters contained in a public common reference string.
This scheme may potentially achieve all desired properties:
\begin{itemize}
	\item Consistency of regular and forced opening can be achieved by using a suitable time-lock puzzle and public-key encryption scheme. 
	\item The commitment is non-interactive.
	\item IND-CCA security follows from the standard Naor-Yung argument.
	\item The time-lock puzzle in the above construction can be instantiated based on repeated squaring \cite{RSW96}, possibly using the variant of \cite{C:MalThy19} that combines repeated squaring with Paillier encryption \cite{EC:Paillier99} to achieve a linear homomorphism. 
	\item Public verifiability can be achieved by using a suitable proof system for $\pi$.
\end{itemize}

Furthermore, in the IND-CCA security proof, we can perform fast opening by decrypting $c_2$ with the secret key of the public key encryption scheme, which is indistinguishable from a forced opening using $c_1$ by the soundness of the proof. 
%
However, it turns out that concretely instantiating this scheme in a way that yields a practical construction is non-trivial and requires a very careful combination of different techniques.

\paragraph{Triple Naor-Yung.}
First of all, note that repeated squaring modulo a composite number $N = PQ$, where $P$ and $Q$ are different primes, is currently the only available choice to achieve a practical time-lock puzzle, hence we are bound to using this puzzle to instantiate $c_1$. Conveniently, this puzzle allows for a linear (\ie, additive) homomorphism by following \cite{C:MalThy19}.
%
Then, in order to be able to instanatiate $\pi$ efficiently, it would be convenient to use a standard Sigma protocol, which can then be made non-interactive via the Fiat-Shamir transform \cite{C:FiaSha86} in the random oracle model, or by leveraging techniques from Libert \etal \cite{Libert2021OneShotFN} in the standard model. Since practically efficient Sigma protocols are only known for algebraic languages, such as that defined by the DDH relation, for example, we have to choose $c_2$ in a way which is ``algebraically compatible'' with $c_1$ and the available proofs $\pi$. If we instantiate $c_1$ with the homomorphic TLP from \cite{C:MalThy19}, then a natural candidate would be to instantiate $c_2$ also with Paillier encryption.  Here we face the first technical difficulty: 
\begin{itemize}
	\item Efficient proof systems for $\pi$ are only available, if both $c_1$ and $c_2$ use the same modulus $N$. Hence, we have to instantiate both with the same modulus.
	\item When arguing that $c_1$ hides the committed message $m$ in the Naor-Yung argument of the security proof, we will have to replace $c_1$ with a random puzzle, using the \emph{strong sequential squaring} (SSS) assumption. At the same time, we have to be able to respond to decommitment queries using the decryption key of $c_2$. But this decryption key is the factorization $P, Q$ of the common modulus $N$, and we cannot reduce to the hardness of SSS while knowing the factorization of $N$.
\end{itemize}
The first candidate approach to overcome this difficulty is to replace the Paillier encryption used in $c_2$ with an encryption scheme that does not require knowledge of the factorization of $N$, such as the ``Paillier ElGamal'' scheme from \cite{C:MalThy19}, which is defined over the subgroup $\Jn$ of elements of $\Zn$ having Jacobi symbol 1, and which uses a discrete logarithm to decrypt but still requires the factorization of $N$ to be hidden in order to be secure.

However, now we run into another difficulty. In the Naor-Yung argument, we will also have to replace $c_2$ with an encryption of a random message, in order to argue that our commitment scheme is hiding. In this part of the proof, we cannot know the secret key of $c_2$, that is, neither the aforementioned discrete logarithm, nor the factorization of $N$. However, we also cannot use $c_1$ to respond to decommitment queries, because then we would have to solve the time-lock puzzle, which cannot be done fast without knowledge of the factorization of $N$.

We resolve this problem by using ``triple Naor-Yung''. In our linearly homomorphic constructions, a commitment to $m$ will have the form $c = (c_1, c_2, c_3, \pi)$, where $c_1$ and $c_2$ are Paillier-ElGamal encryptions of $m$, and $c_3$ is the Paillier-style time-lock puzzle based on repeated squaring from \cite{C:MalThy19}. All are with respect to the same modulus $N$, and thus allow for an efficient Sigma-protocol-based proof $\pi$ that $c_1$, $c_2$, and $c_3$ all contain the same message. In the Naor-Yung-style IND-CCA security proof, we will first replace $c_3$ with a random puzzle, while using the discrete logarithm of the public key that corresponds to $c_1$ to perform fast decommitments. When we then replace $c_2$ with an encryption of a random message, we use the discrete logarithm of the public key that corresponds to $c_2$ to answer decommitment queries. Hence, throughout the argument we never require the factorization of $N$ for fast decommitments.


\paragraph{Standard Naor-Yung works for multiplicative homomorphism.}
Next, we observe that the standard (\ie, ``two-ciphertext'') Naor-Yung approach works, if a \emph{multiplicative} homomorphism (or no homomorphism at all) is required. Concretely, a commitment will have the form $c = (c_1, c_2, \pi)$, where $c_1$ is an ElGamal encryption and $c_2$ uses the ``sequential-squating-with-ElGamal-encryption'' idea of \cite{C:MalThy19}. By replacing the underlying group to the subgroup $\qrn$ of quadratic residues modulo $N$, we can rely on the DDH assumption in $\qrn$ and thus do not require the factorization of $N$ to be hidden when replacing the ElGamal encryption $c_1$ with an encryption of a random message.
While the construction idea and high-level arguments are very similar, the underlying groups and detailed arguments are somewhat different, and thus we have to give a separate proof.




\paragraph{On separate proofs in the standard model and the ROM}
The constructions sketched above can be instantiated relatively efficiently in the standard model, using the one-shot Fiat-Shamir arguments in the standard model by Libert \etal \cite{Libert2021OneShotFN}. However, these proofs repeat the underlying Sigma protocol a logarithmic number of times, and thus it would be interesting to also consider constructions in the random oracle model.
Since the syntactical definitions and properties of proof systems in the random oracle model are slightly different from that in \cite{Libert2021OneShotFN}, we give separate proofs for both random oracle constructions as well. 

\paragraph{Shared randomness} To obtain commitments of smaller size we additionally apply the shared randomness technique from \cite{SCN:BiaMasVen16}, where instead of producing two or three independent encryptions of the same message, we reuse the same randomness for encryption. This allows to save one group element in case of the standard Naor-Yung constructions and two group elements in the case of triple Naor-Yung.

\subsection{Further related work}
Time-lock puzzles based on randomized encodings were introduced in \cite{TCC:BDGM19}, but all known constructions of timed commitments rely on the repeated squaring puzzles of \cite{RSW96}.
Timed commitments are also related to time-lock encryption scheme \cite{liu2018build} and time-released encryption \cite{cryptoeprint:2020/739}, albeit with different properties. The construction in \cite{liu2018build} is based on an external ``computational reference clock'' (instantiated with a public block chain), whose output can be used to decrypt, such that decrypting parties do not have to perform expensive computations by solving a puzzle. The constructions of Chvojka \etal \cite{cryptoeprint:2020/739} are based on repeated squaring, however, the main difference is that the time needed for decryption starts to run from the point when $setup$ is executed and not from the point when ciphertext is created. 


% \subsection{Technical Overview}\label{sec:techoverview}

% % Katz et al:
% % - Slow encryption, have to perform seq squaring in encryption, therefore. Still only for one fixed value of T, because the NIZK language CRS depends on T.

% % - NY proof, well-formed commitment (can be force-openend with suitable parameter),
% % third proof not sufficient to reveal randomness. Three proofs in total....

% % - We need only one, for plaintext equality (sim sound), due to same modulus can be instantiated very efficiently.

% % In our case, T is fixed...

% Similar to the construction by Katz \etal \cite{TCC:KatLosXu20}, our is based on the Naor-Yung double-encryption approach \cite{STOC:NaoYun90}. However, we apply it in a very different way. 


% \todo{Explain why triple encryption}

% \paragraph{Main idea of the construction.}
% The CRS of our commitment scheme essentially consists of a time parameter $T$, an RSA modulus $N$, a generator $g$ of the group $\qrn$ of quadratic residues modulo $N$ and two numbers $h_{1}, h_{2} \in \Z_{N}$, where
% \[
% h_{1} = g^{k} \qquad\text{and}\qquad h_{2} = g^{2^{T}} 
% \]
% for $k \rand \smplset$.
% Note that this $(h_{1}, h_{2})$ can be seen as two ElGamal public keys in the group $\qrn$, where the corresponding secret keys $k$ and $2^{T} \bmod \ord$ are discarded.

% A commitment to a message $m \in \Z_{N}$ contains $c = (c_{0}, c_{1}, c_{2})$ where
% \[
% c_{0} = g^{r}, \qquad c_{1} = h_{1}^{r} \cdot m, \qquad c_{2} = h_{2}^{r} \cdot m,
% \]
% for $r \rand \smplset$.
% Note that $c$ can be viewed as two ElGamal ciphertexts $(c_{0}, c_{1})$ and $(c_{0}, c_{2})$ that share the same randomness $r$ and encrypt the same message $m$ with respect to the two public keys $(h_{1}, h_{2})$.
% ElGamal ciphertext $(c_{0}, c_{2})$ can be force-opened by repeated squaring, by computing $m = c_{2} / c_{0}^{2^{T}}$.
% Furthermore, the fact that there exists a secret key $k$ for ciphertext $(c_{0}, c_{1})$ will allow for answering decommitment queries in the security experiment quickly, that is, without the need to solve a sequential squaring puzzle, by computing $m = c_{1} / c_{0}^{k}$.
% Additionally, a commitment contains a NIZK proof that both  $(c_{0}, c_{1})$ and $(c_{0}, c_{2})$ commit to the same message. 

% Note that the commitment scheme is hiding based on the DDH assumption (for $(c_{0}, c_{1})$) and the strong sequential squaring assumption (for $(c_{0}, c_{2})$) in $\qrn$. It is perfectly binding by the perfect correctness of ElGamal with honestly generated public key in $\qrn$.

% \paragraph{Simplicity and efficiency.}
% We are able to obtain a very significant improvement in computational efficiency and conceptual simplicity when compared to \cite{TCC:KatLosXu20}. We achieve this for the following two reasons.
% \begin{enumerate}
% \item One key advantage of our construction is that both ciphertexts  $(c_{0}, c_{1})$ and $(c_{0}, c_{2})$ are defined over the \emph{same group} $\qrn$. This enables a very simple and efficient Fiat-Shamir-style proof, simply setting $h := h_{1}/h_{2}$ and then proving that
% \[
% \left(g, h, c_{0}, \frac{c_{1}}{c_{2}} \right)
% =
% \left(g, h, g^{r}, \frac{h_{1}^{r} \cdot m}{h_{2}^{r} \cdot m} \right)
% =
% \left(g, h, g^{r}, h^{r} \right)
% \]
% forms a DDH-tuple. Note that this can be achieved using a simple Sigma-protocol-based simulation-sound ZK proof of membership in the language of DDH tuples in $\qrn$ as proposed by Chaum and Pedersen \cite{C:ChaPed92}, which can be made non-interactive using the Fiat-Shamir transform \cite{C:FiaSha86}. Hence, in contrast to  \cite{TCC:KatLosXu20}, who use double encryption with two ciphertexts over different groups modulo $N_{1} \neq N_{2}$, no NIZKs for general NP languages are necessary.
% \item Furthermore, committing to a message essentially consists of a few standard exponentiations in $\qrn$ and computing the Fiat-Shamir NIZK. In particular, in contrast to \cite{TCC:KatLosXu20} it is not necessary to perform $T$ sequential squarings also when \emph{committing} to the message, but only for forced opening.
% \end{enumerate}
 




% %On the technical level our construction has some similarities with the construction of Katz \etal , however we make several observations which are crucial to obtain fast encryption and allows to simplify the construction, which leads to significant efficiency improvements. We recall that the construction by Katz \etal is based on timed-public key encryption (TPKE) which is then generically turned into non-malleable non-interactive timed commitment using two NIZKs. One of the NIZKs is used to guarantee that commitment has been generated properly and the second NIZK is used to verify the validity of the opening. The proposed construction of TPKE is based on the Naor-Yung paradigm combining simulation sound NIZK and two time-locked encryptions produced using two independent RSA modulus $N_1, N_2$. The disadvantage of this construction is an expensive encryption that depends on the hardness parameter $T$ and hence yields an inefficient commitment algorithm. Concretely, one has to compute values $x_i^{2^T} \bmod N_i$ for $i \in [2]$ using $T$ repeated squarings. 

% %Our first observation is that if we can achieve fast encryption and hence fast commitment which is perfectly binding, then we are able to build NITC directly using the Naor-Yung paradigm without using additional NIZKs. Since encryption is fast and perfectly binding, then it is sufficient reveal as opening the message $m$ with the randomness $r$, which were used to produce commitment, and everyone is able to efficiently check that these values are correct. Therefore we do not need any NIZK to prove validity of the opening. Moreover, when relying on the Naor-Yung paradigm the language for the SS-NIZK can be designed in such a way, that it guarantees proper generation of a commitment. Hence, we do not need additional NIZK to enforce this property. 

% %To achieve fast commitment we precompute the value $h_1:=g^{2^T} \bmod N$ where $g$ is an generator of $\qrn$ in trusted setup. This can be done efficiently, since while executing the setup the factorization of $N$ is known. When encrypting a message, we produce an ElGamal-style ciphertext with respect to $g, h_1$. Since the order of the group $\qrn$ is not known, we sample randomness from $\smplset$ which we show is statistically indistinguishable from sampling from the set $[\ord]$. The second ciphertext is ElGamal encryption with respect to secret key $g, h_2: = g^k \bmod N$ where $k$ is chosen uniformly at random from $\smplset$. Lastly, we use SS-NIZK to prove equality of ciphertexts. All of these computations are independent of $T$ and hence the resulting commitment is efficient. In order to be able to prove security of our construction, we need DDH assumption holds in $\qrn$ even if the factorization of $N$ is known. This is however implied by hardness of DDH in the large subgroups of $\Zn^*$ as was shown in \cite{C:CouPetPoi16}.

% %We are able to shorten the ciphertext size by encrypting a message using shared randomness as proposed by Biagioni \etal \cite{SCN:BiaMasVen16}. This adjustment allows to design an efficient Sigma protocol that the commitment is properly generated. Applying the Fiat-Shamir transform \cite{C:FiaSha86} to the Sigma protocol we obtain SS-NIZK. The Sigma protocol essentially proves that the tuple specified by the commitment and the public key is a DDH tuple. Since the order of the group $\qrn$ is not know, we have to design the Sigma protocol in hidden order group. Such protocols are less efficient than Sigma protocols where order of the group is known. 
% %\todo{However, we realize that to obtain SS-NIZK, it is sufficient to prove that the constructed Sigma protocol has negligible soundness and we do not have to care about special soundness. In this way we are able to avoid the strong RSA assumption and therefore we are able to achieve proofs of smaller size. Usually in Sigma protocols in hidden order groups, the first value is sampled from very large interval which increases the overall size of the proof. } Since sampling from $\smplset$ is statistically indistinguishable from sampling from $[\ord]$ this additionally reduces the size of the resulting proof.


% We stress that our construction does not require \emph{special soundness}, which is usually more expensive to achieve in hidden-order groups since to obtain reasonable security guarantees one has to either run the underlying protocols several times in parallel or use large modulus $N$ \cite{SPEED:BKSST}. Instead, for our construction a negligible soundness error is sufficient and therefore our constructions do not suffer with the mentioned drawbacks.
% %so that we can use smaller integers, which reduces computational complexity and proof size further.


%%% Local Variables:
%%% mode: latex
%%% TeX-master: "main"
%%% End:
