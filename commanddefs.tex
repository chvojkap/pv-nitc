%%%%%                   %%%%%%%%%%%%%%%%%%%%%                   %%%%%
%%%%% Take out unneeded commands from this file to avoid bloat. %%%%%
%%%%%                   %%%%%%%%%%%%%%%%%%%%%                   %%%%%

%%%%%%%%%%%%%%%%%%%%%%%%%%%%%%%%%%%%%%%%%%%%%%%%%%%%%%%%%%%%%%%%%%%%%%%%%%%%%%%%%%%%%%%%%%%%%%%%%%%%%%
%%%%                              GROUPS/DISTRIBUTIONS/SETS/LISTS                                 %%%%
%%%%%%%%%%%%%%%%%%%%%%%%%%%%%%%%%%%%%%%%%%%%%%%%%%%%%%%%%%%%%%%%%%%%%%%%%%%%%%%%%%%%%%%%%%%%%%%%%%%%%%
%%%  %%%

% number sets
\newcommand{\x}{\ensuremath{\cdot}}
\newcommand{\Z}{\ensuremath{\mathbb{Z}}\xspace}
\newcommand{\N}{\ensuremath{\mathbb{N}}\xspace}
\newcommand{\Q}{\ensuremath{\mathbb{Q}}\xspace}
\newcommand{\R}{\ensuremath{\mathbb{R}}\xspace}
\newcommand{\F}{\ensuremath{\mathbb{F}}\xspace}

%% Groups, fields and rings
\newcommand{\nats}{\mathbb{N}} % Set of natural numbers
\newcommand{\ints}{\mathbb{Z}} % Set of integers
\newcommand{\prms}{\mathbb{P}} % Set of prime numbers
\newcommand{\Zp}{\ints_p} % Integers modulo p
\newcommand{\Zq}{\ints_q} % Integers modulo q
\newcommand{\Zn}{\ints_N} % Integers modulo N
\newcommand{\Zns}{\ints_{N^2}^*}
\newcommand{\Jn}{\mathbb{J}_N}
\newcommand{\qrn}{\mathbb{QR}_N}
\newcommand{\qrni}{\mathbb{QR}_{N_i}}
\newcommand{\G}{\mathbb{G}} % Group!

% operators
\newcommand{\defeq}{\coloneqq}
%\newcommand{\rand}{\xleftarrow{\smash{\raisebox{-1.75pt}{$\scriptscriptstyle\$$}}}}
\newcommand{\rand}{\stackrel{{\scriptscriptstyle\$}}{\leftarrow}}
\newcommand{\xor}{\oplus}
\newcommand{\concat}{\mathbin\Vert}

%% Distributions
\newcommand{\distr}{\mathfrak{D}}
\newcommand{\ind}{\in_{_{\distr}}}

%% Set-shaped things
\newcommand{\bits}{\{0,1\}} % Bit set
\newcommand{\msgspace}{\mathcal{M}} % Message Space
\newcommand{\tagspace}{\mathcal{T}} % Tag Space. SPPPAAAACEEE!!! SPPPPAAAAAAAAAACCCEEEEE!!!!
\newcommand{\msgspacea}{\msgspace_1} % Message Space Uno!
\newcommand{\msgspaceb}{\msgspace_2} % Message Space Dos!
\newcommand{\repspace}{\mathbf{Y}_{\secpar,\h}} % Message Reprentative Space
\newcommand{\hrepspace}{\mathbf{Y}_{\secpar+\beta,\h}} % Message Reprentative Space
\newcommand{\keyspace}{\mathcal{K}} % Key Space
\newcommand{\sigspace}{\mathfrak{S}} % Signature Space
\newcommand{\sigspacea}{\sigspace_1} % Siganture Space Uno!
\newcommand{\sigspaceb}{\sigspace_2} % Siganture Space Dos!
\newcommand{\inr}{\in_{_R}} % Uniformily random in
\newcommand{\intvalab}[2]{\llbracket #1 , #2 \rrbracket} % Int val a to b
\newcommand{\intval}[1]{\llbracket 1, #1 \rrbracket}  %Int val from 1 to x
\newcommand{\intvalz}[1]{\llbracket 0, #1 \rrbracket} % Int val from 0 to x
\newcommand{\dom}{\mathsf{Domain}} % A domain of some sort.
\newcommand{\ccls}{\mathcal{C}_{\lambda}} % Circuit class
\newcommand{\cclsf}{\mathfrak{F}_{\lambda}} % Circuit class F
\newcommand{\lang}{L}
%\newcommand{\smplset}{[N^2]}
\newcommand{\smplsetqrn}{[\floor{N/4}]}
\newcommand{\smplset}{[\floor{N/2}]}
\newcommand{\hdom}{\mathcal{U}} % domain of a hash function
\newcommand{\himg}{\mathcal{V}} % image of a hash function

%% Lists
\newcommand{\slista}{\mathcal{S}_1} % The list currently known as S
\newcommand{\slistb}{\mathcal{S}_2} % The list currently known as S
\newcommand{\mlista}{\mathcal{M}_1} % The list currently known as M
\newcommand{\mlistb}{\mathcal{M}_2} % The list currently known as M
\newcommand{\qlista}{\mathcal{Q}_1} % The list currently known as Q
\newcommand{\qlistb}{\mathcal{Q}_2} % The list currently known as Q
\newcommand{\alist}{\mathcal{A}} %  The list currently known as A
\newcommand{\clist}{\mathcal{C}} %  The list currently known as F
\newcommand{\hlist}{\mathcal{H}} %  The list currently known as H
\newcommand{\glist}{\mathcal{G}} %  The list currently known as G
\newcommand{\flist}{\mathcal{F}} %  The list currently known as F


%%%%%%%%%%%%%%%%%%%%%%%%%%%%%%%%%%%%%%%%%%%%%%%%%%%%%%%%%%%%%%%%%%%%%%%%%%%%%%%%%%%%%%%%%%%%%%%%%%%%%%
%%%%                                  ADVERSARIES AND SUCH                                        %%%%
%%%%%%%%%%%%%%%%%%%%%%%%%%%%%%%%%%%%%%%%%%%%%%%%%%%%%%%%%%%%%%%%%%%%%%%%%%%%%%%%%%%%%%%%%%%%%%%%%%%%%%
%% Adversaries
\newcommand{\advA}{\mathcal{A}} % Adversary 
\newcommand{\adv}{\mathcal{A}} % Adversary
\newcommand{\advB}{\mathcal{B}} % Simulator
\newcommand{\simr}{\mathcal{B}} % Simulator
\newcommand{\advC}{\mathcal{C}} % Challenger
\newcommand{\chal}{\mathcal{C}} % Challener
\newcommand{\advD}{\mathcal{D}} % Distinguisher
\newcommand{\advE}{\mathcal{E}} % Extractor
\newcommand{\extr}{\mathcal{E}} % Extractor
\newcommand{\dist}{\mathcal{D}} % Distinguiser
\newcommand{\advF}{\mathcal{F}} % Forger
\newcommand{\forger}{\mathcal{F}} % Forger

%% Such
\newcommand{\advtg}{\mathbf{Adv}} % The adversary's advantage
\newcommand{\zk}{\mathbf{ZK}}
\newcommand{\simsnd}{\mathbf{SimSnd}}
\newcommand{\snd}{\mathbf{Snd}}
\newcommand{\qur}{\mathbf{OUR}}

%%%%%%%%%%%%%%%%%%%%%%%%%%%%%%%%%%%%%%%%%%%%%%%%%%%%%%%%%%%%%%%%%%%%%%%%%%%%%%%%%%%%%%%%%%%%%%%%%%%%%%
%%%%                                   ALGORITHMS/PROCEDURES                                      %%%%
%%%%%%%%%%%%%%%%%%%%%%%%%%%%%%%%%%%%%%%%%%%%%%%%%%%%%%%%%%%%%%%%%%%%%%%%%%%%%%%%%%%%%%%%%%%%%%%%%%%%%%
%% Syntax never killed anyody right?
\newcommand{\rnd}{\leftarrow_{\mbox{\tiny{\$}}}} % Randomised algorithm output
\newcommand{\get}{\leftarrow} % Normal Algorithm output
\newcommand{\true}{\mathtt{TRUE}} % TRUE boolean flag
\newcommand{\false}{\mathtt{FALSE}} % FALSE boolean flag

%% Setups and the like.
\newcommand{\setup}{\mathsf{Setup}} % Setup Algorithm, that give us the groups and such.
\newcommand{\keygen}{\mathsf{KeyGen}} % Key Generation Algorithm.
\newcommand{\gen}{\mathsf{Gen}} % Generation Algorithm.
\newcommand{\ggen}{\mathsf{GrpGen}} % Group Generation Algorithm.
\newcommand{\pgen}{\mathsf{PGen}} % Parameters Generation Algorithm.
\newcommand{\kgen}{\mathsf{KGen}} % Key Generation Algorithm.
\newcommand{\genmod}{\mathsf{GenMod}} 

\newcommand{\solve}{\mathsf{Solve}}
\newcommand{\f}{\mathsf{F}}

\newcommand{\preproc}{\mathsf{Preproc}}
\newcommand{\encode}{\mathsf{Encode}}
\newcommand{\decode}{\mathsf{Decode}}


%% Obfuscation
\newcommand{\io}{\mathscript{i}\mathcal{O}}
\newcommand{\dio}{\mathscript{di}\mathcal{O}}
\newcommand{\eo}{\mathscript{e}\mathcal{O}}

%% Encryption Schemes.
\newcommand{\otle}{\mathtt{OTLE}}
\newcommand{\tpke}{\mathtt{TPKE}}
\newcommand{\tlp}{\mathtt{TLP}}
\newcommand{\tlppp}{\mathtt{TLP \mhyphen PP}}
\newcommand{\stlp}{\mathtt{sTLP}}
\newcommand{\wtlp}{\mathtt{wTLP}}
\newcommand{\pptlp}{\mathtt{ppTLP}}
\newcommand{\re}{\mathtt{RE}}
\newcommand{\tre}{\mathtt{TRE}}
\newcommand{\trepp}{\mathtt{TRE \mhyphen PP}}
\newcommand{\hm}{\mathtt{H}}
\newcommand{\htre}{\mathtt{HTRE}}
\newcommand{\ftre}{\mathtt{FTRE}}
\newcommand{\pke}{\mathtt{PKE}}
\newcommand{\he}{\mathtt{HE}}
\newcommand{\e}{\mathtt{E}}
\newcommand{\fhe}{\mathtt{FHE}}
\newcommand{\we}{\mathtt{WE}}
\newcommand{\eowe}{\mathtt{EOWE}}
\newcommand{\owe}{\mathtt{OWE}}
\newcommand{\fwe}{\mathtt{FWE}}
\newcommand{\ofwe}{\mathtt{OFWE}}
\newcommand{\eofwe}{\mathtt{EOFWE}}
\newcommand{\tbe}{\mathtt{TBE}}
%\newcommand{\tbenc}{\mathtt{TBE}}
\newcommand{\penc}{\mathtt{PE}}
% Encryption Algorithms
\newcommand{\enc}{\mathsf{Enc}}
\newcommand{\dec}{\mathsf{Dec}}
\newcommand{\decf}{\mathsf{Dec}_f}
\newcommand{\decs}{\mathsf{Dec}_s}
\newcommand{\eval}{\mathsf{Eval}}
% Tag Encryption Algorithms
\newcommand{\tenc}{\mathsf{TEnc}}
\newcommand{\tdec}{\mathsf{TDec}}
% Puncturable Encryption Algorithms
\newcommand{\punct}{\mathsf{Punct}}
\newcommand{\pdec}{\mathsf{PDec}}
% Functional Encryption Algorithms
\newcommand{\fe}{\mathtt{FE}}
\newcommand{\msk}{\mathsf{msk}}
\newcommand{\FF}{\mathcal{F}}
\newcommand{\ID}{\mathcal{ID}}
\newcommand{\MM}{\mathcal{M}}
\newcommand{\OO}{\mathcal{O}}
\newcommand{\XX}{\mathcal{X}}
\newcommand{\YY}{\mathcal{Y}}
\newcommand{\ff}{f}
\newcommand{\xx}{x}
\newcommand{\yy}{y}
\newcommand{\Sim}{\mathcal{S}}
\newcommand{\simgen}{\mathsf{\widetilde{Gen}}}
\newcommand{\simkeygen}{\mathsf{\widetilde{KeyGen}}}
\newcommand{\simenc}{\mathsf{\widetilde{Enc}}}
\newcommand{\trfe}{\mathsf{TRFE}}


\newcommand{\ro}{\mathsf{RO}}
\newcommand{\gss}{\mathtt{GSS}}
\newcommand{\fac}{\mathtt{Factor}}
\newcommand{\factor}{\mathsf{Factor}}
\newcommand{\dla}{\mathsf{DL}_\adv}
\newcommand{\gnr}{\mathsf{GNR}}
\newcommand{\fail}{\mathsf{FAIL}}
\newcommand{\face}{\mathsf{FACTOR}}
\newcommand{\gsse}{\mathsf{GSS}}
\newcommand{\success}{\mathsf{SUCCESS}}
\newcommand{\forge}{\mathsf{FRG}} 
\newcommand{\event}{\mathsf{E}} 


%% Signature Schemes.
\newcommand{\ots}{\mathtt{OTS}} % One-time Signature Scheme
\newcommand{\sig}{\mathtt{Sig}} % Signature Scheme.
% Signature Algorithms
\newcommand{\sign}{\mathsf{Sign}} % Signing Algorithm.
\newcommand{\verify}{\mathsf{Verify}} % Verification Algorithm.
\newcommand{\vrfy}{\mathsf{Vrfy}} % Verification Algorithm.


%% Commitments
\newcommand{\tc}{\mathtt{TC}}
\newcommand{\nitc}{\mathtt{NITC}}
\newcommand{\com}{\mathsf{Com}}
\newcommand{\cvrfy}{\mathsf{ComVrfy}}
\newcommand{\decom}{\mathsf{Dec}}
\newcommand{\dvrfy}{\mathsf{DecVrfy}}
\newcommand{\fdecom}{\mathsf{FDec}}
\newcommand{\fdvrfy}{\mathsf{FDecVrfy}}


%% NIZK/SNARK
\newcommand{\nizk}{\mathtt{NIZK}}
\newcommand{\niwi}{\mathtt{NIWI}}

%% Proof systems
\newcommand{\poe}{\mathtt{PoE}}
\newcommand{\prv}{\mathsf{Prover}} % Prover
\newcommand{\vrf}{\mathsf{Verifier}} %Verifier
\newcommand{\snark}{\mathtt{SNARK}}
\newcommand{\prove}{\mathsf{Prove}}
%\newcommand{\simul}{\mathsf{Sim}}
\newcommand{\crs}{\mathsf{crs}} % Common Reference String


%% Hashes, Chameleon or otherwise.
\newcommand{\CRH}{\mathsf{CRHF}} % Collision Resistant Hash Function.
% Collision Resitant Hash Algorithms.
\newcommand{\hash}{\mathsf{Hash}} % Hash Algorithm.
\newcommand{\h}{\mathtt{H}}

%% Oracles
\newcommand{\ddhvf}{\mathsf{DDHvf}} % The DDH Oracle
\newcommand{\dss}{\mathsf{DSSvf}} % The Decisional Sequential Squaring oracle
\newcommand{\deco}{\mathsf{DEC}} %decryption oracle
\newcommand{\signo}{\mathsf{SIGN}} %signing oracle
\newcommand{\kgeno}{\mathsf{KEYGEN}} %signing oracle

%%%%%%%%%%%%%%%%%%%%%%%%%%%%%%%%%%%%%%%%%%%%%%%%%%%%%%%%%%%%%%%%%%%%%%%%%%%%%%%%%%%%%%%%%%%%%%%%%%%%%%
%%%%                                         ASSUMPTIONS                                          %%%%
%%%%%%%%%%%%%%%%%%%%%%%%%%%%%%%%%%%%%%%%%%%%%%%%%%%%%%%%%%%%%%%%%%%%%%%%%%%%%%%%%%%%%%%%%%%%%%%%%%%%%%
\newcommand{\dl}{\ensuremath{\mathsf{DLog}}\xspace} % The Discrete Logarithm Assumption
\newcommand{\ddh}{\ensuremath{\mathsf{DDH}}\xspace} % The Decisional Diffie-Hellman Assumption
\newcommand{\dlin}{\ensuremath{\mathsf{DLin}}\xspace} % The Descision Linear Assumption
\newcommand{\gdlin}{\ensuremath{\mathsf{GapDLin}}\xspace} % The Descision Linear Assumption
\newcommand{\ssa}{\ensuremath{\mathsf{SS}}\xspace}
\newcommand{\sss}{\ensuremath{\mathsf{SSS}}\xspace}
\newcommand{\dcr}{\ensuremath{\mathsf{DCR}}\xspace} % The Decisional Diffie-Hellman Assumption

%%%%%%%%%%%%%%%%%%%%%%%%%%%%%%%%%%%%%%%%%%%%%%%%%%%%%%%%%%%%%%%%%%%%%%%%%%%%%%%%%%%%%%%%%%%%%%%%%%%%%%
%%%%                                    KEYS & CONSTANTS                                          %%%%
%%%%%%%%%%%%%%%%%%%%%%%%%%%%%%%%%%%%%%%%%%%%%%%%%%%%%%%%%%%%%%%%%%%%%%%%%%%%%%%%%%%%%%%%%%%%%%%%%%%%%%
%% Keys
\newcommand{\gk}{\mathsf{gk}} % Description of the pairing group
\newcommand{\pk}{\mathsf{pk}} % Public Key.
\newcommand{\sk}{\mathsf{sk}} % Secret Key (Overloads to Signing Key).
\newcommand{\ek}{\mathsf{ek}} % Encryption Key.
\newcommand{\dk}{\mathsf{dk}} % Decryption Key.
\newcommand{\td}{\mathsf{td}} % Trapdoor.
\newcommand{\vk}{\mathsf{vk}} % Verfication Key.
\newcommand{\otsk}{\mathsf{sk_{OT}}} % One Time Signing Key.
\newcommand{\csk}{\mathsf{sk^*_{OT}}} % One Time Signing
\newcommand{\otvk}{\mathsf{vk_{OT}}} % One Time Verification Key.
\newcommand{\cvk}{\mathsf{vk^*_{OT}}} % One Time Verification Key.
\newcommand{\spr}{\mathsf{sp}} % System Parameters.
\newcommand{\pp}{\mathsf{pp}}
\newcommand{\ppe}{\mathsf{pp}_e} % Encryption parameters
\newcommand{\ppd}{\mathsf{pp}_d} % Decryption parameters
\newcommand{\ppei}{\mathsf{pp}_{e,i}} % Encryption parameters
\newcommand{\ppdi}{\mathsf{pp}_{d,i}} % Decryption parameters
\newcommand{\ppej}{\mathsf{pp}_{e,j}} % Encryption parameters
\newcommand{\ppdj}{\mathsf{pp}_{d,j}} % Decryption parameters
\newcommand{\ppeft}{\mathsf{pp}_{e,T_i,\FF'}} % Encryption parameters
\newcommand{\ppdft}{\mathsf{pp}_{d,T_i,\FF'}} % Encryption parameters
\newcommand{\param}{\mathsf{par}} % parameters 
\newcommand{\oracle}{\mathcal{O}}
\newcommand{\funs}{\mathsf{Funs}[\hdom, \himg]}
\newcommand{\rerand}{\mathsf{Rand}}

%% Constants

%%%%%%%%%%%%%%%%%%%%%%%%%%%%%%%%%%%%%%%%%%%%%%%%%%%%%%%%%%%%%%%%%%%%%%%%%%%%%%%%%%%%%%%%%%%%%%%%%%%%%%
%%%%                                          NAMES                                               %%%%
%%%%%%%%%%%%%%%%%%%%%%%%%%%%%%%%%%%%%%%%%%%%%%%%%%%%%%%%%%%%%%%%%%%%%%%%%%%%%%%%%%%%%%%%%%%%%%%%%%%%%%
\newcommand{\hexa}[1]{\mathtt{0x#1}} % Hexadecimal values

%% Security Defs
\newcommand{\ufcma}{\mathsf{UF \mhyphen CMA}}
\newcommand{\ufnma}{\mathsf{UF \mhyphen NMA}}
\newcommand{\rom}{\mathsf{ROM}}
%\newcommand{\sm}{\mathsf{StdM}}

\newcommand{\cpa}{\mathsf{CPA}}
\newcommand{\cca}{\mathsf{CCA}}
\newcommand{\ror}{\mathsf{RoR}}
%%Known Schemes
\newcommand{\ktbe}{\mathtt{KiltzTBE}}





%%%%%%%%%%%%%%%%%%%%%%%%%%%%%%%%%%%%%%%%%%%%%%%%%%%%%%%%%%%%%%%%%%%%%%%%%%%%%%%%%%%%%%%%%%%%%%%%%%%%%%
%%%%                                           MISC                                               %%%%
%%%%%%%%%%%%%%%%%%%%%%%%%%%%%%%%%%%%%%%%%%%%%%%%%%%%%%%%%%%%%%%%%%%%%%%%%%%%%%%%%%%%%%%%%%%%%%%%%%%%%%
\mathchardef\hyphen="2D % Hyphen. Because it has to be done this way for reasons.
\mathchardef\mhyphen="2D % Hyphen. Because it has to be done this way for reasons.
\newcommand{\tab}{\hspace*{2em}} % Tabs for code
\newcommand{\half}{\frac{1}{2}} % Half
\newcommand{\qtr}{\frac{1}{4}} % Quarter
\newcommand{\et}{$e^{\text{th}}$} % e-th cause why not make it easy?
\newcommand{\secpar}{\ensuremath{\lambda\xspace}} % Security Parameter
\newcommand{\seck}{1{^\secpar}} % Unary represntation of our security parameter
\newcommand{\seckt}{1^{2\secpar}} % Unary represntation of our security parameter DOUBLED!
\newcommand{\negl}{\mathsf{negl}}
\newcommand{\poly}{\mathsf{poly}}
\newcommand{\polylog}{\mathsf{polylog}}
\newcommand{\inp}{\mathsf{input}}
\newcommand{\tilT}{\tilde{T}}
\newcommand{\undT}{\underline{T}}
\newcommand{\undE}{\underline{\epsilon}}
\newcommand{\dep}{\mathsf{depth}}
\newcommand{\aux}{\mathsf{aux}}
%\newcommand{\max}{\mathsf{max}}

\newcommand{\win}{\mathsf{Win}}
\newcommand{\query}{\mathsf{Query}}
\newcommand{\games}{\mathsf{G}} % game
\newcommand{\expe}{\mathsf{Exp}} % game
\newcommand{\te}{t_{e}} % run time of encryption
\newcommand{\tfd}{t_{fd}} % run time of fast decryption
\newcommand{\tsd}{t_{sd}} % run time of slow decryption
\newcommand{\st}{\mathsf{st}} % state
\newcommand{\gap}{\epsilon} % gap
\newcommand{\ugap}{\underline{\gap}}

\newcommand{\epke}{\epsilon_{\pke}}
\newcommand{\etbe}{\epsilon_{\tbe}}
\newcommand{\etlp}{\epsilon_{\tlp}}
\newcommand{\eots}{\epsilon_{\ots}}

\newcommand{\simul}{\mathsf{Sim}}

\newcommand{\clock}{\mathcal{C}}
\newcommand{\rel}{\mathcal{R}}

\newcommand{\circC}{C_{\sk}} % circuit with hardcoded secret key
\newcommand{\obfC}{\widetilde{C}_{\sk}} % obfuscated circuit with hardcoded secret key

\newcommand{\plm}{\mathsf{p}(\secpar)}
\newcommand{\qlm}{\mathsf{q}(\secpar)}
%\newcommand{\plm}{\alpha(\lambda)}
%\newcommand{\qlm}{\poly(\lambda,1/(\varepsilon - \alpha(\secpar)))}
\usepackage{stmaryrd}
\newcommand{\heading}[1]{\paragraph{\sc #1}}

\newcommand{\T}{\mathsf{T}}
\newcommand{\np}{\mathtt{NP}}

\newcommand{\ceil}[1]{\left\lceil #1 \right\rceil}
\newcommand{\floor}[1]{\left\lfloor #1 \right\rfloor}
\newcommand{\abs}[1]{\left| #1 \right|}
\newcommand{\sd}{\mathbb{SD}}
\mathchardef\mhyphen="2D
\newcommand{\ord}{\varphi(N)/2}
\newcommand{\ordqrn}{\varphi(N)/4}
\newcommand{\estord}{\floor{N/2}}
\newcommand{\estordqrn}{\floor{N/4}}
\newcommand{\prot}[1]{\langle #1 \rangle}

\newcommand{\fm}{a} % first message
\newcommand{\sm}{c} % second message
\newcommand{\tm}{z} % third message

\newcommand{\frset}{A} %first set
\newcommand{\scset}{\mathcal{C}} %second set
\newcommand{\thset}{\mathcal{Z}} %third set
\newcommand{\sset}{\mathcal{S}}
\newcommand{\wset}{\mathcal{W}}

\newcommand{\pr}{\mathsf{P}} % Prover algorithm
\newcommand{\vr}{\mathsf{V}} %Verifier algorithm



%%%%%%%%%%%%%%%%%%%%%%%%%%%%%%%%%%%%%%%%%%%%%%%%%%%%%%%%%%%%%%%%%%%%%%%%%%%%%%%%%%%%%%%%%%%%%%%%%%%%%%
%%%%                                    COMMENTS AND NOTES                                        %%%%
%%%%%%%%%%%%%%%%%%%%%%%%%%%%%%%%%%%%%%%%%%%%%%%%%%%%%%%%%%%%%%%%%%%%%%%%%%%%%%%%%%%%%%%%%%%%%%%%%%%%%%
\newlength{\strutdepth}%
\settodepth{\strutdepth}{\strutbox}%
\newcommand{\saqib}[1]{%
	\noindent{\bfseries
	\color{orange}{#1}\color{black}}%
    \strut\vadjust{\kern-\strutdepth%
        \vtop to \strutdepth{%
            \baselineskip\strutdepth%
            \vss\llap{{\large\color{blue}Saqib\quad\color{black}}}\null%
        }%
    }%
}
\newcommand{\peter}[1]{%
	\noindent{\bfseries
	\color{orange}{#1}\color{black}}%
    \strut\vadjust{\kern-\strutdepth%
        \vtop to \strutdepth{%
            \baselineskip\strutdepth%
            \vss\llap{{\large\color{green}Peter\quad\color{black}}}\null%
        }%
    }%
}


\newcommand{\daniel}[1]{{\color{blue} \textbf{Daniel:} #1}}
\newcommand{\christoph}[1]{{\color{blue} \textbf{Christoph:} #1}}
\newcommand{\tibor}[1]{{\color{blue} \textbf{Tibor:} #1}}


\newcommand{\etal}{\emph{et~al.}\xspace}
\newcommand{\eg}{\emph{e.g.}\xspace}
\newcommand{\ie}{\emph{i.e.}\xspace}


\newcommand*{\numero}{n\kern-.1em \raise.7ex\vbox{\hbox{\tiny \ensuremath{\circ}}\kern.5pt}}

\newcommand{\todo}[1]{{\marginnote{\textcolor{red}{TODO}}\textcolor{red}{(todo: #1)}}}

%for submission
\crefname{appendix}{supplementary material}{supplementary materials}
\Crefname{appendix}{Supplementary Material}{Supplementary Materials}

%%%%%%%%%%%%%%%%%%%%%%%%%%%%%%%%%%%%%%%%%%%%%%%%%%%%%%%%%%%%%%%%%%%%%%%%%%%%%%%%
\newcommand{\newsequenceofgames}[1]{
  \newcounter{#1}
  \setcounter{#1}{-1}

  \ifx \GameID \undefined
    \newcommand{\GameID}{#1}
  \else
    \renewcommand{\GameID}{#1}
  \fi

  \ifx \PrevLabel \undefined
    \newcommand{\PrevLabel}{\GameID.NULL}
  \else
    \renewcommand{\PrevLabel}{\GameID.NULL}
  \fi

  \ifx \ThisLabel \undefined
    \newcommand{\ThisLabel}{\GameID.NULL}
  \else
    \renewcommand{\ThisLabel}{\GameID.NULL}
  \fi
}


\newcommand{\nextgame}[1]{
  \let\PrevLabel\ThisLabel
  \renewcommand{\ThisLabel}{\GameID.#1}
  \refstepcounter{\GameID}\label{\GameID.#1}
  \paragraph{Game~\arabic{\GameID}.}
}

\newcommand{\thisgame}{{\ref{\ThisLabel}}\xspace}
\newcommand{\prevgame}{{\ref{\PrevLabel}}\xspace}


% workaround for making sure that list/tuples break at the comma
% modified from https://tex.stackexchange.com/questions/19094/allowing-line-break-at-in-inline-math-mode-breaks-citations/19100#19100
\mathchardef\breakingcomma\mathcode`\,
{\catcode`,=\active
    \gdef,{\breakingcomma\discretionary{}{}{}}
}
\newcommand{\mathlist}[1]{\ensuremath{\mathcode`\,=\string"8000 #1}}
%%%%%%%%%%%%%%%%%%%%%%%%%%%%%%%%%%%%%%%%%%%%%%%%%%%%%%%%%%%%%%%%%%%%%%%%%%%%%%%%
%%% Local Variables:
%%% mode: latex
%%% TeX-master: "main"
%%% End:

